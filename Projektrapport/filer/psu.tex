% PSU
\chapter{Power Supply Unit}

Det følgende kapitel omhandler spændingsforsyning til hele Baby Watch systemet. Kapitlet er orienterende for uddybende information [\textit{Se projektdokumentationen pp. 187-194}]. Nedenstående forsynes at et 12 V DC batteri. 12 V batteriet skal nedkonverteres til de 3 nedenstående spændinger: 
\begin{itemize}
\item \textbf{Controller:} Kræver 5 V / 1200 mA (Raspberry Pi + Betjeningspanel)

\item \textbf{Vuggesystem:} Kræver 3.3 V / 500 mA (PSoC + Motorstyringskreds)

\item \textbf{Intelligent Lydmonitor:} Kræver 7.5 V / 1500 mA (BlackFin 533 kit + Mikrofonkreds)
\end{itemize}

Det er den Intelligente Lydmonitor der kræver størst spænding, derfor foretages første nedkonvertering fra 12 V til 7.5 V med en DC-DC konverter og de to resterende spændinger på 3.3 V og 5 V reguleres med spændingsregulatorer. De to regulatorer regulerer fra 7.5 V og ikke 12 V for at sænke effektforbruget i regulatorerne. 

\section{Design}

DC-DC konverteren (LM2696S) er af typen stepdown. Den tager et input på 5-35 V og har justerbart output på 1,25-30 V. Det maksimale strømoutput er på 4 A og overstiger effekten 15 W skal konverteren køles. I dette system forbruges maksimalt 2.5 A, ca. 19 W, der er derfor monteret køleplader på konverteren. Figur \ref{psu:DCkonv} viser DC-DC konverteren, som det ses er der 3 indstillelige potentiometre som indstiller ønsket spænding, strøm. Herudover er der en diode til at indikere hvis strømmen når over et indstillet niveau.  

\figur{0.4}{psu/LM2596S_fabriksbillede}{Fabriksbillede af LM2596S - DC-DC konverter}{psu:DCkonv} 

\newpage

Regulatoren (LM317) benyttes til at regulere fra 7.5 V til hhv. 3.3 V og 5 V. Nedenstående Figur \ref{psu:LM317_schematic} angiver kredsløbsdiagrammet for én LM317.

\figur{0.5}{psu/LM317_schematic.pdf}{Schematic for LM317}{psu:LM317_schematic}

Standardværdierne for komponenterne i \ref{psu:LM317_schematic} er: C1=420uF, C2=220nF, C3=100nF og R1=240$\Omega$. Modstanden R2 er beregnet til $720\Omega$ for 5 V og $390\Omega$ for 3.3 V, se ligningerne \ref{eq:Vout5V} og \ref{eq:Vout3.3V}. Strømmen på 50uA i de to ligningerne er strømmen der løber fra ben 1. De 1,25 V er spændingsforskellen mellem ben 2 og ben 1. 

\begin{equation} 
{ V }_{ out }=1,25V\cdot \left( 1+\frac { 720\Omega  }{ 240\Omega  }  \right) +50uA\cdot 720\Omega =5V
\label{eq:Vout5V}
\end{equation}
\begin{equation} 
{ V }_{ out }=1,25V\cdot \left( 1+\frac { 390\Omega  }{ 240\Omega  }  \right) +50uA\cdot 390\Omega =3,3V
\label{eq:Vout3.3V}
\end{equation}

Controlleren sender signal til PSUen om hvorvidt de 3.3 V og 7.5 V udgange skal være ON. Problematikken om at afbryde forsyningen til disse to udgange løses med et relæ. Relæet der designes efter er et Finder 40.52S. Raspberry Piens GPIO port kan ikke drive relæet, derfor indsættes en transistor som vha. de tilgængelige 7.5 V efter DC-DC konverteren kan løse denne opgave. Figur \ref{psu:relayControl} viser kredsløbsdiagrammet for relæ styringen. Modstanden på $18k\Omega$ er indsat for at begrænse strømmen fra Raspberry Pien til 0.1 mA. 

\figur{0.6}{psu/relayControl.pdf}{Kredsløbsdiagram for relæ styringen}{psu:relayControl}

Det samlede kredsløbsdiagram ses på Figur \ref{psu:PSUsamlet} som herefter er tegnet i EAGLE og sendt til printproduktion. 
\figur{1}{psu/PSUmultisim.pdf}{Samlet kredsløbsdiagram for PSU'en}{psu:PSUsamlet}

\section{Implementering}

Implementeringen ses på figur \ref{psu:PSU_forex_monteret}. Til venste ses bagsiden hvorpå det producerede print og DC-DC konverteren er monteret. Der er desuden placeret en blæser, da effektforbruget er så højt at komponenterne kræver køling. For beregninger af effektforbrug, [\textit{Se projektdokumentationen p. 190}]

\figur{1}{psu/PSU_monteret.pdf}{PSU monteret på Forex-plade.}{psu:PSU_forex_monteret}

