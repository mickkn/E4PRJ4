% Erfaringer
\chapter{Erfaringer}

\section{Hardware erfaringer}
Projektets hardwareudviklingsmæssige problematikker har primært ligget ved Vuggesystemet. I særdeleshed har projektgruppen erfaret vigtigheden af hensigtsmæssigt EMC-design for dette system, idet støj-emission fra motoren kan påvirke operationen af mange andre dele af systemet. 

\section{Software erfaringer}
Softwaremæssigt har vi erfaret vigtigheden af at undersøge og kende den embeddede enhed softwaren i sidste ende skal implementeres på. Specifikt er der ved udviklingen af Intelligent Lydmonitor blevet gjort hårdtkøbte erfaringer med ADSP-BF533 processoren og den fordel der i sidste ende lå i at implementere i kodesproget C frem for C++.

\section{Generelle erfaringer}
Projektstyringsmæssigt har projektgruppen erfaret at det, når der arbejdes sideløbende med et undervisningsforløb på universitet, er uhensigtsmæssigt at bruge SCRUM som arbejdsform. SCRUM kræver at man holder mange statusmøder i løbet af en uge, og hvis projektgruppemedlemmer grundet sideliggende undervisningspres, ikke har lavet projekt relaterede opgaver i flere dage, er det overflødigt at holde disse møder. Gruppen har ligeledes oplevet at de øvrige SCRUM-møder (sprint-planlægning, retrospekt) i mange situationer har været for tids-tunge i forhold til de arbejdstimer der ellers brugt på projektet.