%
\chapter{Projektafgrænsninger}
\label{projektafgraensninger}
Ved projektets afslutning er der en række af de på forhånd specifiserede krav der ikke er opfyldt, og en række moduler der ikke er implementeret færdigt. Her følger en beskrivelse af de forskellige afgrænsninger der har været nødvændige samt hvorfor. 

\subsection{Strømforsyning}
Den i systemet implementerede strømforsyning er modificeret fra designet specificeret i \textit{Power Supply Unit} afsnittet i projektdokumentationen, idet 3V3 linjen blev overflødig da vuggesystemets vinkelsensor blev udskiftet undervejs til en sensor som krævede 5V i stedet. Af denne årsag blev 3V3 linjen lavet om til en ekstra 5V linje ved at ændre værdien af R2 som specificert i \textit{8.1.2 Spændingsregulator (LM317)} -> \textit{3,3V til vuggesystem} således at den matcher modstandsværdien udregnet i afsnittet \textit{5 V til Controller}.

\subsection{Mikrofon kreds i intelligent lysmonitor}
NOGET OM AT DEN MANGLER...

\subsection{Timer og lyd på webserver}
På webserveren er en timer der indikerer tiden babyen har sovet i Baby Watch barnevognen og et alarmlyd på hjemmesiden ikke blevet implementeret. Dette hører under fremtidig udvikling af projektet. 

\subsection{Marginalt stabil regulering af vuggevinkel}
Som et resultat af udstrakte problemer med støj fra motorstyrignskredsløbet, var det ikke muligt at begynde test og kalibrering af vuggesystemets regulering før kort inden projektets afslutning. Problemet med støj blev først forsøgt løst gennem de forskellige emc betragtninger nævnt i EMC rapporten,\footnote{se \citep{cd} \textit{/EMC/EMC Rapport Gruppe 1.pdf}} og da dette viste sig ikke at være nok blev det besluttet at lave galvanisk adskillelse af motorstyringskredsløbet, med eget batteri og optokoblere på styresignalerne, som beskrevet i projektdokumentationen afsnit \textit{7.2.1 Hardware Implementering}. Før dette blev gjort var det ikke muligt at benytte systemets vinkelsensor da der kom støj i systemets stelplan, og denne sensor viste sig meget følsom over for dette. Som en ekstra foranstaltning er selve sensoren også blevet pakket ind i sølvpapir for at skærme den fra feltforstyrrelser fra motoren, hvilket er beskrevet yderligere i afsnittet \vref{fremtidigtArbejde}.\\
Som en konsekvens af dette er det ikke nået at opnå det ønskede niveau af stabilitet i den implementerede regulering. For en gennemgang af den benyttede fremgangsmåde samt tiltænkt procedure for færdig implementering se afsnit \textit{Vuggesystem} -> \textit{7.3.2 Softwareimplementering} -> \textit{regulering} i projektdokumentationen. For at sikre at den implementerede prototype kan fungere på som en samlet enhed på det implementerede niveau er der tilføjet en software begrænsning af de mulige vuggemønstre der kan sendes til vuggesystemet, da dette ikke er stabilt i hele det tiltænkte område.