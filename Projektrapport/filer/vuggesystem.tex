% Vuggesystem
\chapter{Vuggesystem}\label{vuggesys}
Vuggesystemet består af en PSoC4 microcontroller som reguleringsmodul til et mekanisk system der vugger barnevognskurven monteret på Baby Watch barnevognen. \\ PSoC4 microcontrolleren bruger en sensorkreds med indbygget accelerometer og gyroskop placeret på akslen barnevognskurven vugges over til at måle vinklen i forhold til tyngdefeltet. \\ I hver af ende af barnevognen er placeret en enstopsensor. Disse registrerer når det mekaniske vuggesystems ydre vuggegrænse er nået og ved aktivering nødstoppes vuggesystemet. \\ Vuggesystemet kan også kommunikere med Controller dette sker ved hjælp af et I2C-interface. Denne kommunikation består af parametre send til vuggesystemet om hvilken frekvens og hvilken vinkel-amplitude barnevognskurven skal vugges med samt en indikation på eventuelle fejl i vuggesystemet som Controller kan aflæse.

Til at trække det mekaniske vuggesystem er en motor monteret med en tilhørende motorkreds. \\ Denne motorkreds styres af et dobbelt PWM-signal reguleret udfra beregninger lavet på baggrund af barnevognskurvens aktuelle vinkel i forhold til tyngdefeltet og input parametrene sendt fra Controller.

De efterfølgende afsnit beskriver kort vuggesystemets Systemarkitektur, Design  og Implementering samt en kort opremsning af Modultests af vuggesystemet.

\section{Systemarkitektur}
\label{vs_sysark}
Systemarkitekturen for vuggesystemet er udarbejdet ud fra use casene fra kravspecifikationen i Baby Watch projektdokumentationen \textit{Kapitel 3 Kravspecifikation} -> \textit{3.2 Use cases}. Systemarkitekturen er dokumenteret med en BDD, IBD,
\figur{1}{vuggesystem/Vuggesystem_IBD}{IBD diagram for Vuggesystem}{vuggesystem:ibd}


\section{Design}
\label{vs_design}

\figur{1}{vuggesystem/reguleringsMCU_klassediagram.pdf}{Klassediagram for Vuggesystem}{vuggesystem:klassediagram}




\section{HW Design og Implementering}
\label{vs_HW}
Da vuggesystemet er opbygget omkring en PSoC 4 chip er det kun effektdrivende hardware der har været nødvændigt at implementere eksternt, hvorfor det kun er Motorstyringskredsen der er implementeret som et selvstændigt kredsløb, mens systemets sensorer kan kobles direkte til PSoC'en. De to endstop switches er samlet i et signal for at spare en pin. Herunder ses et diagram over det implementerede hardware:\\
\figur{1}{vuggesystem/HW_schematic.pdf}{Samlet kredsløbsdiagram for Vuggesystem}{vuggesystem:hwdesign}
Det ses her at en H-bro med power mosfet's styret af dobbelt siddede MOS drivere benyttes til at styre motoren. Dette kræver et PWM signal som inverteres, med et dødbånd for at sikre mod at der sendes strøm forbi motoren ved skift, hvilket igen er valgt implementeret på PSoC'en i stedet for i ekstern HW. \\ Systemet indeholder desuden to optokoblere som benyttes til at skabe galvanisk adskillelse af motorkredsen som har tendens til støj.
For en dybdegående gennemgang af design og implementering af ovenstående kreds henvises til projektdokumentationens afsnit \textit{7.2.1 Hardware Design} og \textit{7.3.1 Hardware Implementering}.\\



\subsection*{Software}
\label{vs_implementering_sw}


\subsection*{Hardware}
\label{vs_implementering_hw}

\section{Modultests}
\label{vs_modultests}