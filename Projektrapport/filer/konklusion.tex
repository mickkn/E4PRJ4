% Konklusion
\chapter{Konklusion}
Følgende afsnit består først af en fælles konklusion på projektet Baby Watch efterfulgt af individuelle konklusioner.

Gennemgående er alle enige om at projektet har været en succes. Fra start var der enighed om at projektet skulle ende ud i en prototype der var i stand til at udføre kernefunktionaliteterne for Baby Watch systemet. Dette skulle sammenholdes med en løbende projektdokumentation således at dokumentationsbyrden blev fordelt ud over hele semesteret og hele projektet skulle laves udfra en ASE-model der brugte udviklingsværktøjet SCRUM til at styre projektforløbet. Det blev undervejs erfaret at selv en agil projektstyringsmodel som SCRUM kan blive lidt tung for et projekt med så få mennesker, og med en så begrænset timeramme, som et semesterprojekt. En stram definition af grænseflader mellem de forskellige system elementer, sammen med nogle tydelige deadlines har dog resulteret i at vi relativt problemfrit kunne overgå til selvstyring i de forskællige grupper, og opnå en større effektivitet i resten af projektet. 


\section{Individuelle konklusioner}

De følgende afsnit indeholder individuelle konklusioner for alle gruppes medlemmer.

%Konklusion FB
\subsection*{Felix Blix Everberg}
Ved starten af dette semester-projekt, havde jeg som mål at lave et projekt som nåede at blive færdiggjort til et fungerende niveau. Dette er for første gang i mit studie lykkedes, hvilket har været en stor tilfredsstillelse for mig. Både på dette og tidligere semestre, har det været et mål for mig og min gruppe at vi ville implementere løbende frem for til sidst. Dette har fungeret bedre på dette semester, og det tror jeg været medvirkende til at mit første mål også er lykkedes. Derudover har det som håbet også resulteret i en mindre dokumentationsbyrde i projektets sidste fase, idet flere elementer har kunnet færdigdokumenteres tidligere som resultat af den løbende implementering. \\
Slutteligt er det værd at nævne, at det har været lærerigt for mig personligt at projektet på netop 4 semester, hvor man har E4EMC, indeholdt så markante udfordringer omkring kobling som vi oplevede i forbindelse med motorkredsen og vinkelsensoren. \\


%Konklusion JH
\subsection*{Jeppe Hofni}
For mig, har dette semesterprojekt været det mest omfavnende projekt i forhold til at bruge den viden jeg har tilegnet mig på ingeniørhøjskolen. Jeg har i høj grad inddraget min viden fra fagene I4IKN, E4DSA, E4ISU og E4IRT. E4IKN har i særdeleshed været brugbart i forhold til at udvikle http-webserven til Baby Watch hjemmesiden, her har jeg måtte sætte mig ind i flere programmeringssprog for at kunne opsætte en funktionel webserver.\\
I udviklingen af vuggesystemet har jeg primært arbejdet med softwareopsætningen af systemet. Dette har været en iterativ proces der har resulteret i mange design og implementerings ændringer. Disse har hovedsageligt har været baseret på at få optimeret store funktionaliteter og derved mange beregninger ned på en microcontroller platform med begrænset regnekraft og hukommelse.\\
Det skal nævnes at fordybelsen i en udviklingsproces der hele tiden bliver afbrudt af en skoledag ikke i på nogen måde kan sammenlignes med at udvikle fuldtid, dette blev afprøvet i påskeferien med udviklingen af webserveren. Alt i alt har 4. semesterprojekt været det bedste projekt lærings- og resultatmæssigt.

%Kommentar fra Felix:
% Jeg tænker snilt at linje 4 kunne koges noget ned ved at ændre formuleringen, og så 
% måske overveje om det er relevant for den personlige konklusion at valget stod mellem 
% psoc 4 og 5 frem for bare at holde sig til hvad du allerede siger, nemlig at en af 
% hovedudfordringerne var at implementere til en begrænset platform. Resten ser super 
% ud. Har rettet et par stavefejl og indsat nogle linjeskift for læselighedens skyld.

%Konklusion KB
\subsection*{Kristian Boye Jakobsen}
I dette semester-projekt har især viden fra semesterfaget E4DSA været essentielt for mig. Til udviklingen af den Intelligente Lydmonitor har jeg benyttet analyseredskaber givet i E4DSA, samt tilegnet ny viden fra nettet. 
Desuden har jeg selvstændigt tilegnet mig viden om DSP-processering på den embedded platform ADSP-BF533 samt udvikling af projekter i Crosscore Embedded Studio. Desuden har jeg benyttet viden fra faget E4ASD og E4EMC til udvikling af blandt andet mikrofon-forstærkeren.
Samarbejdet med Lukas har fungeret godt, vi har været gode til at supplere hinanden.
For at opsummere har projektet været med til at udvikle mine kompetencer som ingeniør, især inden for digital signal behandling, digital signal analyse og analog signaldesign. 


%Konklusion LH
\section{Lukas Hedegaard Jensen}


%Konklusion MK
\subsection*{Mick Kirkegaard}


%Konklusion PO
\section{Poul Overgaard Pedersen}

Endnu et semesterprojekt er afsluttet og det faglige udbyttet har været rigtigt godt. ASE-modellen har igennem de 2 faser gjort det muligt at have fuld forståelse for det samlede projekt i fællesfasen og samtidigt dykke mere fagligt ned i den fagspecifikke fase. 

Dette års gruppe er smedet sammen af 2 grupper fra sidste års projekt. Det har bevirket at vi skulle lære hinanden at kende og finde sin plads i gruppen. Det gik heldigvis smertefrit! 

Jeg har personligt arbejdet tæt sammen med Mick i den fagspecifikke fase. Det har jeg været glad for og vi supplerer hinanden godt. 

Vi er som gruppe endt ud med et næsten ligeså funktionelt system som vi fra start ønskede, det er jeg stolt over, når man tænker over de problemstillinger vi har mødt undervejs. Semesterets kurser har bidraget til at projektet kunne realiseres med et tilfredsstillende resultat og et lærerigt udbytte. 
