% Udviklingsværktøjer
\chapter{Udviklingsværktøjer}
I det følgende afsnit beskrives de udviklingsværktøjer, der er benyttet i forbindelse med dette projekt. 

\section*{CrossCore Embedded Studio}
CrossCore Embedded Studio er benyttet til programmeringen af Blackfin kittet (Intelligent lydmonitor).

\section*{DIGILENT WaveForms}
DIGILENT WaweForms er software til brug af ANALOG DISCOVERY.

\section*{EAGLE}
EAGLEs PCB layout editor er benyttet til design af prints.

\section*{Eclipse}
Eclipse er benyttet i forbindelse med programmeringen af Raspberry Pien (Controller). Da det i udviklingsfasen var ønskeligt at kunne krydskompilere. 

\section*{Electronic Toolbox}
App af Marcus Roskosch til smartphones og tablets, til at hjælpe med komponent valg på integreret hardware m.m.

\section*{LaTeX og TexMaker}
Både denne rapport og tilhørende dokumentation er skrevet i sproget \LaTeX. \LaTeX har været benyttet før af et par i gruppen og de der ikke havde prøvet det før var villige til at prøve kræfter med det.

\LaTeX \ er et tekstbaseret kodesprog som gør brugeren fri af layout således at fokus kan rettes mod indholdet. Det kræver selvfølgelig at man sætter sig ind i sproget og overholder kodestandarden. 

TexMaker er benyttet som teksteditor til \LaTeX. Denne editor gør det muligt at navigere rundt imellem forskellige .tex-filer og bygge dokumenterne via TexMakers indbyggede pdf-viewer.

\section*{Mathworks MATLAB}
Mathworks MATLAB er benyttet i forbindelse med forundersøgelsen af babygråd. 

\section*{Microsoft Visio}
Microsoft Visio er benyttet til at udarbejde UML- og SysML-diagrammer. 

\section*{Multisim}
National Instruments Multisim er benyttet til kredsløbsdesign og simulering af disse. 

\section*{PSoC Creator}
PSoC Creator fra Cypress er benyttet i forbindelse med softwareprogrammeringen af PSoC4 til Vuggesystemet. 

\section*{Redmine}
Redmine er en fleksibel projektstyring webapplikation, som Ingeniørhøjskolen udbyder. Det er brugt ifm. SCRUM taskboardet og sprints.

\section*{\textit{Filhåndtering}}
Der er benyttet to cloudservices i forbindelse med filhåndteringen for dette projekt

\subsection*{GitHub}
GitHub er benyttet til versionsstyring af projektdokumenter dvs. Alle .tex-filer til dokumentation og rapport, softwarekildekode, UML- og SysML-diagrammer, Multisim-kredsløbsdiagrammer. GitHub sørger for at alle altid har nyeste version. 

\subsection*{Google Drev}
Google Drev er benyttet i flere forskellige forbindelser. Fælles dokumenter, mødereferater, logbog, tidsplan og fælles dokumenter i forbindelse med rette runder.