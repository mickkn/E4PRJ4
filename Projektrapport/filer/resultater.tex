% Resultater
\chapter{Resultater}

Udfra modultestene for de respektive dele af Baby Watch systemet\footnote{For Controller [\textit{Se projektdokumentationen pp.58-67}], for Intelligent lydmonitor [\textit{Se projektdokumentationen pp.121-139}], for Vuggesystemet [\textit{Se projektdokumentationen p.176-186}] og for Strømforsyningen [\textit{Se projektdokumentationen p.193-194}]}, integrationstesten\footnote{[\textit{Se projektdokumentationen p.196-201}]} og accepttesten\footnote{[\textit{Se projektdokumentationen p.203-206}]} er det lykkedes at udarbejde et system der ved at lytte til en babys lyde, kan styre vugningen af en barnevognskurv, opdatere en status-hjemmeside og sende advarselsemails. Lytte-funktionen er implementeret med realtids analyser af frekvensspektre på DSP-processoren Blackfin 533. En status sendes til en controller-enhed, implementeret på en Raspberry Pi, der håndterer afsendelse af emails samt opdateringen af en Flask-server baseret hjemmeside. Controller-enheden sender via I2C parametre til et vuggesystem. Vuggesystemets regulering er implementeret digitalt på en PSoC4. Ud fra de givede parametre sendt fra Controller, og vha. et kombineret gyroskop og accelerometer til vinkelmåling af barnevognskurven, styres af en DC-motor anvendt til at vugge barnevognskurven. Endvidere er der implementeret en strømforsyning, der forsyner resten af systemet.