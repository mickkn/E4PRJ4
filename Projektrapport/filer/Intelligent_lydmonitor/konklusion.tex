% IL sysark
\section{Konklusion}

\textbf{Resultat opsummering}\\
Ved at foretage 100 gennemløb af hver lydfil er der lavet histogrammer, og herudfra er Tabel \ref{IL:result} der viser hvor mange procent af hver BABYCON-satus de enkelte lydfiler producerer. 

For fuld test og beskrivelse henvises til \textit{6.4 Modultest} -> \textit{Modultest: Intelligent Lydmonitor}

\begin{table}[!htbp] \centering
	\caption{Resultattabel}
	\label{IL:result}
	\begin{tabular}{|l|l|l|l|}
	\hline
		\textbf{Kategori} & \textbf{BABYCON1} & \textbf{BABYCON2} & \textbf{BABYCON3}\\\hline
		Højlydt babygråd & 49,50\% & 34,50\% & 16,00\% \\\hline
		Moderat babygråd & 0,00\% & 59,50\% & 40,50\% \\\hline
		Omgivelsesstøj & 0,25\% & 11,00\% & 91,25\% \\\hline
	\end{tabular}
\end{table}

Ud fra histogrammerne i dokumentationen er der beregnet en hit ratio. 
\textbf{Hit ratio}\\
\begin{center}
$HitRatio=\frac { hits }{ total\quad tries } =\frac { 99+119+365 }{ 200+200+400 } =72.9$\%
\end{center}

\textbf{Konklusion}\\
Antages en løs tankegang hvor brugssituationen tages i betragtning, er det langt hen ad
vejen lykkedes at implementere den Intelligente Lydmonitor tilfredsstillende. BABYCON1
detekteres næsten udelukkende ved optagelser af højlydt babygråd med en detektion ved
1\% af Latter tilfældene. BABYCON2 detekteres i højest grad ved moderat babygråd, med
en del detektioner ved højlydt babygråd samt dyrelyde som katte-mjauen en fuglefløjt.
BABYCON3 detekteres mellem 70 og 100\% af tiden ved andre optagelser end babygråd,
hvor fuglefløjt har laveste og trafikstøj højeste procentsats. BABYCON3 detekteres
indimellem ved både højlydt og moderat babygråd, men eftersom denne status er fredelig,
antages denne fejltagelse for mindre væsentlig.

For et mere rigidt udgangspunkt, hvor det antages at de givne optagelser af eksempelvis
”Højlydt babygråd” hver gang bør resultere i BABYCON1, er det lykkedes at implementere
en Intelligent Lydmonitor, der for de angivne lydfiler har en HitRatio = 72.9\%.


 
