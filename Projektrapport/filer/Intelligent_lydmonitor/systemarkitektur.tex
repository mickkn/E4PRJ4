% IL sysark
\section{Systemarkitektur}

Overordnet fungerer den Intelligent Lydmonitor som følger:
\begin{itemize}
	\item Babyens gråd detekteres med en mikrofon og forstærkes med en mikrofonforstærker inden det analoge signal konverteres til diskrete samples med codec AD1836 på Blackfin udviklingsboardet
	\item Den diskrete sample-sekvens forfiltreres. Sample-sekvensen lavpasfiltreres og nedsamples. Det forfiltrerede sample gemmes i bufferen.
	\item Den filtrerede sample-sekvens i bufferen analyseres først for Sound Pressure Level (SPL) for at bedømme hvorvidt der er lyd nok til at den efterfølgende FFT og udglatning kan betale sig. 
	\item Et evt. resultat af FFT'en kategoriseres som én af tre BABYCON-states, som beskrevet i kravspecifikationen 
	\item Den fundne BABYCON state holdes op mod en statistik over tidligere states og den signifikante status findes
	\item Det endelige kategoriseringsresultat sendes til Controller via TwoWireCom
\end{itemize}
En illustration af dette ses i figuren \ref{il_sys_overordnet} herunder. \footnote{Højpasfilteret er en logisk funktionalitet for systemet, men er i realiteten implementeret som side-effekt af den smoothing, der benyttes på det analyserede fft-spektrum. Se \textit{Intelligent lydmonitor -> Softwaredesign- og implementering -> Smoothing af frekvensspektrum -> Anvendelse af transient respons som højpasfiltrering pp. 121-123} i projektdokumentation for beskrivelse.}
\figur{1}{intlydmonitor/Intelligent_Lydmonitor_Systemtegning.pdf}{Overordnet virkemåde for Intelligent Lydmonitor }{il_sys_overordnet}


Der er således tale om en klassisk pipeline arkitektur, hvor resultatet af et givent led videreprocesseres af det næste i rækken.
Systemet opdeles således i følgende række af led:

\textbf{HW:} 
\begin{itemize}
\item Mikrofon
\item Mic Preamp
\item ADC \footnote{Som ADC-led benyttes Blackfin udviklingsboardets codec AD1836. SW for erhvervelse af samples tager udgangspunkt i CrossCore Embedded Studio projektet ''AudioNotchFilter'' benyttet ved undervisning i faget E4DSA forår 2015}
\end{itemize}

\textbf{SW-klasser:} 
\begin{itemize}
\item Prefilter
\item RecBuf
\item Analyzer
\item Statistician
\item Categorizer
\item TwoWireCom
\end{itemize}
Se \textit{Systemarkitektur} afsnittet i projektdokumentation for uddybende HW- og SW-arkitektur.
