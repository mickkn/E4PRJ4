% IL analyzer
\subsection{Analyzer}
Analyzer har til opgave at foretage følgende analyser for at klargøre et frekvensspektrum til analyse ved Categorizer:
\begin{itemize}
	\item Sound Pressure Level
	\item Fast Fourier Transformation
	\item Smoothing af frekvensspektrum
\end{itemize}

Effekten af disse illustreres med en kort beskrivelse af den trinvise klargøring af frekvensspektret. Der benyttes en samplesekvens fra en optagelse af højlydt babygråd.

\figur{1}{intlydmonitor/analyzer_gennemgang}{Analyzers trinvise klargøring af frekvensspektrum. Venstre, originalt FFT frekvensspektrum; Midt, frekvensspektrum i dB; Højre, udglattet frekvensspektrum}{il_an}
Bemærk at y-aksens fft-analyse bins på ovenstående figur er skaleret til frekvenser i Hz.

\subsubsection*{Sound Pressure Level}
For at ikke at benytte unødig processorkraft på beregne FFT af et støj-signal, beregnes først et udtryk for hvor kraftigt signal, der modtages. Er SPL-analysens resultat under en grænseværdi, udføres FFT og udglatning ikke.\footnote{[\textit{Se projektdokumentationen pp. 112-113}] for beskrivelse af implementering}

\subsubsection*{Fast Fourier Transformation}
Ved tungere behandling af data benyttes DSP biblioteksfunktioner, der anvender fixed-point typen \verb+fract32+ for på optimalt vis at multiplicere og akkumulere i filter- og fft-funktioner. Ved generering af frekvensspektrum benyttes funktionerne \verb+rfft_fr32()+ og \verb+fft_magnitude_fr32()+.\footnote{[\textit{Se projektdokumentationen p. 113}] for beskrivelse af implementering} Disse genererer venstre spektrum i figur \ref{il_an}.

\subsubsection*{Smoothing af frekvensspektrum}
For på korrekt vis at kunne bestemme dominante frekvensområder, konverteres frekvensspektrets magnitude til en dB-skala. Produktet af dette ses midterst i \ref{il_an}.
Til midling af spektret, benyttes et rekursivt filter med index 35. Resultatet af denne filtrering ses til højre i \ref{il_an}. Bemærk at indsvingningen af filtret resulterer i en højpas-lignende effekt før frekvensbin 35, svarende til 818Hz. \footnote{[\textit{Se projektdokumentationen pp. 115-116}]}. 

Der genereres således et udglattet frekvensspektrum, der er klar til kategorisering ved næste led i rækken, Categorizer.

For en fuld gennemgang af Analyzer, [\textit{Se projektdokumentationen pp. 111-117}].