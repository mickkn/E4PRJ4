%Konklusion FB
\subsection*{Felix Blix Everberg}
Ved starten af dette semester projekt havde jeg som mål at lave et projekt som nåede at blive færdiggjort til et fungerende niveau. Dette er for første gang i mit studie lykkedes, hvilket har været en stor tilfredsstillelse for mig. Både på dette og tidligere semestre har det været et mål for mig og min gruppe at vi ville implemenetere løbende frem for til sidst. Dette har fungeret bedre på dette semester, og har tror jeg været medvirkende til at mit første mål også er lykkedes. Derudover har det som håbet også resulteret i en mindre dokumentationsbyrde i projektets sidste fase, idet flere elementer har kunnet færdigdokumenteres tidligere som resultat af den løbende implementering. \\
Vores projektstyrings model har vi i dette projekt skiftet undervejs. Dette kom som konsekvens af at scrum var begyndt at være meget mere tungt end hvad der kunne retfærdiggøres ud fra fordelene. Som udgangspunkt tror jeg at dette skyldes at selv en letvægts projektstyringsmodel som Scrum er lidt for meget til et projekt som skal køre sideløbende med andre fag, og hvor der ikke er mere end seks mennesker indvoldveret. \\
Slutteligt er det værd at nævne at det har været lærerigt for mig personligt at projektet på netop 4 semester, hvor man har EMC, indeholdt så markante udfordringer omkring kobling som vi oplevede i forbindelse med motorkredsen og vinkelsensoren. \\
Alt i alt har det været et meget tilfredsstillende projekt, med en række gode udfordringer både fagligt og ift. projektstyring og process.\\