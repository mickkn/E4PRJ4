%Konklusion LH
\subsection*{Lukas Hedegaard Jensen}

I udviklingsprocessen har jeg fagligt beskæftiget mig fortrinsvis med digital signalbehandling, og har ved implementeringen af den Intelligente Lydmonitor benyttet mig af stort set hele pensum for kurset E4DSA, være dette sample-konvertering, filter-design, frekvensdomæne-analyser eller tal-repræsentationer, samt redskaber uden for kursets pensum. Særligt har jeg opnået erfaring med DSP-processing på en embedded platform som ADSP-BF533, som jeg ikke opnåede ved semesterfagene. Derudover har jeg draget erfaringer med design af- og printudlæg til en mikrofon-forstærker, og dermed draget nytte af undervisningen i Analog Signal Design [E4ASD] og E4EMC. Hvad angår projektarbejdet i en seksmandsgruppe, er det ved dette semesterprojekt i højere grad end tidligere lykkedes effektivt at strukturere arbejdet. Dette tillægger jeg stadig bedre udnyttelse af revisionsstyring, lettere opdeling af tekstafsnit via Latex, samt en grundlæggende skarp separation af systemets moduler med lav kobling og høj intern samhørighed.