%Konklusion MK
\subsection*{Mick Kirkegaard}

I dette semesterprojekt har jeg næsten udelukkende arbejdet med at kode vores Linux program til Raspberry Pien. Da Raspberry Pien var et nyt område for mig, har det været utroligt spændene at sætte sig ind i. Der har dog været nogle hurdles under vejen, da vi i starten gerne ville krydskompilere fra Windows. Og jo længere vi kom frem i processen, blev vi nødsaget til at skifte over til Linux.

Jeg har igennem hele forløbet arbejdet tæt sammen med min faste samarbejdspartner Poul, som jeg har sparet meget med ift. hvordan programmet skulle opbygges og fungere. Vi har også sammen udviklet strømforsyningen til Baby Watch, hvor Poul har haft ansvaret og jeg har hjulpet til med printudlæg og fejlfinding.

Jeg har savnet lidt, at have kunnet inkludere flere 4. semester i arbejdet. Og det er en skam at ISU ikke ligger tidligere i forløbet, så man kunne gøre mere brug af faget i projektet. 