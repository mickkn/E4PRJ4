%Konklusion JH
\subsection*{Jeppe Hofni}
For mig, har dette semesterprojekt været det mest omfavnende projekt i forhold til at bruge den viden jeg har tilegnet mig på ingeniørhøjskolen. Jeg har i høj grad inddraget min viden fra fagene I4IKN, E4DSA, E4ISU og E4IRT. E4IKN har i særdeleshed været brugbart i forhold til at udvikle http-webserven til Baby Watch hjemmesiden, her har jeg måtte sætte mig ind i flere programmeringssprog for at kunne opsætte en funktionel webserver.\\
I udviklingen af vuggesystemet har jeg primært arbejdet med softwareopsætningen af systemet. Dette har været en iterativ proces der har resulteret i mange design og implementerings ændringer der hovedsageligt har været baseret på at få optimeret mange funktioner og derved mange beregninger ned på en microcontroller platform med begrænset regnekraft og hukommelse. Valget af platform har således været til diskussion et par gange, da en opgradering fra PSoC4 til PSoC5 har været mulig, men i sidste ende ikke nødvendig i denne version af projektet.\\
Udviklingsprocessen med at bruge SCRUM i starten og senere da det udefrakommende pres\footnote{Ikke fra det alm. skolearbejde, men fra praktiksøgningen til næste semester} blev for stort at skifte over til en mere personlig udviklingsproces har været meget givende for arbejdsfokusset. Det skal dog nævnes at fordybelsen i en udviklingsproces der hele tiden bliver afbrudt af en skoledag ikke i på nogen måde kan sammenlignes med at udvikle fuldtid, dette blev afprøvet i påskeferien med udviklingen af webserveren. Alt i alt har 4. semesterprojekt været det bedste projekt lærings- og resultatmæssigt.

%Kommentar fra Felix:
% Jeg tænker snilt at linje 4 kunne koges noget ned ved at ændre formuleringen, og så 
% måske overveje om det er relevant for den personlige konklusion at valget stod mellem 
% psoc 4 og 5 frem for bare at holde sig til hvad du allerede siger, nemlig at en af 
% hovedudfordringerne var at implementere til en begrænset platform. Resten ser super 
% ud. Har rettet et par stavefejl og indsat nogle linjeskift for læselighedens skyld.