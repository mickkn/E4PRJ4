% Opgaveformulering
\chapter{Systembeskrivelse}
Som beskrevet i rapportens resume afsnit \vref{resume} består Baby Watch systemet af tre dele monteret på en almindelig barnevogn. Disse dele er en Controller med et betjeningspanel, et babymonitorerings system kaldet Intelligent Lydmonitor og et mekanisk system der kan vugge barnevognskurven på den alm. barnevogn kaldet Vuggesystem.

Controlleren er den centrale styreenhed i Baby Watch systemet. Via dens betjeningspanel kan en bruger benytte systemet til at monitorer og vugge en baby i søvn. Controlleren sørger for alt kommunikation mellem systemets andre dele. Controlleren hoster ligeledes en hjemmeside med babyens tilstand og ved alarmerende baby tilstand, beskrevet længere nede, sendes e-mails til brugeren. Controlleren har derfor internet forbindelse.

Den Intelligente Lydmonitor lytter og analysere lydene i og omkring barnevognen. På baggrund af analysen sendes en baby-tilstand til controlleren. Controlleren sender herefter parametre til Vuggesystemet som angiver en vuggefrekvens og vuggevinkel\footnote{Vuggefrekvensen angiver frekvens som barnevognskurven bliver vugget med. Vuggevinklen er en vinkel-amplitude som barnevognskurven vugges med, vinklen her er relativ til tyngdefeltet}. Ud fra disse parametre og en vinkelmåler regulerer Vuggesystemet en motor, der trækker barnevognskurven i den givne vuggefrekvens og -vinkel.  

Der skelnes imellem 3 niveauer, kaldet BABYCON\footnote{Baby Condition - babyens humørtilstand kategoriseret på baggrund af lydoptagelser}. BABYCON1 er en alarm-tilstand hvor babyen behøver tilsyn. BABYCON2 er en tilstand hvor babyen vugges, da babyen ud fra lydanalysen menes at være urolig. BABYCON3 er en tilstand hvor babyen overvåges uden vugning. Systemet har også en ''Manuel start'' tilstand til at vugge babyen i søvn, monitorering er i denne tilstand sat på stand by i et kort tidsrum.

\figur{0.55}{Systemtegning_edit}{Systemtegning der illustrer ovenstående beskrivelse}{op:systegning} 
