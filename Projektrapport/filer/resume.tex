% Resume
\chapter{Resumé}

Denne rapport og tilhørende projektdokumentation omhandler arbejdet i forbindelse med 4. semesterprojektet på elektronikingeniøruddannelsen på Aarhus School of Engineering. Læringsmålene for projektet er bl.a. at benyttet tilegnet viden om emner og metoder fra undervisningen på dette semester. Rapporten indeholder de benyttede arbejdsmetoder der er brugt fra idé til produkt, det udarbejdet produkt samt udviklingen af dette. En mere teknisk og udtømmende udgave forefindes i den tilhørende dokumentation. 

Det udviklede system tager udgangspunkt i babyers middagslur, som ofte foregår i en barnevogn. Systemet skal ud fra lydoptagelse detektere hvorvidt en baby græder og starte vuggemekanismen. Babypasseren opdateres via hjemmeside og e-mail ved alarm eller fejl.

ASE-modellen er benyttet til at styre udviklingsforløbet. Produktet består af 3 hoveddele samt en strømforsyning. De 3 hoveddele er en Controller, som står for brugerinteraktionen. En Intelligent lydmonitor som står for lydanalysen. Et Vuggesystem som står for at styre og regulere vugningen af barnevognen. 

Det lykkedes i sidste ende at få systemet op at køre. Dog har der været nogle tidsmæssige begrænsninger som kan læses i kapitlet Projektafgrænsninger på side \pageref{projektafgraensninger}.