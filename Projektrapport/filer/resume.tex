% Resume
\chapter{Resumé} \label{resume}

Nuværende babyalarmsystemer fungerer ved at lade forældre høre deres børn uden at være tilstede, hvilket giver dem større mulighed for at udnytte den tid hvor der soves til middag.
Denne rapport og tilhørende projektdokumentation beskriver et 4. semesterprojekt på elektronikingeniøruddannelsen ved Ingeniørhøjskolen Århus, hvori det forsøges at udvikle et udvidet babymonitoreringssytem.\\
Det udviklede system monitorerer babyens lyde og analyserer disse for at vurdere barnets tilstand indenfor en række kategorier. Denne information gøres tilgængelig for forældrene gennem en hjemmeside, samtidig med at det forsøges at berolige barnet gennem automatiseret vugning.
Produktet består af tre hoveddele samt en strømforsyning: En Controller, som står for brugerinteraktionen, en Intelligent lydmonitor som står for lydoptagelse og -analyse samt et Vuggesystem som står for at regulere vugningen af barnevognen. \\
Systemet var ved projektets afslutning i stand til at opfylde kravene til dets grundlæggende funktionalitet, med nogle mindre begrænsninger. Overvejelser omkring udbedringer af systemets mangler samt oplagte systemudvidelser forefindes som del af rapporten.\\