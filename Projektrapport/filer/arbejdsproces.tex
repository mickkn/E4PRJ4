% Arbejdsproces

\chapter{Arbejdsproces}

\section{Udviklingsmodeller}

Projektarbejdet for dette semesterprojekt har udgangspunkt i ASE-modellen, se figur \ref{ASE_modellen}. ASE-modellen er benyttet fordi den har været gavnlig på tidligere semestre samt at modellen danner grundlag for flere læringsmål for dette semesterprojekt.

ASE-modellen er hovedsagligt opbygget af 2 faser. En fællesfase hvor gruppen samlet udarbejder projektformulering, kravspecifikation og et fælles grundlag for systemarkitekturen. Herefter skiftes til den fagspecifikke fase, hvor gruppen har arbejdet i tre mindre grupper bestående af to personer pr. gruppe. I denne fase arbejdes der med design og implementering inden gruppen skifter tilbage i en fællesfase hvor accepttest og færdiggørelse af dokumentation og rapport udarbejdes. 

\figur{1}{ASE_modellen.pdf}{ASE modellen}{ASE_modellen}

I starten af projektet, blev det besluttet at bruge SCRUM. Der blev oprettet et taskboard på redmine til håndteringer af de forskellige tasks og sprints. SCRUM er et agilt værktøj til projektstyring. 

4. semester har været et intenst semester og bl.a. derfor har vi i projektets sidste 1,5 måned lagt SCRUM på hylden, for at bruge tiden anderledes og arbejde mere fokuseret på at nå målet, fremfor at bruge tid på SCRUM og planlægningen i forbindelse dertil. 

Vi har som gruppe haft god erfaring med at holde ca. 3 ugentligt "stå-op-møder" hvor gruppen mødes hvor status, problemer og eventuelt vendes i gruppen. 

Vi har haft møde med vores vejleder cirka hver 14. dag, som har givet en god hjælp. Især da vi kom til EMC-problematikkerne i systemet, har vi haft god sparring med vores vejleder. Udover det har vi også brugt andre undervisere fra diverse fag på 3. og 4. semester, når andre problemer opstod.

Gruppen har desuden frivilligt indvilget i at gennemgå en anden gruppes kravspecifikation og systemarkitektur, mod at den anden gruppe så ville gennemgå vores tilgengæld. Det gav en række problemstillinger og spørgsmål begge veje, hvilket må konkluderes at være gavnligt for begge parter.

Gruppen har desuden afholdt gruppemøde, mindst hver anden uge. Her er der afklaret en lang række problemstillinger og gruppens medlemmer har haft et bedre grundlag for at danne sig et samlet overblik over projektet. 

\section{Arbejdsfordeling}

Gruppen har som beskrevet arbejdet i tre mindre teams i den fag-specifikke fase. 

\subsection*{Felix og Jeppe}
Felix(FB) og Jeppe(JH) har haft ansvaret for udviklingen af Vuggesystemet jf. afsnit \vref{vuggesys}. De primære opgaver for Felix har været motorstyringskredsen og vinkelsensor software mens Jeppe's hovedopgaver var den overordnede programstruktur i vuggesystemets MCU og I2C interfacet til Controller-enheden.\\ 
Hardwareopgaverne har bestået af udvikling af et mekanisk vuggesystem med tilhørende motor(FB+JH), en motorstyrings-kreds(FB) og en opsætningen af to endstop-sensorer(JH).\\
Softwareopgaverne har bestået af udviklingen af et reguleringsmodul på en microcontroller med tilhørende interface til en vinkelsensor(FB), kommunikationsinterface til Baby Watch Controller(JH), beregninger af reguleringen(FB+JH), styringen af et PWM-signal til hardware motorkredsen(FB+JH) og et mainprogram til at styre alle disse funktionaliteter(JH). \\

\subsection*{Kristian og Lukas}
Kristian og Lukas har haft ansvaret for udviklingen af den Intelligente Lydmonitor. Herunder er udvikling af en mikrofon-kreds til måling  og en software implementering af optagelse, analyse og kategorisering af baby-lyd på et Blackfin 533 (BF533) udviklings-board. Design af et 6. ordens analogt Butterworth lavpasfilter til sikring af høj signal-to-noise-ratio på indgangen af udviklingsboardets AD1836 codec. Som forundersøgelse for software-implementeringen er foretaget analyse af diverse almindelige lyde fra et babyvugnings-scenarie, hvor DSP-redskaberne Windowing, Fast Fourier Transform, Tonal Power Ratio og spektrogram (Short-time DFT) benyttes. Disse redskaber indgår ligeledes ved implementering på BF533-boardet, hvor decimering, analyse af Sound Pressure Level samt flere buffering metoder også tages i brug. For at realisere projektet er der fundet en måde at tilgå udviklingsboardets AD1836 codec samt BF533 GPIOer til kommunikation med Controller.  \\

\subsection*{Mick og Poul}
Mick(MK) og Poul(PO) har haft ansvaret for udviklingen af Controlleren, som i hovedtræk er styreenheden i systemet, der skal kommunikere imellem alle delene i Baby Watch. Derudover har Mick(MK) og Poul(PO) haft ansvaret for strømforsyningen, til forsyningen af Vuggecontrolleren og den Intelligente lydmonitor. Så der har både været software og hardware opgaver at tage vare på. Softwaredelen er baseret på en Raspberry Pi, hvor linux har været i højsædet, der har været en del forundersøgelse i hvordan der kunne kommunikeres ud til hardwaren fra computeren samt hvordan det kunne gøres nemmere at compile og debugge, ydermere har \iic forbindelsen budt på lidt udfordringer. På 4. semester har vi stiftet bekendtskab med trådet programmering, og dette er der blevet draget nytte af i Controller-programmet også. Der er lavet minimalt hardware til Controlleren, men strømforsyningen har derimod budt på udfordringer i printudlæg, som har været nyt for begge.

\subsubsection*{Jeppe}
Jeppe har også haft til opgave i at udvikle en webserver til Raspberry Pi'en med tilhørende klasse "Website" der kan styre denne webserver via. main programmet på Controlleren.