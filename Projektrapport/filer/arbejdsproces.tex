% Arbejdsproces

\chapter{Arbejdsproces}

\section{Udviklingsmodeller}

Projektarbejdet for dette semesterprojekt har udgangspunkt i ASE-modellen, se figur \ref{ASE_modellen}. ASE-modellen er benyttet fordi den har været gavnlig på tidligere semestre samt at modellen danner grundlag for flere læringsmål for dette semesterprojekt.

ASE-modellen er hovedsagligt opbygget af 2 faser. En fællesfase hvor gruppen samlet udarbejder projektformulering, kravspecifikation og et fælles grundlag for systemarkitekturen. Herefter skiftes til den fagspecifikke fase, hvor gruppen har arbejdet i 3 mindre grupper bestående af 2 personer pr. gruppe. I denne fase arbejdes der med design og implementering inden gruppen skifter tilbage i fællesfase hvor accepttest og færdiggørelse af dokumentation og rapport udarbejdes. 

\figur{1}{ASE_modellen.pdf}{ASE modellen}{ASE_modellen}

I starten af projektet, blev det besluttet at bruge SCRUM. Der blev oprettet et taskboard på redmine til håndteringer af de forskellige tasks og sprint. TILFØJ NOGET MERE

4. semester har været et intenst semester og bl.a. derfor har vi i projektets sidste 1,5 måned lagt SCRUM på hylden, for at bruge tiden anderledes og arbejde mere fokuseret på at nå målet, fremfor at bruge tid på SCRUM og planlægningen i forbindelse dertil. 

Vi har som gruppe haft god erfaring med at holde ca. 3 ugentligt "stå-op-møder" hvor gruppen lige mødes og status, problemer og evt. vendes i gruppen. 

Vejledermøder - andre vejledere til specifikke problemer ?

Gruppen har desuden frivilligt indvilget i at reviewe en anden gruppes kravspecifikation og dele af systemarkitekturen mod at den anden gruppe reviewede os tilgengæld. Det gav en række problemstillinger og spørgsmål begge veje, hvilket må konkluderes at være gavnligt for begge parter.

Gruppen har desuden afhold møde mindst hver anden uge. Her er der afklaret en lang række problemstillinger og gruppens medlemmer har haft et bedre grundlag for at danne sig et samlet overblik over projektet. 


 \section{Arbejdsfordeling}

Gruppen har som beskrevet arbejdet i 3 mindre teams i den fagspecifikke fase. 

\subsection*{Felix og Jeppe}


\subsection*{Kristian og Lukas}


\subsection*{Mick og Poul}
Mick og Poul har haft ansvaret for Controlleren. Controlleren består primært af software afviklet på en Raspberri Pi. Opgaverne har bl.a. indbefattet kommunikation imellem både Intelligent lydmonitor og Vuggesystemet, netværksforbindelse og e-mail afsendelse. 
Hardware mæssigt består Controlleren af et betjeningspanel med 3 LEDs, én ON/OFF kontakt samt en ''Manuel start''-knap. Betjeningspanelet er Babypasserens mulighed for at interagere med systemet
Mick og Poul har desuden stået for PSU'en til det samlede Baby Watch system


