% Opgaveformulering
\section{Systembeskrivelse}

Systemet indholder indeholde en Intelligent lydmonitor som står for at optage og analyse lyd fra barnevogns kassen. Ud fra denne analyse sendes signal til en Controller som som står for at handle på det modtaget signal og herpå kommunikere via \iic til Vuggesystemet. Controlleren står ydermere for brugerinteraktionen via et betjeningspanel samt opdatering af hjemmeside og e-mail afsendelse ved alarm eller fejl. Vuggesystemet skal på baggrund af modtaget data fra Controlleren starte vuggemekanismen og regulere således barnevogns kassen vugger med ønsket frekvens og vinkeludsving. Der skelnes mellem 3 niveauer, kaldet BABYCON. BABYCON1 er alarm tilstanden og babyen behøver tilsyn. BABYCON2 er en tilstand hvor babyen vugges, da babyen ud fra lydoptagelser viser sig at være urolig. BABYCON3 er en tilstand hvor babyen blot overvåges, og vuggen vugger derfor ikke. 

''Manuel start'' kan aktiveres når babyen ligges i barnevognen og herved starter vuggen, som skal sikre at babyen falder i søvn uden der står en og aktivt vugger barnevognen. 

Figur \ref{op:systegning} illustrer ovenstående beskrivelse. 

\figur{0.8}{Systemtegning_edit}{Systemtegning}{op:systegning} 
