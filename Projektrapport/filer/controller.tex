% Controller
\chapter{Controller}

Controllerens funktion er at modtage et BABYCON-niveau fra den Intelligente lydmonitor og handle på det. Controlleren består af et betjeningspanel og en Raspberry Pi. Raspberry Pien fungerer som hjernen i Controlleren og styrer betjeningspanelets dioder og handler på panelets knapper. Raspberry Pien handler på det modtagne BABYCON-niveau ved at sende data over \iic til Vuggesystemet. Raspberry Pien indeholder desuden en webserver samt en mail klient som hhv. opdaterer en hjemmeside med BABYCON-niveau og meddeler Babypasseren via e-mail ved alarm eller fejl.

De følgende afsnit \ref{ctrl_sysark}, \ref{ctrl_design} og \ref{ctrl_implementering} deler Controlleren op i Systemarkitektur, Design og Implementering.

\section{Systemarkitektur}
\label{ctrl_sysark}

Softwarearkitekturen er designet ud fra de udarbejdede use cases. Der er udarbejdet tre sekvensdiagrammer, som er beskrevet herunder. [\textit{Se projektdokumentationen p. 30}]

\begin{itemize}
\item \textbf{BABYCON-niveauer:} Beskriver Controllerens opførsel ved de forskellige BABYCON-niveauer.
\item \textbf{''Manuel start'':} Beskriver forløbet når Babypasseren aktiverer ''Manuel start''
\item \textbf{Power Off:} Beskriver forløbet når Controllerens ON/OFF-kontakt afbrydes og Baby Watch systemet slukkes 
\end{itemize}

Hardwarearkitekturen er ligeledes udarbejdet ud fra use cases. Der er udarbejdet et internt blokdiagram, se figur \ref{controller:ibd}. Til det interne blokdiagram er der udarbejdet en forklarende signaltabel [\textit{Se projektdokumentationen p. 34}].

\figur{1}{controller/Controller_IBD}{IBD diagram for Controller}{controller:ibd}


\section{Design}
\label{ctrl_design}

Softwarefunktionaliteten for Controlleren deles op i klasser og ud fra denne opdeling er nedenstående klassediagram udarbejdet, se figur \ref{controller:klassediagram}. Klassebeskrivelserne er beskrevet i projektdokumentationen [\textit{Se projektdokumentationen p. 37}]

\figur{0.9}{controller/Klassediagram.pdf}{Klassediagram for Controller}{controller:klassediagram}

Hardwaredesignet for Controlleren består af betjeningspanelet samt stik til forbindelser til/fra Controlleren. Det samlede kredsløbsdiagram med GPIO-forbindelser til Raspberry Pi'en kan ses på figur \ref{controller:hwdesign}.

\figur{1}{controller/controller_samlet.pdf}{Samlet kredsløbsdiagram for Controller}{controller:hwdesign}

\section{Implementering}
\label{ctrl_implementering}

\subsection*{Software}

Hovedprogrammet er udviklet i Eclipse til Linux \citep{website:eclipsekepler}, hvor der krydskompileres direkte til Raspberry Pi via ssh\footnote{Secure Shell - krypteret fjernadgang}.

Der er lavet et \verb+GPIOsocket+ funktionsbibliotek til at oprette filbiblioteker til de enkelte GPIO-porte. I biblioteket er der 3 funktioner der varetager oprettelse og nedlæggelse af et GPIO-ben, samt opsætning af retning (input/output). Funktionerne bliver brugt af de klasser der bruger GPIO-benene. \verb+GPIOsocket+ biblioteket gør brug af wiringPi biblioteket \citep{website:wiringpi}. Og det er dens funktionalitet der gøres brug af når der oprettes og nedlægges gpio-ben.

HTTP-webserveren er implementeret vha. af en Flask-server \citep{website:flask}.

En detaljeret gennemgang af implementeringen kan findes i Controllerens implementeringsafsnit i projektdokumentationen [\textit{Se projektdokumentationen p. 47}].

\subsection*{Hardware}

Hardwaren består som sagt af en Raspberry Pi Model B, derudover er der udarbejdet et print til lysdioderne hvor de beregnede for-modstande er monteret. Alt dette er bygget i en kasse lavet i materialet Forex\footnote{PVC plastic plade i 5 mm tykkelse}. På kassen er monteret 2 kontakter til hhv. ''Manuel start'' og tænd/sluk.

\section{Konklusion}

Vi er kommet godt i mål med hovedenheden Controller. Vi har måtte tage brug af fag vi har haft på tidligere semestre, men har dog også gjort brug af ISU\footnote{I4ISU - Inlejrede Systemudvikling v/ Søren Hansen} til den trådede del af systemet. Vi blev dog opmærksomme på, sent i forløbet at ISU kunne have været en meget større del af vores software udvikling af Controlleren. Hvor systemet kunne være blevet lavet som et event-drevet system. Men grundet manglende tid blev dette ikke realiseret.