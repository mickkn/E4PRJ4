% Abstract
\chapter{Abstract}

Current baby-alarm-systems work by letting the parents hear their children without having to be present, thus allowing them to better utilize the time when their children are napping.
This report and the associated project documentation describes the fourth semester-project in the electronic engineering programme at Aarhus School of Engineering, wherein it is sought to develop an expanded baby-monitoring-system.\\
The developed system monitors the sounds of the baby and analyses them in order to evaluate the state of the baby within a set of predefined categories. This information is made available to the parents through a website, while also being used to calm the baby through automatized rocking of the baby-carriage.\\
The product consists of three main parts, along with a power supply: A controller responsible for the user interface, an intelligent sound-monitor responsible for sound-recording and -analysis, and a rocking-system controlling the rocking of the baby-carriage.\\
By the end of the project, the system was capable of meeting the requirements to the fundamental system functionality, with some minor constraints. Considerations about how to mend the shortcomings of the system, as well as meaningful system expansions are also to be found in the report.\\

%Nuværende babyalarmsystemer fungerer ved at lade forældre høre deres børn uden at være tilstede, hvilket giver dem større mulighed for at udnytte den tid hvor der soves til middag.
%Denne rapport og tilhørende projektdokumentation beskriver et 4. semesterprojekt på elektronikingeniøruddannelsen ved Ingeniørhøjskolen Århus, hvori det forsøges at udvikle et udvidet babymonitoreringssytem.\\
%Det udviklede system monitorerer babyens lyde og analyserer disse for at vurdere barnets tilstand indenfor en række kategorier. Denne information gøres tilgængelig for forældrene gennem en hjemmeside, samtidig med at det forsøges at berolige barnet gennem automatiseret vugning.
%Produktet består af tre hoveddele samt en strømforsyning: En Controller, som står for brugerinteraktionen, en Intelligent lydmonitor som står for lydoptagelse og -analyse samt et Vuggesystem som står for at regulere vugningen af barnevognen. \\
%Systemet var ved projektets afslutning i stand til at opfylde kravene til dets grundlæggende funktionalitet, med nogle mindre begrænsninger. Overvejelser om udbedringer af systemets mangler, samt oplagte systemudvidelser er ligeledes at finde i rapporten.\\