%controller

\subsection{Systemarkitektur for Controller}




\figur{1}{sysark/Controller_IBD}{IBD diagram for Controller}{IBD:controller}

Signal tabel ?

\begin{center}
\label{table:Signaltabel}
\begin{longtable}{|p{2,5cm}|p{1,8cm}|p{2,6cm}|p{2,8cm}|p{3cm}|}
\hline
\textbf{Signal-navn}	&\textbf{Type} 		&\textbf{Port 1} 	&\textbf{Port 2} 			&\textbf{Kommentar} \\ \hline
\endfirsthead
\multicolumn{5}{l}{...fortsat fra forrige side} \\ \hline 
\textbf{Signal-navn}	&\textbf{Type} 		&\textbf{Port 1} 	&\textbf{Port 2} 			&\textbf{Kommentar} \\ \hline
\endhead

%% Signal-navn
%% &Type
%% &Område
%% &Port1
%% &Port2
%% &Kommentar
%% \\\hline

manStartOn
&touch
&Betjeningspanel \newline (manStart\_in)
&Babypasser \newline (touch)
&Babypasser der laver et tryk på knappen
\\\hline

babyWatchOn
&touch
&Betjeningspanel \newline (onOff\_in)
&Babypasser \newline (touch)
&Babypasser der laver et tryk på knappen
\\\hline

power
&5 VDC
&Rasberry Pi 1 B \newline(micro\_USB) \newline
	Betjeningspanel \newline (pow\_in)
&Controller \newline(powC\_in)
&5 V forsyning til controller
\\\hline

powerOn			
&TTL 0-3,3 VDC	
&Rasberry Pi 1B \newline (GPIO23)				
&Controller \newline (powOn\_out) 			
&Benyttes til at tænde og slukket for strømforsyningen   				\\\hline

I2C			
&Seriel		
&Rasberry Pi 1 B \newline (I2CRPI) \newline	
	GPIO2: SDA \newline	
	GPIO3: SCL \newline	
&Controller \newline (I2CController) 			
&Seriel kommunikation
\\\hline

USB		
&Seriel				
&Rasberry Pi 1 B \newline (USB\_in) 			
&Wi-Fi dongle \newline (USB\_out) \newline	
&Seriel kommunikation for Wi-Fi forbindelse
\\\hline

ledControl
&TTL 0-3,3 VDC		
&RaspberryPi \newline (GPIO14) \newline
 RaspberryPi \newline (GPIO15) \newline
 RaspberryPi \newline (GPIO18)
&Betjeningspanel \newline (onOffLed\_in) \newline
 Betjeningspanel \newline (manLed\_in) \newline
 Betjeningspanel \newline (wifiLed\_in)
&Logisk signal til at styre LED på Betjenningspanel
\\\hline

butPress
&TTL 0-3,3 VDC		
&RaspberryPi \newline (GPIO17) \newline
 RaspberryPi \newline (GPIO22)
&Betjeningspanel \newline (onOff\_out) \newline
 Betjeningspanel \newline (manStart\_out)
&Logisk signal til kontrol af knapper på betjeningspanel
\\\hline

\end{longtable}
\end{center}


\subsection{Design af Controller}


\subsubsection{Lysdioder}

Controlleren består som beskrevet af 3 lysdioder. Én grøn, én gul og én rød der hhv. indikerer at Baby Watch er tændt/slukket, at "Manuel start" er aktiveret/deaktiveret samt Wi-Fi status. 

Der benyttes 5mm dioder fra komponentrummet: 

\begin{itemize}
	\item Grøn 5mm LED: KINGBRIGHT L-53 GD
	\item Gul 5mm LED: KINGBRIGHT L-53 YD
	\item Rød 5mm LED: KINGBRIGHT L-53 HD
\end{itemize}


\figur{1}{controller/L53_LEDS}{Udsnit af datablad for KINGBRIGHT L53 HD, GD og YD }{controller:ledSpec}

Ud fra figur \ref{controller:ledSpec} ses strømmen som funktion af spændingen over dioderne. Den indtegnede blå linje på hver af 3 kurver angiver spændingsfaldet over hver diode når strømmen er sat til 10 mA. Ud fra aflæsning på kurverne beregnes for modstandene for dioderne

\figur{0.5}{controller/L53_resistor}{For modstandsberegninger for de 3 dioder}{controller:resistors}

Ud fra modstandsberegningerne i figur \ref{controller:resistors} er kredsløbsdiagrammet, se figur \ref{controller:schematic}, opbygget. De to knapper for ON/OFF samt "Manuel start" er medtaget sammen med deres gpio porte på Rasberry Pi model b.

\figur{1}{controller/controller_multisim}{Kredsløbsdiagram for Controller}{controller:schematic}

\subsection{Implementering af Controller}
