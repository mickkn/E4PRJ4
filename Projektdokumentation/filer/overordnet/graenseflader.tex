%Grænseflader
\section{Grænseflader}
I dette afsnit beskrives specifikke hardware grænseflader og protokoller for kommunikation mellem systemets moduler. Disse grænseflader er således modul-udviklernes kontrakt for fyldestgørelse af modulets funktionalitet.

\subsection{Grænseflade mellem Intelligent Lydmonitor og Controller}
\label{overordnet:graenseflade_IL}
Kommunikationen mellem Intelligent Lydmonitor og Controller foregår via 2 signal ledere, der forbindes til 2 GPIO'er på Controller. Den intelligente Lydmonitor opdaterer via denne forbindelse BABYCON niveauet.  

\subsubsection{Kommunikationsprotokol}
\label{overordnet:Kommunikationsprotokol}

Tabellerne \ref{Overordnet:Kommunikationsprotokol/BABYCON1} til \ref{Overordnet:Kommunikationsprotokol/IL_fejl} viser kommunikationsprotokollen for 2WireTTL forbindelsen mellem Intelligent Lydmonitor og Controller. De 2 GPIO porte på controllenren repræsenteres af hhv MSB og LSB, som mest og mindst betydende bit af 2WireTTL forbindelsen, som skal være aktiv-høj.

\begin{table}[H]
	\caption{BABYCON1 }
	\centering
	\begin{tabular}{|l|c|c|}
		\hline 
			&\textbf{2WireTTL\_MSB} & \textbf{2WireTTL\_LSB}     \\ 
		\hline 
		\textbf{Bit værdi} &0 & 1     \\ 
		\hline
	\end{tabular} 
	\label{Overordnet:Kommunikationsprotokol/BABYCON1}
\end{table}

\begin{table}[H]
	\caption{BABYCON2 }
	\centering
	\begin{tabular}{|l|c|c|}
		\hline 
			&\textbf{2WireTTL\_MSB} & \textbf{2WireTTL\_LSB}     \\ 
		\hline 
		\textbf{Bit værdi} &1 & 0     \\ 
		\hline
	\end{tabular} 
	\label{Overordnet:Kommunikationsprotokol/BABYCON2}
\end{table}


\begin{table}[H]
	\caption{BABYCON3 }
	\centering
	\begin{tabular}{|l|c|c|}
		\hline 
			&\textbf{2WireTTL\_MSB} & \textbf{2WireTTL\_LSB}     \\ 
		\hline 
		\textbf{Bit værdi} &1 & 1     \\ 
		\hline
	\end{tabular} 
	\label{Overordnet:Kommunikationsprotokol/BABYCON3}
\end{table}


\begin{table}[H]
	\caption{Fejl fra Intelligent Lydmonitor }
	\centering
	\begin{tabular}{|l|c|c|}
		\hline 
			&\textbf{2WireTTL\_MSB} & \textbf{2WireTTL\_LSB}     \\ 
		\hline 
		\textbf{Bit værdi} &0 & 0     \\ 
		\hline
	\end{tabular} 
	\label{Overordnet:Kommunikationsprotokol/IL_fejl}
\end{table}

2WireTTL forbindelsen har som ovenstående tabeller \ref{Overordnet:Kommunikationsprotokol/BABYCON1} til \ref{Overordnet:Kommunikationsprotokol/IL_fejl} viser 4 kombinationsmuligheder. 3 af dem til de 3 BABYCON niveauer, samt en til fejlindikation fra den Intelligente Lydmonitors side. 

\subsection{Grænseflade mellem Vuggesystem og Controller}
Kommunikationen mellem Vuggesystemet og Controllen foregår via I2C. Der benyttes 5 skifteregistre til kommunikationen, disse registre beskrives i kommunikationsprotokollen herunder.

\subsubsection{Kommunikationsprotokol}
Forbindelserne ud og ind af dette delsystem er samlet i en I2C bus, som er beskrevet på næste side...

\begin{center}
\begin{table}[H]
\label{overordnet:i2c_tabel}
\caption{Specifikation af I2C grænseflade}
\begin{tabular}{|p{1cm}|p{2.4cm}|p{1cm}|p{5.5cm}|p{2.5cm}|}
\hline 
\multicolumn{5}{|l|}{\textbf{I2C Adresse:} 0b1111000X  (Write: 0xF0, Read: 0xF1} \\ 
\hline 
\multicolumn{5}{|l|}{\textbf{I2C Frekvens:} 100kHz} \\ 
\hline 
{Reg\#} & \textbf{Navn} & \textbf{Type} & \textbf{Beskrivelse} & \textbf{Startværdi} \\ 
\hline 
0x00 & ID & \textbf{R} & Indeholder et id som kan benyttes til at identificere denne enhed, eller til at teste forbindelsen til denne. & 0xFB \\ 
\hline 
\multirow{5}{*}{0x01} & Status & \textbf{R} & Indeholder en bitsekvens som indikerer systemets status. Registeret indeholder følgende: [\textbf{ERR} \textbf{STALL} \textbf{END\_STP} \textbf{SD\_RDY} X X X X] & 0b0000XXXX \\ \cline{2-5}

	&\multicolumn {2}{p{3cm} |} {\textbf{ERR}} & \multicolumn {2}{p{8cm} |} {Indikerer at der er opstået en fejl i systemet. Kendes årsagen til fejlen indikeres denne i STALL og END\_STP} \\ \cline{2-5}

	&\multicolumn {2}{p{3cm} |} {\textbf{STALL}} & \multicolumn {2}{p{8cm} |} {Indikerer at systemet har været ude af stand til at drive motoren, formegentlig pga for stor belastning} \\ \cline{2-5}

	&\multicolumn {2}{p{3cm} |} {\textbf{END\_STP}} & \multicolumn {2}{p{8cm} |} {Indikerer at vuggen har ramt en af sine mekaniske yderpositioner, og vuggesystemet er deaktiveret indtil der er blevet genstartet.} \\ \cline{2-5}

	&\multicolumn {2}{p{3cm} |} {\textbf{SD\_RDY}} & \multicolumn {2}{p{8cm} |} {Indikerer at systemet er klar til at få afbrudt strømmen} \\ \hline
0x02 & ON\_OFF & \textbf(R/W) & Dette register benyttes til at tænde og slukke for systemet. Skrives et nul til dette register begynder systemet at lukke ned. Strømmen til systemet bør ikke afbrydes før SD\_RDY i status registeret er skiftet til et. Hvis systemet er tændt indeholder registeret en værdi forskellig fra 0. & 0xFF \\ \hline

0x03 & Frekvens & \textbf{R/W} & Værdien i dette register styrer frekvensen hvormed der vugges. Område: \SI{0}{\hertz} = \SI{2.550}{\hertz},  1 LSB = \SI{10}{\milli\hertz}. & 0x00 \\ \hline

0x04 & Vinkeludsving & \textbf{R/W} & Værdien i dette register kontrollerer størrelsen af vuggens udsving i grader. Område: +/- \SI{12.75}{\degree}, 1 LSB = \SI{0.05}{\degree}. & 0x00 \\ \hline

\hline 
\end{tabular}
\end{table}
\end{center} 

\subsection{Grænseflade mellem Controllers servere og ekstern internet enhed}
\subsubsection*{Fysiske porte}

Raspberry Pi'en der er Controlleren hovedenhed, er forbundet til netværket via en trådløs Wi-Fi dongle. Netværket består af en router der er koblet op til internettet.

\subsubsection*{Kommunikationsprotokol}

Standard Ethernet TCP/IP protokol