\section{Systemarkitektur}

%Indledningstekst her
Det følgende afsnit beskriver den overordnede systemarkitektur for Baby Watch. Afsnittet består af forskellige diagrammer og tabeller med tilhørende forklaringer. Diagrammerne er opbygget efter SysML standarden.

%Domænemodel
\subsection{Domænemodel}
\figur{1}{overordnet/sysark/domaene_model}{Domænemodel for Baby Watch}{overordnet:domaene_model}

Figur \ref{overordnet:domaene_model} viser domænemodellen for Baby Watch. Modellen beskriver systemets funktionalitet og indbyrdes sammenhæng. Multipliciteter på 1 til 1 er udeladt i diagrammet. Babypasseren kan tjekke status på flere enheder (antal simultane enheder begrænses af HTTP-serverens kapacitet). Systemet er som udgangspunkt kun designet til én baby og multipliciteten for denne er derfor også 1 til 1.

\subsection{System-states}
\figur{1}{overordnet/sysark/Overordnet_STM.pdf}{Overordnede states for Baby Watch}{overordnet:STM}

Systemets har tre overordnede states: Monitoringstilstand, ''Manuel start''-tilstand og Undtagelsestilstand. \\\textbf{Monitoreringstilstand} har to substates: 
\begin{itemize}
	\item \textbf{Stilstand}, der indikerer BABYCON3 niveau med rolig baby. Vuggesystemet er for denne tilstand inaktivt.
	\item \textbf{Vugning}, der indikerer BABYCON2 niveau med urolig baby, og hvori barnevognen vugger automatisk.
\end{itemize}

\textbf{''Manuel start''-tilstand} igangsættes manuelt af babypasseren. Her sættes barnevognen til at vugge i 5 minutter, hvorefter systemet går i Monitoreringstilstand. Efter to minutter afbryder en måling af barneaktivitet svarende til BABYCON1 også ''Manuel start''-tilstand. Systemet går herefter i Undtagelsestilstand.

\textbf{Undtagelsestilstand} indikerer BABYCON1 niveau med højeste alarmberedskab. Her stoppes barnevognens vuggefunktion, der sendes en advarsels e-mail til Babypasser og hjemmeside opdateres til BABYCON1, som igangsætter afspilning af en alarmlyd. 


\subsection{BDD}
\figur{1}{overordnet/sysark/Overordnet_BDD.pdf}{Overordnet BDD for Baby Watch}{sysark:overordnet_BDD}
Figur \ref{sysark:overordnet_BDD} viser det overordnede BDD for Baby Watch systemet. Som figuren viser, består systemet af en intelligent lydmonitor, et vuggesystem, en power blok, samt en controller, der indeholder en HTTP server, en E-mail klient samt et betjeningspanel.


\subsection{IBD}
\figur{1}{overordnet/sysark/Overordnet_IBD.pdf}{Overordnet IBD for Baby Watch}{sysark:overordnet_IBD}

Figur \ref{sysark:overordnet_IBD} viser det overordnede IBD for Baby Watch systemet.  Figuren viser de interne forbindelser for blokkene i figur \ref{sysark:overordnet_BDD}. For yderlige specifikation af porte og signaler se signalbeskrivelsen \ref{overordnet:signalbeskrivelse}.

