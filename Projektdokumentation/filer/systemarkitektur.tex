\chapter{Systemarkitektur}

%Indledningstekst her

%Domænemodel
%\figur{Bredden}{Pdfnavn}{Billedtekst}{Label}

%Statediagram
\subsection{Overordnede system states}

\figur{1}{sysark/Overordnet_STM.pdf}{Overordnede states for systemet}{Overordnet_STM}

Systemets har tre overordnede states: Monitoringstilstand, Puttetilstand og Undtagelsestilstand. \\\textbf{Monitoreringstilstand} har to substates: 
\begin{itemize}
	\item \textbf{Stilstand}, der indikerer laveste alarmtilstand med rolig baby. Vuggesystemet er for denne tilstand inaktivt
	\item \textbf{Vugning}, der indikerer en let aktiv baby og hvori barnevognen vugger automatisk.
\end{itemize}

\textbf{Puttetilstand} igangsættes manuelt af babypasseren. Her sættes barnevognen til at vugge i 5 minutter, hvorefter systemet går i Monitoring mode. Efter to minutter afbryder en måling af barneaktivitet svarende til BABYCON1 også puttetilstand. Systemet går herefter i undtagelsestilstand.

\textbf{Undtagelsestilstand} indikerer højeste alarmberedskab. Her stoppes barnevogens vuggefunktion, sendes en advarselsemail til Babypasser og opdateres hjemmeside til BABYCON1, som igangsætter afspilning af en alarmlyd. 

\figur{1}{BabyWatchBDD}{Overordnet BDD for Baby Watch}{sysark:overordnet_BDD}
Figur \ref{sysark:overordnet_BDD} viser det overordnede BDD for Baby Watch systemet. Som figuren viser består systemet af en inteligent lydmonitor, en controller, et vuggesystem, en server samt en power blok. 

\figur{1}{BabyWatchIBD}{Overordnet IBD for Baby Watch}{sysark:overordnet_IBD}

Figur \ref{sysark:overordnet_IBD} viser det overordnede IBD for Baby Watch systemet.  Figuren viser de interne forbindelser for blokkene i figur \ref{sysark:overordnet_BDD}. For yderlige specifikation af porte og signaler se signaltabellen REF!!!

%signal beskrivelse

\begin{table}[H]
\subsection{Signalbeskrivelse}
TEKST..
 
\caption{Tabel over signaler med terminaler}
\begin{small}
\begin{tabular}{|p{2cm}|p{2cm}|p{2cm}|p{2cm}|p{2cm}|p{2,2cm}|}
\hline

\textbf{Signal-navn}	&\textbf{Type} 		&\textbf{Område} &\textbf{Port 1} 	&\textbf{Port 2} 			&\textbf{Kommentar} \\ \hline

monitorOn 			&Digital signal  	&Høj/lav repræsentation af tændt/slukket 				
&Controller (mon\_out)		&Intelligent lydmonitor (mon\_in)			&Indikerer hvorvidt monitoreringsenheden skal være aktive eller ej					 \\\hline

babyState 			&uint8				&0-255 lineær repræsentation af babyens uro-niveau fra helt stille til det højeste detekter bare niveau 	&Intelligent lydmonitor (bs\_out) &Controller (bs\_in)			&Dette signal er en indikation af babyens vurderede urolighedsniveau på baggrund lydanalysen		\\\hline

serverCmd 			&Wi-Fi				&?	 	&Controller (server\_out)  &Server (server\_in				
&Denne forbindelse benyttes til at opdatere hjemmesiden samt sende emails ved BABYCON1					\\\hline
					
vuggeFrek			&uint8				&0-255 lineær repræsentation af den ønskede vugge frekvens 0-2 Hz  	&Controller (vs\_c)  &Vuggesystem (vs\_v)		&Indeholder den ønskede frekvens til reguleringen 	    				\\\hline

vuggeAmp				&uint8 				&0-255 lineær repræsentation af den ønskede vugge amplitude 0-10°   &Controller (vs\_c)  &Vuggesystem (vs\_v)		&Indeholder den ønskede amplitude til reguleringen   				\\\hline

vuggeOn			&Digital signal 			&Høj/lav repræsentation af tændt/slukket 
&Controller (vs\_c) 	&Vuggesystem (vs\_v)			&Indikerer hvorvidt vuggesystemet skal være aktivt eller ej 				\\\hline

overlast			&Digital signal			&??   
&Controller (vs\_v) 	&Vuggesystem (vs\_c)			&  				\\\hline

powerOn			&Digital signal				&Høj/lav repræsentation af tændt/slukket 
&Controller (pow\_out) 			&Power (pow\_in)	&Benyttes til at tænde og slukket for strømforsyningen   				\\\hline

powerStby		&Digital signal 				&0-255 lineær repræsentation af den ønskede vugge amplitude 0-10°   &Power (powStby\_out)			&Controller (powStby\_in)	& Forsyner Controller med strøm også når systemet er slukket   				\\\hline

power			&power 				&Forsyning
&power 			&power				&Forsyner systemet med strøm   				\\\hline

psu				&230VAC 				&230VAC
&Server 	(psu\_in)	&230VAC elnet			&Netspændingsforsyning til server   				\\\hline

\end{tabular}
\end{small}
\label{table:Signaltabel}
\end{table}

\newpage
%% Hardware
\section{Hardwarebeskrivelse}
%Hardwarebeskrivelse

\clearpage
%% Sowftware
\section{Softwarebeskrivelse}
%Softwarebeskrivelse


