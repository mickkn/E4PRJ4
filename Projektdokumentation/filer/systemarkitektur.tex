\chapter{Systemarkitektur}

%Indledningstekst her

%Domænemodel
%\figur{Bredden}{Pdfnavn}{Billedtekst}{Label}

%Statediagram
\subsection{Overordnede system states}
\figur{1}{Overordnet_STM.png}{Overordnede states for systemet}{Overordnet_STM}

Systemets har to overordnede states: Monitoring og Snuggle. \textbf{Monitoring} har tre substates: 
\begin{itemize}
\item \textbf{BABYCON 3}, der indikerer laveste alarmtilstand med rolig baby. \item \textbf{BABYCON 2}, der indikerer en let aktiv baby og hvori barnevognen vugger automatisk.
\item \textbf{BABYCON 1}, der indikerer alarmberedskab og hvor en alarmlyd afspilles. 
\end{itemize}

\textbf{Snuggle} er en specialtilstand, som børnepasseren kan igangsætte manuelt. Her sættes barnevognen til at vugge i 5 minutter, hvorefter systemet går i Monitoring mode. En måling af barneaktivitet svarende til BABYCON 1, afbryder også snuggle mode.


\figur{1}{BabyWatchBDD}{Overordnet BDD for Baby Watch}{sysark:overordnet_BDD}
Figur \ref{sysark:overordnet_BDD} viser det overordnede BDD for Baby Watch systemet. Som figuren viser består systemet af en inteligent lydmonitor, en controller, et vuggesystem, en server samt en power blok. 

\figur{1}{BabyWatchIBD}{Overordnet IBD for Baby Watch}{sysark:overordnet_IBD}

Figur \ref{sysark:overordnet_IBD} viser det overordnede IBD for Baby Watch systemet.  Figuren viser de interne forbindelser for blokkene i figur \ref{sysark:overordnet_BDD}. For yderlige specifikation af porte og signaler se signaltabellen REF!!!

\newpage
%% Hardware
\section{Hardwarebeskrivelse}
\input{filer/systemarkitektur/hardware}

\clearpage
%% Sowftware
\section{Softwarebeskrivelse}
\input{filer/systemarkitektur/software}


