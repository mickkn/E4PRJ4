\chapter{Termliste}

% Husk at indsætte i alfabetisk orden!!

\begin{description}
\item[AASH] Antal af samtidige hændelser
\item[Baby] Aktøren som står for at generere lyd til systemet
\item[Baby Watch] Navnet på systemet
\item[BABYCON niveau] En skala for babys humørtilstand (1, 2 og 3)
\item[BABYCON3: Rolig] Lyden indikerer at baby sover
\item[BABYCON2: Urolig] Lyden indikerer at baby er vågen
\item[BABYCON1: Alarmerende] Lyden indikerer at baby er i en tilstand der kræver tilsyn
\item[Babypasser] Brugeren som ønsker at benytte systemet
\item[Betjeningspanel]Et panel bestående af to knapper(En ON/OFF-knap og en Manuelstart-knap) og tre LED'er(En ON/OFF-LED, en ). 
%\item[E-mail Alarmerende] Alarm meddelelse ved BABYCON1
%\item[E-mail Fejl ved lydoptagelse] Fejl meddelelse ved fejl i lydoptagelse
%\item[E-mail Fejl i vuggesystem] Fejl meddelelse ved overbelastning af vuggesystem 
\item[Intelligent lydmonitor] Del af system som står for indsamling/behandling af lyd
\item[Manuel start] Knap med funktion der manuelt starter vuggemekanismen
\item[Monitorerings-tilstand] En tilstand hvor der systemet kører efter hensigten
\item[Passiv monitorering] Dækker over en tilstand hvor der vugges manuelt og monitoreres for BABYCON1
\item[Putte-tilstand] En tilstand hvor barnevognen vugger uden at blive styret af lydinput
\item[SPL] Sound Preasure Level (dB)
\item[Undtagelses-tilstand] Dækker over tilstanden hvor barnevogn stopper og går i BABYCON1
\item[Vugge-indsving] Skal forstås som at når barnevognen begynder en vugning så skal vuggesystemet give det en blid opstart
\item[Wi-Fi-LED]Diode der lyser når WiFi forbindelsen er afbrudt





\end{description}