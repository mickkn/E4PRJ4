\chapter{Kravspecifikation}

\section{Indledning}

Formålet med projektet er at lave en prototype af en intelligent babymonitor til barnevogne med dertilhørende vuggesystem, statushjemmeside og e-mail notifikation.

Systemet har tre tilstande en putte-tilstand, hvor barnet lægges til at sove, en monitorerings-tilstand og undtagelse-tilstand.
I \textbf{putte-tilstanden}, skal systemet vugge babyen i et fastsat tidsinterval, hvorefter monitoreringen overtager.
I \textbf{monitorerings-tilstanden} styres barnevognens vuggefunktion på baggrund af analysen af den aktuelle baby-lydoptagelse.
I \textbf{undtagelses-tilstanden} låses systemet. Barnevognen skal ikke vugges. En e-mail afsendes til den registrerede babypasser hvert 30. sekund, indtil systemet resettes manuelt ude ved barnevognen. 

Statushjemmesiden opdateres løbende, med status på baby, indledt i tre konditioner vist som en BABYCON statusbar.

\begin{itemize}
\item \textbf{BABYCON3} 
\newline På dette niveau kategoriseres lydsignalet fra barnet som roligt. Heraf skal barnevognen ikke vugges. Systemet indsamler lyd og afventer en ændring i lydsignalet, som vil medføre en ændring i niveau. 

\item \textbf{BABYCON2}
\newline På dette niveau kategoriseres lydsignalet fra barnet som uroligt. Heraf skal barnevognen vugge med en forudbestemt frekvens. Systemet indsamler lyd og afventer en ændring i lydsignalet, som vil medføre en ændring i niveau. 

\item \textbf{BABYCON1}
\newline På dette niveau kategoriseres lydsignalet fra barnet som alarmerende. Undtagelses-tilstanden aktiveres. 
\end{itemize}
 
\subsection*{Systemtegning}
\figur{1}{Systemtegning}{Skitse af systemet Baby Watch}{kravspec:systemtegning}

Figur \ref{kravspec:systemtegning} viser illustrativt systemet. Barnevognen er udstyret med en controller, der styrer den intelligente lydmonitor samt vuggesystemet. Controlleren forbindes via Wi-Fi til en server. Babypasseren modtager information på en hjemmeside og modtager desuden mail når systemets tilstand skifter til BABYCON1.
\newpage
\section{Use cases}

I dette afsnit specificeres brugssituationer for systemet Baby Watch.

Kravspecifikationen er udfærdiget med basis i en use case-baseret tilgang. Den indeholder en beskrivelse over relevante aktører for systemet samt fully dressed use cases.

\figur{1}{kravspec/usecase_diagram}{Usecase diagram for Baby Watch}{kravspec:usecase_diagram}

Usecase diagrammet giver et overblik aktørerne og deres rolle i use cases.

\section{Aktører}

%Indsæt aktørdiagram og aktør beskrivelser

%\figur{Bredden}{Pdfnavn}{Billedtekst}{Label}

\begin{table}[!htbp] \centering
	\begin{tabular}{|p{2.5cm}|p{11.5cm}|}
	\hline
		\textbf{Aktør navn} & \textbf{Beskrivelse} \\\hline
		Barn &Skriv her 
		\\\hline
		Børnepasser & Skriv her 
		\\\hline
	\end{tabular}
\end{table}
\newpage
\section{Fully dressed use cases}
\begin{center} \centering \label{kravspec:uc1}
	\begin{longtable}{|p{5cm}|p{9cm}|}  %% Longtable = forsætter på næste side
	\hline
		\multicolumn{2}{|l|}{\textbf{UC1: Igangsæt vugning manuelt}} \\\hline %% HUSK USECASE NAVN
		\endfirsthead
		
		\multicolumn{2}{l}{...fortsat fra forrige side} \\ \hline %% Til LONGTABLE
		\multicolumn{2}{|l|}{\textbf{UC1: Igangsæt vugning manuelt}} \\\hline %%  HUSK USECASE NAVN
		\endhead	
		
		\textbf{Mål}							&At igangsætte vugning af barnevogn manuelt 		\\\hline
		\textbf{Initialisering}				&Babypasser		\\\hline
		\textbf{Aktører og Stakeholders}		&Babypasser(Primær)		\\\hline 
		\textbf{Referencer}					&UC2, UC3, UC4		\\\hline
		\textbf{AASH}						&1		\\\hline
		\textbf{Efterfølgende tilstand}		&UC3: Monitorer baby igangsat		\\\hline
		\textbf{Hovedforløb}					
			&\begin{enumerate}
	
				\item Børnepasser igangsætter manuel vugning ved tryk på "Manuel start"-knap på Baby Watch controller
				
				\item \label{kravspec:uc1_vugning}Systemet starter vugning iht. UC2
				
				\item \label{kravspec:uc1_timeout2}Efter 2 minutter igangsætter systemet passiv monitorering af baby jf. UC3			
								
				\item \label{kravspec:uc1_timeout5}Efter 5 minutter igangsætter systemet automatisk monitorering af baby jf. UC3	
				\newline [Und: \ref{kravspec:uc1_timeout5}.a Alarmerende baby detekteret indenfor 5 min]				
				
			\end{enumerate}
		\\\hline
		\textbf{Undtagelser}
			&\begin{enumerate}[label=\ref{kravspec:uc1_vugning}.a]
			\item Alarmerende baby detekteret indenfor 5 min
					\begin{itemize}
					
					\item Systemet igangsætter undtagelsestilstand jf. UC4
					
					\end{itemize}
			\end{enumerate}
			
		\\\hline
	\end{longtable} 
\end{center}

%% TIPS:
%% LABEL TIL PUNKT: \label{labelnavn}
%% REFERENCE: \ref{labelnavn}
\newpage
\begin{center} \centering \label{kravspec:uc2}
	\begin{longtable}{|p{5cm}|p{9cm}|}  %% Longtable = forsætter på næste side
	\hline
		\multicolumn{2}{|l|}{\textbf{UC2: Vug barnevogn}} \\\hline %% HUSK USECASE NAVN
		\endfirsthead
		
		\multicolumn{2}{l}{...fortsat fra forrige side} \\ \hline %% Til LONGTABLE
		\multicolumn{2}{|l|}{\textbf{UC2: Vug barnevogn}} \\\hline %%  HUSK USECASE NAVN
		\endhead	
		
		\textbf{Mål}							&At vugge barnevognen i et tidsinterval		\\\hline
		\textbf{Initialisering}				&UC1, UC3 \\\hline
		\textbf{Aktører og Stakeholders}		&Baby(Sekundær)		\\\hline 
		\textbf{Referencer}					&UC1, UC3, UC4		\\\hline
		\textbf{AASH}						&1		\\\hline
		\textbf{Efterfølgende tilstand}		&Barnvognen er i stilstand	\\\hline
		\textbf{Hovedforløb}					
			&\begin{enumerate}
	
				\item Systemet igangsætter vugge-indsvingning
				
				\item \label{kravspec:uc2_vugning}Systemet opretholder vugning ved en fastsat frekvens
				\newline [Und: \ref{kravspec:uc2_vugning}.a Overbelastning af vuggesystem]	
				
				\item \label{kravspec:uc2_stopvugning}Fra UC3 eller UC4 	gives besked om stop af vugning	
				
				\item Systemet dæmper vugning til stilstand
				
			\end{enumerate}
		\\\hline
		\textbf{Undtagelser}
			&\begin{enumerate}[label=\ref{kravspec:uc2_vugning}.a]
			\item Overbelastning af vuggesystem
					\begin{itemize}
					
					\item Afsend overbelastnings e-mail
					\item Systemet igangsætter undtagelsestilstand jf. UC4
					
					\end{itemize}
			\end{enumerate}
		\\\hline
	\end{longtable} 
\end{center}

%% TIPS:
%% LABEL TIL PUNKT: \label{labelnavn}
%% REFERENCE: \ref{labelnavn}
\newpage
\begin{center} \centering \label{kravspec:uc1}
	\begin{longtable}{|p{5cm}|p{9cm}|}  %% Longtable = forsætter på næste side
	\hline
		\multicolumn{2}{|l|}{\textbf{UC1: Monitorér baby}} \\\hline %% HUSK USECASE NAVN
		\endfirsthead
		
		\multicolumn{2}{l}{...fortsat fra forrige side} \\ \hline %% Til LONGTABLE
		\multicolumn{2}{|l|}{\textbf{UC1: Monitorér baby}} \\\hline %%  HUSK USECASE NAVN
		\endhead	
		
		\textbf{Mål}							&BABYCON tilstand er opdateret		\\\hline
		\textbf{Initialisering}				&Babypasser		\\\hline
		\textbf{Aktører og Stakeholders}		&Baby(Primær), Babypasser(Primær/Sekundær)		\\\hline 
		\textbf{Referencer}					&Ingen		\\\hline
		\textbf{AASH}						&1		\\\hline
		\textbf{Efterfølgende tilstand}		&Monitorér barn fortsættes		\\\hline
		\textbf{Hovedforløb}					
			&\begin{enumerate}
	
				\item Børnepasseren starter system
				
				\item Lyd-detekteringssystem indsamler lyd i et bestemt tidsinterval 				
				
				\item Systemet analyserer indsamlet lydsample 
				
				\item Systemet opdaterer BABYCON tilstand på hjemmeside på baggrund af analyse 				
				
				\item Systemet opdaterer BABYCON tilstand på baggrund af analyse
				
				\item \textit{UC1 Monitorér barn} fortsættes fra punkt 2
				
			\end{enumerate}
		\\\hline
		\textbf{Undtagelser}
			&\begin{enumerate}
			
				\item Punkt 1

				\item Punkt 2
				
				\item Punkt 3

			\end{enumerate}
		\\\hline
	\end{longtable} 
\end{center}

%% TIPS:
%% LABEL TIL PUNKT: \label{labelnavn}
%% REFERENCE: \ref{labelnavn}
\newpage
\begin{center} \centering \label{kravspec:uc4}
	\begin{longtable}{|p{5cm}|p{9cm}|}  %% Longtable = forsætter på næste side
	\hline
		\multicolumn{2}{|l|}{\textbf{UC4: Igangsæt undtagelsestilstand}} \\\hline %% HUSK USECASE NAVN
		\endfirsthead
		
		\multicolumn{2}{l}{...fortsat fra forrige side} \\ \hline %% Til LONGTABLE
		\multicolumn{2}{|l|}{\textbf{UC4: Igangsæt undtagelsestilstand}} \\\hline %%  HUSK USECASE NAVN
		\endhead	
		
		\textbf{Mål}							&At stoppe vugning og alarmere Babypasser		\\\hline
		\textbf{Initialisering}				&UC2, UC3		\\\hline
		\textbf{Aktører og Stakeholders}		&Babypasser(Sekundær)	\\\hline 
		\textbf{Referencer}					&UC2, UC3, UC5		\\\hline
		\textbf{AASH}						&1		\\\hline
		\textbf{Efterfølgende tilstand}		&Vugning indstillet, e-mail afsendt og hjemmeside opdateret 		\\\hline
		\textbf{Hovedforløb}					
			&\begin{enumerate}
				
				\item Systemet stopper vugning jf. UC2 punkt \ref{kravspec:uc2_stopvugning} 							
				
				\item \label{kravspec:uc4_opdaterweb}Systemet opdaterer hjemmeside til BABYCON1				
				\newline [Und: \ref{kravspec:uc4_opdaterweb}.a Ingen netværksforbindelse]
				
				\item \label{kravspec:uc4_emailafsend}System afsender e-mail til Babypasser hvert 30. sekund
				\newline [Und: \ref{kravspec:uc4_emailafsend}.a Ingen netværksforbindelse]
				
				\item System afventer genstart fra Babypasser
				
			\end{enumerate}
		\\\hline
		\textbf{Undtagelser}
			&\begin{enumerate}[label=\ref{kravspec:uc4_opdaterweb}.a]
			\item Ingen netværksforbindelse
					\begin{itemize}				
						\item Wi-Fi-LED lyser
					\end{itemize}
			\end{enumerate}
			\begin{enumerate}[label=\ref{kravspec:uc4_emailafsend}.a]
			\item Ingen netværksforbindelse
					\begin{itemize}				
						\item Wi-Fi-LED lyser
					\end{itemize}
			\end{enumerate}
		\\\hline
	\end{longtable} 
\end{center}

%% TIPS:
%% LABEL TIL PUNKT: \label{labelnavn}
%% REFERENCE: \ref{labelnavn}
\section{Ikke-funktionelle krav}

\subsection*{Mikrofon}
For at kunne opfange et tilstrækkeligt signal til analyse af babyens gråd, skal systemets mikrofon opfylde følgende krav:
\begin{itemize}
\item Mikrofonen skal være en cardioid mikrofon for at minimere uønskede baggrundslyde
\item Mikrofonen skal have en max SPL rating på min. \SI{120}{\dB}.\footnote{FIXME indsæt reference til studie om gråd volumen}
\item Mikrofonen skal have en jævn frekvens respons på maksimalt +/- \SI{5}{\dB} fra \SI{40}{\hertz} til \SI{20}{\kilo\hertz}
\end{itemize}

\subsection*{Vuggemekanisme}

Systemets vugge mekanisme skal overholde følgende krav for at sikre en blid vugning:
\begin{itemize}
\item Vuggen skal kunne vippe det plan hvorpå babyen ligger med op til \SI{10}{\degree} i hver retning fra dets vandrete udgangspunkt, med en fejlmargin på \SI{2}{\degree}.
\item Vuggen skal kunne variere frekvensen hvormed der vugges fra \SI{0}{\hertz} til \SI{2}{\hertz}, med en fejlmargin på \SI{0.2}{\hertz}.
\item Vuggen skal vende tilbage til en vinkel indenfor \SI{5}{\degree} af dens udgangspunkt når systemet lukkes ned.
\item Vuggen skal have en begrænsning på vinkelfrekvensen ved \SI{20}{\degree\per\second}, med en fejlmargin på \SI{10}{\percent}.
\item Vuggens vinkel acceleration skal være begrænset ved \SI{20}{\degree\per\square\second}, med en fejlmargin på \SI{10}{\percent}.
\end{itemize}

\subsection*{Baby status}
For at sikre at vurderingen af babyens status er pålidelig, samt rettidigt tilgængelig for brugeren skal systemet overholde følgende:
\begin{itemize}
\item Systemets babystatus(babycon niveau) skal opdateres minimum hvert 5. sekund.
\item Status hjemmesiden skal være opdateret senest 5 sekunder efter opdateret babystatus.
\end{itemize}

\section{Web-skitser}

%\figur{Bredden}{Pdfnavn}{Billedtekst}{Label}

De følgende figurer \ref{kracspec:BABYCON_GUI_3} , \ref{kracspec:BABYCON_GUI_2} samt \ref{kracspec:BABYCON_GUI_1} ses skitser af hjemmesidens udseende. Hjemmesiden viser aktuel tid, samlet tid som barnet har været i ro, et billede af barnet med tilhørende navn og besked, BABYCON skala fra 1-3 og information omkring hvornår hjemmesiden sidst er opdateret.

\figur{1}{GUI_BABYCON_3}{BABYCON3}{kracspec:BABYCON_GUI_3}

Figur \ref{kracspec:BABYCON_GUI_3} illustrerer hjemmesiden når BABYCON niveauet er 3, det niveau hvor barnet er i ro.

\figur{1}{GUI_BABYCON_2}{BABYCON2}{kracspec:BABYCON_GUI_2}

Figur \ref{kracspec:BABYCON_GUI_2} illustrerer hjemmesiden når BABYCON niveauet er 2, det niveau hvor barnet er utilpas, men ikke nok til at udløse alarm til barnepasser.

\figur{1}{GUI_BABYCON_1}{BABYCON1}{kracspec:BABYCON_GUI_1}

Figur \ref{kracspec:BABYCON_GUI_1} illustrerer hjemmesiden når BABYCON niveauet er 1. Niveau 1 er alarm niveauet hvor barnet er så utilfreds at barnepasseren modtager en email og selv må trøste barnet.

\section{E-mail beskeder}

\subsection*{BabyWatch: Alarm!}

Emne: BabyWatch: Alarm! \newline
Tekst: Baby Watch har registreret en alarmerende baby (Vugning er stoppet)

\subsection*{BabyWatch: Fejl!}

Emne: BabyWatch: Fejl! \newline
Tekst: Fejl i vuggesystem, afventer genstart.



