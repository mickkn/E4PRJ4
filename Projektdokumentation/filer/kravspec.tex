\chapter{Kravspecifikation}

\chapter{Indledning}

Når EMC reglerne skal imødegås er det vigtigt at overveje EMC-problematikkerne allerede i de indledende faser af designet og implementering af printene. Dette gøres for at finde løsninger imens det stadig er billigt at implementere. Igennem en udviklingsproces af et hardwareprodukt, vil prisen på eventuelle EMC-mæssige rettelser stige eksponentiel i takt med hver fase i udviklingen. Det vil derfor være uheldigt ikke at tage hensyn til disse, da en rettelse i så fald kan betyde at systemet at bliver mange gange dyrere, end hvis man havde gennemtænkt sit design fra start af.

Den største EMC-problematik i Baby Watch systemet består af en DC-motor der bruges til at trække vugningen af barnevognskurven samt det dertilhørende motorkredsløb. Ydermere består Baby Watch af en Controller-del bestående af en hovedcomputer\footnote{Raspberry Pi} med GPIO porte, en DSP\footnote{Digital Signal Processor(Blackfin)} med dertilhørende mikrofon forstærker til optagelse og analyse af lyd og en strømforsyning drevet af et 12 V batteri, til at forsyne alle dele af systemet.

Rapporten er delt op i samme form for afsnit som Projektdokumentationen, så hvert projektdel har sit eget afsnit og EMC-overvejelser. Den er sat op i LaTeX, som gør at Rapporten starter sine kapitler på næste højre side, da det er meningen at der printes på begge sider i udskriften, dette medfører en del blanke sider.

%\figur{0.7}{emc}

\newpage
\section{Use cases}

I dette afsnit specificeres brugssituationer for systemet Baby Watch.

Kravspecifikationen er udfærdiget med basis i en use case-baseret tilgang. Den indeholder en beskrivelse over relevante aktører for systemet samt fully dressed use cases.

\figur{1}{kravspec/usecase_diagram}{Usecase diagram for Baby Watch}{kravspec:usecase_diagram}

Usecase diagrammet giver et overblik aktørerne og deres rolle i use cases.

\section{Aktører}

%Indsæt aktørdiagram og aktør beskrivelser

%\figur{Bredden}{Pdfnavn}{Billedtekst}{Label}

\begin{table}[!htbp] \centering
	\begin{tabular}{|p{2.5cm}|p{11.5cm}|}
	\hline
		\textbf{Aktør navn} & \textbf{Beskrivelse} \\\hline
		Babypasser 
		& En person som ønsker at benytte systemet til at 
		  berolige Baby til at falde i søvn samt monitorerer Babys tilstand elektronisk
		\\\hline
		Baby 
		& Spædbarn som monitoreres og beroliges af system
		\\\hline
		Installatør
		& Tekniker, der opsætter systemet (optræder i den ikke-implementerede UC5)
		\\\hline
	\end{tabular}
\end{table}
\newpage
\section{Fully dressed use cases}
\begin{center} \centering \label{kravspec:uc1}
	\begin{longtable}{|p{5cm}|p{9cm}|}  %% Longtable = forsætter på næste side
	\hline
		\multicolumn{2}{|l|}{\textbf{UC1: Igangsæt vugning manuelt}} \\\hline %% HUSK USECASE NAVN
		\endfirsthead
		
		\multicolumn{2}{l}{...fortsat fra forrige side} \\ \hline %% Til LONGTABLE
		\multicolumn{2}{|l|}{\textbf{UC1: Igangsæt vugning manuelt}} \\\hline %%  HUSK USECASE NAVN
		\endhead	
		
		\textbf{Mål}							&At igangsætte vugning af barnevogn manuelt 		\\\hline
		\textbf{Initialisering}				&Babypasser		\\\hline
		\textbf{Aktører og Stakeholders}		&Babypasser(Primær)		\\\hline 
		\textbf{Referencer}					&UC2, UC3, UC4		\\\hline
		\textbf{AASH}						&1		\\\hline
		\textbf{Efterfølgende tilstand}		&UC3: Monitorer baby igangsat		\\\hline
		\textbf{Hovedforløb}					
			&\begin{enumerate}
	
				\item Børnepasser igangsætter manuel vugning ved tryk på "Manuel start"-knap på Baby Watch controller
				
				\item \label{kravspec:uc1_vugning}Systemet starter vugning iht. UC2
				
				\item \label{kravspec:uc1_timeout2}Efter 2 minutter igangsætter systemet passiv monitorering af baby jf. UC3			
								
				\item \label{kravspec:uc1_timeout5}Efter 5 minutter igangsætter systemet automatisk monitorering af baby jf. UC3	
				\newline [Und: \ref{kravspec:uc1_timeout5}.a Alarmerende baby detekteret indenfor 5 min]				
				
			\end{enumerate}
		\\\hline
		\textbf{Undtagelser}
			&\begin{enumerate}[label=\ref{kravspec:uc4_autovugning}.a]
			\item Alarmerende baby detekteret indenfor 5 min
					\begin{itemize}
					
					\item Systemet igangsætter undtagelsestilstand jf. UC4
					
					\end{itemize}
			\end{enumerate}
			
		\\\hline
	\end{longtable} 
\end{center}

%% TIPS:
%% LABEL TIL PUNKT: \label{labelnavn}
%% REFERENCE: \ref{labelnavn}
\newpage
\begin{center} \centering \label{kravspec:uc2}
	\begin{longtable}{|p{5cm}|p{9cm}|}  %% Longtable = forsætter på næste side
	\hline
		\multicolumn{2}{|l|}{\textbf{UC2: USECASENAVN}} \\\hline %% HUSK USECASE NAVN
		\endfirsthead
		
		\multicolumn{2}{l}{...fortsat fra forrige side} \\ \hline %% Til LONGTABLE
		\multicolumn{2}{|l|}{\textbf{UC2: USECASENAVN}} \\\hline %%  HUSK USECASE NAVN
		\endhead	
		
		\textbf{Mål}							&Skriv her		\\\hline
		\textbf{Initialisering}				&Skriv her		\\\hline
		\textbf{Aktører og Stakeholders}		&Skriv her		\\\hline 
		\textbf{Referencer}					&Skriv her		\\\hline
		\textbf{AASH}						&Skriv her		\\\hline
		\textbf{Efterfølgende tilstand}		&Skriv her		\\\hline
		\textbf{Hovedforløb}					
			&\begin{enumerate}
	
				\item Punkt1
				
				\item Punkt2				
				
				\item Punkt3
				
			\end{enumerate}
		\\\hline
		\textbf{Undtagelser}
			&\begin{enumerate}
			
				\item Punkt 1

				\item Punkt 2
				
				\item Punkt 3

			\end{enumerate}
		\\\hline
	\end{longtable} 
\end{center}

%% TIPS:
%% LABEL TIL PUNKT: \label{labelnavn}
%% REFERENCE: \ref{labelnavn}
\newpage
\begin{center} \centering \label{kravspec:uc1}
	\begin{longtable}{|p{5cm}|p{9cm}|}  %% Longtable = forsætter på næste side
	\hline
		\multicolumn{2}{|l|}{\textbf{UC1: Monitorér baby}} \\\hline %% HUSK USECASE NAVN
		\endfirsthead
		
		\multicolumn{2}{l}{...fortsat fra forrige side} \\ \hline %% Til LONGTABLE
		\multicolumn{2}{|l|}{\textbf{UC1: Monitorér baby}} \\\hline %%  HUSK USECASE NAVN
		\endhead	
		
		\textbf{Mål}							&BABYCON tilstand er opdateret		\\\hline
		\textbf{Initialisering}				&Babypasser		\\\hline
		\textbf{Aktører og Stakeholders}		&Baby(Primær), Babypasser(Primær/Sekundær)		\\\hline 
		\textbf{Referencer}					&Ingen		\\\hline
		\textbf{AASH}						&1		\\\hline
		\textbf{Efterfølgende tilstand}		&Monitorér barn fortsættes		\\\hline
		\textbf{Hovedforløb}					
			&\begin{enumerate}
	
				\item Børnepasseren starter system
				
				\item Lyd-detekteringssystem indsamler lyd i et bestemt tidsinterval 				
				
				\item Systemet analyserer indsamlet lydsample 
				
				\item Systemet opdaterer BABYCON tilstand på hjemmeside på baggrund af analyse 				
				
				\item Systemet opdaterer BABYCON tilstand på baggrund af analyse
				
				\item \textit{UC1 Monitorér barn} fortsættes fra punkt 2
				
			\end{enumerate}
		\\\hline
		\textbf{Undtagelser}
			&\begin{enumerate}
			
				\item Punkt 1

				\item Punkt 2
				
				\item Punkt 3

			\end{enumerate}
		\\\hline
	\end{longtable} 
\end{center}

%% TIPS:
%% LABEL TIL PUNKT: \label{labelnavn}
%% REFERENCE: \ref{labelnavn}
\newpage
\begin{center} \centering \label{kravspec:uc4}
	\begin{longtable}{|p{5cm}|p{9cm}|}  %% Longtable = forsætter på næste side
	\hline
		\multicolumn{2}{|l|}{\textbf{UC4: USECASENAVN}} \\\hline %% HUSK USECASE NAVN
		\endfirsthead
		
		\multicolumn{2}{l}{...fortsat fra forrige side} \\ \hline %% Til LONGTABLE
		\multicolumn{2}{|l|}{\textbf{UC4: USECASENAVN}} \\\hline %%  HUSK USECASE NAVN
		\endhead	
		
		\textbf{Mål}							&Skriv her		\\\hline
		\textbf{Initialisering}				&Skriv her		\\\hline
		\textbf{Aktører og Stakeholders}		&Skriv her		\\\hline 
		\textbf{Referencer}					&Skriv her		\\\hline
		\textbf{AASH}						&Skriv her		\\\hline
		\textbf{Efterfølgende tilstand}		&Skriv her		\\\hline
		\textbf{Hovedforløb}					
			&\begin{enumerate}
	
				\item Punkt1
				
				\item Punkt2				
				
				\item Punkt3
				
			\end{enumerate}
		\\\hline
		\textbf{Undtagelser}
			&\begin{enumerate}
			
				\item Punkt 1

				\item Punkt 2
				
				\item Punkt 3

			\end{enumerate}
		\\\hline
	\end{longtable} 
\end{center}

%% TIPS:
%% LABEL TIL PUNKT: \label{labelnavn}
%% REFERENCE: \ref{labelnavn}
\section{Ikke-funktionelle krav}

\subsection*{Mikrofon}
For at kunne opfange et tilstrækkeligt signal til analyse af babyens gråd, skal systemets mikrofon opfylde følgende krav:
\begin{itemize}
\item Mikrofonen skal have en max SPL rating på min. \SI{120}{\dB}.\footnote{FIXME indsæt reference til studie om gråd volumen}
\item Mikrofonen skal have en jævn frekvens respons på maksimalt +/- \SI{5}{\dB} fra \SI{40}{\hertz} til \SI{10}{\kilo\hertz}.
\end{itemize}

\subsection*{Vuggemekanisme}

\textbf{Nivellering}: \label{kravspec:ikke_funk_nivellering}
\begin{itemize}
	\item Vuggesystemet skal kunne nivellere planet, hvorpå babyen ligger, til vandret position indenfor \SI{2}{\degree}.
	\item Barnevognens understel må stå på et plan med op til \SI{5}{\degree} hældning.
	\item Når systemet er tændt, men ikke skal vugge, nivelleres planet, hvorpå babyen ligger, automatisk til vandret.
\end{itemize}

Ved vugning jf. UC2 gennemgår vugningen af barnet.
Systemets vugge mekanisme skal overholde følgende krav for at sikre en blid vugning:
\begin{itemize}
\item Vuggen skal kunne vippe planet, hvorpå babyen ligger, med op til \SI{10}{\degree} i hver retning fra dets vandrette udgangspunkt, med en fejlmargin på \SI{2}{\degree}.
\item Vuggen skal kunne variere frekvensen, hvormed der vugges fra \SI{0}{\hertz} til \SI{2}{\hertz}, med en fejlmargin på \SI{0.2}{\hertz}.
\item Vuggen skal vende tilbage til vandret indenfor en vinkel på \SI{2}{\degree}, når systemet lukkes ned.
\item Vuggen skal have en begrænsning på vinkelfrekvensen ved \SI{50}{\degree\per\second}. \footnote{De valgte grænseværdier for hastighed og accelleration er bestemt på baggrund af en undersøgelse af almindelig vugning, som dokumenteret på \citep{cd} under textit{/andet/teknologiundersoelse\_af\_alm\_vugning.pdf}}
\item Vuggens vinkel acceleration skal være begrænset ved \SI{250}{\degree\per\square\second}. \footnote{Se fodnote til bestemmelse af grænseværdi for vinkelacceleration.}
\end{itemize}

\textbf{Vuggetilstande}: \label{kravspec:ikke_funk_vuggetilstande}
Ved vugning jf. UC2 gennemgår vugningen af barnet en sekvens af tre vuggetilstande med et interval på 2 min.
\begin{enumerate}

\item Vugning foregår med en frekvens på 0,5 Hz og en vinkel-amplitude på \SI{10}{\degree} +/- \SI{2}{\degree}
\item Vugning foregår med en frekvens på 1 Hz og en vinkel-amplitude på \SI{6}{\degree} +/- \SI{2}{\degree}
\item Vugning foregår med en frekvens på 2 Hz og en vinkel-amplitude på \SI{4}{\degree} +/- \SI{2}{\degree}
\end{enumerate}

Udover de tre sekventielle tilstande har Baby Watch; en manuel vuggetilstand der vugger med en frekvens på 0,75 Hz og en vinkel-amplitude på \SI{8}{\degree} +/- \SI{2}{\degree} og en ''nul''-vuggetilstand, der ikke vugger men bare nivellerer planet, hvorpå barnet ligger til \SI{0}{\degree} +/- \SI{2}{\degree}.

\subsection*{Baby status}
For at sikre at vurderingen af babyens status er pålidelig samt rettidigt tilgængelig for brugeren, skal systemet overholde følgende:
\begin{itemize}
\item Systemets BABYCON-statusbar (se illustration nedenfor) skal opdateres mindst hvert 10. sekund.
\item Statushjemmesiden skal være opdateret, senest 5 sekunder efter controlleren har opdateret babystatus.
\item Når BABYCON-statusbaren opdateres til BABYCON1-niveau skal hjemmesiden afspille en alarmlyd.
\item I undtagelsestilstand afsendes der mails med maksimum 20 sekunders interval. 
\end{itemize}

\subsection*{Controller}
Krav til Controllerens advisering og virkemåder:
\begin{itemize}
\item Wi-Fi-LEDen skal tænde maksimum 15 sekunder efter afbrydelse af netværket.
\item Wi-Fi-LEDen skal slukke maksimum 25 sekunder efter re-etablering af netværket.
\end{itemize}
\section{Web-skitser} \label{Ikke-funk:webskitser}

%\figur{Bredden}{Pdfnavn}{Billedtekst}{Label}

De følgende figurer \ref{kracspec:BABYCON_GUI_3} , \ref{kracspec:BABYCON_GUI_2} samt \ref{kracspec:BABYCON_GUI_1} skitser statushjemmesidens udseende. Hjemmesiden viser aktuel tid, samlet tid som barnet har været i ro, et billede af barnet med tilhørende navn og besked, BABYCON skala fra 1-3 og information omkring hvornår hjemmesiden sidst er opdateret.

\figur{1}{kravspec/GUI_BABYCON_3}{BABYCON3}{kracspec:BABYCON_GUI_3}

Figur \ref{kracspec:BABYCON_GUI_3} illustrerer hjemmesiden når BABYCON niveauet er 3, det niveau hvor Babyen er rolig.

\figur{1}{kravspec/GUI_BABYCON_2}{BABYCON2}{kracspec:BABYCON_GUI_2}

Figur \ref{kracspec:BABYCON_GUI_2} illustrerer hjemmesiden når BABYCON niveauet er 2, det niveau hvor Babyen er urolig, men ikke nok til at udløse alarm til Babypasser.

\figur{1}{kravspec/GUI_BABYCON_1}{BABYCON1}{kracspec:BABYCON_GUI_1}

Figur \ref{kracspec:BABYCON_GUI_1} illustrerer hjemmesiden når BABYCON niveauet er 1. BABYCON1 er niveauet hvor Babyen er alarmerende utilfreds. Babypasseren modtager en e-mail og skal selv trøste Baby. På dette niveau skal hjemmesiden afspille en alarmlyd og det BABYCON røde felt skal blinke.

\section{E-mail beskeder}

\subsection*{BabyWatch: Alarm!}

Emne: Baby Watch: Alarm! \newline
Tekst: Baby Watch har registreret en alarmerende baby (Undtagelsestilstand)

\subsection*{BabyWatch: Fejl!}

Emne: Baby Watch: Fejl! \newline
Tekst: Fejl i vuggesystem, afventer genstart.



