% Ikke-funktionelle krav

\section{Ikke-funktionelle krav}

De ikke-funktionelle krav er testet i forbindelse med de forskellige modultests.


%\begin{center}
%\label{accepttest:ikkefunktionellekrav} 
%\begin{longtable}{|p{0,5cm}|p{3,8cm}|p{3,8cm}|p{2,2cm}|p{2,2cm}|} % l for left, c for center, r for right 
%\hline
%\multicolumn{5}{|l|}{\textbf{Ikke-funktionelle krav}} \\ \hline
%\multicolumn{1}{|c|}{} &
%\textbf{Test} &
%\textbf{Forventet \newline Resultat} &
%\textbf{Resultat} &
%\textbf{Godkendt/ \newline Kommentar} \\ \hline 
%\endfirsthead
%
%\multicolumn{5}{l}{...fortsat fra forrige side} \\ \hline 
%\multicolumn{1}{|c|}{} &
%\textbf{Test} &
%\textbf{Forventet \newline Resultat} &
%\textbf{Resultat} &
%\textbf{Godkendt/ \newline Kommentar} \\ \hline 
%\endhead


%%%%% Tabel Opsætning
%
%% &Test
%% &Forventet
%% &Resultat
%% &Godkendt/Kommentar
%
%
%&\multicolumn{4}{|c|}{\textbf{Vuggesystem:}} \\ \hline
%
%\textbf{1}	&Vuggens vinkel måles med et digitalt vaterpas når vuggen er i højeste vugge niveau (aktiveret med manuel vuggestart), og det samlede udsving udregnes ud fra den største og mindste målte vinkel.
%
%			&Det målte vinkel udsving ligger mellem \SI{36}{\degree} og \SI{44}{\degree}.
%			
%			&Skriv her
%			&Skriv her 
%			\\\hline
%			 
%			 
%\textbf{2}	&Vuggen indstilles til at vippe med hhv. 0.5, 1, 1.5, og \SI{2}{\hertz}, og der tælles hvor mange udsving vuggen gør over en periode på 30 sekunder. 
%
%			&Det samlede antal udsving for de 4 optællinger ligger inden for \SI{10}{\percent} af hhv. 15, 30, 45, og 60.
%			
%			&Skriv her
%			&Skriv her
%			\\\hline
%
%
%\textbf{3}  &Vuggen slukkes mens vuggen er nær ved fuldt udsving, og vinklen af vuggen måles når vuggen er faldet til ro. 
%			&Den målte vinkel skal være mellem \SI{-5}{\degree} og \SI{5}{\degree}.
%			&Skriv her
%			&Skriv her
%			\\ \hline
%
%
%\textbf{4}  &FIXME: Test af begrænsning i vinkelfrekvens, og vinkelacceleration er svær at udtænke. TODO: snak med Carl
%			&Skriv her
%			&Skriv her
%			&Skriv her
%			\\ \hline
%			
%			
%&\multicolumn{4}{|c|}{\textbf{Baby status}} \\ \hline
%
%\textbf{1}  &Systemet konfigureres til at skrive til en log ved opdatering af baby status, og får lov at køre i \SI{1}{\minute}. Mens systemet kører overvåges hjemmesiden, og opdateringstidspunkterne noteres. Til sidst sammenholdes loggen med de noterede tider, og forsinkelsen beregnes som differencen fra opdatering af babystatus, til opdatering af hjemmesiden.
%			&Den længste pause mellem to opdateringer af baby status, og den største forsinkelse til opdatering af hjemmesiden er under \SI{5}{\second}.
%			&Skriv her
%			&Skriv her
%			\\ \hline
%			
%\end{longtable}
%\end{center}
%
%
