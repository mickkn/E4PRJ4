\section{Implementering}
\label{Vuggesystem: Implementering} 
Implementeringsafsnittet for vuggesystemet beskriver først hardware implementeringen efterfulgt af software implementeringen til sidst i afsnittet beskrives og diskuteres afgrænsninger af implementeringen.
\subsection{Hardware implementering}
\subsubsection{Motorstyringskreds og endstop kreds}
Motorstyringskredsløbet er det kredsløb i hele systemet som trækker mest effekt, og har derfor været underlagt en række særlige betragtninger for at sikre at kredsløbet ikke forstyrer resten af systemet. Den fysiskke kreds er implementeret på et print sammen med endstop kredsløbet. Layoutet er som følger:

\figur{0.5}{vuggesystem/impl/motorkreds_layout.pdf}{Print layoutet for motorstyringskredsen og endstop kontakterne}{vugsys:motorkreds_layout}

De to kredsløb er placeret på samme print udelukkende for at spare plads. Dette er vurderet problemfrit udfra den batragting at endstop kredsen er stort set ufølsom overefor støj.\\

En mere detaljeret beskrivelse af designbetragningerne for endstop kredsløbet kan findes i EMC rapporten. De to mest centrale ting er afkoblingskondensatorerne (markeret med gult i figur \ref{vugsys:motorkreds_layout}) og brugen af groundplan (makeret med stiplet rødt omrids). \\

Kredsløbets eksterne forbindelser er lavet med Harwin pins til signalforbindelser (sv1 og sv2 i diagrammet), og skrueterminaler til effektforbindelserne (X1 og X2)

\newpage
\subsubsection{Endstopsensor}\label{Vuggesystem: Implementering_ES} 
Endstop sensorene er blevet realiseret ved hjælp af to end stop switches. En endstop switch fungere som en alm. switch der ved aktivering skaber forbindelse mellem to punkter. Figuren under viser en tilsvarende udgave af en end stop switch som dem brugt i projektet.
\figur{0.2}{vuggesystem/impl/ES_SWITCH.pdf}{Generisk end stop switch}{VS_Im_ES_SWITCH}
 Disse end stop switches er placeret således at når barnevognskurven på Baby Watch systemet går ud i en yder vippeposition så aktivers en end stop switch.
\figur{0.5}{vuggesystem/impl/Endstop_Sensor.pdf}{Monteret endstop sensor}{VS_Im_ES_Sensor}
Endstop sensorne er herefter forbundet til endstop kredsløbet som anvist i Vuggesystem Design afsnit \vref{Vuggesystem: Design_ES}. \\
Implementeringen af endstop sensor kredsløbet kan ses i det foregående Motorstyringskreds afsnit på figur \vref{vugsys:motorkreds_layout}
\newpage
\subsubsection{Vuggeudsving sensor}
\label{Vuggesystem: Implementering_VuggeudsvingSensor}
Vuggeudsvingssensoren er implementeret som beskrevet i Vuggesystem Design afsnittet \vref{Vuggesystem: Design_Vugggeudsving_sensor} ved hjælp af en MPU6050 chip med indbygget accelerometer, gyroskop og I2C bus interface. Den er placeret på undersiden af barnevognskurven ud for aksen der vippes henover. Nedenfor kan ses et billede af implementeringen. 
\figur{0.5}{vuggesystem/impl/Vuggeudsving_sensor.pdf}{Monteret Vuggeudsving sensor}{VS_Im_Vuggeudsving_Sensor}
Chippen er delvist skjult bag en beskyttelse. Den gule streg på billedet markerer aksen barnevognskurven og derved chippen vipper henover. 
De parsnoede ledning er for at beskytte I2C kommunikationen mod støj.


\subsubsection{Motor positionssensor}
Motor positionssensoren bliver ikke implementeret i denne iteration af projektet.

\newpage
\subsubsection{Mekanisk vuggesystem}
\label{Vuggesystem: Implementering_MV} 
Det mekaniske vuggesystem er implementeret ud fra skitsen i Vuggesystem Design Mekanisk Vuggesystem afsnit \vref{Vuggesystem: Design_MekVug}. \\
\textit{Motor:}\\
Motoren er placeret under barnevognskurvens tyngde center med en påsat gearing på 1:16 samt en kæde der giver en yderligere gearing på 1:8. Dette betyder at ved en hel vugning af barnevognskurven (fra yder position til yder position) roterer motoren 8 gange.
\figur{0.5}{vuggesystem/impl/Mekanisk_Vuggesystem_Motor.pdf}{Motor til det mekaniske vuggesystem med vist gearing}{VS_Im_Gear}
\figur{0.5}{vuggesystem/impl/Mekanisk_Vuggesystem_Akse.pdf}{Tandhjul påsat motoren til det mekaniske vuggesystem}{VS_Im_Akse}

\textit{Vuggeudsving:}
Som det ses på de tre nedenstående figurer så opfylder vuggesystemet kravspecifikationens ikke-funktionelle krav afsnit \ref{kravspec:ikke_funk_nivellering} om at barnevognen skal kunne vugge med mindst +/- 10 grader. Det implementerede vuggesystem  kan ideelt vippe ud til +/- 24,5 grader. Dette gør at Baby Watch systemet vil være i stand til at stå på en skrånende flade på op til 14,5 grader. Dette bliver ikke afprøvet og tested i denne iteration af projektet. 
\figur{0.35}{vuggesystem/impl/Mekanisk_Vuggesystem_Vandret.pdf}{Baby Watch barnevogn i vandret position}{VS_MV_Vandret}
\figur{0.35}{vuggesystem/impl/Mekanisk_Vuggesystem_Hojre.pdf}{Baby Watch barnevogn i højre yderposition}{VS_MV_Hojre}
\figur{0.35}{vuggesystem/impl/Mekanisk_Vuggesystem_Venstre.pdf}{Baby Watch barnevogn i venstre yderposition}{VS_MV_Venstre}
\textit{Stabilisering af barnevogn}:
Barnevogne har som standard fjedrende led. Disse led er blevet stabiliseret med spændebånd for at gøre barnevognens overføringsfunktion mere lineær. 
\figur{0.35}{vuggesystem/impl/Mekanisk_Vuggesystem_Opspaend.pdf}{Billede af stabiliseret barnevognsled}{VS_MV_Led}

\newpage
\subsection{Software implementering}
\subsubsection{Regulerings MCU} \label{Vuggesystem: R_MCU} 
Regulerings MCU'en er implementeret på et Cypress CY8CKIT-042 PSoC4 Pioneer Kit \citep{website:Cypress}. Denne platform består af en ARM Cortex-M0 CPU forbundet med en simpel FPGA der alt sammen kan programmes og styres med Cypress' egen udviklings suite, PSoC Creator 3 \citep{website:PSoC_Creator}. Dette giver en fleksibel micro controller platform med mulighed for at bruge flere specialiseret embedded hardware FPGA blokke der kan styrer diverse interfaces, GPIO'er, Timer, interrupts mm. \\
På nedenstående Top Design for fra PSoC Creator kan hardware setup'et for Regulerings MCU'en ses. Hver indrammende boks markerer en specialiseret funktionalitet håndteret af en eller flere hardware blokke.
\figur{1}{vuggesystem/impl/SW_ReguleringsMCU_Top_Design.JPG}{Top Design for Vuggesystems Regulerings MCU}{VS_impl_Reg_MCU}
Følgende vil gennemgå funktionaliteten for hver boks fra start fra venstre mod højre:
\begin{center}
    \begin{tabular}{| p{14.5cm} |}
    \hline
    \textit{Sensor input:} \\ \hline
    Denne boks består af en I2C master blok der sørger for et I2C-interface mellem Regulerings MCU'en og Vuggeudsving sensoren. Dette bruges i aflæsningen af sensoren. I dette setup er overføringshastigheden sat til Normal-Mode dvs. med en bitrate på 100 kHz. \\ \hline
    \end{tabular}
\end{center}

\begin{center}
    \begin{tabular}{| p{14.5cm} |}
    \hline
    \textit{Debug:} \\ \hline
    Debug boksen sørger for et UART interface mellem PSoC4 micro controlleren og en computerterminal samt at sætte en GPIO-pin høj. Begge dele er til debugging af Regulerings MCU'en. \\ \hline
    \end{tabular}
\end{center}

\begin{center}
    \begin{tabular}{| p{14.5cm} |}
    \hline
    \textit{Motor styring:} \\ \hline
    Motor styring består af en PWM blok med to PWM outputs der går til henholdvis Mos driver A og Mos driver B i motorstyringskredsen, se Vuggesystem Design Mosfet driver afsnit \vref{Vuggesytem:HW_DESIGN_MOSDRIVER}. Outputtene er inverterede i forhold til hinanden og via PWMblokken er der lavet et deadband mellem de to PWMsignaler. For yderligere beskrivelse af PWM-styresignalet kan afsnit \vref{Vuggesystem:Styresignal} i Vuggesystem Design Styresignal ses. \\ \hline
    \end{tabular}
\end{center}


\begin{center}
    \begin{tabular}{| p{14.5cm} |}
    \hline
    \label{vuggesystem: impl,Loop timing} \textit{Loop timing:} \\ \hline
    Denne funktionalitet bruges til at styre loop timingen for main funktionen i Regulerings MCU'en, også kaldet \verb+vuggeControl+. Boksen indeholder tre hardware blokke, en Clock, en Timer og et Interrupt. Disse sørger tilsammen for at loop timingen for \verb+vuggeControl+ bliver 200 gange i sekundet. \\   \hline
    \end{tabular}
\end{center}

\begin{center}
    \begin{tabular}{| p{14.5cm} |}
    \hline
    \textit{Shutdown switch:} \\ \hline
    Denne GPIO-pin sættes høj (3.3V) hvis forbindelsen til motoren skal lukkes. Dette er ikke implementeret i denne iteration af projektet.  \\ \hline
    \end{tabular}
\end{center}

\begin{center}
    \begin{tabular}{| p{14.5cm} |}
    \hline
    \textit{Controller I2C:} \\ \hline
    Denne boks består af et I2C-Slave interface der sørger for kommunkationen mellem Regulerings MCU'en og Baby Watch Controller. Denne I2C bus forbindelse kører ligeledes med Normal-Mode. \\ \hline
    \end{tabular}
\end{center}

\begin{center}
    \begin{tabular}{| p{14.5cm} |}
    \hline
    \textit{Endstop switch:} \\ \hline
   Endstop switch boksen består af en pulled-up GPIO-pin forbundet med et Interrupt. Denne funktionalitet bruges til Endstopsensorne, afsnit \vref{EndstopS_IBD} i Vuggesystem Systemarkitektur Hardware arkitektur. Ved aktivering af en Endstop sensor trækkes signalet lav og interruptet aktives. \\ \hline
    \end{tabular}
\end{center}

\textbf{Beskrivelse af fysiske output fra PSoC4 Pioneer Kittet:} \\
Følgende table beskriver sammenhængen mellem vuggesystemets Signalbeskrivelse under Vuggesystem System Arkitektur Grænsefladebeskrivelse\vref{signalbeskr_vugge_tabel} og de tildelte fysiske porte i PSoC Creator 3.
\begin{center}
    \begin{tabular}{| c | c | c |}
    \hline
    \textbf{Navn (Signalbeskrivelse)}	& \textbf{Navn (PSoC Creator 3)} & \textbf{Port (PSoC)} \\ \hline
    MotorPWM & PWM A & P2[0] \\ \hline
    MotorPWM & PWM B & P2[1] \\ \hline
    MotorRelae & ShutdownSwitch & P3[4] \\ \hline
    VuggeV & sensorI2C:scl & P0[4] \\ \hline
    VuggeV & sensorI2C:sda & P0[5] \\ \hline
    MotHast & Ikke implementeret & - \\ \hline
    ESSwitch & ESSwitch & P1[5] \\ \hline
    I2CVuggesystem & I2CVuggesystem:scl & P4[0] \\ \hline
    I2CVuggesystem & I2CVuggesystem:sda & P4[1] \\
  	\hline
    \end{tabular}
\end{center}
Til debugging er der tilføjet to ekstra porte. En til at styrer en GPIO-pin og en til et UART Tx output. De er beskrevet på forrige side under hardware boksen \textit{Debug}
\begin{center}
    \begin{tabular}{| c | c |}
    \hline
    \textbf{Navn (PSoC Creator 3)} & \textbf{Port (PSoC)} \\ \hline
    activityDebug & P3[6] \\ \hline
    Tx 1 & P1[4] \\
  	\hline
    \end{tabular}
\end{center}
Portene er tildelt således at der tages hensyn til den interne fysiske kobling i PSoC4 chippen. Dette gøres for at reducerer støjen fra opadliggende porte så signaler såsom I2C ikke bliver påvirket af PWM signalerne eller andre I2C signaler, og GPIO interrupts ikke aktiveres tilfældigt. Nedenstående figur viser den interne kobling i PSoC4 chippen. De brugte porte er markeret med sort.

\figur{1}{vuggesystem/impl/PSOC_INTERNKOBLING.pdf}{Billede af den interne kobling i PSoC4}{vuggesystem:intern_kobling}

Vuggesystemet anvender fire interrupt service rutiner, disse er blevet prioritet for at sikre systemet bedst muligt mod kritiske fejl. De fire ISR'er er som følger med højeste prioritet først:
\begin{itemize}
\item \textbf{Endstop\_isr} er en del af \textit{Endstop switch} hardware boksen\footnotemark  fra Regulerings MCU'ens Top Design. Den reagerer når systemets mekaniske vuggemekanisme har nået en af yderpositionerne og derfor hurtigt skal lukke vuggesystemet ned. Dette medfører at GPIO-pinen går lav (0V) hvilket trigger ISR'en
\item \textbf{I2CVuggesystem\_SCB\_IRQ} foregår som et internt interrupt til \textit{Controller I2C} hardware boksen. Denne ISR har fået anden højeste prioritet da vigtige informationer som f.eks at vuggesystemet skal lukke ned kan blev sendt
\item \textbf{sensorI2C\_SCB\_IRQ} er også et internt interrupt. Den tilhører \textit{Sensor input} hardware boksen. 
\item \textbf{loop\_isr} er et timer-styret interrupt der tilhører \textit{Loop timing} hardware boksen. 
\end{itemize}

\footnotetext{Beskrivelser af hardware boksene kan findes i afsnit \vref{Vuggesystem: R_MCU}}

\figur{0.2}{vuggesystem/impl/VS_ISR_PRIOTERING.pdf}{Billede af interrupt service rutine prioriteringen i PSoC 4'en}{vuggesystem:ISR_PRIORITERING}

\textbf{Beskrivelse af C kode filerne i Regulerings MCU:} \\
Koden er implementeret som beskrevet i applikations modellen fra Vuggesystem Systemarkitektur afsnit \vref{Vuggesystem:SD} og klassediagrammet fra Vuggesystem Design Software design \vref{VuggeSystem_Klassediagram}. \\


\textbf{vuggeControl (main):} \\
Tanken bag vuggesystemets main-funktion \verb+VuggeControl+, er at den under opstart skal initier alle hardware blokkene og deres funktionaliteter igennem initierings kald til de andre  filer i Regulerings MCU. Derefter skal den kører i loop som beskrevet i sekvens diagrammet i afsnit \vref{Vuggesystem:SD}. Loopet kører med en frekvens på 200Hz ved hjælp af \textit{Loop timing} hardware boksen beskrevet i ovenstående afsnit \vref{vuggesystem: impl,Loop timing}. \\ 

\verb+VuggeControl+filen indeholder få hjælpefunktioner som hver især sørger for tjekke systemkritiske informationer, debugging eller nødstop af motoren(Hvilket ikke er fuldt implementeret i denne iteration af projektet). \\

Af systemkritiske informationer kan nævnes; et eksternt interrupt om at en Endstop Sensor er aktiveret og at systemet derfor skal gå i nødstop, at Baby Watch Controller vil slukke vuggesystemet og at regulerings status skal tjekkes for fejl i reguleringen.\\ 

Nødstop-funktionen der lukker vuggesystemet er egentligt tiltænkt at skulle slukke et relæ der styrer strømtilførelsen til motoren men som nævnt tidligere er denne funktionalitet ikke implementeret i denne udgave af projektet. Funktionen sørger istedet for at neutraliserer PWM-signalet der styrer motoren så motoren derigennem ikke får strøm. \\ 

Til at styrer main-loopets 200Hz hastighed bruges et timer-interrupt der sætter et flag højt hver gang et nyt gennemløb af loopet skal køres. I main-loopet kaldes der først førnævnte systemkritiske tjek efterfulgt af en opdatering af de nyeste vuggeudsving- og vuggefrekvensværdier fra Baby Watch Controller, herefter hentes og beregnes barnevognskurvens vinkel relativ til tyngdefeltet. Denne vinkel videresendes til reguleringen som beregner et output som lægges ind til motorstyringens PWM blok. Loopet slutter med at resette flaget så det er klart til et nyt gennemløb. Koden kan ses på \citep{cd} under \verb+vuggeControl.c+ \\

\label{vugsys:impl, i2cKommunikation}\textbf{i2cKommunikation} \\ \verb+i2cKommunikation+ filen sørger for initierer den førnævnte Controller I2C hardware der indeholdte et I2C Slave interface og derefter bruge denne til at opdaterer et I2C register med informationer som skal kommunikeres mellem de to moduler jf. Grænseflade mellem Vuggesystem og Controller i den overordnede Baby Watch Systemarkitektur afsnit \ref{overordnet:i2c_tabel}. Som sikkerhed i kommunikationen er der sat begrænsning på at register 0x04 der indeholder vuggesystemet status ikke kan ændres udfra dvs. fra Baby Watch Controller. Kode ses på: \citep{cd} under \verb+i2cKommunikation.c+ og \verb+i2cKommunikation.h+\\

\label{vugsys:impl, sensFortolk}\textbf{sensorFortolker}  \\
\verb+sensorFortolker+ står for alt kommunikation mellem sensorer (Endstop Sensor og Vuggeudsving sensor) og Regulerings MCU samt beregninger af data fra disse der giver barnevognskurvens nuværende vinkel. Dette gøres, som nævnt under Regulerings MCUs hardware moduler, vha. hjælp af et I2C Master interface og et GPIO styret interrupt. 
\textbf{Sensor konfiguration:} \\
Da den valgte sensorkreds (MPU6050) er i stand til at køre på mange forskellige måder kræves der en indledende opsætning af sensoren før den kan benyttes. Dette gøres ved at skrive til en række registre på enheden gennem I2C bussen \citep{I2C}. Der skrives til følgende registre:
\begin{center}
    \begin{tabular}{| c | c | p{8cm} |}
    \hline
    \textbf{Register}	& \textbf{værdi} & \textbf{kommentar} \\ \hline
    0x19 & 0b00000100 & sætter dataraten til 200Hz \\
    0x1A & 0b00000101 & sætter knækfrekvensen for sensorens digitale lavpasfilter til 10 Hz \\
    0x1A & 0b00000101 & sætter knækfrekvensen for sensorens digitale lavpasfilter til 10 Hz \\
    0x1B & 0b00011000 & gyropskopet indstilles til +/- 2000 grader i sekundet  \\
    0x1C & 0b00001000 & accelerometeret indstilles til +/- 4g \\
    0x1B & 0b00011000 & gyropskopet indstilles til +/- 2000 grader i sekundet  \\
    0x1C & 0b00001000 & accelerometeret indstilles til +/- 4g \\
    0x6B & 0b00000001 & chippens clk sættes til at være x gyroskopets oscilator, da denne er mere præcis \\ \hline
    \end{tabular}
\end{center}
\textbf{Dataindsamling:} \\
MPU6050 chippen genererer en puls på dataInt benet, hvorefter der læses fra følgende I2C registre:
\figur{0.75}{vuggesystem/impl/i2cDataReg.pdf}{udsnit af i2c register oversigt fra datablad}{vugsys:dataReg}
Herefter bitshiftes de høje databytes (\_H) 8 bits til vænstre og adderes med de tilhørende lave databytes.\\
textbf{Databehandling:} \\

Før de indsamlede data fra de to sensorer kan samles udregnes først den estimerede vinkel for hver af de to sensorer hver for sig. Da Accellerometerets data kun kan bruges så længe sensoren ikke bliver udsat for nogen accelleration, benyttes accelerometer målingen kun så længe accellerationsvektoren har en længde næsten lig 1g (tyngdevektoren). For at kunne udregne dette uden at belaste processoren for meget benyttes følgende antagelse:\\
	$\sqrt { { x }^{ 2 }+{ y }^{ 2 } } \sim \quad \left| x \right| +\left| y \right|$ \\
Da denne antagelse er ret upræcis, sættes grænserne lidt bredere end egentlig ønsket, og accellerometer dataene vurderes gældende hvis estimatet af accellerationsvektoren er i intervallet [0.9;1.4]. Dette interval er vurderet på baggrund af følgende måling: (fig \ref{vugsys:forceMag})\\
\figur{1}{vuggesystem/impl/forceMag.pdf}{plot af acc vector længde estimat ved alm vug og rytelser}{vugsys:forceMag}
Her ses et plot af accellerometer vinkelen (blå) og den estimerede længde af accellerationsvektoren (rød) først ved almindelig vugning, og derefter hvor barnevognen rystes, begyndende ved ca 2400 samples. De valgte grænser er markeret med gul.\\
Hvis målingen vurderes gyldig udregnes vinklen ved brug af arcus tangens for Z- og Y-aksens data, hvorved man får en vinkel, som vist herunder:\\
	${ \theta  }_{ acc }=\quad atan2({ a }_{ z },\quad { a }_{ y })\\ \\ hvor:\quad atan2(y,\quad x)\quad =\quad \begin{cases} \begin{matrix} \arctan 
	{ \frac { y }{ x }  } \quad \quad \quad \quad \quad \quad \quad \quad \quad \quad \quad  \\ \arctan { \frac { y }{ x } \quad +\quad \pi  } \quad 
	\quad \quad \quad \quad \quad \quad  \\ \arctan { \frac { y }{ x } \quad -\quad \pi  } \quad \quad \quad \quad \quad \quad \quad  \end{matrix}
	\begin{matrix} x\quad >\quad 0\quad \quad \quad \quad \quad \quad \quad  \\ y\quad \ge \quad 0,\quad x\quad <\quad 0 \\ y\quad \ge \quad 0,\quad 
	x\quad <\quad 0 \end{matrix} \\ \begin{matrix} +\frac { \pi  }{ 2 } \quad \quad \quad \quad \quad \quad \quad \quad \quad \quad \quad \quad \quad  
	\\ -\frac { \pi  }{ 2 } \quad \quad \quad \quad \quad \quad \quad \quad \quad \quad \quad \quad \quad  \\ undefined\quad \quad \quad \quad \quad 
	\quad \quad \quad \quad  \end{matrix}\begin{matrix} y\quad >\quad 0,\quad x\quad =\quad 0 \\ y\quad <\quad 0,\quad x\quad =\quad 0 \\ y\quad 
	=\quad 0,\quad x\quad =\quad 0 \end{matrix} \end{cases}$ \\ \\
Da denne beregning skal køre på psoc'en er der benyttet en fixed point implementering af cordic algoritmens atan2 funktion. For en mere detaljeret beskrivelse af Jim Shima's atan2 algoritme se dspGuru's artikkel \citep{website:atan}. \\
Gyroskopets data omregnes til en vinkel ved at integrere den målte vinkelfrekvens over hver sample perode som vist her:\\
	${ \theta  }_{ gyro }\quad =\quad { \theta  }_{ est\left[ t-1 \right]  }\quad +\quad { Gyro }_{ x }\quad *\quad { T }_{ s }$ \\
${ \theta  }_{ est\left[ t-1 \right]  }$ er den estimerede vinkel fra sidste sample periode. Grunden til at benytte den splejsede måling frem for at gyroskop- og accellerometer data holdes seperat, er at gyroskopets fejl ellers vil akkumulere, hvorved fordelen ved splejsning ville gå tabt.\\
Når de to vinkler er udregnes tages et vægtet gennemsnit af de to. Da gyroskopet giver de pæneste data, og accellerometeret hovedsageligt bruges at at eliminere drift vægtes sensor dataene 9:1, og man får følgende formel:\\
	${ \theta  }_{ est }\quad =\quad 0.9\cdot { \theta  }_{ gyro }\quad +\quad 0.1\cdot { \theta  }_{ acc }$ \\
Hermed er den estimerede vinkel af barnevognen udregnet.

SensorFortolker koden ses på: \citep{cd} under \verb+sensorFortolker.c+ og \verb+sensorFortolker.h+\\

\textbf{regulering} \\
Denne fil sørger for at beregne regulerings outputtet, initierer PWM blokken og styrer denne PWM blok. Reguleringen sørger også for at lave fejltjek på på de vuggeudsvings- og vuggefrekvensværdier fra Baby Watch Controller før de bruges. Dette gøres for at øge datasikkerheden. Reguleringen bruger disse værdier sammen med den nuværende vinkel angivet i et fixed16.16 dataformat til at beregne et output til motorstyringens PWM. 
\\ MERE \\
Kode ses på: \citep{cd} under \verb+regulering.c+ og \verb+regulering.h+\\







\newpage