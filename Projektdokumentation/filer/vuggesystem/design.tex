\section{Design}
\subsection{Hardware design}

\subsubsection{Motorkreds}

\subsubsection{Endstopsensor}

\subsubsection{Vuggeudsving sensor}

\subsubsection{Motor positions sensor}

\subsubsection{Mekanisk vuggesystem}
Nedenståede blokdiagram er en skitsering af den valgte teoretiske model for det mekaniske vuggesystem
\figur{1}{vuggesystem/design/teoretisk_blokdiagram_vuggeregulering.JPG}{Skitse af teoretiske model for mekanisk vuggesystem}{VuggeSystem_teoretiskmodel}


Beskrivelses af modellen:
\begin{itemize}
\item \textbf{R(s)} skal ses som ankerspændingen der tilføres systemet
\item \textbf{Ankermodstand} og \textbf{Motorkonstant} udgør DC-motoren
\item \textbf{Inerti} er vuggesystemets mekaniske rotationsmoment
\item \textbf{Væskedæmpning} er friktionen der stammer fra aksen der vugges over i systemet
\item \textbf{Det Inverse pendul} er et ulineært element der stammer fra at barnevognskurvens masse ligger højere end aksen der vugges over. Dette element er ustabilt af natur og kræver påvirkning for at holdes stabilt. Dette er især gældende forstørre udfaldsvinkler, men da niveauet der skal vugges over maximalt udgør ±10 grader i forhold til tyngdefeltet kan dette element gøres tilnærmelsesvist lineært
\item Fra højre mod venstre skal punktet efter \textbf{Interator2} ses som hastigheden af barnevognens rotation og punktet efter \textbf{Integrator1} skal ses som accelerationen af barnevognens rotation
\item \textbf{C(s)} er vuggeudsvinget/vippevinklen for barnevognskurven i forhold til tyngdefeltet

\end{itemize}

\newpage
\subsection{Software design}
Softwaredesignet for Vuggesystemet udarbejdes vha. et klassediagram med tilhørende funktionsbeskrivelser lavet på baggrund af applikationsmodellen fra Vuggesystemets software systemarkitektur jf. afsnit \ref{Vuggesystem: Software}

\subsubsection*{Klassediagram}
\figur{1}{vuggesystem/sysark/reguleringsMCU_klassediagram.pdf}{UML klassediagram for Regulerings MCU}{VuggeSystem_SD}

\subsubsection*{Funktionsbeskrivelser}
{\centering
\textbf{vuggeControl}\par
}
\textbf{Ansvar:} Main klasse, styrer de andre klasser i reguleringsMCU'en. \

\begin{center}
    \begin{tabular}{ | l | p{11,8cm} |}
    \hline
    \textbf{Funktion}	& \verb+void reguleringsStatus(uint8 status) +						\\ \hline
    \textbf{Parametre} 	& uint8 status angiver status		\\ \hline
    \textbf{Returværdi}	& Ingen 								\\ \hline
    \textbf{Beskrivelse}	& Tjekker reguleringsklassens driftstatus og i tilfælde af fejl videre sendes denne fejl til i2cKommunikationsklassen, med fejlindikator værdierne 0b1000000 for error og 0b01000000 for stall.		\\ \hline
    \end{tabular}
\end{center}

\begin{center}
    \begin{tabular}{ | l | p{11,8cm} |}
    \hline
    \textbf{Funktion}	& \verb+void lukSystem() +						\\ \hline
    \textbf{Parametre} 	& Ingen		\\ \hline
    \textbf{Returværdi}	& Ingen 								\\ \hline
    \textbf{Beskrivelse}	& Lukker strømforsyningen til det mekaniske vuggesystem og videre sender fejlmeddelse; 0b0001000 til i2cKommunikationsklassen. Kaldes i tilfælde hvor det mekaniske vuggesystem skal lukkes		\\ \hline
    \end{tabular}
\end{center}

\begin{center}
    \begin{tabular}{ | l | p{11,8cm} |}
    \hline
    \textbf{Funktion}	& \verb+void checkEndstop(BOOL) +						\\ \hline
    \textbf{Parametre} 	& BOOL: indikere det mekaniske vuggesystem har nået sin maximale mekaniske vuggegrænse		\\ \hline
    \textbf{Returværdi}	& Ingen 								\\ \hline
    \textbf{Beskrivelse}	& Kaldes i tilfælde hvor endstop værdi i sensorFortolkerklassen er TRUE. Funktionen kalder lukSystem() og videre sender fejlmeddelse; 0b00100000 til i2cKommunikationsklassen.		\\ \hline
    \end{tabular}
\end{center}


{\centering
\textbf{i2cKommunikation}\par
}
\textbf{Ansvar:} Står for I2C kommunikationen til mellem vuggesystemet og Baby Watchs Controller. \

\begin{center}
    \begin{tabular}{ | l | p{11,8cm} |}
    \hline
    \textbf{Funktion}	& \verb+void initI2C() +									\\ \hline
    \textbf{Parametre} 	& Ingen														\\ \hline
    \textbf{Returværdi}	& Ingen 													\\ \hline
    \textbf{Beskrivelse}	& Initiere I2C-kommunikationen til operationel funktion	\\ \hline
    \end{tabular}
\end{center}

\begin{center}
    \begin{tabular}{ | l | p{11,8cm} |}
    \hline
    \textbf{Funktion}	& \verb+uint8 sendConKom() +								\\ \hline
    \textbf{Parametre} 	& Ingen														\\ \hline
    \textbf{Returværdi}	& Returner pointer til i2cKommunikationsklassens TxBuffer											\\ \hline
    \textbf{Beskrivelse}	& Returner pointer fra i2cKommunikationsklassens TxBuffer til vuggeControlklassen	\\ \hline
    \end{tabular}
\end{center}

{\centering
\textbf{regulering}\par
}
\textbf{Ansvar:} Står for regulering og styring af det mekaniske vuggesystem ud fra beregner baseret på sensorværdier. \

\begin{center}
    \begin{tabular}{ | l | p{11,8cm} |}
    \hline
    \textbf{Funktion}	& \verb+void initRegulering() +								\\ \hline
    \textbf{Parametre} 	& Ingen														\\ \hline
    \textbf{Returværdi}	& Ingen														\\ \hline
    \textbf{Beskrivelse}	& Initerer reguleringen til operationel status samt starter reguleringen	\\ \hline
    \end{tabular}
\end{center}

\begin{center}
    \begin{tabular}{ | l | p{11,8cm} |}
    \hline
    \textbf{Funktion}	& \verb+void regParameter(uint8 vuggefrekvens, uint8 vuggeudsving) +								\\ \hline
    \textbf{Parametre} 	& uint8 vuggefrekvens indeholder en binærværdi fra 0-255 der udfra en skala svare til en vuggefrekvens jf. Kommunikationsprotokol \ref{overordnet:Kommunikationsprotokol}, uint8 vuggeudsving indeholder en binærværdi fra 0-255 der ud fra en skala svare til et vuggeudsving jf. Kommunikationsprotokol \ref{overordnet:Kommunikationsprotokol}						\\ \hline
    \textbf{Returværdi}	& Ingen														\\ \hline
    \textbf{Beskrivelse}	& Gemmer ny vuggefrekvens og vuggeudsving til at regulerer udfra	\\ \hline
    \end{tabular}
\end{center}


\begin{center}
    \begin{tabular}{ | l | p{11,8cm} |}
    \hline
    \textbf{Funktion}	& \verb+void beregnRegulering() +								\\ \hline
    \textbf{Parametre} 	& Ingen						\\ \hline
    \textbf{Returværdi}	& Ingen														\\ \hline
    \textbf{Beskrivelse}	& Beregner reguleringen for vuggesystem via angivet værdier vuggefrekvens, vuggeudsving og sensordata samt overføringsfunktionen for vuggesystemet og opdaterer fixedPointReg med en værdi X? ud fra et 16.16 fixed-point real number tabel opslag(IRT undervisning først)	\\ \hline
    \end{tabular}
\end{center}

\begin{center}
    \begin{tabular}{ | l | p{11,8cm} |}
    \hline
    \textbf{Funktion}	& \verb+void PWMStyring(int fixedPointReg) +				\\ \hline
    \textbf{Parametre} 	& int fixedPointReg: Værdi til at styrer duty-cyclen for PWM modulet 		\\ \hline
    \textbf{Returværdi}	& Ingen														\\ \hline
    \textbf{Beskrivelse}	& Styrer udadgående PWM udfra fixedPointReg værdi X? (IRT uv)	\\ \hline
    \end{tabular}
\end{center}

{\centering
\textbf{sensorFortolker}\par
}
\textbf{Ansvar:} Står for grænsefladen ud til Vuggeudsving sensor, Endstop sensors og Motor positionssensoren. Udfra Vuggeudsving sensorens parametre beregnes desuden den absolutte vinkel af barnevognskurven i forhold til tyngdefeltet. \

\begin{center}
    \begin{tabular}{ | l | p{11,8cm} |}
    \hline
    \textbf{Funktion}	& \verb+void initSensor() +				\\ \hline
    \textbf{Parametre} 	& Ingen							 		\\ \hline
    \textbf{Returværdi}	& Ingen									\\ \hline
    \textbf{Beskrivelse}	& Initiere alt sensor-kommunikation til operationel funktion	\\ \hline
    \end{tabular}
\end{center}

\begin{center}
    \begin{tabular}{ | l | p{11,8cm} |}
    \hline
    \textbf{Funktion}	& \verb+void getMotorPosSens() +				\\ \hline
    \textbf{Parametre} 	& Ingen							 		\\ \hline
    \textbf{Returværdi}	& Ingen									\\ \hline
    \textbf{Beskrivelse}	& Opdaterer sensorData arrays plads 0 med den nyeste motor position angivet i værdi fra 0-255	\\ \hline
    \end{tabular}
\end{center}

\begin{center}
    \begin{tabular}{ | l | p{11,8cm} |}
    \hline
    \textbf{Funktion}	& \verb+void getVuggeUdsvingSens() +				\\ \hline
    \textbf{Parametre} 	& Ingen							 		\\ \hline
    \textbf{Returværdi}	& Ingen									\\ \hline
    \textbf{Beskrivelse}	& Opdaterer vuggeudsvingArray plads 0-5 med de nyeste gyroskop- og accellerometersensor målinger angivet i værdier fra 0-255	\\ \hline
    \end{tabular}
\end{center}

\begin{center}
    \begin{tabular}{ | l | p{11,8cm} |}
    \hline
    \textbf{Funktion}	& \verb+void calcAbsVuggeUdsving(char* vuggeudsvingArrayPtr) +				\\ \hline
    \textbf{Parametre} 	& char* vuggeudsvingArrayPtr: Pointer der peger ind i vuggeudsvingArray							 		\\ \hline
    \textbf{Returværdi}	& Ingen									\\ \hline
    \textbf{Beskrivelse}	& Udfra accellerometer- og gyroskopmålingerne i vuggeudsvingArray beregnes og opdater sensorData arrays plads 1 med den absolutte vinkel for barnevognskurven angivet i værdi X?	\\ \hline
    \end{tabular}
\end{center}

\begin{center}
    \begin{tabular}{ | l | p{11,8cm} |}
    \hline
    \textbf{Funktion}	& \verb+int* getSensorData() +				\\ \hline
    \textbf{Parametre} 	& Ingen							 		\\ \hline
    \textbf{Returværdi}	& int*: sfSensorDataPtr der peger ind i sensorData arrayet									\\ \hline
    \textbf{Beskrivelse}	& Sender pointer sfSensorDataPtr der peger ind i sensorData arrayet til klassen regulering	\\ \hline
    \end{tabular}
\end{center}

\begin{center}
    \begin{tabular}{ | l | p{11,8cm} |}
    \hline
    \textbf{Funktion}	& \verb+void endstopGPIO(BOOL)) +				\\ \hline
    \textbf{Parametre} 	& Ingen							 		\\ \hline
    \textbf{Returværdi}	& BOOL: TRUE ved aktiveret endstop						\\ \hline
    \textbf{Beskrivelse}	& Ved ændring fra 1 til 0 i endstopsensorGPIOpin og opdateres endstop og checkEndstop(BOOL) funktionen kaldes	\\ \hline
    \end{tabular}
\end{center}

