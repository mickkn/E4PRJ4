\section{Design}
\subsection{Hardware design}

\subsubsection{Motorstyringskreds}
I denne sektion beskrives det kredsløb der benyttes til at styre ankerspændingen for vuggesystemets motor. Kredsløbet modtager et PWM signal fra controlleren som styrer ankerspændingen. Udover styresignalet og motor terminalerne, er kredsløbet koblet til systemets 12V forsyning(VCC i diagram) hvorfra motoren forsynes, og systemets 3.3V forsyning som leverer strøm til de logiske kredsløb.\\

\figur{0.2}{vuggesystem/design/stik.pdf}{Forbindelser til motorkreds}{motor_kreds_forb}

\subsubsection*{H-bro}
\label{Vuggesytem:HW_DESIGN_HBRO} 
Motoren styres med en H-bro implementeret med 4 N-ch power MOSFETs, som vist herunder:

\figur{1}{vuggesystem/design/h_bro.pdf}{N-ch enh. Power MOSFET H-bro}{h_bro_kreds}

De fire N-channel er koblet med en lille modstand foran gaten, og en pull-down modstand fra gaten til source. Den lille modstand sikrer at der ikke opstår or meget ring som resultat af seriekobling af ledningens induktans og gatens kapacitet, og pull-down modstanden sikrer at gaten bliver trukket lav hvis MOS driveren skulle fejle, således at transistoren afbryder. \\ 
Signalerne M\_AH, M\_AL, M\_BH, og M\_BL udgør således H-broens styresignaler til hhv. den høje og lave gate i A og B siden af broen. Terminalerne MOT\_A og MOT\_B går til motorens terminaler. Hvis A siden af broen er tændt påtrykkes en spænding med MOT\_A som den positive terminal, og omvendt for B siden. \\
Kondensatorerne C3 og C4 er placeret så tæt som muligt på H-broens positive og negative forsyning, og hjælper til at afkoble motoren, således at der ikke introduceres for store forstyrrelser på forsyningen. Dimensionering af disse er beskrevet nærmere i EMC afsnittet (LAV ORDENTLIG REF).\\

\subsubsection*{MOSFET driver}
\label{Vuggesytem:HW_DESIGN_MOSDRIVER} 
H-broens MOSFETs bliver drevet ved hjælp af 2 dobbelt sidede MOSFET drivere, som er koblet som vist herunder:

\figur{0.75}{vuggesystem/design/mos_driver.pdf}{Dual MOSFET driver, A siden}{MOS_driver}

Driveren kan drive en MOSFET der sidder til ground, og en MOSFET der sidder fra VCC, og de to drivere er koblet ens til hhv. A siden og B siden. signalet til den lave MOSFET, og det interne logik drives fra VCC, mens signalet til den høje MOSFET drives fra en bootstrap kreds bestående af C1 og D1. Kredsen fungerer ved at C1 oplades til en diode spænding under VCC når den lave MOSFET fra den anden side af broen er tændt. Når den lave MOSFET slukkes vil potentialet ved VB således ligge ca. 11,3 V over potentialet ved source benet af den høje MOSFET, og der er således den nødvændige spænding for at trække gaten i mætning. \\
Driverens input er et digitalt input, med en fast overgang fra lav til høj mellem 0.8 og 3 V, og kan således styres med 3.3V logik uafhængigt af hvilken forsyningsspænding driveren er koblet med. \\
Kondensatoren C2 hjælper til at afkoble driveren fra forsyningen. \\

\subsubsection*{Styresignal}
Broen styres, som nævn ovenfor, fra et PWM signal. Dette gøres ved at A sidens MOSFETs modtager PWM signalet direkte, mens B siden modtager det inverterede signal. Benyttes halvtreds procent duty cycle vil de to sidder af broen være tænd i lige lang tid, og middelspændingen over motoren vil således blive 0V. Hæves duty cyclen til over 50 procent opnås en gradvis større middelspænding med MOT\_A som den positive terminal, og omvændt ved duty cycles under halvtreds procent. \\
Da MOSFET transistorne tager en hvis tid om at slukke, skal der introduceres en dødtid i styresignalerne, således at signalerne alle går kortvarigt til 0 før de skifter. Dette sikrer at to transistorer fra hver deres side af broen ikke er tændt samtidig, hvilket ville resultere i en kortslutning fra VCC til GND gennem de to transistorer. 

\figur{0.75}{vuggesystem/design/deadband_eksempel.pdf}{Deadband eksempel}{deadband_eksempel}

Dødbåndet skabes ved at introducere en forsinkelse på den stigende kant af både signalet og det inverterede signal, som det ses på figuren herover. Forsinkelsens størrelse skal være tilpasset, således at den har samme længde som den tid det tager transistorne i broen at slukke helt. På denne måde sikres det at transistorne ikke er tændt samtidig, men også at der spildes mindst muligt af PWM perioden på skift.
På ovenstående eksempel ses det også at duty cyclen ikke ændres, idet de to signaler stadig er tændt i en lige stor del af perioden.\\
Denne forsinkelse på den stigende kant introduceres vha et RC led og en and gate, som vist nedenfor.

\figur{0.5}{vuggesystem/design/deadband_kreds.pdf}{Deadband kredsløb}{deadband_kreds}

Den første inverter fungerer som en buffer for at sikre at indgangssignalet møder en høj impedans, mens den anden inverter skaber et ikke inverteret udtag til PWM signalet. Modstanden er dimensioneret således at der højst trækkes en milliampere fra inverteren, og kondensatoren kan så dimensioneres til at give en passende tidsforsinkelse. Da MOSFET driveren kortslutter gate signalet til source benet når transistoren skal afbrydes, kan slukketiden estimeres ved at anskue transistoren og dens formodstand, som et RC led der skal aflades for at slukke transistoren. Gate kapaciteten oplyses i databladet som 370pF, og formodstanden er på 10 Ohm, og man får således en tidskonstant på 3.7*10\^-9. Transistoren anskues som afbrudt efter 5 * tau = 18.5 ns, hvilket med den valgte modstand på 3.3kOhm giver en kondensator værdi på 1.12 pF hvilket rundes op til 1.2pF. \\
Til sidst er de ubrugte logiske elementer er koblet som følger:

%TEX HJÆLP! nedenstående figur er udkommenteret fordi jeg ikke kan få den til at fungere.
%\figur{0.5}{vuggesystem/design/ubrugte_elementer.pdf}{Ubrugte logiske elementer}{ub_logik_elem}

\subsubsection{strømforbrug}
Det dominerende strømforbrug i systemet stammer fra vuggesystemets motor. Denne strøm trækkes direkte fra systemets batteri, og har derfor ikke indflydelse på dimensioneringen af systemets reguleringskreds. Tilgengæld er den vigtig for at kunne vurdere hvor stort et batteri der skal til for at drive systemet en given tid.\\ 
Et tidligt skøn over motorens strømforbrug er gjort ved at sætte et manuelt styret PWM signal på motor kredsen, og så vugge barnevognen med vægt i, så godt som muligt ved manuel regulering. Herunder ses strømmålingen for 60 sekunders test. Målingen er gjort med en /10 probe på spændingsfaldet over en 0.1 Ohm's modstand, og den målte spænding svarer således direkte til strømmen.\\

%TEX HJÆLP! nedenstående figur er udkommenteret fordi jeg ikke kan få den til at fungere.
%\figur{0.75}{vuggesystem/design/stroemforbrug.pdf}{Test af strømforbrug for motor}{strøm_graf}

\subsubsection{Endstopsensor}\label{Vuggesystem: Design_ES} 
Endstopsensor kredsløbet er designet som et pull-up kredsløb hvor to mekaniske switches kan parallelopkobles. Designet er vist på figuren nedenfor.
\figur{0.25}{vuggesystem/design/Endstopsensor_design.pdf}{Endstop sensor schematic}{ES_schematic}
Kredsløbet opererer med 3.3V og ved aktivering af en af endstop sensorene trækkes signalet ned til 0V (GND).
\subsubsection{Vuggeudsving sensor}
\label{Vuggesystem: Design_Vugggeudsving_sensor} 

\figur{0.25}{vuggesystem/design/VuggeudsvingSensor_mpu6050.pdf}{MPU6050}{MP6050}

\subsubsection{Motor positions sensor}
Motor positions sensoren er ikke med i denne iteration af projektet.
\subsubsection{Mekanisk vuggesystem}\label{Vuggesystem: Design_MekVug} 
Det mekaniske vuggesystem er designet ud fra følgende skitse: \\



Nedenståede blokdiagram er en skitsering af den valgte teoretiske model for det mekaniske vuggesystem
\figur{1}{vuggesystem/design/teoretisk_blokdiagram_vuggeregulering.JPG}{Skitse af teoretiske model for mekanisk vuggesystem}{VuggeSystem_teoretiskmodel}


Beskrivelses af modellen:
\begin{itemize}
\item \textbf{R(s)} skal ses som ankerspændingen der tilføres systemet
\item \textbf{Ankermodstand} og \textbf{Motorkonstant} udgør DC-motoren
\item \textbf{Inerti} er vuggesystemets mekaniske rotationsmoment
\item \textbf{Væskedæmpning} er friktionen der stammer fra aksen der vugges over i systemet
\item \textbf{Det Inverse pendul} er et ulineært element der stammer fra at barnevognskurvens masse ligger højere end aksen der vugges over. Dette element er ustabilt af natur og kræver påvirkning for at holdes stabilt. Dette er især gældende forstørre udfaldsvinkler, men da niveauet der skal vugges over maximalt udgør ±10 grader i forhold til tyngdefeltet kan dette element gøres tilnærmelsesvist lineært
\item Fra højre mod venstre skal punktet efter \textbf{Interator2} ses som hastigheden af barnevognens rotation og punktet efter \textbf{Integrator1} skal ses som accelerationen af barnevognens rotation
\item \textbf{C(s)} er vuggeudsvinget/vippevinklen for barnevognskurven i forhold til tyngdefeltet

\end{itemize}

\newpage
\subsection{Software design}
Softwaredesignet for Vuggesystemet udarbejdes vha. et klassediagram med tilhørende funktionsbeskrivelser lavet på baggrund af applikationsmodellen fra Vuggesystemets software systemarkitektur jf. afsnit \ref{Vuggesystem: Software}

\subsubsection*{Klassediagram}
\figur{1}{vuggesystem/sysark/reguleringsMCU_klassediagram.pdf}{UML klassediagram for Regulerings MCU}{VuggeSystem_SD}

\subsubsection*{Funktionsbeskrivelser}
{\centering
\textbf{vuggeControl}\par
}
\textbf{Ansvar:} Main klasse, styrer de andre klasser i reguleringsMCU'en. \

\begin{center}
    \begin{tabular}{ | l | p{11,8cm} |}
    \hline
    \textbf{Funktion}	& \verb+void initVuggesystem() +						\\ \hline
    \textbf{Parametre} 	& Ingen		\\ \hline
    \textbf{Returværdi}	& Ingen 								\\ \hline
    \textbf{Beskrivelse}	& Kalder funktionerne \verb+void initI2C() +, \verb+void initSensor() + og \verb+void initRegulering() + samt initiere i2cPtr vha. \verb+sendConKom()+ og reguleringStatusPtr vha. \verb+getReguleringsStatus() +		\\ \hline
    \end{tabular}
\end{center}

\begin{center}
    \begin{tabular}{ | l | p{11,8cm} |}
    \hline
    \textbf{Funktion}	& \verb+void reguleringsStatus() +						\\ \hline
    \textbf{Parametre} 	& Ingen		\\ \hline
    \textbf{Returværdi}	& Ingen 								\\ \hline
    \textbf{Beskrivelse}	& Tjekker reguleringsklassens driftstatus og i tilfælde af fejl videresendes denne fejl til i2cKommunikationsklassen, med fejlindikator værdierne 0b1000000 for error og 0b11000000 for stall. Herefter kaldes lukSystem()		\\ \hline
    \end{tabular}
\end{center}

\begin{center}
    \begin{tabular}{ | l | p{11,8cm} |}
    \hline
    \textbf{Funktion}	& \verb+void lukSystem() +						\\ \hline
    \textbf{Parametre} 	& Ingen		\\ \hline
    \textbf{Returværdi}	& Ingen 								\\ \hline
    \textbf{Beskrivelse}	& Lukker strømforsyningen til det mekaniske vuggesystem og videresender fejlmeddelse; 0b1001000 til i2cKommunikationsklassen. Kaldes i tilfælde hvor det mekaniske vuggesystem skal lukkes		\\ \hline
    \end{tabular}
\end{center}

\begin{center}
    \begin{tabular}{ | l | p{11,8cm} |}
    \hline
    \textbf{Funktion}	& \verb+void checkEndstop() +						\\ \hline
    \textbf{Parametre} 	& Ingen		\\ \hline
    \textbf{Returværdi}	& Ingen 								\\ \hline
    \textbf{Beskrivelse}	& Tjekker om endstop værdi i sensorFortolkerklassen er TRUE ved at kalde funktionen \verb+getEndstopGPIO()+. Ved TRUE kalder funktionen lukSystem() og videresender fejlmeddelse; 0b10100000 til i2cKommunikationsklassen.		\\ \hline
    \end{tabular}
\end{center}


{\centering
\textbf{i2cKommunikation}\par
}
\textbf{Ansvar:} Står for I2C kommunikationen til mellem vuggesystemet og Baby Watchs Controller. \

\begin{center}
    \begin{tabular}{ | l | p{11,8cm} |}
    \hline
    \textbf{Funktion}	& \verb+void initI2C() +									\\ \hline
    \textbf{Parametre} 	& Ingen														\\ \hline
    \textbf{Returværdi}	& Ingen 													\\ \hline
    \textbf{Beskrivelse}	& Initiere I2C-kommunikationen til operationel funktion	\\ \hline
    \end{tabular}
\end{center}

\begin{center}
    \begin{tabular}{ | l | p{11,8cm} |}
    \hline
    \textbf{Funktion}	& \verb+uint8 sendConKom() +								\\ \hline
    \textbf{Parametre} 	& Ingen														\\ \hline
    \textbf{Returværdi}	& Returner pointer til i2cKommunikationsklassens I2CBuf											\\ \hline
    \textbf{Beskrivelse}	& Returner pointer fra i2cKommunikationsklassens I2CBuf til vuggeControlklassen	\\ \hline
    \end{tabular}
\end{center}

{\centering
\textbf{regulering}\par
}
\textbf{Ansvar:} Står for regulering og styring af det mekaniske vuggesystem ud fra beregner baseret på sensorværdier. \

\begin{center}
    \begin{tabular}{ | l | p{11,8cm} |}
    \hline
    \textbf{Funktion}	& \verb+void initRegulering() +								\\ \hline
    \textbf{Parametre} 	& Ingen														\\ \hline
    \textbf{Returværdi}	& Ingen														\\ \hline
    \textbf{Beskrivelse}	& Initerer reguleringen til operationel status samt starter reguleringen	\\ \hline
    \end{tabular}
\end{center}

\begin{center}
    \begin{tabular}{ | l | p{11,8cm} |}
    \hline
    \textbf{Funktion}	& \verb+void updateRegParameter(uint8 vuggefrek, uint8 vuggeud) +								\\ \hline
    \textbf{Parametre} 	& uint8 vuggefrekvens indeholder en binærværdi fra 0-255 der udfra en skala svare til en vuggefrekvens jf. Kommunikationsprotokol \ref{overordnet:Kommunikationsprotokol}, uint8 vuggeudsving indeholder en binærværdi fra 0-255 der ud fra en skala svare til et vuggeudsving jf. Kommunikationsprotokol \ref{overordnet:Kommunikationsprotokol}						\\ \hline
    \textbf{Returværdi}	& Ingen														\\ \hline
    \textbf{Beskrivelse}	& Gemmer og validere ny vuggefrekvens og vuggeudsving til at regulerer udfra. 	\\ \hline
    \end{tabular}
\end{center}


\begin{center}
    \begin{tabular}{ | l | p{11,8cm} |}
    \hline
    \textbf{Funktion}	& \verb+double beregnRegulering(double currentAngle) +								\\ \hline
    \textbf{Parametre} 	& double currentAngle					\\ \hline
    \textbf{Returværdi}	& double													\\ \hline
    \textbf{Beskrivelse}	& Beregner reguleringen for vuggesystem via angivet værdier vuggefrekvens, vuggeudsving og parameter currentAngle. Returnere en double med en værdi ud fra et 16.16 fixed-point nummer som svare til et output fra en overføringsfunktion "C(t)"	\\ \hline
    \end{tabular}
\end{center}

\begin{center}
    \begin{tabular}{ | l | p{11,8cm} |}
    \hline
    \textbf{Funktion}	& \verb+void PWMStyring(double C) +				\\ \hline
    \textbf{Parametre} 	& double C 		\\ \hline
    \textbf{Returværdi}	& Ingen														\\ \hline
    \textbf{Beskrivelse}	& Styrer udadgående PWM udfra fixedPointReg værdi C (En output værdi fra overføringfunktionen for vuggesystemet "C(t)")	\\ \hline
    \end{tabular}
\end{center}


\begin{center}
    \begin{tabular}{ | l | p{11,8cm} |}
    \hline
    \textbf{Funktion}	& \verb+int* getReguleringsStatus() +				\\ \hline
    \textbf{Parametre} 	& Ingen 		\\ \hline
    \textbf{Returværdi}	& Int*														\\ \hline
    \textbf{Beskrivelse}	& Returnere pointer fra \verb+int reguleringsStatus + 	\\ \hline
    \end{tabular}
\end{center}

{\centering
\textbf{sensorFortolker}\par
}
\textbf{Ansvar:} Står for grænsefladen ud til Vuggeudsving sensor, Endstop sensors og Motor positionssensoren. Udfra Vuggeudsving sensorens parametre beregnes desuden den absolutte vinkel af barnevognskurven i forhold til tyngdefeltet. \

\begin{center}
    \begin{tabular}{ | l | p{11,8cm} |}
    \hline
    \textbf{Funktion}	& \verb+void initSensor() +				\\ \hline
    \textbf{Parametre} 	& Ingen							 		\\ \hline
    \textbf{Returværdi}	& Ingen									\\ \hline
    \textbf{Beskrivelse}	& Initiere alt sensor-kommunikation til operationel funktion	\\ \hline
    \end{tabular}
\end{center}

\begin{center}
    \begin{tabular}{ | l | p{11,8cm} |}
    \hline
    \textbf{Funktion}	& \verb+int getMotorPosSens() +				\\ \hline
    \textbf{Parametre} 	& Ingen							 		\\ \hline
    \textbf{Returværdi}	& int									\\ \hline
    \textbf{Beskrivelse}	& Returner sensor data fra Motorpositions sensoren angivet i værdi fra -256 til 255	\\ \hline
    \end{tabular}
\end{center}

\begin{center}
    \begin{tabular}{ | l | p{11,8cm} |}
    \hline
    \textbf{Funktion}	& \verb+double getVuggeUdsvingSens() +				\\ \hline
    \textbf{Parametre} 	& Ingen							 		\\ \hline
    \textbf{Returværdi}	& double								\\ \hline
    \textbf{Beskrivelse}	& Returnere en fix16.16 double med den nyeste vinkel beregnet udfra gyroskop- og accellerometersensor måling	\\ \hline
    \end{tabular}
\end{center}




\begin{center}
    \begin{tabular}{ | l | p{11,8cm} |}
    \hline
    \textbf{Funktion}	& \verb+void endstopGPIO() +				\\ \hline
    \textbf{Parametre} 	& Ingen							 		\\ \hline
    \textbf{Returværdi}	& Ingen						\\ \hline
    \textbf{Beskrivelse}	& ISR funktion som sætter flaget \verb+int endstop+ til 1	\\ \hline
    \end{tabular}
\end{center}

\begin{center}
    \begin{tabular}{ | l | p{11,8cm} |}
    \hline
    \textbf{Funktion}	& \verb+bool getEndstopGPIO() +				\\ \hline
    \textbf{Parametre} 	& Ingen							 		\\ \hline
    \textbf{Returværdi}	& int*									\\ \hline
    \textbf{Beskrivelse}	& Returner bool fra \verb+ int endstop+ til klassen vuggeControl	\\ \hline
    \end{tabular}
\end{center}