\section{Systemarkitektur}

I dette afsnit beskrives systemarkitekturen for vuggesystemet.

\subsection*{Overordnet virkemåde}
Vuggesystemet fungerer overordnet som følger:
\begin{itemize}
	\item Vuggesystemets Regulerings MCU står for at vugge barnevognens kurv med en bestemt vuggefrekvens og et bestemt vinkeludsving. Dette sker på baggrund af værdier for disse modtaget fra Controller
	\item Regulerings MCU'en regulerer vuggesystemet således at vugningen altid foregår ud fra en vandret akse samt at de modtagne værdier for vuggefrekvensen og vinkeludsvinget overholdes  
	\item Kommunikation mellem Regulerings MCU og Controlleren foregår via I2C protokollen
\end{itemize}

\subsection{Hardware arkitektur}

\figur{1}{vuggesystem/sysark/Vuggesystem_BDD.pdf}{BDD for Vuggesystem}{Vug_BDD}

Vuggesystemet består af tre dele
\begin{description}
\item[Motorsystem block:] Består af en motor til at drive vuggebevægelsen samt en motorkreds til strømstyringen af motoren. Motorkredsen sørger for at motorens ankerspænding styres vha dutycycle på et PWM-signal og retningen på motoren styres af et logisk signal.
\item[Regulerings MCU block:] Styringsenheden for vuggesystemet. Denne sørger for reguleringen af vuggesystemet samt kommunikationen til og fra controlleren.
\item[Sensor block:] Består af fire sensorer; to Endstop sensorer som måler om barnevognens kurv har nået den mekaniske vuggegrænse, Vuggeudsving sensor måler kurvens absolutte vinkel i forhold til tyngdefelt og Motor positionssensoren måler motorens position.
\end{description}

Følgende beskriver vuggesystemets kobling og grænseflade.

\figur{1}{vuggesystem/sysark/Vuggesystem_IBD.pdf}{IBD for Vuggesystem}{Vug_IBD}

\vspace{5mm} 

\figur{0.5}{vuggesystem/sysark/Endstop_sensor_IBD.pdf}{IBD for Endstop sensorer}{EndstopS_IBD}
\textbf{Endstop sensorer} består af to ens sensorer, hhv. \textbf{Endstop sensor front}, placeret til at detekterer hvis barnevognskurven når den mekaniske vuggegrænse ved fremad vuggeretning, og \textbf{Endstop sensor rear}, placeret til at detekterer hvis barnevognskurven når den mekaniske vuggegrænse ved bagud vuggeretning. Sensorerne giver det samme signal uanset om detekteringen sker i front eller bag.

\vspace{5mm}

\figur{0.5}{vuggesystem/sysark/Vuggeudsving_sensor_IBD.pdf}{IBD for Vuggeudsving sensor}{VugUdsving_IBD}
\textbf{Vuggeudsving sensor} er placeret så den måler barnevognskurvens plan i forhold til jordens tyngdefelt. 

\vspace{5mm}

\figur{0.5}{vuggesystem/sysark/motor_positions_sensor_IBD.pdf}{IBD for Motor positionssensor}{MotoPosS_IBD}
\textbf{Motor positionssensor} giver et frekvenssignal relativ til motorens nuværende hastighed. 

\subsection{Grænsefladebeskrivelse}
Herunder findes en beskrivelse af grænsefladen både til denne del af systemet, samt de interne forbindelser. 

\subsection*{I2C beskrivelse}
Forbindelserne ud og ind af dette delsystem er samlet i en I2C bus, som er beskrevet her:

\begin{center}
\begin{longtable}{|p{1cm}|p{2cm}|p{1cm}|p{5.5cm}|p{2.5cm}|}
\caption[i2cBeskrivelse]{Specifikation af I2C grænseflade} 

\label{i2c_tabel} \\

\hline 

\multicolumn{5}{|l|}{\textbf{I2C Adresse:} 0b1111000X  (Write: 0xF0, Read: 0xF1)} \\ \hline
\multicolumn{5}{|l|}{\textbf{I2C Frekvens:} 100kHz} \\ \hline

\textbf{Reg\#} & \textbf{Navn} & \textbf{Type} & \textbf{Beskrivelse} & \textbf{Startværdi} \\
\hline 
\endfirsthead


\multicolumn{5}{c}{{\bfseries \tablename\ \thetable{} --> fortsat fra forrige side}} \\
\multicolumn{5}{c}{} \\

\hline
\textbf{Reg\#} & \textbf{Navn} & \textbf{Type} & \textbf{Beskrivelse} & \textbf{Startværdi} \\
\hline 
\endhead


\multicolumn{5}{r}{{fortsættes på næste side -->}} \\ 
 
\endfoot

0x00 & ID & \textbf{R} & Indeholder et id som kan benyttes til at identificere denne enhed, eller til at teste forbindelsen til denne. & 0xFB \\ \hline

0x01 & Status & \textbf{R} & Indeholder en bitsekvens som indikerer systemets status. Registeret indeholder følgende: [ERR STALL END\_STP SD\_RDY X X X X] & 0b0000XXXX \\ \hline

\multicolumn{2}{|r|}{\textbf{ERR}} & \multicolumn {3}{p{8cm} |} {Indikerer at der er opstået en fejl i systemet. Kendes årsagen til fejlen indikeres denne i STALL og END\_STP} \\ \hline

\multicolumn{2}{|r|}{\textbf{STALL}} & \multicolumn {3}{p{8cm} |} {Indikerer at systemet har været ude af stand til at drive motoren, formegentlig pga for stor belastning} \\ \hline

\multicolumn{2}{|r|}{\textbf{END\_STP}} & \multicolumn {3}{p{8cm} |} {Indikerer at vuggen har ramt en af sine mekaniske yderpositioner, og vuggesystemet er deaktiveret indtil der er blevet genstartet.} \\ \hline

\multicolumn{2}{|r|}{\textbf{SD\_RDY}} & \multicolumn {3}{p{8cm} |} {Indikerer at systemet er klar til at få afbrudt strømmen} \\ \hline

0x02 & ON\_OFF & \textbf(R/W) & Dette register benyttes til at tænde og slukke for systemet. Skrives et nul til dette register begynder systemet at lukke ned. Strømmen til systemet bør ikke afbrydes før SD\_RDY i status registeret er skiftet til et. Hvis systemet er tændt indeholder registeret en værdi forskellig fra 0. & 0xFF \\ \hline

0x03 & Frekvens & \textbf{R/W} & Værdien i dette register styrer frekvensen hvormed der vugges. Område: \SI{0}{\hertz} = \SI{2.550}{\hertz},  1 LSB = \SI{10}{\milli\hertz}. & 0x00 \\ \hline

0x04 & Vinkeludsving & \textbf{R/W} & Værdien i dette register kontrollerer størrelsen af vuggens udsving i grader. Område: +/- \SI{12.75}{\degree}, 1 LSB = \SI{0.05}{\degree}. & 0x00 \\ \hline

\hline \hline
\endlastfoot

\end{longtable}
\end{center}

\subsection*{Signalbeskrivelse}
\begin{center}
\begin{longtable}{|p{3cm}|p{2cm}|p{6cm}|}
\caption[Signalbeskrivelse for vuggesystem]{Signalbeskrivelse} 

\label{signalbeskr_vugge_tabel} \\

\hline 

\multicolumn{1}{|p{3cm}|}{\textbf{Signal}} & 
\multicolumn{1}{p{2cm}|}{\textbf{Type}} & 
\multicolumn{1}{p{6cm}|}{\textbf{Kommentar}} \\
\hline 
\endfirsthead


\multicolumn{3}{c}%
{{\bfseries \tablename\ \thetable{} --> fortsat fra forrige side}} \\

\multicolumn{1}{|p{3cm}|}{\textbf{Signal}} & 
\multicolumn{1}{p{2cm}|}{\textbf{Type}} & 
\multicolumn{1}{p{6cm}|}{\textbf{Kommentar}} \\
\hline
\endhead


\hline \multicolumn{3}{|r|}{{fortsættes på næste side -->}} \\ \hline
\endfoot

\hline \hline
\endlastfoot

VB & 12V DC & Dette signal kommer direkte fra systemets batteri.\\
\hline
VS & 0V DC & Dette er batteriforsyningens retur\\
\hline
VCC & 3.3V DC & Dette er forsyningen til PSoC og andet logik.\\
\hline
GND & 0V stel & Stelforbindelse til PSoC og andet logik\\
\hline
MotRetning & Digital & HIGH = CW, LOW = CCW\\
\hline
MotSpænding & PWM & PWM som styrer motor spændingen ved VB * \%dc. PWM signalet 0-3.3V CMOS med f = [???]\\
\hline
VuggeV & I2C & Angiver vinklen af vuggen relativt til tyngdefeltet.\\
\hline
MotHast & ?? & Angiver hastigheden af motoren\\
\hline
ESSwitch & Digital & Endstop status, ES-for OR ES-bag, HIGH = ikke ramt LOW = ramt.\\
\hline
I2CVuggesystem & I2C & Styresignal til vuggesystemet\\
\hline
MotForbind & DCMotorSig & Motor tilkobling\\
\hline

\end{longtable}
\end{center}



\newpage
\subsection{Software arkitektur}

I følgende afsnit beskrives softwarearkitekturen for Vuggesystem delen af Baby Watch. Softwarearkitekturen er udarbejdet på baggrund af projektformuleringen og kravspecifikationen. 

\vspace{5mm}

Softwarearkitekturen for Vuggesystemet består af: 
\begin{itemize}
\item Identifikation af problemer, klasser og metoder med udgangspunkt i en domænemodel for system og dets softwaremoduler  
\item Oprettelse af skelet for videre implementering af Vuggesystemets SW vha. et sekvensdiagram for program-flow 
\end{itemize}

Det er valgt at lave applikationsmodeller for modulerne/modulgrupperne internt frem for direkte Use Case baseret. Applikationsmodellen indeholder således funktionaliteter for op til flere Use Cases, men kun afgrænset til Vuggesystemet. 

\vspace{5mm}
\subsubsection*{Domænemodel} 

\figur{1}{vuggesystem/sysark/Domain_Klasse_kommunikation.pdf}{Domain kommunikations model for Vuggesystem}{VuggeSystem_Domain}

\vspace{1mm}
Figur \ref{VuggeSystem_Domain} viser en domænemodel for kommunikationen mellem systemets interne SW-moduler. Blokken Baby Watch Controller tilhører ikke Vuggesysmet, men er sat på for identificerer udadgående grænseflader derfor er denne markeret bag en stiplet linje.

\subsubsection*{Klasseidentifikation}
På baggrund af domænemodellen for softwaremoduler, identificeres følgende klasser for Vuggesystem:
Vuggesystemet sørger for at vugge barnevognskurven med en bestemt vuggefrekvens og vuggeudsving angivet af Baby Watchs Controller. Denne vugning reguleres ud fra sensorinput der angiver; den absolutte vinkel for vuggeudsvinget i forhold til tyngdefeltet, kurvens position og om det yderste vuggeudsving er overskredet. Vuggesystemet skal også give besked til den overordnede Baby Watch Controller om eventuelle fejl i denne regulering.

\begin{itemize}
\item Regulerings MCU - klassenavn: \textbf{vuggeControl} er Vuggesystems control-klasse. 
\item I2C kommunikation - klassenavn: \textbf{i2cKommunikation} står for I2C interfacet ud til Baby Watchs Contoller.
\item Motorregulering - klassenavn: \textbf{regulering} står for beregningerne til reguleringen af motoren der trækker vugningen af barnevognenskurv
\item Sensor Fortolker - klassenavn: \textbf{sensorFortolker} står for at behandle sensorinput fra Motor positionssensor, Endstop sensors og Vuggeudsving sensor
\end{itemize}

Følgende sekvensdiagram og klassediagram  identificerer programflow og funktioner  for \textbf{Regulerings MCU} blokken

\subsubsection*{Sekvensdiagram}

\figur{1}{vuggesystem/sysark/reguleringsMCU_SD.pdf}{Sekvensdiagram for Regulerings MCU}{VuggeSystem_SD}

\subsubsection*{Klassediagram}

\figur{1}{vuggesystem/sysark/reguleringsMCU_klassediagram.pdf}{Klassediagram for Regulerings MCU}{VuggeSystem_SD}

\subsubsection*{Funktionsbeskrivelser}
\textit{vuggeControl} \\
\begin{center}
    \begin{tabular}{ | l | p{10cm} |}
    \hline
    \textbf{Funktion}	 	& IKKE FÆRDIG								\\ \hline
    \textbf{Beskrivelse} 	& Gemmer diskret sample i RecordingBuffer					\\ \hline
    \textbf{Parametre}		& int Sample: Diskret værdi som repræsenterer det momentane spændingsniveau på Blackfin 533 ADC indgang														 		\\ \hline
    \textbf{Returværdi} 	& Ingen		 												\\ \hline
    \end{tabular}
\end{center}
\textit{i2cKommunikation} \\

\begin{center}
    \begin{tabular}{ | l | p{10cm} |}
    \hline
    \textbf{Funktion}	 	& void sendConKom(char* reg[2], char* reg[3], char* reg[4])								\\ \hline
    \textbf{Beskrivelse} 	& Sender tre char pointere der peger på registrene reg[2], reg[3] og reg[4] til vuggeControl klassen					\\ \hline
    \textbf{Parametre}		& char * reg[2]: peger på ????, char * reg[3]: peger på ???? og char * reg[4]: peger på ????				 		\\ \hline
    \textbf{Returværdi} 	& Ingen		 												\\ \hline
    \end{tabular}
\end{center}

\textit{reguelering} \\

\begin{center}
    \begin{tabular}{ | l | p{10cm} |}
    \hline
    \textbf{Funktion}	 	& IKKE FÆRDIG								\\ \hline
    \textbf{Beskrivelse} 	& Gemmer diskret sample i RecordingBuffer					\\ \hline
    \textbf{Parametre}		& int Sample: Diskret værdi som repræsenterer det momentane spændingsniveau på Blackfin 533 ADC indgang														 		\\ \hline
    \textbf{Returværdi} 	& Ingen		 												\\ \hline
    \end{tabular}
\end{center}

\textit{sensorFortolker} \\

\begin{center}
    \begin{tabular}{ | l | p{10cm} |}
    \hline
    \textbf{Funktion}	 	& char getMotorPosSens(void)								\\ \hline
    \textbf{Beskrivelse} 	& Returnerer den nyeste motorposition fra motorpositionssensoren.					\\ \hline
    \textbf{Parametre}		& Ingen								 		\\ \hline
    \textbf{Returværdi} 	& char: Den nyeste position angivet i værdi fra 0-255		 												\\ \hline
    \end{tabular}
\end{center}

\begin{center}
    \begin{tabular}{ | l | p{10cm} |}
    \hline
    \textbf{Funktion}	 	& char getVuggeUdsving(void)										\\ \hline
    \textbf{Beskrivelse} 	& Returnerer den nyeste måling fra Vuggeudsvingssensoren.		\\ \hline
    \textbf{Parametre}		& Ingen			 										\\ \hline
    \textbf{Returværdi} 	& char: Den nyeste måling fra Vuggeudsvingssensoren angivet i ??? accelerometer og gyroskop ???		 											\\ \hline
    \end{tabular}
\end{center}

\begin{center}
    \begin{tabular}{ | l | p{10cm} |}
    \hline
    \textbf{Funktion}	 	& int calcAbsVuggeUdsving(char udsving)										\\ \hline
    \textbf{Beskrivelse} 	& Udfra acceletormeter- og gyroskopmålingerne beregnes og returners den absolutte vinkel for barnevognskurven. 		\\ \hline
    \textbf{Parametre}		& char: Udsving er målinger fra Vuggeudsvingssensoren 		\\ \hline
    \textbf{Returværdi} 	& int: Den absolutte vinkel for barnevognskurven			\\ \hline
    \end{tabular}
\end{center}



\begin{center}
    \begin{tabular}{ | l | p{10cm} |}
    \hline
    \textbf{Funktion}	 	& void checkEndstop(bool)										\\ \hline
    \textbf{Beskrivelse} 	& Tjekker endstop sensorernes værdi.		\\ \hline
    \textbf{Parametre}		& Ingen			 										\\ \hline
    \textbf{Returværdi} 	& wav		 											\\ \hline
    \end{tabular}
\end{center}
