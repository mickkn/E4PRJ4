\section{Systemarkitektur}

I dette afsnit beskrives systemarkitekturen for vuggesystemet.

\subsection*{Overordnet virkemåde}
Vuggesystemet fungerer overordnet som følger:
\begin{itemize}
	\item Vuggesystemets Regulerings MCU står for at vugge barnevognens kurv med en bestemt vuggefrekvens og et bestemt vinkeludsving. Dette sker på baggrund af værdier for disse modtaget fra Controller
	\item Regulerings MCU'en regulerer vuggesystemet således at vugningen altid foregår ud fra en vandret akse samt at de modtagne værdier for vuggefrekvensen og vinkeludsvinget overholdes  
	\item Kommunikation mellem Regulerings MCU og Controlleren foregår via I2C protokollen
\end{itemize}

\subsection{Hardware arkitektur}

\figur{1}{vuggesystem/sysark/Vuggesystem_BDD.pdf}{BDD for Vuggesystem}{Vug_BDD}

Vuggesystemet består af tre dele

\textbf{Motorsystem block:} Består af en motor til at drive vuggebevægelsen samt en motorkreds til strømstyringen af motoren. Motorkredsen sørger for at motorens ankerspænding styres vha dutycycle på et PWM-signal og retningen på motoren styres af et logisk signal.\\
\textbf{Regulerings MCU block:} Styringsenheden for vuggesystemet. Denne sørger for reguleringen af vuggesystemet samt kommunikationen til og fra controlleren.\\
\textbf{Sensor block:} Består af fire sensorer; to Endstop sensorer som måler om barnevognens kurv har nået den mekaniske vuggegrænse, Vuggeudsving sensor måler kurvens absolutte vinkel i forhold til tyngdefelt og Motor positionssensoren måler motorens position.


Følgende beskriver vuggesystemets kobling og grænseflade.

\figur{1}{vuggesystem/sysark/Vuggesystem_IBD.pdf}{IBD for Vuggesystem}{Vug_IBD}

\vspace{5mm} 

\figur{0.5}{vuggesystem/sysark/Endstop_sensor_IBD.pdf}{IBD for Endstop sensorer}{EndstopS_IBD}
\textbf{Endstop sensorer} består af to ens sensorer, hhv. \textbf{Endstop sensor front}, placeret til at detektere hvis barnevognskurven når den mekaniske vuggegrænse ved fremad vuggeretning, og \textbf{Endstop sensor rear}, placeret til at detekterer hvis barnevognskurven når den mekaniske vuggegrænse ved bagud vuggeretning. Sensorerne giver det samme signal uanset om detekteringen sker i front eller bag.

\vspace{5mm}

\figur{0.5}{vuggesystem/sysark/Vuggeudsving_sensor_IBD.pdf}{IBD for Vuggeudsving sensor}{VugUdsving_IBD}
\textbf{Vuggeudsving sensor} er placeret så den måler barnevognskurvens plan i forhold til jordens tyngdefelt. 

\vspace{5mm}

\figur{0.5}{vuggesystem/sysark/motor_positions_sensor_IBD.pdf}{IBD for Motor positionssensor}{MotoPosS_IBD}
\textbf{Motor positionssensor} giver et frekvenssignal relativ til motorens nuværende hastighed. 

\newpage
\subsection{Grænsefladebeskrivelse}
Herunder findes en beskrivelse af de interne forbindelser. 

\subsection*{Signalbeskrivelse}
\begin{center}
\begin{longtable}{|p{3cm}|p{2cm}|p{6cm}|}
\caption[Signalbeskrivelse for vuggesystem]{Signalbeskrivelse} 

\label{signalbeskr_vugge_tabel} \\

\hline 

\multicolumn{1}{|p{3cm}|}{\textbf{Signal}} & 
\multicolumn{1}{p{2cm}|}{\textbf{Type}} & 
\multicolumn{1}{p{6cm}|}{\textbf{Kommentar}} \\
\hline 

\endhead




\hline \hline
\endlastfoot

VB & 12V DC & Dette signal kommer direkte fra systemets batteri\\
\hline
VS & 0V DC & Dette er batteriforsyningens retur\\
\hline
VCC & 3.3V DC & Dette er forsyningen til PSoC og andet logik\\
\hline
GND & 0V stel & Stelforbindelse til PSoC og andet logik\\
\hline
MotorPWM & PWM & PWM som styrer motor spændingen ved VB samt retningen for motoren. Fra 0-50\%duty-cycle generer en negativ motorspænding, fra 50-100 \%duty-cycle generer en positiv motorspænding.  PWM signalet 0-3.3V CMOS med f = [20kHz]\\
\hline
MotorRelae & Digital & Kontakt som holder motoren tændt,  HIGH = Tændt LOW = slukket  \\
\hline
VuggeV & I2C & Angiver vinklen af vuggen relativt til tyngdefeltet\\
\hline
MotHast & Mcount & Angiver hastigheden af motoren(Undersøges færdig)\\
\hline
ESSwitch & Digital & Endstop status, ES-for OR ES-bag, HIGH = ikke ramt LOW = ramt\\
\hline
I2CVuggesystem & I2C & Styresignal til vuggesystemet\\
\hline
MotForbind & DCMotorSig & Motor tilkobling\\
\hline

\end{longtable}
\end{center}

\newpage
\subsection{Software arkitektur}
\label{Vuggesystem: Software}

I følgende afsnit beskrives softwarearkitekturen for Vuggesystem delen af Baby Watch. Softwarearkitekturen er udarbejdet på baggrund af projektformuleringen og kravspecifikationen. 

\vspace{5mm}

Softwarearkitekturen for Vuggesystemet består af: 
\begin{itemize}
\item Identifikation af problemer, klasser og metoder med udgangspunkt i en domænemodel for system og dets softwaremoduler  
\item Oprettelse af skelet for videre implementering af Vuggesystemets SW vha. et sekvensdiagram til klasseidentifikation og programflow
\end{itemize}

Applikationsmodellen indeholder funktionaliteter for op til flere Use Cases, men er kun afgrænset til Vuggesystemet. 

\vspace{2mm}
\subsection*{Applikationsmodel} 
\subsubsection*{Domænemodel} 

\figur{1}{vuggesystem/sysark/Domain_Klasse_kommunikation.pdf}{Domain kommunikations model for Vuggesystem}{VuggeSystem_Domain}

\vspace{1mm}
Figur \ref{VuggeSystem_Domain} viser en domænemodel for kommunikationen mellem systemets interne SW-moduler. Blokken Baby Watch Controller tilhører ikke Vuggesystemet, men er sat på for identificerer udadgående grænseflader derfor er denne markeret bag en stiplet linje.

\subsubsection*{Klasseidentifikation}
På baggrund af domænemodellen for softwaremoduler og følgende beskrivelse, udarbejdes et sekvensdiagram der identificerer programflow og klasser  for \textbf{Regulerings MCU} blokken.\\
\vspace{1mm}

Beskrivelse af system:\\
Vuggesystemet sørger for at vugge barnevognskurven med en bestemt vuggefrekvens og vuggeudsving angivet af Baby Watchs Controller. Denne vugning reguleres ud fra sensorinput der angiver; den absolutte vinkel for vuggeudsvinget i forhold til tyngdefeltet, motorens position og om den mekaniske grænse for vuggeudsvinget er nået. Vuggesystemet skal også give besked til den overordnede Baby Watch Controller om eventuelle fejl i denne regulering.


\subsubsection*{Sekvensdiagram}

\figur{1}{vuggesystem/sysark/reguleringsMCU_SD.pdf}{Sekvensdiagram for Regulerings MCU}{VuggeSystem_SD}

Blokkene i \textbf{Regulerings MCU} fra domænemodellen ændres til følgende klassenavne: \\
\begin{itemize}
\item Regulerings MCU - klassenavn: \textbf{vuggeControl} er Vuggesystems control-klasse. 
\item I2C kommunikation - klassenavn: \textbf{i2cKommunikation} står for I2C interfacet ud til Baby Watchs Contoller.
\item Motorregulering - klassenavn: \textbf{regulering} står for beregningerne til reguleringen af motoren der trækker vugningen af barnevognenskurv
\item Sensor Fortolker - klassenavn: \textbf{sensorFortolker} står for at behandle sensorinput fra Motor positionssensor, Endstop sensors og Vuggeudsving sensor
\end{itemize}

Motorposition sensordata, Vuggeudsving sensordata og endstopGPIO skal forståes som interruptbaserede funktioner. Disse opdaterer automatisk sensordataene fra de tre forskellige sensorer.