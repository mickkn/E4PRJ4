\section{Modultest}
Modultest for vuggesystemet beskriver først modultests af hardware modulerne efterfulgt af tests af softwaremodulerne. 
\subsection{Hardware modultest}
Følgende afsnit beskriver tests for de individuelle hardwaremoduler i Vuggesystem-delen af Baby Watch.
\subsubsection{Motorkreds}
\textbf{Formål} \\
Formålet med testen er at måle om motorkredsen kan styrer en 12V motor der trækker op til 10A ved hjælp af et PWM kontrolleret input. For at testen kan godkendes skal motoren "vippe" en barnevognskurv over en horisontal akse jf. afsnit \vref{kravspec:ikke_funk_nivellering} i Ikke-funktionelle krav fra -12 grader position til 12 grader tre gange i sekvens. Der foretages to udgaver af testen en uden mervægt og en med en mervægt på 10kg placeret i barnevognskurven.

\textbf{Overordnet opstilling af Motorkreds test}

\begin{itemize}
	\item Testen forgår i et rum med et horisontal nivelleret gulv
	\item Til testen skal Baby Watch Vuggesystemets motor være monteret jf. afsnit \vref{Vuggesystem: Implementering_MV} Mekanik vuggesystem i Implementering for Vuggesystemet
	\item Motorens plus og minus indgange kobles til henholdsvis MOT\_A og MOT\_B jf. afsnit \vref{Vuggesytem:HW_DESIGN_HBRO} i Hardware Design for Vuggesystemet
	\item Et PSoC 4 Pioneer Kit \citep{website:Cypress} programmeres med programmet \textit{vuggeSinus} på bilags cd'en \citep{cd} hvilket kan findes under \textit{/SW/Vuggesystem/Testprogrammer/vuggeSinus} og de to PWM output kobles til henholdsvis Mos Driver A indgang M\_A og Mos Driver B indgang M\_B på motorkredsen jf. afsnit \vref{Vuggesytem:HW_DESIGN_MOSDRIVER} i Hardware Design for Vuggesystemet
	\item I første udgave af testen placerets der ingen ekstra vægt i barnevognskurven. I anden udgave placeres en 10kg vægt på barnevognskurvens centertyngdepunkt.
\end{itemize}

Figur billede af opstillingen

\textbf{Testbeskrivelse}
\begin{itemize}
	\item Opsæt systemet som beskrevet ovenfor
	\item Programmer PSoC4 Pioneer Kittet med VuggeSinus programmet
	\item Testen dokumenteres med video og billeder
	\item Testen gentages med en mervægt på 10kg i barnevognskurven
\end{itemize}

\textbf{Forventet resultat} \\
Det forventes at motorkredsløbet er i stand til at "vippe" barnevognen som beskrevet i Formålet tre gange i sekvens fra yder position til yder position både med og uden mervægt.

\textbf{Resultat} \\
// Tekst \\

// Figur 2 - Billede af resultat og reference til BILAG med video.

\textit{Testen er godkendt/ikke-godkendt}



\subsubsection{Endstopsensor}
\textbf{Formål} \\
Formålet med testen er at se om Vuggesystemet melder om fejl ved aktivering af en af Endstopsensorerne. Beskrivelse af Endstopsensor kan ses i afsnit \vref{Vuggesystem: Design_ES}.

\textbf{Overordnet opstilling af Enstopsensor test}

\begin{itemize}
	\item Testen foregår i et rum med et horisontalt nivelleret gulv
	\item Baby Watch barnevognen monteres med Endstopsensorne jf.\vref{Vuggesystem: Implementering_ES} i Implementeringen for Vuggesystemet 
	\item Baby Watch barnevogns kurven sættes i et vandret niveau
	\item PSoC4 Pioneer Kittet er tilsluttet en computer
	\item Computeren har startet et terminal program som har oprettet forbindelse til den givne com port PSoC4 Pioneer Kittet sidder på
	\item PSoC4 Pioneer Kittet programmeres med VuggeControl programmet
\end{itemize}

// Figur med billede af opstilling

\textbf{Testbeskrivelse}
\begin{itemize}
	\item Opsæt systemet som beskrevet ovenfor
	\item Baby Watch barnevognskurven sættes ud i en yder position så Endstopsensoren aktives
	\item Der kontrollers på udskrift fra terminal programmet på computeren om Endstopsensoren aktiver en advarsel om aktivering. Tekst: "Endstop aktiveret"
	\item Testen gentages men denne gang med aktivering af Endstopsensoren på den modsatte yder position
\end{itemize}

\textbf{Forventet resultat} \\
Det forventes at der ved aktivering ved begge yder positioner skrive "Endstop aktiveret" i terminal vinduet på computeren \\
\textbf{Resultat} \\
// Tekst \\

// Figur med billede af terminal udskriften samt et billede af den vippede barnevogn

\textit{Testen er godkendt/ikke-godkendt}



\subsubsection{Mekanisk vuggesystem}
\textbf{Formål} \\
Formålet med denne modultest er at observere om det mekaniske vuggesystem er i stand til at "vippe" en barnevognskurv over en horisontal akse jf. afsnit \vref{kravspec:ikke_funk_nivellering} i Ikke-funktionelle krav fra den ene yder position til den anden tre gange i sekvens. 

\textbf{Overordnet opstilling af Mekanisk vuggesystem test}

\begin{itemize}
	\item Testen forgår i et rum med et horisontal nivelleret gulv
	\item Motoren er spændt på Baby Watch Vuggesystemet jf. afsnit Mekanik vuggesystem i Implementering for Vuggesystemet
	\item Motorens plus og minus indgange kobles til henholdsvis MOT A og MOT B jf. afsnit \ref{Vuggesytem:HW_DESIGN_HBRO} i Hardware Design for Vuggesystemet
	\item Et PSOC4 Pioneer Kit \citep{website:Cypress} programmeres med testprogrammet VuggeSinus jf. BILAG?? og de to PWM output kobles til henholdsvis Mos Driver A indgang M A og Mos Driver B indgang M B på motorkredsen jf. afsnit \vref{Vuggesytem:HW_DESIGN_MOSDRIVER} i Hardware Design for Vuggesystemet
\end{itemize}

Figur billede af opstillingen

\textbf{Testbeskrivelse}
\begin{itemize}
	\item Opsæt systemet som beskrevet ovenfor
	\item Programmer PSoC4 Pioneer Kittet med VuggeSinus programmet
	\item Testen dokumenteres med video og billeder
\end{itemize}

\textbf{Forventet resultat} \\
Det forventes at det mekaniske vuggesystem er i stand til at "vippe" barnevognen som beskrevet i Formålet tre gange i sekvens fra yder position til yder position.

\textbf{Resultat} \\
// Tekst \\

// Figur 2 - Billede af resultat og reference til BILAG med video.

\textit{Testen er godkendt/ikke-godkendt}




\subsection{Software modultest}
Dette afsnit beskriver tests for de individuelle hardwaremoduler i Vuggesystem-delen af Baby Watch.
\subsubsection{Vuggeudsvingssensor}
\textbf{Formål} \\
Formålet er at teste om Vuggeudsvingssensor monteret på en barnevognskurv der "vipper" over en horisontal akse jf. afsnit \vref{kravspec:ikke_funk_nivellering} i Ikke-funktionelle krav er i stand til at detektere vinklen i forhold til tyngdefeltet. For at testen kan godkendes forventes en præcision på mindst +/- 2 grader.

Testen dokumenteres med en sammenligning af den målte vinkel fra Vuggeudsvingssensor relativ til den målte vinkel af et elektronisk vaterpas. Vinklen måles fra den mekaniske yderposition til den anden mekaniske yderposition i 5 graders intervaller.

\textbf{Overordnet opstilling af Vuggeudsvingssensor test}

\begin{itemize}
	\item Testen foregår i et rum med et horisontalt nivelleret gulv
	\item Til testen skal bruges Baby Watch barnevognen monteret med Vuggeudsvingssensor jf.\vref{Vuggesystem: Implementering_VuggeudsvingSensor} i Implementeringen for Vuggesystemet
	\item Et PSoC 4 Pioneer Kit af typen CY8CKIT-042 
	\item En mobiltelefon af typen Samsung Galaxy S4 mini \citep{website:Samsung} med vaterpas-programmet Clinometer \citep{website:Clinometer} og streamming programmet Mobizen \citep{website:Mobizen}
	\item En Windows 8.1 \citep{website:Windows_8_1} computer med et terminalprogram, med USB-porte og Microsoft Excel\citep{website:Excel} installeret 
	\item Programmene Clinometer og Mobizen startes på mobiltelefonen og Mobizen startes ligeledes på computeren
	\item PSoC4 Kittet tilsluttes en computer via USB
	\item Terminalprogrammet startes på computeren og indstilles til PSoC4 kittes USB Com-port
	\item Baby Watch barnevogns kurven sættes i en mekanisk yderposition
	\item PSoC4 Pioneer Kittet er tilsluttet en computer
	\item PSoC4 Pioneer Kittet programmeres med programmet \textit{vinkelsensor test} på bilags cd'en \citep{cd} hvilket kan findes under \textit{/SW/Vuggesystem/Testprogrammer/vinkelsensor test}
\end{itemize}

\textbf{Testbeskrivelse}
\begin{itemize}
\item Opsæt systemet som beskrevet ovenfor
\item Baby Watch barnevognskurven vippes med håndkraft fra yderposition til yderposition i intervallet -24.7, -20, -15, -10, -5, 0, 5, 10, 15, 20, 24.5 grader\footnotemark målt med det elektriske vaterpas 
\item På terminalprogrammet på computeren udskrives data som angiver vinklen i fix16.16 format ved hvert af intervallerne
\item Disse data ligges ind i Microsoft Excel og sammenlignes
\end{itemize}

\textbf{Forventet resultat} \\
Det forventes at de målte data fra Vinkeludvingssensor passer overens med det elektroniske vaterpas med max +/- 2 grader difference.
 \\

\footnotetext{Der forventes en kontinuert kurve der begynder i 0 graders vinkel som efterfølgende skifter karakter til en trekantkurve der går fra -24.7 grader til 24.8 grader vinkel og til sidst stopper på 0 grader vinkel igen.}

\textbf{Resultat} \\
Resultatet er som forventet:
\figur{0.8}{vuggesystem/modultest/vugUdsving_modulData.pdf}{Samlings tabel af målte Vuggeudsving Sensor vinkeler og elektronsk vaterpas værdier}{vuggesystem:ModulT_VugUdSens}

\textit{Testen er godkendt}

\subsubsection{Vuggesystemet}
\textbf{Formål} \\
Formålet er at teste hele vuggesystemet som reguleringsmodul for Baby Watch systemet. Testen udføres i seks udgaver alle med en 10 kg mervægt i barnevognskurven. Testen dokumenteres med video.

\textit{Testudgaver}*
\begin{itemize}
	\item Vuggesystemet regulerer barnevognskurven i en 0 graders vinkel-amplitude med en vippe-frekvens på 0 Hz
	\item Vuggesystemet regulerer barnevognskurven i sekvens af først 10 graders vinkel-amplitude med en vippe-frekvens på 0.5 Hz efterfulgt af 6 graders vinkel-amplitude med en vippe-frekvens på 1 Hz og til sidst af 4 graders vinkel-amplitude med en vippe-frekvens på 2 Hz
	\item Vuggesystemet regulerer barnevognskurven i sekvens af først 6 graders vinkel-amplitude med en vippe-frekvens på 1 Hz efterfulgt af 10 graders vinkel-amplitude med en vippe-frekvens på 0.5 Hz og til sidst af 4 graders vinkel-amplitude med en vippe-frekvens på 2 Hz
	\item Vuggesystemet regulerer barnevognskurven i 8 graders vinkel amplitude med en vippe frekvens på 0.75 Hz
	\item Barnevognskurven presses ned i en yderpositon så en endstop sensor aktives
	\item Barnevognskurven presses ned i modsatte yderpositon så den anden endstop sensor aktives
\end{itemize}
*Alle vinkler er i forhold til tyngdefeltet.


\textbf{Overordnet opstilling af Vuggesystem modultest}

\begin{itemize}
	\item Testen foregår i et rum med et horisontalt nivelleret gulv
	\item Vuggesystemet er opsat jf. implementeringen af Vuggesystem afsnit \vref{Vuggesystem: Implementering}
	\item En mobiltelefon af typen Samsung Galaxy S4 Mini \citep{website:Samsung} er opsat og klar til optagelse
\end{itemize}
// Figur med billede af testopstillingen

\textbf{Testbeskrivelse}
\begin{itemize}
	\item Testen opsættes om beskrevet ovenfor
	\item Hvert punkt i \textit{Testudgaver} under Formål udføres og dokumenteres med video
\end{itemize}

\textbf{Forventet resultat} \\
Der forventes at:
\begin{itemize}
	\item Barnevognskurven holdes i en vandret på position i forhold til tyngdefeltet
	\item Barnevognskurven vugges i en sekvens af først 10 graders vinkel-amplitude med en vippe-frekvens på 0.5 Hz efterfulgt af 6 graders vinkel-amplitude med en vippe-frekvens på 1 Hz og til sidst af 4 graders vinkel-amplitude med en vippe-frekvens på 2 Hz
	\item Barnevognskurven vugges i en sekvens af først 6 graders vinkel-amplitude med en vippe-frekvens på 1 Hz efterfulgt af 10 graders vinkel-amplitude med en vippe-frekvens på 0.5 Hz og til sidst af 4 graders vinkel-amplitude med en vippe-frekvens på 2 Hz
	\item Barnevognskurven vugges i en 8 graders vinkel amplitude med en vippe frekvens på 0.75 Hz
	\item Når barnevognskurven presses ned i en yderpositon aktives en endstop sensor og vuggesystemet stopper med alt vugning 
	\item Når barnevognskurven presses ned i den modsatte yderpositon aktives en endstop sensor og vuggesystemet stopper med alt vugning 
\end{itemize}
*Alle vinkler er i forhold til tyngdefeltet.

\textbf{Resultat} \\
// Resultatet beskrives og der henvis til optagelserne \\


\textit{Testen er godkendt/ikke-godkendt}