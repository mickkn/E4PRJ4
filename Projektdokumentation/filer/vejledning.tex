\chapter{Læsevejledning}

Da dokumentationen er opbygget i delelementer, følger her en lille vejledning i hvordan den bør læses.

\textbf{\textit{Kravspecifikationen}} er opbygget som en alm. kravspecifikation der i fully dressed use cases beskrive ideén og funktionaliteten i projektet.

\textbf{\textit{Baby Watch}} Består af den samlede systemarkitektur og alt hvad der er overordnet for Baby Watch. Grænseflader imellem delelementerne er også beskrevet her.

\textbf{\textit{Controller, Intelligent lydmonitor, Vuggesystem og PSU}} Er afsnit for sig selv, som er opbygget med hver deres systemarkitektur, design, implementering og modultest. Dette valg er taget for at det vil være nemmere at finde frem til lige netop det afsnit man vil fortælle om til eksamen. Vi oplevede det var et irritationsmoment på sidste semester at finde rundt i et afsnit med alle delelementer.