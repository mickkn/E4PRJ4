\section{Systemarkitektur}

I dette afsnit beskrives systemarkitekturen for den intelligente lydmonitor.

\subsection*{Overordnet virkemåde}
Overordnet skal den intelligente lydmonitor fungere som følger:
\begin{itemize}
	\item Babyens gråd detekteres med en mikrofon og forstærkes med en mikrofonforstærker inden det analoge signal konverteres til diskrete samples med Blackfin's ADC
	\item Den diskrete sample-sekvens i bufferen analyseres med hensyn til power og frekvensindhold og resultatet heraf gemmes i endnu en buffer med analyseresultater
	\item Disse analyseresultater kategoriseres som tre BABYCON-states, som beskrevet i \ref{kravspec:indledning_babycon_states} (OBS: Find enighed om kategorisering)
	\item Kategoriseringsresultatet sendes til Controller via en I2C forbindelse
\end{itemize}

\figur{1}{intlydmonitor/sysark/Intelligent_Lydmonitor_Systemtegning.pdf}{Overordnet virkemåde for Intelligent Lydmonitor}{IL_virmåde}

\newpage
\subsection{Hardware arkitektur}
\figur{1}{intlydmonitor/sysark/Intelligent_Lydmonitor_IBD.pdf}{IBD for Intelligent Lydmonitor}{IL_IBD}

\textbf{Intelligent Lydmonitor} består af tre dele: 
\begin{itemize}
\item \textbf{Mikrofon} optager signalet babyLyd, som er lyden Baby producerer. 
\item \textbf{Mic Preamp} modtager og forstærker signalet opLyd fra Mikrofon, hvilket er den lyd Mikrofon har optaget. 
\item \textbf{Blackfin} modtager forstærket lyd, forLyd, fra Mic Preamp. Blackfin analyserer lyden og kategoriserer denne, inden den via I2C sender kategoriseringen til Controller. 
\end{itemize}

\newpage
\subsection{Software arkitektur}
\figur{1}{intlydmonitor/sysark/Intelligent_Lydmonitor_SD.pdf}{SD for Intelligent Lydmonitor}{IL_SD}

Som det ses i sekvensdiagrammet, er der opstillet nogle klasser for Intelligent Lydmonitor. Diagrammet viser hvordan de forskellige klasser interagerer med hinanden, og i hvilken rækkefølge. 

\textbf{RecordingBuffer} er en buffer hvor en lydsample optaget af mikrofonen gemmes.
\textbf{Controller} er hovedklassen som styrer de andre klasser. Controller benytter først funktionen \textit{getRec()} til at hente en optagelse fra RecordingBuffer. Herefter sendes optagelsen til klassen \textbf{NoiseFilter} med funktionen \textit{filterRec()}. I NoiseFilter frafiltreres støj og den støjfrie lydsample returneres til Controller. Med funktionen \textit{analyzePitch()} sendes den filtrerede lydsample til klassen \textbf{PitchAnalyzer} som analyserer lydsamplen for pitch. Resultatet returneres til Controller. Resultatet sendes fra Controller til en buffer kaldet \textbf{ResultBuffer} hvor det gemmes. Dette gentages indtil ResultBuffer er fyldt. \\ \\
Når ResultBuffer er fyldt igangsætter Controller en kategorisering af resultaterne med funktionen \textit{categorizeRec()}. Klassen \textbf{RecordingCategorizer} henter med funktionen \textit{getResults()} et array med resultaterne fra ResultBuffer. Herefter kategoriserer RecordingCategorizer resultaterne, kategorien sendes så til Controller. Controller sender med \textit{sendMsg()} kategorien til klassen \textbf{I2Csocket} som er klassen der håndterer den videre forsendelse af kategorien. 

\figur{1}{intlydmonitor/sysark/Intelligent_Lydmonitor_UML.pdf}{UML for Intelligent Lydmonitor}{IL_UML}
Klasser og deres tilhørende funktioner samt parametre er specificeret i ovenstående UML for Intelligent Lydmonitor. 




\subsection*{Grænseflade til Controller}