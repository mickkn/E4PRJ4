\newpage
\section{Systemarkitektur}

I dette afsnit beskrives systemarkitekturen for den Intelligente lydmonitor.

\subsection*{Overordnet virkemåde}
Overordnet skal den Intelligent lydmonitor fungere som følger:
\begin{itemize}
	\item Babygråd detekteres med en mikrofon og forstærkes med en mikrofonforstærker inden det analoge signal konverteres til diskrete samples med codec AD1836 på Blackfin udviklingsboardet
	\item Den diskrete sample-sekvens forfiltreres. Sample-sekvensen lavpasfiltreres og nedsamples. Det forfiltrerede sample gemmes i bufferen
	\item Den filtrerede sample-sekvens i bufferen analyseres først for Sound Pressure Level (SPL) for at bedømme, hvorvidt der er lyd nok til at den efterfølgende FFT og udglatning kan betale sig
	\item Et evt. resultat af FFT'en kategoriseres som én af tre BABYCON-states, som beskrevet i \ref{kravspec:indledning_babycon_states} 
	\item Den fundne BABYCON-state holdes op mod en statistik over tidligere states, og den signifikante status findes.
	\item Det endelige kategoriseringsresultat sendes til Controller via TwoWireCom
\end{itemize}

\figur{1}{intlydmonitor/sysark/Intelligent_Lydmonitor_Systemtegning.pdf}{Overordnet virkemåde for Intelligent Lydmonitor}{IL_virmåde}

\newpage
\subsection{Hardware arkitektur}
I dette afsnit beskrives hardware arkitekturen for Intelligent lydmonitor. Den er skitseret IBD'et herunder:
\figur{0.7}{intlydmonitor/sysark/Intelligent_Lydmonitor_IBD.pdf}{IBD for Intelligent lydmonitor}{IL_IBD}

Som det ses af Figur \ref{IL_IBD} består \textbf{Intelligent lydmonitor} af tre dele: 
\begin{itemize}
\item \textbf{Mikrofon} optager signalet babyLyd, som er lyden babyen producerer
\item \textbf{Mic Preamp} modtager og forstærker signalet opLyd fra Mikrofon, hvilket er den lyd Mikrofon har optaget 
\item \textbf{Blackfin} modtager forstærket lyd, forLyd, fra Mic Preamp. Blackfin analyserer lyden og kategoriserer denne, inden den via to signalledere sender kategoriseringen til Controller
\end{itemize}

\subsection{Software arkitektur}
I dette afsnit beskrives software-arkitekturen for Intelligent lydmonitor. Det vil tage udgangspunkt i \ref{IL_virmåde} og består af et sekvensdiagram, et klassediagram og en efterfølgende funktionsbeskrivelse. Overordnet set kan arkitekturen beskrives som en pipeline-arkitektur med audio-indsamling i den ene side af samlebåndet og en BABYCON-status som produkt i enden.

\newpage
\subsubsection*{Sekvensdiagram}
Herunder ses sekvensdiagrammet for en komplet lyd-analyse i Intelligent lydmonitor.
\figur{1}{intlydmonitor/sysark/Intelligent_Lydmonitor_SD.pdf}{SD for Intelligent lydmonitor}{IL_SD}

\ref{IL_SD} viser interaktionen mellem Intelligent lydmonitors subklasser.

Klassen \textbf{PreFilter} står for forfiltreringen af sample-sekvensen. PreFilter har to funktioner, \textit{lowpass()} som lavpasfiltrerer sample-sekvensen, og  \textit{downsample()} som nedsampler sekvensen. Disse to danner tilsammem en decimering, der sænker samplefrekvensen og dermed letter den efterfølgende processering. Et højpasfilter var desuden del af klassen ved første udkast, men da midlingsfilteret, der senere benyttes på det genererede fft spektrum, har en naturlig indsvingning med højpaskarakter, kunne en højpasfiltrering i forfiltreringen undgås. PreFilter kalder efter sit gennemløb RecBuf's funktion \textit{setData()} for at gemme samplen.
\textbf{RecBuf} er en multiple buffer. Det er altså to buffere, hvilket muliggøre påfyldning af data i en buffer, mens aflæsning foregår af den anden buffer. Når RecBuf har indsamlet nye samples svarende til 1/24 sekund optagelse kaldes \textit{analyze()}. \textbf{Analyzer} analyserer nu data i RecBuf via funktionerne \textit{calcSPL()} og dernæst \textit{calcFFT}, hvis SPL er over threshold. Resultatet sendes til \textbf{Categorizer} via funktionen \textit{categorize}, der kategoriserer resultatet indenfor de tre BABYCON-statuser (1,2 og 3). Funktionen \textit{calcSignificans} i \textbf{Statistician} kaldes hernæst og den nyligt fundne BABYCON-status sammenlignes med tidligere statuser for at sænke risiko for en evt. fejlmelding. Var den analyserede SPL-værdi under et threshold, kaldes Statistician's calSignificanas() direkte med status BABYCON3. Til slut sendes det fundne BABYCON-niveau til Controller via \textbf{TwoWireCom}. 

\subsubsection*{Klassediagram}
Softwaren er oprindelig tænkt som klasser i C++. Dette skabte dog en del uforudsete problemer ved aflæsning af data fra ADC'en på Blackfin533. Derfor er implementeringen foretaget i C. UML-klassediagrammet er dog bibeholdt, da det giver et godt abstraktionsniveau. I klassediagrammet ses attributter og funktioner samt deres indbyrdes relationer, som er vist som en pointer. I virkeligheden kommer de blot til at kalde hinandens funktioner indbyrdes. 

\figur{1}{intlydmonitor/sysark/Intelligent_Lydmonitor_UML.pdf}{UML for Intelligent lydmonitor}{IL_UML}
Funktionsbeskrivelser findes i det efterfølgende afsnit. 

\subsubsection*{Funktionsbeskrivelse}

\textit{PreFilter} \\
\textbf{Ansvar:} Indeholder funktionaliteten til forfiltrering og nedsampling af den diskrete sampling-sekvens modtaget fra ADC'en.


\begin{center}
    \begin{tabular}{ | l | p{11,8cm} |}
    \hline
    \textbf{Funktion}	& \verb+void pf_initDecima(void) +						\\ \hline
    \textbf{Parametre} 	& Ingen 											\\ \hline
    \textbf{Returværdi}	& Ingen																				\\ \hline
    \textbf{Beskrivelse}	& Initialiserer FIR lavpasfilter og nedsampling med en faktor 2		\\ \hline
    \end{tabular}
\end{center}

\begin{center}
    \begin{tabular}{ | l | p{11,8cm} |}
    \hline
    \textbf{Funktion}	& \verb+void pf_sampleRdy(fract32 sample) +						\\ \hline
    \textbf{Parametre} 	& fract32 sample, diskret værdi som repræsenterer det momentane spændingsniveau på Blackfin 533 ADC indgang 	\\ \hline
    \textbf{Returværdi}	& Ingen 								\\ \hline
    \textbf{Beskrivelse}	& Kaldes ved erhvervelse af nyt sample. Decimerer input og kalder \verb+rb_storeData()+ med filtreret sample som parameter\\ \hline
    \end{tabular}
\end{center}

\textit{RecBuf} \\
\textbf{Ansvar:} For at kunne opsamle data, mens der bliver behandlet andet data, er der valgt at benytte multiple buffering. Der fyldes samples i en buffer til denne er fuld, hvorefter indholdet sættes til behandling. Mens behandlingen af den ene buffer foretages, fyldes samples i den anden buffer. Når denne er fuld gentages proceduren igen med første buffer.

\begin{center}
    \begin{tabular}{ | l | p{11,8cm} |}
    \hline
    \textbf{Funktion}	& \verb+void rb_init(void) +						\\ \hline
    \textbf{Parametre} 	& Ingen		\\ \hline
    \textbf{Returværdi}	& Ingen 								\\ \hline
    \textbf{Beskrivelse}& Her sættes attributter rb\_currActiveBuffer, rb\_index og rb\_ready til 0. Herudover initieres de to buffere og alle pladser sættes til 0 		\\ \hline
    \end{tabular}
\end{center} 

\begin{center}
    \begin{tabular}{ | l | p{11,8cm} |}
    \hline
    \textbf{Funktion}	& \verb+void storeData(fract32 sample ) +						\\ \hline
    \textbf{Parametre} 	& fract32 sample: Diskret værdi som repræsenterer det momentane spændingsniveau på Blackfin 533 ADC indgang efter filtrering	\\ \hline
    \textbf{Returværdi}	& Ingen 								\\ \hline
    \textbf{Beskrivelse}& Gemmer diskret sample i RecBuf med pladsen cg\_index		\\ \hline
    \end{tabular}
\end{center} 



\textit{Analyzer} \\
\textbf{Ansvar:} Analyzer indeholder analyse-funktionerne.

\begin{center}
    \begin{tabular}{ | l | p{11,8cm} |}
    \hline
    \textbf{Funktion}	& \verb+void an_analyze(bool curBufRdy) +			\\ \hline
    \textbf{Parametre} 	& bool curBufRdy, nummer på nuværende fyldt buffer i RecBuf-klassen.	\\ \hline
    \textbf{Returværdi}	& Ingen 													\\ \hline
    \textbf{Beskrivelse}& Sætter variablen bool og kalder funktionerne \verb+calcSPL()+. Hvis returneret SPL er under an\_splThresh, kaldes \verb+ss_calcSignificans()+ direkte med status BABYCON3. Hvis returneret SPL er over an\_splThresh anvendes funktionerne \verb+an_calcFFT()+, \verb+an_calcTPR()+ og \verb+an_smooth()+ inden \verb+cg_categorize()+ kaldes til slut	\\ \hline
    \end{tabular}
\end{center} 

\begin{center}
    \begin{tabular}{ | l | p{11,8cm} |}
    \hline
    \textbf{Funktion}	& \verb+int an_calcSPL(void) +						\\ \hline
    \textbf{Parametre} 	& Ingen		\\ \hline
    \textbf{Returværdi}	& float, SPL-værdi som dBV							\\ \hline
    \textbf{Beskrivelse}& Beregner Sound Pressure Level af samplesekvensen fra RecBuf \\ \hline
    \end{tabular}
\end{center} 

\begin{center}
    \begin{tabular}{ | l | p{11,8cm} |}
    \hline
    \textbf{Funktion}	& \verb+void an_calcFFT(void) +						\\ \hline
    \textbf{Parametre} 	& Ingen		\\ \hline
    \textbf{Returværdi}	& Ingen							\\ \hline
    \textbf{Beskrivelse}& Beregner frekvensspektret via en fast fourier transformation af samplesekvensen fra RecBuf. Frekvensspektret gemmes i fract32 arrayet an\_freqSpec\\ \hline
    \end{tabular}
\end{center}

\begin{center}
    \begin{tabular}{ | l | p{11,8cm} |}
    \hline
    \textbf{Funktion}	& \verb+void an_gain2dB(void) +						\\ \hline
    \textbf{Parametre} 	& Ingen		\\ \hline
    \textbf{Returværdi}	& Ingen							\\ \hline
    \textbf{Beskrivelse}& Konverterer frekvensspektret i arrayet an\_freqSpec fra gg til dB og gemmer det som arrayet float an\_freqSpecdB \\ \hline
    \end{tabular}
\end{center}  

\begin{center}
    \begin{tabular}{ | l | p{11,8cm} |}
    \hline
    \textbf{Funktion}	& \verb+void an_smooth(void) +						\\ \hline
    \textbf{Parametre} 	& Ingen		\\ \hline
    \textbf{Returværdi}	& Ingen							\\ \hline
    \textbf{Beskrivelse}& Udglatter frekvensspektret i arrayet an\_freqSpecdB vha. et rekursivt midlingsfilter og gemmer det som arrayet an\_freqSpecSmooth  \\ \hline
    \end{tabular}
\end{center}

\begin{center}
    \begin{tabular}{ | l | p{11,8cm} |}
    \hline
    \textbf{Funktion}	& \verb+void an_calcTPR(void) +						\\ \hline
    \textbf{Parametre} 	& Ingen		\\ \hline
    \textbf{Returværdi}	& Ingen							\\ \hline
    \textbf{Beskrivelse}& Beregner TPR (Tonal power ratio) for frekvensspektret an\_freqSpec  \\ \hline
    \end{tabular}
\end{center} 


\textit{Categorizer} \\
\textbf{Ansvar:} Categorizer står for BABYCON-kategorisering af resultaterne fra Analyzer.

\begin{center}
    \begin{tabular}{ | l | p{11,8cm} |}
    \hline
    \textbf{Funktion}	& \verb+void cg_init(void)+ \\ \hline
    \textbf{Parametre} 	& Ingen\\ \hline
    \textbf{Returværdi}	& Ingen	 								\\ \hline
    \textbf{Beskrivelse}& Sætter alle attributter til de korrekte værdier. Udregner værdier der skal bruges til at bestemme typen af indkommen lyd	\\ \hline
    \end{tabular}
\end{center}

\begin{center}
    \begin{tabular}{ | l | p{11,8cm} |}
    \hline
    \textbf{Funktion}	& \verb+void cg_categorize(void) +						\\ \hline
    \textbf{Parametre} 	& Ingen\\ \hline
    \textbf{Returværdi}	& Ingen	 								\\ \hline
    \textbf{Beskrivelse}& Kategoriserer BABYCON-niveau på baggrund af frekvensspektrum og TPR. Kalder Statisticians calcSignificans() med BABYCON-niveau		\\ \hline
    \end{tabular}
\end{center}

\begin{center}
    \begin{tabular}{ | l | p{11,8cm} |}
    \hline
    \textbf{Funktion}	& \verb+bool cg_checkBC1(void) +						\\ \hline
    \textbf{Parametre} 	& Ingen 								\\ \hline
    \textbf{Returværdi}	& bool, true hvis BABYCON1 	 								\\ \hline
    \textbf{Beskrivelse}& Returnerer true hvis BABYCON1		\\ \hline
    \end{tabular}
\end{center}

\begin{center}
    \begin{tabular}{ | l | p{11,8cm} |}
    \hline
    \textbf{Funktion}	& \verb+bool cg_checkBC2(void) +						\\ \hline
    \textbf{Parametre} 	& Ingen 								\\ \hline
    \textbf{Returværdi}	& bool, true hvis BABYCON2 	 								\\ \hline
    \textbf{Beskrivelse}& Returnerer true hvis BABYCON2		\\ \hline
    \end{tabular}
\end{center}



\textit{Statistician} \\
\textbf{Ansvar:} Statistician står for at beregne mest signifikante BABYCON-status. Giver besked til TwoWireCom. 

\begin{center}
    \begin{tabular}{ | l | p{11,8cm} |}
    \hline
    \textbf{Funktion}	& \verb+void ss_init(void) +						\\ \hline
    \textbf{Parametre} 	& Ingen		\\ \hline
    \textbf{Returværdi}	& Ingen	 								\\ \hline
    \textbf{Beskrivelse}& Sætter relevante attributter og initiere ss\_sigContainer med BABYCON3 som default	\\ \hline
    \end{tabular}
\end{center}

\begin{center}
    \begin{tabular}{ | l | p{11,8cm} |}
    \hline
    \textbf{Funktion}	& \verb+void calcSignificans(int bc) +						\\ \hline
    \textbf{Parametre} 	& int bc, BABYCON-niveau		\\ \hline
    \textbf{Returværdi}	& Ingen	 								\\ \hline
    \textbf{Beskrivelse}& Indsætter den modtagne BABYCON i FIFO-kø og beregner mest signifikante BABYCON-status, kalder TwoWireCom's send-funktion		\\ \hline
    \end{tabular}
\end{center}


\textit{TwoWireCom} \\
\textbf{Ansvar:} TwoWireCom står for at sende BABYCON-niveau til Controller jf. \ref{overordnet:graenseflade_IL}.

\begin{center}
    \begin{tabular}{ | l | p{11,8cm} |}
    \hline
    \textbf{Funktion}	& \verb+void tw_init(void) +						\\ \hline
    \textbf{Parametre} 	& Ingen			\\ \hline
    \textbf{Returværdi}	& Ingen	 								\\ \hline
    \textbf{Beskrivelse}& Sætter retning på GPIO pins og kalder tw\_send med BABYCON3 som default		\\ \hline
    \end{tabular}
\end{center}

\begin{center}
    \begin{tabular}{ | l | p{11,8cm} |}
    \hline
    \textbf{Funktion}	& \verb+void tw_send(int bc) +						\\ \hline
    \textbf{Parametre} 	& int bc: Nuværende BABYCON-niveau			\\ \hline
    \textbf{Returværdi}	& Ingen	 								\\ \hline
    \textbf{Beskrivelse}& Sender BABYCON-niveau til Controller		\\ \hline
    \end{tabular}
\end{center}


