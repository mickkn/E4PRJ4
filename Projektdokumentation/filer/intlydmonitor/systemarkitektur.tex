\newpage
\section{Systemarkitektur}

I dette afsnit beskrives systemarkitekturen for den intelligente lydmonitor.

\subsection*{Overordnet virkemåde}
Overordnet skal den intelligente lydmonitor fungere som følger:
\begin{itemize}
	\item Babyens gråd detekteres med en mikrofon og forstærkes med en mikrofonforstærker inden det analoge signal konverteres til diskrete samples med Blackfin's ADC
	\item Den diskrete sample-sekvens forfiltreres. Sample-sekvensen lavpasfiltreres og nedsamples for herefter at blive højpasfiltreret. Det forfiltrerede sample gemmes i bufferen, der skal holde 5 sekunders optagelse.
	\item Den filtrerede sample-sekvens i bufferen analyseres først for Sound Pressure Level (SPL) for at bedømme hvorvidt der er lyd nok til at den efterfølgende FFT kan betale sig. 
	\item Et evt. resultat af FFT'en kategoriseres som én af tre BABYCON-states, som beskrevet i \ref{kravspec:indledning_babycon_states} 
	\item Den fundne BABYCON state holdes op mod en statistik over tidligere states for at sikre mod evt. fejl-state.
	\item Det endelige kategoriseringsresult sendes til Controller via TwoWireCom
\end{itemize}

\figur{1}{intlydmonitor/sysark/Intelligent_Lydmonitor_Systemtegning.pdf}{Overordnet virkemåde for Intelligent Lydmonitor}{IL_virmåde}

\newpage
\subsection{Hardware arkitektur}
I dette afsnit beskrives hardware arkitekturen for Intelligent Lydmonitor. Den er skitseret IBD'et herunder:
\figur{1}{intlydmonitor/sysark/Intelligent_Lydmonitor_IBD.pdf}{IBD for Intelligent Lydmonitor}{IL_IBD}

Som det ses af Figur \ref{IL_IBD} består \textbf{Intelligent Lydmonitor} af tre dele: 
\begin{itemize}
\item \textbf{Mikrofon} optager signalet babyLyd, som er lyden Baby producerer. 
\item \textbf{Mic Preamp} modtager og forstærker signalet opLyd fra Mikrofon, hvilket er den lyd Mikrofon har optaget. 
\item \textbf{Blackfin} modtager forstærket lyd, forLyd, fra Mic Preamp. Blackfin analyserer lyden og kategoriserer denne, inden den via to signalledere sender kategoriseringen til Controller. 
\end{itemize}

\newpage
\subsection{Software arkitektur}
I dette afsnit beskrives software-arkitekturen for Intelligent Lydmonitor. Det vil tage udgangspunkt i \ref{IL_virmåde} og består af et sekvensdiagram, et klassediagram og en efterfølgende funktionsbeskrivelser.

\subsubsection*{Sekvensdiagram}
Herunder ses sekvensdiagrammet for en komplet lyd-analyse i Intelligent Lydmonitor.
\figur{1}{intlydmonitor/sysark/Intelligent_Lydmonitor_SD.pdf}{SD for Intelligent Lydmonitor}{IL_SD}

\ref{IL_SD} viser interaktionen mellem Intelligent Lydmonitors subklasser.

Klassen \textbf{PreFilter} står for forfiltreringen af sample-sekvensen. PreFilter har tre funktioner, \textit{lowpass()} som lavpasfiltrerer sample-sekvensen, \textit{downsample()} som nedsampler sekvensen og \textit{highpass()} som højpasfiltrerer den nedsamplede sekvens af hensyn til lavfrekvent støj fra vind og vejr. PreFilter kalder efter sit gennemløb RecBuf's funktion \textit{set()} for at gemme samplen.
\textbf{RecBuf} er cirkulær buffer med mulighed for midlertidig locking af bufferområder. Det er altså en buffer der muliggør påfyldning af data ét sted mens aflæsning foregår fra et låst segment. Se \ref{intlyd:sw:circ_buf} for yderligere beskrivelse. Når RecBuf har indsamlet nye samples svarende til 5 sekunders optagelse kaldes \textit{analyze()}. \textbf{Analyzer} analyserer nu data i RecBuf via funktionerne \textit{calcSPL()} og dernæst \textit{calcFFT}, hvis SPL er over threshold. Resultatet sendes til \textbf{Categorizer} via funktionen \textit{categorize}, der kategoriserer resultatet indenfor de tre BABYCON niveauer (1,2 og 3). Funktionen \textit{calcSignificans} i \textbf{Statistician} kaldes hernæst og det nyligt fundne BABYCON-niveau sammenlignes med tidligere niveauer for at finde sænke risiko for en evt. fejlmelding. Til slut sendes det fundne BABYCON-niveau til Controller via \textbf{TwoWireCom}. 

\subsubsection*{Klassediagram}
I UML-klassediagrammet herunder ses en oversigt over system-klassernes attributter og funktioner samt deres indbyrdes relationer. 
\figur{1}{intlydmonitor/sysark/Intelligent_Lydmonitor_UML.pdf}{UML for Intelligent Lydmonitor}{IL_UML}
Funktionsbeskrivelser findes i det efterfølgende afsnit. 

\subsubsection*{Funktionsbeskrivelse}

\textit{PreFilter} \\
\textbf{Ansvar:} Klassen indeholder funktionaliteten til forfiltrering og nedsampling af den diskrete sampling-sekvens.


\begin{center}
    \begin{tabular}{ | l | p{11,8cm} |}
    \hline
    \textbf{Funktion}	& \verb+void init(int lpFreq, int hpFreq, int factor, int *curSample) +						\\ \hline
    \textbf{Parametre} 	& int lpFreq knækfrekvens for LP-filter,\\&
    						  int hpFreq knækfrekvens for HP-filter,\\&
    						  int factor nedsamplingsfaktor, \\&
    						  int *curSample pointer til placering af nyeste sample		\\ \hline
    \textbf{Returværdi}	& Ingen																				\\ \hline
    \textbf{Beskrivelse}	& Initialiserer filtre og downsampling, ved at kalde deres init-funktioner		\\ \hline
    \end{tabular}
\end{center}


\begin{center}
    \begin{tabular}{ | l | p{11,8cm} |}
    \hline
    \textbf{Funktion}	& \verb+void sampleRdy(void) +						\\ \hline
    \textbf{Parametre} 	& Ingen		\\ \hline
    \textbf{Returværdi}	& Ingen 								\\ \hline
    \textbf{Beskrivelse}	& Kaldes ved erhvervelse af nyt sample. Kalder funktionerne lowpass(), highpass() og downsample()		\\ \hline
    \end{tabular}
\end{center}

\begin{center}
    \begin{tabular}{ | l | p{11,8cm} |}
    \hline
    \textbf{Funktion}	& \verb+void initLP(int freqLP) +						\\ \hline
    \textbf{Parametre} 	& int lpFreq knækfrekvens for LP-filter		\\ \hline
    \textbf{Returværdi}	& Ingen 								\\ \hline
    \textbf{Beskrivelse}	& Initialiserer LP-filter		\\ \hline
    \end{tabular}
\end{center}

\begin{center}
    \begin{tabular}{ | l | p{11,8cm} |}
    \hline
    \textbf{Funktion}	& \verb+void initHP(int freqHP) +						\\ \hline
    \textbf{Parametre} 	& int hpFreq knækfrekvens for HP-filter		\\ \hline
    \textbf{Returværdi}	& Ingen 								\\ \hline
    \textbf{Beskrivelse}	& Initialiserer HP-filter		\\ \hline
    \end{tabular}
\end{center}

\begin{center}
    \begin{tabular}{ | l | p{11,8cm} |}
    \hline
    \textbf{Funktion}	& \verb+void initDown(int factor) +						\\ \hline
    \textbf{Parametre} 	& int factor faktor for nedsampling		\\ \hline
    \textbf{Returværdi}	& Ingen 								\\ \hline
    \textbf{Beskrivelse}	& Initialiserer downsampling		\\ \hline
    \end{tabular}
\end{center}

\begin{center}
    \begin{tabular}{ | l | p{11,8cm} |}
    \hline
    \textbf{Funktion}	& \verb+int lowpass(int sample) +						\\ \hline
    \textbf{Parametre} 	& Int sample, sample til LP-filtrering		\\ \hline
    \textbf{Returværdi}	& int, værdi efter filtrering 								\\ \hline
    \textbf{Beskrivelse}& Lavpasfiltrerer sample		\\ \hline
    \end{tabular}
\end{center}

\begin{center}
    \begin{tabular}{ | l | p{11,8cm} |}
    \hline
    \textbf{Funktion}	& \verb+int highpass(int sample) +						\\ \hline
    \textbf{Parametre} 	& Int sample, sample til HP-filtrering		\\ \hline
    \textbf{Returværdi}	& int, værdi efter filtrering 								\\ \hline
    \textbf{Beskrivelse}& Højpasfiltrerer sample		\\ \hline
    \end{tabular}
\end{center}

\begin{center}
    \begin{tabular}{ | l | p{11,8cm} |}
    \hline
    \textbf{Funktion}	& \verb+bool downsample(void) +						\\ \hline
    \textbf{Parametre} 	& Ingen		\\ \hline
    \textbf{Returværdi}	&  bool, 1 hvis det givne sample skal beholdes og 0 hvis det skal frasorteres 								\\ \hline
    \textbf{Beskrivelse}& Frasorterer samples jf. valgt nedsampling		\\ \hline
    \end{tabular}
\end{center}


\textit{PingPongBuf} \\
\textbf{Ansvar:} For at kunne opsamle data, mens at der bliver behandlet andet data, er det valgt at benytte pingpong-buffering. Dermed er det unødvendigt at kopiere bufferen når denne er fyldt. I stedet sendes til den klasse, som skal behandle data, en pointer til den fyldte buffer. Mens data behandles, kan nyt data fyldes i den anden buffer.

\begin{center}
    \begin{tabular}{ | l | p{11,8cm} |}
    \hline
    \textbf{Funktion}	& \verb+void store(int sample ) +						\\ \hline
    \textbf{Parametre} 	& int sample: Diskret værdi som repræsenterer det momentane spændingsniveau på Blackfin 533 ADC indgang		\\ \hline
    \textbf{Returværdi}	& Ingen 								\\ \hline
    \textbf{Beskrivelse}& Gemmer diskret sample i PingPongBuf. Hvis bufToFill er sat til 0 påfyldes data i atributten buf0. Ellers fyldes data i buf1		\\ \hline
    \end{tabular}
\end{center} 

\begin{center}
    \begin{tabular}{ | l | p{11,8cm} |}
    \hline
    \textbf{Funktion}	& \verb+void handleFullBuf(void) = 0 +						\\ \hline
    \textbf{Parametre} 	& Ingen		\\ \hline
    \textbf{Returværdi}	& Ingen 								\\ \hline
    \textbf{Beskrivelse}& Virtuel funktion		\\ \hline
    \end{tabular}
\end{center} 



\textit{RecBuf} \\
\textbf{Ansvar:} RecBuf er en afledt klasse af PingPongBuf. Data i RecBuf benyttes af Analyzer.

\begin{center}
    \begin{tabular}{ | l | p{11,8cm} |}
    \hline
    \textbf{Funktion}	& \verb+void handleFullBuf(void) +						\\ \hline
    \textbf{Parametre} 	& Ingen		\\ \hline
    \textbf{Returværdi}	& Ingen 								\\ \hline
    \textbf{Beskrivelse}& Kalder klassen Analyzers funktion bufRdy(int[ ][ ]*) med en pointer til den buffer som er fyldt. Herudover toggles bufToFill		\\ \hline
    \end{tabular}
\end{center} 


\textit{ResBuf} \\
\textbf{Ansvar:} ResBuf er en afledt klasse af PingPongBuf. Data i ResBuf benyttes af Categorizer til at bestemme hvilket BABYCON-niveau vi befinder os på. 

\begin{center}
    \begin{tabular}{ | l | p{11,8cm} |}
    \hline
    \textbf{Funktion}	& \verb+void handleFullBuf(void) +						\\ \hline
    \textbf{Parametre} 	& Ingen		\\ \hline
    \textbf{Returværdi}	& Ingen 								\\ \hline
    \textbf{Beskrivelse}& Kalder klassen Categorizers funktion bufRdy(int[ ][ ]*) med en pointer til den buffer som er fyldt. Herudover toggles bufToFill		\\ \hline
    \end{tabular}
\end{center} 


\textit{Analyzer} \\
\textbf{Ansvar:} Klassen Analyzer indeholder analyse-funktionerne for beregning af Tonal Power Ratio og Sound Pressure level. 

\begin{center}
    \begin{tabular}{ | l | p{11,8cm} |}
    \hline
    \textbf{Funktion}	& \verb+void analyze(int[ ][ ]* recBufPtr) +						\\ \hline
    \textbf{Parametre} 	& int[ ][ ]* recBufPtr, pointer til RecBuf		\\ \hline
    \textbf{Returværdi}	& Ingen 								\\ \hline
    \textbf{Beskrivelse}& Gemmer pointeren til RecBuf kalder funktionerne calcTPR() og calcSPL()		\\ \hline
    \end{tabular}
\end{center} 

\begin{center}
    \begin{tabular}{ | l | p{11,8cm} |}
    \hline
    \textbf{Funktion}	& \verb+int calcTPR(void) +						\\ \hline
    \textbf{Parametre} 	& Ingen		\\ \hline
    \textbf{Returværdi}	& int, værdi af TPR 								\\ \hline
    \textbf{Beskrivelse}& Beregner Tonal Power Ratio		\\ \hline
    \end{tabular}
\end{center} 

\begin{center}
    \begin{tabular}{ | l | p{11,8cm} |}
    \hline
    \textbf{Funktion}	& \verb+int calcSPL(void) +						\\ \hline
    \textbf{Parametre} 	& Ingen		\\ \hline
    \textbf{Returværdi}	& int, værdi af SPL 								\\ \hline
    \textbf{Beskrivelse}& Beregner Sound Pressure Level		\\ \hline
    \end{tabular}
\end{center} 


\textit{Categorizer} \\
\textbf{Ansvar:} Klassen Categorizer står for BABYCON kategorisering af resultaterne fra ResBuf

\begin{center}
    \begin{tabular}{ | l | p{11,8cm} |}
    \hline
    \textbf{Funktion}	& \verb+void init(int threshTPRmax,+ \\ &
    						 \verb+int threshTPRavg int threshSPLmax, int threshSPLavg) +						\\ \hline
    \textbf{Parametre} 	& int threshTPRmax, int threshTPRavg, int threshSPLmax, int threshSPLavg		\\ \hline
    \textbf{Returværdi}	& Ingen	 								\\ \hline
    \textbf{Beskrivelse}& Sætter threshholds for TPR og SPL		\\ \hline
    \end{tabular}
\end{center}

\begin{center}
    \begin{tabular}{ | l | p{11,8cm} |}
    \hline
    \textbf{Funktion}	& \verb+void categorize(int[ ][ ]* resBufPtr) +						\\ \hline
    \textbf{Parametre} 	& int[ ][ ]* resBufPtr, pointer til ResBuf		\\ \hline
    \textbf{Returværdi}	& Ingen	 								\\ \hline
    \textbf{Beskrivelse}& Gemmer pointeren til ResBuf, kalder funktionerne smooth(), catTPR(),catSPL() og compThresh()		\\ \hline
    \end{tabular}
\end{center}

\begin{center}
    \begin{tabular}{ | l | p{11,8cm} |}
    \hline
    \textbf{Funktion}	& \verb+void smooth(void) +						\\ \hline
    \textbf{Parametre} 	& Ingen		\\ \hline
    \textbf{Returværdi}	& Ingen	 								\\ \hline
    \textbf{Beskrivelse}& Udglatter data-sekvensen gemt i ResBuf		\\ \hline
    \end{tabular}
\end{center}

\begin{center}
    \begin{tabular}{ | l | p{11,8cm} |}
    \hline
    \textbf{Funktion}	& \verb+void catTPR(int* maxTPR, int* avgTPR) +						\\ \hline
    \textbf{Parametre} 	& int* maxTPR pointer til maksimum TPR\\&
    						  int* avgTPR pointer til average TPR			\\ \hline
    \textbf{Returværdi}	& Ingen	 								\\ \hline
    \textbf{Beskrivelse}& Finder maksimum og average Tonal Power Ratio værdi		\\ \hline
    \end{tabular}
\end{center}

\begin{center}
    \begin{tabular}{ | l | p{11,8cm} |}
    \hline
    \textbf{Funktion}	& \verb+void catSPL(int* maxSPL, int* avgSPL) +						\\ \hline
    \textbf{Parametre} 	& int* maxSPL pointer til maksimum SPL\\&
    						  int* avgSPL pointer til average SPL			\\ \hline
    \textbf{Returværdi}	& Ingen	 								\\ \hline
    \textbf{Beskrivelse}& Finder maksimum og average Sound Pressure Level værdi		\\ \hline
    \end{tabular}
\end{center}

\begin{center}
    \begin{tabular}{ | l | p{11,8cm} |}
    \hline
    \textbf{Funktion}	& \verb+int compThresh(int maxTPR, int avgTPR, int maxSPL, avgSPL) +						\\ \hline
    \textbf{Parametre} 	& int maxTPR, int avgTPR, int maxSPL, avgSPL			\\ \hline
    \textbf{Returværdi}	& int, BABYCON-niveau mellem 1-3 								\\ \hline
    \textbf{Beskrivelse}& Sammenligner modtagne værdier med foruddefinerede thresholds, for at bestemme BABYCON-niveau		\\ \hline
    \end{tabular}
\end{center}


\textit{TwoWireCom} \\
\textbf{Ansvar:} Klassen TwoWireCom står for at sende BABYCON-niveau til Controller jf. \ref{overordnet:graenseflade_IL}

\begin{center}
    \begin{tabular}{ | l | p{11,8cm} |}
    \hline
    \textbf{Funktion}	& \verb+void send(int bc) +						\\ \hline
    \textbf{Parametre} 	& int bc: Nuværende BABYCON-niveau			\\ \hline
    \textbf{Returværdi}	& Ingen	 								\\ \hline
    \textbf{Beskrivelse}& Sender BABYCON-niveau til Controller		\\ \hline
    \end{tabular}
\end{center}


