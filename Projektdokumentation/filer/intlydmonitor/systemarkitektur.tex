\newpage
\section{Systemarkitektur}

I dette afsnit beskrives systemarkitekturen for den intelligente lydmonitor.

\subsection*{Overordnet virkemåde}
Overordnet skal den intelligente lydmonitor fungere som følger:
\begin{itemize}
	\item Babyens gråd detekteres med en mikrofon og forstærkes med en mikrofonforstærker inden det analoge signal konverteres til diskrete samples med Blackfin's ADC
	\item Den diskrete sample-sekvens forfiltreres. Sample-sekvensen lavfiltreres og nedsamples for herefter at blive højfiltreret. 
	\item Den filtrerede sample-sekvens i bufferen analyseres først for Tonal Power Ratio for herefter at blive analyseret for Sound Pressure Level (SPL). Resultatet heraf gemmes i endnu en buffer med analyseresultater.
	\item Disse analyseresultater kategoriseres som tre BABYCON-states, som beskrevet i \ref{kravspec:indledning_babycon_states} 
	\item Kategoriseringsresultatet sendes til Controller via Two wire interface
\end{itemize}

\figur{1}{intlydmonitor/sysark/Intelligent_Lydmonitor_Systemtegning.pdf}{Overordnet virkemåde for Intelligent Lydmonitor}{IL_virmåde}

\newpage
\subsection{Hardware arkitektur}
\figur{1}{intlydmonitor/sysark/Intelligent_Lydmonitor_IBD.pdf}{IBD for Intelligent Lydmonitor}{IL_IBD}

\textbf{Intelligent Lydmonitor} består af tre dele: 
\begin{itemize}
\item \textbf{Mikrofon} optager signalet babyLyd, som er lyden Baby producerer. 
\item \textbf{Mic Preamp} modtager og forstærker signalet opLyd fra Mikrofon, hvilket er den lyd Mikrofon har optaget. 
\item \textbf{Blackfin} modtager forstærket lyd, forLyd, fra Mic Preamp. Blackfin analyserer lyden og kategoriserer denne, inden den via to signalledere sender kategoriseringen til Controller. 
\end{itemize}

\newpage
\subsection{Software arkitektur}
\subsubsection*{Sekvensdiagram}
\figur{1}{intlydmonitor/sysark/Intelligent_Lydmonitor_SD.pdf}{SD for Intelligent Lydmonitor}{IL_SD}

Som det ses i sekvensdiagrammet, er der opstillet nogle klasser for Intelligent Lydmonitor. Diagrammet viser hvordan- og i hvilken rækkefølge de forskellige klasser interagerer med hinanden. 

\textbf{RecordingBuffer} er en buffer hvor de diskrete lydsamples gemmes.
\textbf{Main} er hovedklassen som styrer de andre klasser. Main benytter først funktionen \textit{getRec()} til at hente en optagelse fra RecordingBuffer. Herefter sendes optagelsen til klassen \textbf{NoiseFilter} med funktionen \textit{filterRec()}. I NoiseFilter frafiltreres støj og den støjfrie samplesekvens returneres til Main. Med funktionen \textit{analyzePitch()} sendes den filtrerede samplesekvens til klassen \textbf{PitchAnalyzer} som analyserer sekvensen for dominerende toneindhold. Resultatet returneres til Main og sendes herfra til en buffer kaldet \textbf{ResultBuffer} hvor det gemmes. Dette gentages indtil ResultBuffer er fyldt. \\ \\
Når ResultBuffer er fyldt igangsætter Main en kategorisering af resultaterne med funktionen \textit{categorizeRec()}. Klassen \textbf{RecordingCategorizer} henter med funktionen \textit{getResults()} et array med resultaterne fra ResultBuffer. Herefter kategoriserer RecordingCategorizer resultaterne, kategorien sendes så til Main. Main sender med \textit{sendMsg()} kategorien til klassen \textbf{I2Csocket} som er klassen der håndterer den videre forsendelse af kategorien. 

\subsubsection*{Klassediagram}
\figur{1}{intlydmonitor/sysark/Intelligent_Lydmonitor_UML.pdf}{UML for Intelligent Lydmonitor}{IL_UML}
Klasser og deres tilhørende funktioner samt parametre er specificeret i ovenstående UML for Intelligent Lydmonitor. 

\subsubsection*{Funktionsbeskrivelse}
\textit{Main} \\
Main indeholder ingen funktioner.

\textit{RecordingBuffer} \\

\begin{center}
    \begin{tabular}{ | l | p{10cm} |}
    \hline
    \textbf{Funktion}	 	& void storeSample(int sample)								\\ \hline
    \textbf{Beskrivelse} 	& Gemmer diskret sample i RecordingBuffer					\\ \hline
    \textbf{Parametre}		& int Sample: Diskret værdi som repræsenterer det momentane spændingsniveau på Blackfin 533 ADC indgang														 		\\ \hline
    \textbf{Returværdi} 	& Ingen		 												\\ \hline
    \end{tabular}
\end{center}

\begin{center}
    \begin{tabular}{ | l | p{10cm} |}
    \hline
    \textbf{Funktion}	 	& wav getRec(void)										\\ \hline
    \textbf{Beskrivelse} 	& Returnerer indhold af RecordingBuffer som wav-fil		\\ \hline
    \textbf{Parametre}		& Ingen			 										\\ \hline
    \textbf{Returværdi} 	& wav		 											\\ \hline
    \end{tabular}
\end{center}

\textit{NoiseFilter} \\
\begin{center}
    \begin{tabular}{ | l | p{10cm} |}
    \hline
    \textbf{Funktion}	 	& wav filterRec(wav rec)								\\ \hline
    \textbf{Beskrivelse} 	& Frafiltrerer støj fra optagelse						\\ \hline
    \textbf{Parametre}		& wav rec: Ufiltreret optagelse		 					\\ \hline
    \textbf{Returværdi} 	& wav: Filtreret optagelse							\\ \hline
    \end{tabular}
\end{center}

\textit{PitchAnalyzer} \\
\begin{center}
    \begin{tabular}{ | l | p{10cm} |}
    \hline
    \textbf{Funktion}	 	& uint signalPitch analyzePitch(wav rec)						\\ \hline
    \textbf{Beskrivelse} 	& Analyserer optagelse for dominant tone					    \\ \hline
    \textbf{Parametre}		& wav rec: Optagelse						 					\\ \hline
    \textbf{Returværdi} 	& uint: Dominant tone i Hz										\\ \hline
    \end{tabular}
\end{center}

\textit{ResultBuffer} \\
\begin{center}
    \begin{tabular}{ | l | p{10cm} |}
    \hline
    \textbf{Funktion}	 	& void saveAnalysis(uint signalPitch)								\\ \hline
    \textbf{Beskrivelse} 	& Gemmer signalPitch i ResultBuffer									\\ \hline
    \textbf{Parametre}		& uint signalPitch: Dominant tone i Hz		 					    \\ \hline
    \textbf{Returværdi} 	& Ingen																\\ \hline
    \end{tabular}
\end{center}

\begin{center}
    \begin{tabular}{ | l | p{10cm} |}
    \hline
    \textbf{Funktion}	 	& uint[ ] getResults(void)								\\ \hline
    \textbf{Beskrivelse} 	& Returnerer indhold af ResultBuffer som et array		\\ \hline
    \textbf{Parametre}		& Ingen							 					    \\ \hline
    \textbf{Returværdi} 	& uint[ ]: array af analyseresultater 					\\ \hline
    \end{tabular}
\end{center}

\textit{RecordingCategorizer} \\
\begin{center}
    \begin{tabular}{ | l | p{10cm} |}
    \hline
    \textbf{Funktion}	 	& int categorizeRec(void)								\\ \hline
    \textbf{Beskrivelse} 	& Henter signalPitches fra ResultBuffer og kalder funktionen categorize()					\\ \hline
    \textbf{Parametre}		& Ingen							 					    \\ \hline
    \textbf{Returværdi} 	& int: Værdi svarende til BABYCON-niveau, 1-3			\\ \hline
    \end{tabular}
\end{center}

\begin{center}
    \begin{tabular}{ | l | p{10cm} |}
    \hline
    \textbf{Funktion}	 	& int categorize(uint[ ] signalPitches)								\\ \hline
    \textbf{Beskrivelse} 	& Returnerer kategorisering af resultater				\\ \hline
    \textbf{Parametre}		& uint[ ] signalPitches: Array af signalPitches							 					    \\ \hline
    \textbf{Returværdi} 	& int: Værdi svarende til BABYCON-niveau, 1-3			\\ \hline
    \end{tabular}
\end{center}

\textit{I2Csocket} \\
\begin{center}
    \begin{tabular}{ | l | p{10cm} |}
    \hline
    \textbf{Funktion}	 	& void storeBabycon(int babycon)								\\ \hline
    \textbf{Beskrivelse} 	& Gemmer det kategoriserede BABYCON-niveau						\\ \hline
    \textbf{Parametre}		& int babycon: Nuværende BABYCON-niveau							 					    \\ \hline
    \textbf{Returværdi} 	& Ingen														     \\ \hline
    \end{tabular}
\end{center}
