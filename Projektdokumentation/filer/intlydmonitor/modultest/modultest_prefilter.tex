%Modultest Prefilter
\subsection{Modultest Prefilter}

\textbf{Formål} \\
Testens formål er at eftervise korrekt decimeringen, herunder nedsampling fra 48kHz til 24kHz og lavpasfiltrering tilstrækkelig til at dæmpe frekvenser over nyquist-frekvensen, 12kHz, med 50dB. 

\textbf{Opstilling til test af Prefilter test}

\begin{itemize}
	\item Blackfin 533 er sluttet til 230VAC
	\item Blackfin 533 er tilsuttet PC via USB-port
	\item CrossCore Embedded Studio (CCES) er installeret på PC
	\item Analog Discovery er tilsluttet Blackfin 533's Audio left\_in\_channel0 og left\_out\_channel0
\end{itemize}

\textbf{Testprocedure}
\begin{enumerate}
	\item Projektet med Prefilter testen eksekveres med CCES 
	\item Prefilter sættes til bypass-mode
	\item Digilent Waveforms Network analyzer sættes til at generere et frekvenssweep fra 10Hz til 50kHz på Analog Discovery. 
	\item Bodeplottet aflæses i Network analyzer. 
	\item Prefilter sættes aktiv
	\item Pkt. 3 og 4 gentages
\end{enumerate}

\textbf{Forventet resultat} \\
Ved aktivering af Prefilter forventes en dæmpning på 50dB for frekvenser over 12 kHz. Det forventes at se en knækfrekvens for lavpasfilteret ved 10 kHz.

\textbf{Resultat} \\
Herunder ses bodeplots genereret af Waveform's netværksanalyse:
\figur{1}{intlydmonitor/modultest/prefilter_bode_bypass}{Målt bodeplot for Prefilter i bypass-mode}{pf_bypass}
\figur{1}{intlydmonitor/modultest/prefilter_bode_active}{Målt bodeplot for Prefilter i active-mode}{pf_active}

Ved aktivering af Prefilter, ses en dæmpning af frekvenser over 8 kHz med godt 53dB. Imidlertid findes et frekvens-peak omkring 20kHz. Afvigelsen fra den forventede knækfrekvens og magnitudeboblen ved 20kHz skyldes formodentlig en kvantiseringsfejl af filter-koefficienterne. Der forefindes ikke vigtige analysepunkter for bestemmelse af babygråd over 6kHz. Desuden er båndbredden  af den anvendte mikrofon, MCE100, 50Hz-10kHz, og det vil derfor være et minimum af signaler i 20kHz området. Den opnåede filtrering regnes derfor som tilstrækkelig

\textit{Testen er godkendt}





