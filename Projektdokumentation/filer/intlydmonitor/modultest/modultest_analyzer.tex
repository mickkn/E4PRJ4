%Modultst analyzer
\subsection{Modultest Analyzer}

\textbf{Formål} \\
Formålet med at teste Analyzer er at tjekke at de individuelle funktioner i Analyzer fungerer som de skal. Ved test af Analyzer benyttes en fuld RecBuf med data læst med ADC'en og filtreret med Prefiltret. 
Disse funktioner er:
\begin{itemize}
	\item Beregning af Sound Pressure Level
	\item Beregning af Fast Fourier Transformation
	\item Beregning af Tonal Power Ratio
	\item Omregning fra gain til dB
	\item Smoothing af FFT-spektrum
\end{itemize}

\textbf{Opstilling til test af Analyzer}

\begin{itemize}
	\item Blackfin 533 er sluttet til 230VAC
	\item Blackfin 533 er tilsuttet PC via COM-port
	\item CCES er kørende på PC
	\item En sinus-generator er tilsluttet Blackfin 533's Audio In left\_channel0
\end{itemize}

\textbf{Testprocedure for SPL}
\begin{enumerate}
	\item Den tilkoblede sinusgenerator generere en sinus med amplituden 1V og frekvensen 880Hz.	
	\item RecBuf fyldes med data læst med ADC'en og filtreret med Prefiltret.
	\item Projektet med Analyzer testen eksekveres 
	\item CCES memory browser benyttes til at aflæse variablen an\_spl. 
	\item Testen gentages fra punkt 1 dog med sinus med en amplitude på 0.5V.
\end{enumerate}

\textbf{Forventet resultat} \\
Det forventes at magnituden er 50dB mindre end for sinustonen. 


\textbf{Resultat} \\


På Figur \ref{FirstRecBufTest} ses FFT'en for de Prefitrerede samples 

\figur{0.3}{intlydmonitor/modultest/First_output}{Første udskrift i kommandoprompten}{FirstRecBufTest}




\textit{Testen er godkendt}





