%Modultst analyzer
\subsection{Modultest Analyzer}

\textbf{Formål} \\
De individuelle analyse-funktioner i Analyzer skal valideres. Disse funktioner er:
\begin{itemize}
	\item Beregning af Sound Pressure Level
	\item Beregning af Fast Fourier Transformation
	\item Beregning af Tonal Power Ratio
	\item Omregning fra gain til dB
	\item Smoothing af FFT-spektrum
\end{itemize}

\textbf{Opstilling til test af Analyzer}
\begin{itemize}
	\item Blackfin 533 er sluttet til 230VAC
	\item Blackfin 533 er tilsuttet PC via COM-port
	\item CrossCore Embedded Studio (CCES) er kørende på PC
	\item En sinus-generator er tilsluttet Blackfin 533's Audio In left\_channel0
\end{itemize}

\textit{\textbf{Test af Sound Pressure Level}}\\
\textbf{Testprocedure}
\begin{enumerate}
	\item Den tilkoblede sinusgenerator genererer en sinus med amplituden 1V og frekvensen 880Hz.	
	\item Projektet med Analyzer testen eksekveres 
	\item RecBuf fyldes med data læst fra ADC'en og filtreret med Prefiltret.
	\item Projektet køres indtil breakpoint, sat umiddelbart efter benyttelsen af funktionen \verb+an_calcSPl()+
	\item CCES memory browser benyttes til at aflæse variablen an\_spl. 
	\item Testen punkt 1-5 gentages med en sinus med en amplitude 100mV.
\end{enumerate}

\textbf{Forventet resultat} \\
Ved input-sinus med amplituden 1V forventes at variablen \verb+an_spl+ er 0dBV +/- 1dBV. Ved amplituden 100mV forventes en værdi på -20dBV +/- 1dBV.

\textbf{Resultat} \\

%\figur{0.3}{intlydmonitor/modultest/First_output}{Første udskrift i kommandoprompten}{FirstRecBufTest}

\textit{Testen er godkendt}





