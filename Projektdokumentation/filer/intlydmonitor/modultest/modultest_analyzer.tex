%Modultst analyzer
\subsection{Modultest Analyzer}

\textbf{Formål} \\
De individuelle analyse-funktioner i Analyzer skal valideres. Disse funktioner er:
\begin{itemize}
	\item Beregning af Sound Pressure Level
	\item Beregning af Fast Fourier Transformation
	\item Beregning af Tonal Power Ratio
	\item Omregning fra gain til dB
	\item Smoothing af FFT-spektrum
\end{itemize}

\textbf{Opstilling til test af Analyzer}
\begin{itemize}
	\item Blackfin 533 er sluttet til 230VAC
	\item Blackfin 533 er tilsuttet PC via COM-port
	\item CrossCore Embedded Studio (CCES) er kørende på PC
	\item En sinus-generator er tilsluttet Blackfin 533's Audio In left\_channel0
\end{itemize}

\textbf{\textit{Test af Sound Pressure Level}}\\
\textbf{SPL: Testprocedure}
\begin{enumerate}
	\item Den tilkoblede sinusgenerator genererer en sinus med amplituden 1V og frekvensen 880Hz	
	\item Projektet med Analyzer testen startes i debug-mode
	\item Et breakpoint sættes i "Analyzer.c", linje 36 
	\item Der klikkes på "resume"
	\item Projektet køres indtil breakpoint
	\item CCES memory browser benyttes til at aflæse variablen an\_spl 
	\item Testen punkt 1-5 gentages med en sinus med en amplitude 100mV
\end{enumerate}

\textbf{SPL: Forventet resultat} \\
Jævnfør beregningerne i \ref{an_spl} forventes ved input-sinus med amplituden 1V at variablen \verb+an_spl+ er 0dBV. Ved amplituden 100mV forventes en værdi på -20dBV. Eftersom funktionen blot er tænkt som grovfiltrering, accepteres en afvigelse på +/- 2dBV.

\textbf{SPL: Resultat} \\
\figur{1}{intlydmonitor/modultest/an_spl_1V_880Hz}{Aflæst SPL-værdi ved 880Hz sinus med 1 V amplitude}{an_spl_1V}
\figur{1}{intlydmonitor/modultest/an_spl_100mV_880Hz}{Aflæst SPL-værdi ved 880Hz sinus med 100 mV amplitude}{an_spl_100mV}
Som det ses af ovenstående figurer, \ref{an_spl_1V} og \ref{an_spl_100mV}, måles hhv. 0.53 dBV ved input med 1V amplitude og -18.76 dBV ved input på 100mV amplitude. SPL er således målt med afvigelse på hhv. 0.53 dBV og 1.24 dBV, hvilket er inden for den specificerede tollerence.
 
\textit{Testen af Sound Pressure Level er godkendt}

\textbf{\textit{Test af Fast Fourier Transformation}}\\
\textbf{FFT: Testprocedure}
\begin{enumerate}
	\item Der genereres et to-tonet testsignal bestående af to overlejrede sinuser med 1V amplitude og frekvenserne 1kHz og 6kHz.	
	\item Projektet med Analyzer testen startes i debug-mode
	\item Et breakpoint sættes i "Analyzer.c", linje 40 
	\item Der klikkes på "resume"
	\item Projektet køres indtil breakpoint
	\item CCES memory browser benyttes til at eksportere arrayet \verb+an_freqSpec+ til en .dat-fil 
	\item .dat-filen loades i matlab og printes med x-akse svarende til de enkelte pladsers repræsenterede frekvenser.
\end{enumerate}
Se BILAG Analyzer\_test.m for matlab testkode. Et screenshot af data-eksport ses herunder:
\figur{1}{intlydmonitor/modultest/an_freqSpec_data_export}{Eksport af an\_freqSpec[]}{an_freqSpec_data_exp}

\textbf{FFT: Forventet resultat} \\
På plottet af arrayet \verb+an_freqSpec[]+ forventes markante frekvenspeaks at befinde sig ved hhv. 1kHz og 6kHz. 

\textbf{FFT: Resultat} \\
\figur{0.7}{intlydmonitor/modultest/an_freqSpec_print}{Genereret frekvensspektrum med x-aksens frekvensbins repræsenteret som frekvenser}{an_freqSpec_print}
Som vi kan se på \ref{an_freqSpec_print}, forefindes som forventet to tydelige frekvenspeaks ved 1kHz og 4kHz. Imidlertid ses også 2. og 3. harmoniske af 1kHz-sinusen. Idet test-signalet blev genereret i Matlab og dermed outputtet via PC'ens lavkvalitets DAC, tilstrives denne harmoniske forvrængning DAC'en.
\textit{Testen af Fast Fourier Transform er godkendt}

\textbf{\textit{Test af omregning fra gain til dB}}\\
\textbf{FFT: Testprocedure}
\begin{enumerate}
	\item Der genereres et to-tonet testsignal bestående af to overlejrede sinuser med 1V amplitude og frekvenserne 1kHz og 6kHz.	
	\item Projektet med Analyzer testen startes i debug-mode
	\item Et breakpoint sættes i "Analyzer.c", linje 41 
	\item Der klikkes på "resume"
	\item Projektet køres indtil breakpoint
	\item CCES memory browser benyttes til at eksportere arrayet \verb+an_freqSpecdB+ til en .dat-fil 
	\item .dat-filen loades i matlab og printes med x-akse svarende til de enkelte pladsers repræsenterede frekvenser.
\end{enumerate}

\textbf{g2dB: Forventet resultat} \\
De to peaks fra "Test af Fast Fourier Transformation" har en gain målt til hhv. 0.03998 gg og 0.02673 gg. I dB giver dette:
\begin{center} 
$peak1kHZdB=20\log _{ 10 }{ \left( 0.03998 \right)  } =-27.96\quad dB $ \\
$peak4kHzdB=20\log _{ 10 }{ \left( 0.02673 \right)  } =-31,46\quad dB $
\end{center}
Disse værdier forventes set på frekvensspektret.

\textbf{g2dB: Resultat} \\
\figur{0.7}{intlydmonitor/modultest/an_freqSpecdB_print}{Genereret frekvensspektrum med x-aksens frekvensbins repræsenteret som frekvenser}{an_freqSpecdB_print}
Som det ses af \ref{an_freqSpecdB_print} opnås præcis de forventede peak-værdier.
 
\textit{Testen er godkendt}

\textbf{\textit{Test af Smoothing af FFT-spektrum}}\\
\textbf{Smooth: Testprocedure}
\begin{enumerate}
	\item Optagelsen "Hoejlydt\_babygraad2" afspilles fra computeren.
	\item Projektet med Analyzer testen startes i debug-mode
	\item Et breakpoint sættes i "Analyzer.c", linje 43 
	\item Der klikkes på "resume"
	\item Projektet køres indtil breakpoint
	\item CCES memory browser benyttes til at eksportere arrayet \verb+an_freqSpecdB+ til en .dat-fil 
	\item .dat-filen loades i matlab og printes med x-akse svarende til de enkelte pladsers repræsenterede frekvenser.
	\item CCES memory browser benyttes til at eksportere arrayet \verb+an_freqSpecSmooth+ til en .dat-fil 
	\item .dat-filen loades i matlab og printes med x-akse svarende til de enkelte pladsers repræsenterede frekvenser.
\end{enumerate}

\textbf{Smooth: Forventet resultat} \\
Der forventes at se et markant glattere signal med en indsvingning fra 0-818 Hz. Der forventes ligeledes at karakteristiske peaks bibeholdes, dog forskubbet omtrent 800 Hz til højre.

\textbf{Smooth: Resultat} \\
\figur{1}{intlydmonitor/modultest/an_freqSpecSmooth_hoejlydt2_pre_post}{Hoejlydt\_baby2 FFT med og uden smoothing}{an_freqSpecSmooth_print}
Der opnås betydelig udglatning af signalet, der som forventet har en filter-indsvingning fra 0-818 Hz og er forskubbet omtret 800 Hz.

\textit{Testen er godkendt}

\textbf{\textit{Analyzer samlet konklusion}}\\
Det er således lykkedes at analysere data med hensyn til SPl, FFT, g2dB og Smooth, og et categoriserbart datasæt er klargjort.