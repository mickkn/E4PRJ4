\section{Modultest: Intelligent Lydmonitor}

\subsection{Formål}
I dette afsnit følger den endelige modultest for Intelligent Lydmonitor. Den endelig funktionalitetsgrad og træfsikkerhed skal bestemmes. Dette gøres ved at lave histogram med 100 på hinanden følgende BABYCON statuser for hver lydfil. Intelligent lydmonitor testes med følgende lydfiler:
\begin{itemize}
	\item Hoejlydt\_babygraad1.wav
	\item Hoejlydt\_babygraad2.wav
	\item Moderat\_babygraad1.wav
	\item Moderat\_babygraad2.wav
	\item Trafik.wav
	\item Cat.wav
	\item Fuglefloejt.wav
	\item Latter.wav
\end{itemize}

\subsection{Testopstilling}
\begin{itemize}
	\item Blackfin 533 er sluttet til 230VAC
	\item Blackfin 533 er tilsuttet PC via COM-port
	\item CrossCore Embedded Studio (CCES) samt en lydafspiller er kørende på PC
	\item Lydoutputtet på en PC er tilstluttet Blackfin 533's Audio In left\_channel0
\end{itemize}

Data til histogrammet indsamles med Statistician-klassens test-funktion \verb+hitRatioTest()+, der kaldes ved hver ny BABYCON erhvervelse. Funktionen er vist herunder:
\begin{verbatim}void hitRatioTest(int bc)
{
    ss_Histogram[ss_hist_count++] = bc;
    if(ss_hist_count == SS_HIST_SIZE)
    {
        ss_hist_count = 0;
        printf("Statistician: Histogram ready\n");
    }
}
\end{verbatim}

Herunder ses et screenshot af eksport af arrayet \verb+ss_Histogram[]+ via CrossCore Embedded Studio's "memory browser" 
\figur{1}{intlydmonitor/modultest_endelig/hist_export}{Eksport af ss\_Histogram data}{IL_hist_export}

Et matlab-program til plot af disse data ses herunder:
\begin{verbatim}%**** HISTOGRAM PLOT *****************************************
load data.dat;
x = data;
N = length(x);

edges = [1 2 3];
figure(2), hist(x, edges)
xlabel('BABYCON')
ylabel('Count')
title('Data: Histogram over 100 målinger')
\end{verbatim}

\subsection{Testprocedure}
Følgende procedure gentages for alle specificerede lydfiler:
\begin{enumerate}
	\item Lydfilen afspilles i "loop-mode" på PC
	\item Projektet "Intelligent\_Lydmonitor startes i debug-mode
	\item Et breakpoint sættes i "Statistician.c", linje 124 i funktionen \verb+hitRatioTest()+ efter linjen \verb+printf("Statistician: Histogram ready\n");+ 
	\item Der klikkes på "resume"
	\item Projektet køres indtil breakpoint
	\item CCES memory browser benyttes til at eksportere arrayet \verb+ss_Histogram+ til en .dat-fil
	\item Testen loades i matlab projektet "a\_hist\_plot.m" og eksekveres
	\item Matlab eksekveres og et histogram printes 
\end{enumerate}

\subsection{Forventede resultater} 
\textit{Højlydt babygråd:} Det forventes af optagelserne af ''Hoejlydt\_babygraad1.wav'' og ''Hoejlydt\_babygraad2.wav'' primært resulterer i BABYCON1. Dog forventes en mængde BABYCON2 detektioner, idet en baby's gråd er svært kvantiserbar og ofte vil ligge i et mellemleje mellem BABYCON1 og BABYCON2. 

\textit{Moderat babygråd:} Det forventes at optagelserne ''Moderat\_babygraad1.wav'' og ''Moderat\_babygraad2.wav'' primært resulterer i BABYCON2. Dog forventes ligeledes nogle få BABYCON1 detektioner, samt en del BABYCON3 detektioner, grundet optagelsernes dynamiske karakter. 

\textit{Omgivelsesstøj:} Det forventes at optagelserne ''Trafik.wav'', ''Cat.wav'', ''Fuglefloejt.wav'' og ''Latter.wav'' primært resulterer i BABYCON3. Dog forventes nogle få BABYCON2 detektioner. Kun et minimum BABYCON1 detektioner forventes og tolereres, eftersom denne status er af disruptiv karakter for systemet.

\subsection{Resultater} 
I følgende afsnit ses histogrammer for BABYCON detektioner af tidligere beskrevne lydfiler. Ligeledes forefindes en opsummering af detektions-resultater på tabelform samt en endelig "hit rate" (træfsikkerhed) for systemet.

\subsubsection{Histogrammer}
\figur{1}{intlydmonitor/modultest_endelig/hoejlydt_1_2_hist}{Histogram over BABYCON-status for 100 optagelser af lydfilerne ''Hoejlydt\_babygraad1.wav'' og ''Hoejlydt\_babygraad2.wav''}{IL_test_hoej_hist}
\figur{1}{intlydmonitor/modultest_endelig/moderat_1_2_hist}{Histogram over BABYCON-status for 100 optagelser af lydfilerne ''Moderat\_babygraad1.wav'' og ''Moderat\_babygraad2.wav''}{IL_test_mod_hist}
\figur{1}{intlydmonitor/modultest_endelig/omgivelser_hist}{Histogram over BABYCON-status for 100 optagelser af lydfilerne ''Trafik.wav'', ''Cat.wav'', ''Fuglefloejt.wav'' og ''Latter.wav''}{IL_test_omg_hist}

\subsubsection{Resultat opsummering}
Resultater af testen er opsummere i de tre kategorier ''Højlydt babygråd'', ''Moderat babygråd'' og ''Omgivelsesstøj'' med en procentvis detektering af de tre BABYCON statuser.

\begin{center}
    \begin{tabular}{ | l | l | l | l |}     					
    \hline
	\textbf{Kategori}	& \textbf{BABYCON1}	& \textbf{BABYCON2}	& \textbf{BABYCON3}  \\ \hline
    Højlydt babygråd	& 49,50\%			& 34,50\%			& 16,00\%   		\\ \hline
    Moderat babygråd 	& 0,00\%			& 59,50\%			& 40,50\%   		\\ \hline
    Omgivelsesstøj 		& 0,25\%			& 11,00\%			& 91,25\%   		\\ \hline
    \end{tabular}
\end{center}


\subsubsection{Hit ratio}
Idet det antages at en optagelse af ''Højlydt babygråd'' altid bør resultere i et BABYCON1, og ''Moderat babygråd'' og ''Omgivelsesstøj'' ligeledes bør resultere i hhv. BABYCON2 og BABYCON3, kan en hit ratio for den enkelte detektionstype aflæses direkte af tabellen afsnittet herover. Den samlede hit-radio bliver:

\begin{center}
$ HitRatio=\frac { hits }{ total\quad tries } =\frac { 99+119+365 }{ 200+200+400 } =72.9 $\%
\end{center}


\subsubsection{Konklusion}
Antages en løs tankegang hvor brugssituationen tages i betragtning, er det langt hen ad vejen lykkedes at implementere den Intelligente Lydmonitor tilfredsstillende. \textit{BABYCON1} detekteres næsten udelukkende ved optagelser af højlydt babygråd med en detektion ved 1\% af Latter tilfældene. \textit{BABYCON2} detekteres i højest grad ved moderat babygråd, med en del detektioner ved højlydt babygråd samt dyrelyde som katte-mjauen en fuglefløjt. \textit{BABYCON3} detekteres mellem 70 og 100\% af tiden ved andre optagelser end babygråd, hvor fuglefløjt har laveste og trafikstøj højeste procentsats. BABYCON3 detekteres indimellem ved både højlydt og moderat babygråd, men eftersom denne status er fredelig, antages denne fejltagelse for mindre væsentlig.

For et mere rigidt udgangspunkt, hvor det antages at de givne optagelser af eksempelvis ''Højlydt babygråd'' hver gang bør resultere i BABYCON1,  er det lykkedes at implementere en Intelligent Lydmonitor, der for de angivne lydfiler har en \textit{HitRatio = 72.9\%}.

