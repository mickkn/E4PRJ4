%IL: forundersøgelse
\section{Forundersøgelse}

I dette afsnit undersøges og analyseres udvalgte lyde fra babygråd og typisk omgivelsesstøj såsom billarm og fuglefløjt.

\subsection{Metode}
Der findes optagelser af de udvalgte lyde i formatet .wav. Lydene analyseres i Matlab med hensyn til:
\begin{itemize}
	\item Generelt frekvensindhold
	\item Dominant tone
	\item Grad af tonalitet
\end{itemize} 

Disse parametre analyseres med redskaberne Short-Time DFT, en max-funktion, Tonal Power Spectrum, samt en smoothing-funktion.

\textbf{Short-Time DFT} \\
FFT af kortere optagelsessegmenter. Produktet af dette, kan plottes som et spektogram, der således viser optagelsens frekvensindhold som variation af tiden.

\textbf{Max} \\
Matlabs funktion "max", finder den højeste værdi i matrixen. I denne analyse, vil den benyttes på FFT'en af de enkelte optagelsessegmenter (Short-Time DFT'ens output).

\textbf{Tonal Power Spectrum} \\
Tonal Power Spectrum benyttes til beregne tonaliteten i optagelsen, og er altså et udtryk for forholdet mellem dominante toner og det samlede energiindhold. 
\begin{center}
${ v }_{ Tpr }=\frac { { E }_{ T }(n) }{ \sum _{ i=0 }^{ K/2-1 }{ { \left| X(k,n) \right|  }^{ 2 } }  } $
\end{center}

Hvor nævneren er det totale spektrale energiindhold og tælleren, E\begin{tiny}T\end{tiny}, er det tonale indhold. Det tonale indhold beregnes ved at tage FFT'en af de enkelte optagelsessegmenter (Short-Time DFT'ens output) og summere alle de bins, der:
\begin{itemize}
	\item Er lokalt maximum: ${ \left| X(k-1,n) \right|  }^{ 2 }\le { \left| X(k,n) \right|  }^{ 2 }\le { \left| X(k+1,n) \right|  }^{ 2 }$ og
	\item Ligger over en forudbestemt grænseværdi ${ G }_{ T }$.
\end{itemize} 	

Resultatet af vil ligge mellem $0\le { v }_{ Tpr }\le 1$

\textbf{Smoothing} \\
Til at udjævne analysesignalerne benyttes matlabs "smooth" funktion. Default filteret, moving average, benyttes.

\subsection{Situationer}
Der udføres analyser på lyd-optagelser af følgende situationer:
\begin{itemize}
	\item Højlydt babygråd
	\item Moderat babygråd
	\item Fuglefløjt
	\item Trafik
\end{itemize} 

\newpage
\subsection{Analysekode}
\begin{verbatim}
%******************************************************************%
% Project: E4PRJ4 BABY WATCH                                       %
% Authors: Lukas Hedegaard og Kristian Boye                        %
% Task: Forundersoegelse af fire udvalgte lydfiler,                %
% Hoejlydt babygraad, Moderat babygraad, Fuglefloejt og Trafik     %
%******************************************************************%

%**** EXTRACT SAMPLES *********************************************%
[rec, f_s] = wavread('Hoejlydt_babygraad');
%[rec, f_s] = wavread('Moderat_babygraad');   
%[rec, f_s] = wavread('Fuglefloejt');
%[rec, f_s] = wavread('Trafik');  

x = rec(:,1)';  %extract left channel'

%**** Set sample window *******************************************%
frame_sec = 0;          %Set frame size from recordning length
                        %0 for full recording
if frame_sec == 0 
    N = length(x);
    shift_N = 1;
else
    N = frame_sec*f_s;
    shift_sec = 1.1;    %Set time shift
    shift_N = shift_sec*f_s;
end

x_win = x(shift_N:N-1+shift_N);             %extract framed sample
x_win = x_win .* hanning(length(x_win))';   %window with hamming

%**** Bandpass prefiltrering **************************************%
n = 4;                                      % Order
bs_low_cutoff = 100;                        % Cut off freq (Hz)
bs_low_cutoff_rad = bs_low_cutoff/(0.5*f_s);% Cut off freq (rad/sample)
bs_high_cutoff = 10000; 
bs_high_cutoff_rad = bs_high_cutoff/(0.5*f_s); 
[b,a] = butter(n, [bs_low_cutoff_rad bs_high_cutoff_rad]); % Butter BP

x_win = filter(b,a,x_win);

%**** Play sample *************************************************%
sound(x_win, f_s)

%**** Plot sample *************************************************%
offset = 0;
x = x_win + offset;     %use offset to fit both sampleplot and TPR

figure(1)
subplot(3,1,1)
plot(x)
xlabel('Tid/Samples')
ylabel('Amplitude')
title('Rec: Sample-plot')
grid on

%**** FAST FOURIER TRANSFORM **************************************% 
NFFT = 2^nextpow2(N);   %Find next power of 2
X = fft(x_win,NFFT);

%**** Find Maximal FFT frequency **********************************%
X_max = max(abs(X));


%**** SHORT-TIME DFT (Spectogram) *********************************%
segmentlen = 256;
nOverlap = 60;
NFFT = 1280;

figure(1)
subplot(3,1,3)
spectrogram(x_win, segmentlen, nOverlap, NFFT, f_s, 'yaxis');
specMinFreq = 0;
specMaxFreq = 5000;
axis([-inf inf specMinFreq specMaxFreq])
colormap bone
%colorbar

%Repeat and save data for use in tonal analysis functions:
[s,f,t,p] = spectrogram(x_win, segmentlen, nOverlap, NFFT, f_s,'yaxis'); 


%**** TONAL POWER RATIO *******************************************% 
TPR = FeatureSpectralTonalPowerRatio(abs(s), f_s, 5*10^-2)';
TPR = TPR*5;   % upscaling

TPR_smooth = smooth(TPR(:),0.1,'moving');   % Smooth data

%**** Plot Tonal Power Ratio **************************************%
xAxis = 1:(length(x_win)/length(TPR)):length(x_win);

figure(1)
subplot(3,1,2)
plot(xAxis,TPR,'r','linewidth',1)
hold on
plot(xAxis,TPR_smooth,'k','linewidth',2)
hold off
xlabel('Tid/sample')
ylabel('Tonal Power Ratio')
title('Tonal Power Ratio')
grid on

%**** FREQ WITH MAX POWER *****************************************%
[q,nd] = max(10*log10(p));  %q is used for x-axis in later plot

maxPwr_smooth = smooth(f(nd),0.05,'moving');     % Smooth data


%**** FREQUENCY ANALYSIS PLOTS ************************************%
%**** Discard values beneath Tonal Power Ratio Threshold *******
cutThresh = 0;
maxPwr_smooth_cut = maxPwr_smooth;
maxPwr_cut = f(nd);

for i = 1 : length(maxPwr_smooth)
    %determine spectral energy at freq
    if ((TPR_smooth(i) < cutThresh) || (TPR(i) == 0))
        maxPwr_smooth_cut(i) = NaN(1);
        maxPwr_cut(i) = NaN(1);
    end
end

%**** Plot frequency analysis *************************************%
figure(1)
subplot(3,1,3)

hold on
plot3(t,maxPwr_smooth_cut,q,'r','linewidth',2)  %maxPwr_smooth plot
scatter(t,maxPwr_cut,5,'b')
view(2)
hold off

\end{verbatim}

\newpage
\subsection{Analyser}
Resultatet af analysekoden er 3-subplots:
\begin{enumerate}
	\item Sampleplot af det givne lydsignal (blå)
	\item Tonal Power Ratio (rød) og midlet TPR (sort)
	\item Spectogram (colormap: bone), dominerende tone (blå prikker), midlet dominerende tone (rød streg)
\end{enumerate}

Bemærk for subplot 3, at data for dominerende frekvens er fjernet ved TPR på 0.


\textbf{Højlydt babygråd} \\
Karakteristisk ved højlydt babygråd er, at dele af Tonal Power Ratio (TPR) ligger over en værdi på 0.13 (se subplot 2). Herudover ses det på subplot 3 at de dominerende frekvenser typisk ligger mellem 700 og 3000 Hz (blå prikker). Midles disse værdier, ligger det dominerende frekvensindhold mellem 900 Hz og 2100 Hz (røde streger). 

\figur{0.9}{intlydmonitor/forundersoegelse/analyse_baby_crying}{Analyseresultat af optagelsen Højlydt babygråd}{intlyd:forundersoegelse:hoj_baby}


\newpage
\textbf{Moderat babygråd}\\
Karakteristisk ved optagelsen af moderat babygråd er, at store dele af TPR ligger over en værdi på 0.13 (subplot 2). Af subplot 3 fremgår det at de dominerende frekvenser typisk ligger mellem 500 og 3500 Hz (blå prikker). Midles disse værdier, ligger det dominerende frekvensindhold mellem 900 Hz og 1700 Hz (røde streger). 

\figur{0.9}{intlydmonitor/forundersoegelse/analyse_baby_cooing}{Analyseresultat af optagelsen Moderat babygråd}{intlyd:forundersoegelse:mod_baby}


\newpage
\textbf{Fuglefløjt}\\
Karakteristik for fuglefløjt og omgivelsesstøj fra natur er, at TPR ligger på en værdi over 0.11, men under 0.13 (subplot 2). På spektrogrammet (subplot3) ses det at de dominerende frekvenser ligger meget spredt: det meste af tiden dominerer vind og vejr (ca. 100 Hz), og andre gange dominerer fuglefløjt med frekvenser fra 1800 Hz til 5000 Hz. Den midlede dominerende tone ligger mellem 35 Hz og 2500 Hz.

\figur{0.9}{intlydmonitor/forundersoegelse/analyse_bird}{Analyseresultat af optagelsen Fuglefløjt}{intlyd:forundersoegelse:fugl}

\newpage
\textbf{Trafikstøj}\\
Karakteristisk for trafikstøj er, at TPR ligger 0.10, men under 0.13 (subplot 2). Af spektrogrammet (subplot 3) fremgår det at frekvenserne ligger spredt mellem 0 og 1500 Hz med en middel-værdi på ca 700 Hz.

\figur{0.9}{intlydmonitor/forundersoegelse/analyse_street_traffic}{Analyseresultat af optagelsen Trafikstøj babygråd}{intlyd:forundersoegelse:trafik}

\newpage
\textbf{Sammenligning af maximal TPR}	\\
Matlabs funktion til at finde maksimum-værdien for at array er benyttet til at finde den højeste værdi af TPR for de forskellige situationer. Resultaterne kan ses i tabellen herunder. 

\begin{center}
    \begin{tabular}{ | l | l |}
    \hline
    \textbf{Situation} & \textbf{Maximal TPR}  \\ \hline
    Højlydt babygråd & 0.144   \\ \hline
    Moderat babygråd & 0.146  \\ \hline
    Fuglefløjt & 0.127  \\ \hline
    Trafikstøj & 0.120  \\
    \hline
    \end{tabular}
\end{center}

Babygråd producerer altså en højere TPR end typisk omgivelsesstøj, hvad enten det drejer sig om moderat eller højlyd gråd.

\textbf{Sammenligning af dominerende frekvensindhold}	\\
Minimum- og maksimumfrekvens for analysen af dominerende toner er vist i tabellen herunder:
\begin{center}
    \begin{tabular}{ | l | l | l |}
    \hline
    \textbf{Situation} &	\textbf{Min frekvens}	& 	\textbf{Max frekvens}  \\ \hline
    Højlydt babygråd & 		770 Hz &					1932 Hz\\ \hline
    Moderat babygråd & 		676 Hz &					1613 Hz\\ \hline
    Fuglefløjt & 			69  Hz &					3120 Hz\\ \hline
    Trafikstøj & 			38  Hz &					710 Hz\\
    \hline
    \end{tabular}
\end{center}

Fuglefløjt har største båndbredde, med udfald mellem 69 Hz og 3120 Hz. Trafikstøjens dominerende frekvenser bevæger sig ikke over 710 Hz. Babygråd har sit dominante frekvensindhold mellem 600 og 2000 Hz. Trafikstøj har således lavere frekvensindhold end babygråd. Frekvensindholdet for fuglefløjt har en større båndbredde end babygråd og har sit indhold både over- og under båndbredden for babygråd.

\subsection{Konklusion}
Ved hjælp af \textit{Tonal Power Ratio} er det altså muligt at kende forskel på babygråden og almindelige støjsignaler som trafikstøj og fuglefløft/natur. \\
Af \textit{frekvensanalysen} ses det, at højlydt babygråd vil have dominant-toneindhold der ligger omtrent 300 Hz højere end det for moderat babygråd. Forekommer dominant toneindhold (midlet) over 2500 Hz, antages tonen ikke at komme fra babyen. \\
Forskellen i frekvensindhold mellem højlydt og moderat gråd er dog ikke signifikant, og en \textit{dB-måling} af grådens lydstyrke kunne evt være nødvendig for at skelne mellem højlydt- og moderat babygråd.