%IL: forundersøgelse
\section{Forundersøgelse}

I dette afsnit undersøges og analyseres udvalgte lyde fra babygråd og typisk omgivelsesstøj såsom trafikstøj og fuglefløjt.

\subsection*{Situationer}
Der udføres analyser på lyd-optagelser af følgende situationer:
\begin{itemize}
	\item Højlydt babygråd
	\item Moderat babygråd
	\item Fuglefløjt
	\item Trafik
	\item Katte
\end{itemize} 
Disse analyseres således som beskrevet i efterfølgende afsnit.

\subsection{Metode}
Der analyseres tre optagelser af hver af de udvalgte lyde i formatet .wav. Lydene analyseres i både lange (~5 sek) og korte (~5 msek) optagelsessegmenter i Matlab med hensyn til:
\begin{itemize}
	\item Frekvensindhold
	\item Dominant tone
	\item Grad af tonalitet (andel af signalets effekt, der tilskrives dominante toner frem for generel støj) 
\end{itemize} 

Disse parametre analyseres med redskaberne Short-Time DFT, en max-funktion, Tonal Power Spectrum, samt en smoothing-funktion.

\textbf{Short-Time DFT} \\
FFT af kortere optagelsessegmenter. Produktet af dette, kan plottes som et spektogram, der således viser optagelsens frekvensindhold som variation af tiden.

\textbf{Max} \\
Matlabs funktion "max", finder den højeste værdi i matrixen. I denne analyse, vil den benyttes på FFT'en af de enkelte optagelsessegmenter (Short-Time DFT'ens output).

\textbf{Tonal Power Ratio} \\
Tonal Power Ratio benyttes til beregne tonaliteten i optagelsen, og er altså et udtryk for forholdet mellem dominante toner og det samlede energiindhold. Hvid støj vil således have en TPR = 0 og en ren sinus-tone vil have en TPR = 1. TPR beregnes som følger: 
\begin{center}
${ v }_{ Tpr }=\frac { { E }_{ T }(n) }{ \sum _{ i=0 }^{ K/2-1 }{ { \left| X(k,n) \right|  }^{ 2 } }  } $
\end{center}

Hvor nævneren er det totale spektrale energiindhold og tælleren, E\begin{tiny}T\end{tiny}, er det tonale indhold. Det tonale indhold beregnes ved at tage FFT'en af de enkelte optagelsessegmenter (Short-Time DFT'ens output) og summere alle de bins, der:
\begin{itemize}
	\item Er lokalt maximum: ${ \left| X(k-1,n) \right|  }^{ 2 }\le { \left| X(k,n) \right|  }^{ 2 }\le { \left| X(k+1,n) \right|  }^{ 2 }$
	\item Ligger over en forudbestemt grænseværdi ${ G }_{ T }$.
\end{itemize} 	

Resultatet af vil ligge mellem $0\le { v }_{ Tpr }\le 1$

\textbf{Smoothing} \\
Til at udjævne analysesignalerne benyttes matlabs "smooth" funktion. Default filteret, moving average, benyttes.

%\newpage
\subsection{Analysekode gennemgang}
Endnu ikke skrevet
\begin{verbatim}

\end{verbatim}

%\newpage
\subsection{Analyser}
Resultatet af korttids-analysen er 3-plots:
\begin{enumerate}
	\item Sampleplot af det givne lydsignal (blå)
	\item Tonal Power Ratio (rød) og midlet TPR (sort)
	\item Spectogram (colormap: bone), dominerende tone (blå prikker), midlet dominerende tone (rød streg)
\end{enumerate}
Bemærk for plot 3, at data for dominerende frekvens er fjernet ved TPR på 0.

Resultatet af langtids-analysen er:
\begin{enumerate}
	\item DFT-spektrum: Originalt (blå), moderat filtreret (rød), kraftigt filtreret (sort)
	\item Global TPR-værdi
	\item Global dominerende tone
\end{enumerate}

\newpage
\begin{center}  \textit{\textbf{Højlydt babygråd}}  \end{center}
\textbf{Korttids-analyse} \\
Karakteristisk ved højlydt babygråd er, at dele af Tonal Power Ratio (TPR) når over en værdi på 0.13 (se plot 2). Herudover ses det på plot 3 at de dominerende frekvenser (blå prikker) typisk ligger omkring 900, 3100 og 5000 Hz. Midles disse værdier over længere tid, ligger det dominerende frekvensindhold mellem 900 Hz og 3500 Hz (røde streger). 

\figur{1}{intlydmonitor/forundersoegelse/analyse_hoejlydt_babygraad_kort}{Korttids-analyse af højlydt babygråd optagelser}{intlyd:forundersoegelse:hoj_baby_kort}

\newpage
\textbf{Langtids-analyse} \\
Karakteristisk ved langtidsanalysen for højlydt babygråd er, at den dominerende tone (sort streg) ligger omkring 900 Hz. Næste dominante frekvens ligger mellem 3 kHz og 3,1 kHz. Endvidere ses at magnituden af frekvenserne mellem 1 kHz og 6 kHz er relativt ligeligt distribuerede med magnitudeforskel på laveste og højeste frekvens (på sort, midlet kurve) på omtrent 25 dB. \\
De tre optagelser samlet (nederste højre graf) har en TPR = 0.336.

\figur{1}{intlydmonitor/forundersoegelse/analyse_hoejlydt_babygraad_lang_samling}{Langtids-analyser af højlydt babygråd optagelser. Nederste venstre graf er analyse af de tre optagelser samlet}{intlyd:forundersoegelse:hoj_baby_lang}


\newpage
\begin{center} \textit{\textbf{Moderat babygråd}} \end{center}
\textbf{Korttids-analyse} \\
Karakteristisk ved optagelsen af moderat babygråd er, at store dele af TPR ligger over en værdi på 0.13 (plot 2). Af plot 3 fremgår det at de dominerende frekvenser typisk ligger mellem 500 og 3500 Hz (blå prikker). Midles disse værdier, ligger det dominerende frekvensindhold mellem 900 Hz og 1500 Hz (røde streger). 

\figur{1}{intlydmonitor/forundersoegelse/analyse_moderat_babygraad_kort}{Korttids-analyse af moderat babygråd optagelser}{intlyd:forundersoegelse:mod_baby_kort}

\newpage
\textbf{Langtids-analyse} \\
Ved langtidsanalysen for højlydt babygråd ligger den dominerende dybe tone (sort streg) på hhv. 1,5 kHz (optagelse 1, øverst venstre), 1 kHz (optagelse 2, øverst højre) og 600 Hz (optagelse 3, nederst venstre). Næste dominante frekvens ligger omkring 2,5 kHz for optagelse 2 og 3. For optagelse 1, findes intet tydeligt peak. Der er altså ikke tydelig korrelation for placering af frekvenspeaks. Magnituden af frekvenserne mellem 1 kHz og 6 kHz er relativt ligeligt distribuerede med magnitudeforskel på laveste og højeste frekvens (sort, midlet kurve) på omtrent 20 dB. \\
De tre optagelser samlet (nederste højre graf) har TPR = 0.374.

\figur{1}{intlydmonitor/forundersoegelse/analyse_moderat_babygraad_lang_samling}{Langtids-analyser af moderat babygråd optagelser. Nederste venstre graf er analyse af de tre optagelser samlet}{intlyd:forundersoegelse:mod_baby_lang}

\newpage
\begin{center} \textit{\textbf{Fuglefløjt}} \end{center}
\textbf{Korttids-analyse} \\
Karakteristik for fuglefløjt og omgivelsesstøj fra natur er, at TPR ligger på en værdi over 0.11, men under 0.13 (plot 2). På spektrogrammet (plot3) ses det at de dominerende frekvenser ligger meget spredt; det meste af tiden dominerer vind og vejr (ca. 100 Hz), og andre gange dominerer fuglefløjt med frekvenser fra 1,8 kHz til 6 kHz. Den midlede dominerende tone ligger mellem 35 Hz og 2500 Hz.

\figur{1}{intlydmonitor/forundersoegelse/analyse_fuglefloejt_kort}{Korttids-analyse af fuglefløjt optagelser}{intlyd:forundersoegelse:fugl_kort}

\newpage
\textbf{Langtids-analyse} \\
\figur{1}{intlydmonitor/forundersoegelse/analyse_fuglefloejt_lang_samling}{Langtids-analyser af fuglefløft optagelser. Nederste venstre graf er analyse af de tre optagelser samlet}{intlyd:forundersoegelse:fugl_lang}
%TPR_global = 0.291

\newpage
\begin{center}\textit{\textbf{Trafikstøj}}\end{center}
\textbf{Korttids-analyse} \\
Karakteristisk for trafikstøj er, at TPR ligger over 0.10, men under 0.13 (plot 2). Af spektrogrammet (plot 3) fremgår det at frekvenserne ligger spredt mellem 0 og 1500 Hz med en middel-værdi på ca 700 Hz.

\figur{0.9}{intlydmonitor/forundersoegelse/analyse_trafik_kort}{Korttids-analyse af trafikstøjs-optagelser}{intlyd:forundersoegelse:trafik_kort}

\newpage
\textbf{Langtids-analyse} \\
\figur{0.9}{intlydmonitor/forundersoegelse/analyse_trafik_lang_samling}{Langtids-analyser af trafikstøjs-optagelser. Nederste venstre graf er analyse af de tre optagelser samlet}{intlyd:forundersoegelse:trafik_lang}
%TPR_global = 0.269

\newpage
\begin{center}\textit{\textbf{Kat}}\end{center}
\textbf{Korttids-analyse} \\

\figur{1}{intlydmonitor/forundersoegelse/analyse_kat_kort}{Korttids-analyse af katte-optagelser}{intlyd:forundersoegelse:kat_kort}

\newpage
\textbf{Langtids-analyse} \\

\figur{1}{intlydmonitor/forundersoegelse/analyse_kat_lang_samling}{Langtids-analyser af katte-optagelser. Nederste venstre graf er analyse af de tre optagelser samlet}{intlyd:forundersoegelse:kat_lang}
%TPR_global = 0.295

\newpage
\begin{center}\textit{\textbf{Latter}}\end{center}
\textbf{Korttids-analyse} \\

\figur{1}{intlydmonitor/forundersoegelse/analyse_latter_kort}{Korttids-analyse af latter-optagelser}{intlyd:forundersoegelse:latter_kort}

\newpage
\textbf{Langtids-analyse} \\

\figur{1}{intlydmonitor/forundersoegelse/analyse_latter_lang_samling}{Langtids-analyser af latter-optagelser. Nederste venstre graf er analyse af de tre optagelser samlet}{intlyd:forundersoegelse:latter_lang}
%TPR_global = 0.253


\newpage
\begin{center} \textit{\textbf{Korttids-analyse delkonklusion}} \end{center}
\textbf{Sammenligning af maximal TPR}	\\
Matlabs funktion til at finde maksimum-værdien for at array er benyttet til at finde den højeste værdi af TPR for de forskellige situationer. Resultaterne kan ses i tabellen herunder. 

\begin{center}
    \begin{tabular}{ | l | l |}
    \hline
    \textbf{Situation} & \textbf{Maximal TPR}  \\ \hline
    Højlydt babygråd & 0.144   \\ \hline
    Moderat babygråd & 0.146  \\ \hline
    Fuglefløjt & 0.127  \\ \hline
    Trafikstøj & 0.120  \\
    \hline
    \end{tabular}
\end{center}

Babygråd producerer altså en højere TPR end typisk omgivelsesstøj, hvad enten det drejer sig om moderat eller højlyd gråd.

\textbf{Sammenligning af dominerende frekvensindhold}	\\
Minimum- og maksimumfrekvens for analysen af dominerende toner er vist i tabellen herunder:
\begin{center}
    \begin{tabular}{ | l | l | l |}
    \hline
    \textbf{Situation} &	\textbf{Min frekvens}	& 	\textbf{Max frekvens}  \\ \hline
    Højlydt babygråd & 		770 Hz &					1932 Hz\\ \hline
    Moderat babygråd & 		676 Hz &					1613 Hz\\ \hline
    Fuglefløjt & 			69  Hz &					3120 Hz\\ \hline
    Trafikstøj & 			38  Hz &					710 Hz\\
    \hline
    \end{tabular}
\end{center}

Fuglefløjt har største båndbredde, med udfald mellem 69 Hz og 3120 Hz. Trafikstøjens dominerende frekvenser bevæger sig ikke over 710 Hz. Babygråd har sit dominante frekvensindhold mellem 600 og 2000 Hz. Trafikstøj har således lavere frekvensindhold end babygråd. Frekvensindholdet for fuglefløjt har en større båndbredde end babygråd og har sit indhold både over- og under båndbredden for babygråd.

\begin{center} \textit{\textbf{Langtids-analyse delkonklusion}} \end{center}

\subsection{Konklusion}
Ved hjælp af \textit{Tonal Power Ratio} er det altså muligt at kende forskel på babygråden og almindelige støjsignaler som trafikstøj og fuglefløjt. \\
Af \textit{frekvensanalysen} ses det, at højlydt babygråd vil have dominant-toneindhold der ligger omtrent 300 Hz højere end det for moderat babygråd. Forekommer dominant toneindhold (midlet) over 2500 Hz, antages tonen ikke at komme fra babyen. \\
Forskellen i frekvensindhold mellem højlydt og moderat gråd er dog ikke signifikant, og en \textit{dB-måling} af grådens lydstyrke kunne evt være nødvendig for at skelne mellem højlydt- og moderat babygråd.