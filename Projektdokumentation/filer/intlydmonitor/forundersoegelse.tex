%IL: forundersøgelse
\section{Forundersøgelse}

I dette afsnit undersøges og analyseres udvalgte lyde fra babygråd og typisk omgivelsesstøj såsom billarm og fuglefløjt.

\subsection{Metode}
Der findes optagelser af de udvalgte lyde i formatet .wav. Lydene analyseres i Matlab med hensyn til:
\begin{itemize}
	\item Generelt frekvensindhold
	\item Dominant tone
	\item Grad af tonalitet
\end{itemize} 

Disse parametre analyseres med redskaberne Short-Time DFT, en max-funktion, Tonal Power Spectrum, samt en smoothing-funktion.

\textbf{Short-Time DFT} \\
FFT af kortere optagelsessegmenter. Produktet af dette, kan plottes som et spektogram, der således viser optagelsens frekvensindhold som variation af tiden.

\textbf{Max} \\
Matlabs funktion "max", finder den højeste værdi i matrixen. I denne analyse, vil den benyttes på FFT'en af de enkelte optagelsessegmenter (Short-Time DFT'ens output).

\textbf{Tonal Power Spectrum} \\
Tonal Power Spectrum benyttes til beregne tonaliteten i optagelsen, og er altså et udtryk for forholdet mellem dominante toner og det samlede energiindhold. 
\begin{center}
${ v }_{ Tpr }=\frac { { E }_{ T }(n) }{ \sum _{ i=0 }^{ K/2-1 }{ { \left| X(k,n) \right|  }^{ 2 } }  } $
\end{center}

Hvor nævneren er det totale spektrale energiindhold og tælleren, E\begin{tiny}T\end{tiny}, er det tonale indhold. Det tonale indhold beregnes ved at tage FFT'en af de enkelte optagelsessegmenter (Short-Time DFT'ens output) og summere alle de bins, der:
\begin{itemize}
	\item Er lokalt maximum: ${ \left| X(k-1,n) \right|  }^{ 2 }\le { \left| X(k,n) \right|  }^{ 2 }\le { \left| X(k+1,n) \right|  }^{ 2 }$ og
	\item Ligger over en forudbestemt grænseværdi ${ G }_{ T }$.
\end{itemize} 	

Resultatet af vil ligge mellem $0\le { v }_{ Tpr }\le 1$

\textbf{Smoothing} \\
Til at udjævne analysesignalerne benyttes matlabs "smooth" funktion. Default filteret, moving average, benyttes.

\subsection{Situationer}
Der udføres analyser på lyd-optagelser af følgende situationer:
\begin{itemize}
	\item Højlydt babygråd
	\item Moderat babygråd
	\item Fuglefløjt
	\item Trafik
\end{itemize} 

\subsection{Analyser}
\textbf{Højlydt babygråd} \\
\figur{0.7}{intlydmonitor/forundersoegelse/analyse_baby_crying}{Analyseresultat af optagelsen Højlydt babygråd}{intlyd:forundersoegelse:hoj_baby}

Karakteristisk ved Højlydt babygråd er at dele af Tonal Power Ratio (TPR) ligger over en værdi på 0.5. Herudover ses det på spektrogrammet at frekvenserne ligger over de 1000Hz og helt op til 2000Hz. 



\newpage
\textbf{Moderat babygråd}\\
\figur{0.7}{intlydmonitor/forundersoegelse/analyse_baby_cooing}{Analyseresultat af optagelsen Moderat babygråd}{intlyd:forundersoegelse:mod_baby}

For Moderat babygråd viser det sig ligeledes at store dele af TPR ligger over en værdi på 0.5. Af spektrogrammet fremgår det at frekvenserne ligger lig omkring de 1000Hz.

\newpage
\textbf{Fuglefløjt}\\
\figur{0.7}{intlydmonitor/forundersoegelse/analyse_bird}{Analyseresultat af optagelsen Fuglefløjt}{intlyd:forundersoegelse:fugl}

Ved Fuglefløjt ses det at TPR ligger langt under en værdi på 0.5. På spektrogrammet ses det at frekvenserne ligger lavt, under 1000Hz. 

\newpage
\textbf{Trafikstøj}\\
\figur{0.7}{intlydmonitor/forundersoegelse/analyse_street_traffic}{Analyseresultat af optagelsen Trafikstøj babygråd}{intlyd:forundersoegelse:trafik}

For Trafikstøj gælder det at TPR ligger under en værdi på 0.5. Af spektrogrammet fremgår det at frekvenserne ligger lavt som ved Fuglefløjt, under 1000Hz. 


Matlabs funktion til at finde maksimum-værdien for at array er benyttet til at finde den højeste værdi af TPR for de forskellige situationer. Resultaterne kan ses i tabellen herunder. 

\begin{center}
    \begin{tabular}{ | l | l |}
    \hline
    \textbf{Situation} & \textbf{maxTPR}  \\ \hline
    Højlydt babygråd & 0.6697   \\ \hline
    Moderat babygråd & 0.6884  \\ \hline
    Fuglefløjt & 0.2911  \\ \hline
    Trafikstøj & 0.4995  \\
    \hline
    \end{tabular}
\end{center}

\subsection{Konklusion}
Konklusionen er at babygråd har en Tonal Power Ratio over eller lige omkring 0.5, hvorimod støjsignalerne har en Tonal Power Ratio som ligger under 0.5. Det er altså muligt at kende forskel på babygråden og støjsignalerne ved at kigge på TPR. For herefter at kende forskel på Højlydt babygråd og Moderat babygråd, skal vi kigge på frekvensindholdet. Findes frekvenser i området omkring 2000Hz har vi at gøre med Højlydt babygråd og ellers er det Moderat babygråd. 