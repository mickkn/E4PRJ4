%IL: forundersøgelse
\section{Forundersøgelse}

I dette afsnit undersøges og analyseres udvalgte lyde fra babygråd og typisk omgivelsesstøj såsom billarm og fuglefløjt.

\subsection{Metode}
Der findes optagelser af de udvalgte lyde i formatet .wav. Lydene analyseres i Matlab med hensyn til:
\begin{itemize}
	\item Generelt frekvensindhold
	\item Dominant tone
	\item Grad af tonalitet
\end{itemize} 

Disse parametre analyseres med redskaberne Short-Time DFT, en max-funktion, Tonal Power Spectrum, samt en smoothing-funktion.

\textbf{Short-Time DFT} \\
FFT af kortere optagelsessegmenter. Produktet af dette, kan plottes som et spektogram, der således viser optagelsens frekvensindhold som variation af tiden.

\textbf{Max} \\
Matlabs funktion "max", finder den højeste værdi i matrixen. I denne analyse, vil den benyttes på FFT'en af de enkelte optagelsessegmenter (Short-Time DFT'ens output).

\textbf{Tonal Power Spectrum} \\
Tonal Power Spectrum benyttes til beregne tonaliteten i optagelsen, og er altså et udtryk for forholdet mellem dominante toner og det samlede energiindhold. 
\begin{center}
${ v }_{ Tpr }=\frac { { E }_{ T }(n) }{ \sum _{ i=0 }^{ K/2-1 }{ { \left| X(k,n) \right|  }^{ 2 } }  } $
\end{center}

Hvor nævneren er det totale spektrale energiindhold og tælleren, E\begin{tiny}T\end{tiny}, er det tonale indhold. Det tonale indhold beregnes ved at tage FFT'en af de enkelte optagelsessegmenter (Short-Time DFT'ens output) og summere alle de bins, der:
\begin{itemize}
	\item Er lokalt maximum: ${ \left| X(k-1,n) \right|  }^{ 2 }\le { \left| X(k,n) \right|  }^{ 2 }\le { \left| X(k+1,n) \right|  }^{ 2 }$ og
	\item Ligger over en forudbestemt grænseværdi ${ G }_{ T }$.
\end{itemize} 	

Resultatet af vil ligge mellem $0\le { v }_{ Tpr }\le 1$

\textbf{Smoothing} \\
Til at udjævne analysesignalerne benyttes matlabs "smooth" funktion. Default filteret, moving average, benyttes.

\subsection{Situationer}
Der udføres analyser på lyd-optagelser af følgende situationer:
\begin{itemize}
	\item Højlydt babygråd
	\item Moderat babygråd
	\item Fuglefløjt
	\item Trafik
	\item Forsamlingsstøj
\end{itemize} 

\subsection{Analyser}
På de følgende sider ...

\newpage
\textbf{Højlydt babygråd} \\
\figur{0.9}{intlydmonitor/forundersoegelse/analyse_baby_crying}{Analyseresultat af optagelsen 'Højlydt babygråd'}{intlyd:forundersoegelse:hojbaby}

\newpage
\textbf{Moderat babygråd}\\
\figur{0.9}{intlydmonitor/forundersoegelse/analyse_baby_cooing}{Analyseresultat af optagelsen 'Moderat babygråd'}{intlyd:forundersoegelse:mod_baby}

\newpage
\textbf{Fuglefløjt}\\
\figur{0.9}{intlydmonitor/forundersoegelse/analyse_bird}{Analyseresultat af optagelsen 'Fuglefløjt'}{intlyd:forundersoegelse:fugl}

\newpage
\textbf{Trafikstøj}\\
\figur{0.9}{intlydmonitor/forundersoegelse/analyse_street_traffic}{Analyseresultat af optagelsen 'Trafikstøj'}{intlyd:forundersoegelse:trafik}

\newpage
\textbf{Forsamlingsstøj}\\
\figur{0.9}{intlydmonitor/forundersoegelse/analyse_crowd_talking.png}{Analyseresultat af optagelsen 'Forsamlingsstøj'}{intlyd:forundersoegelse:forsamling}


\subsection{Konklusion}

Noget om typiske kraftigste frekvenser. Almindelig støj ligger oftest under 1000 Hz

Tonal Power Ratio over 0.1 indikerer tonalt indhold højt nok til at det kan være baby-gråd.
