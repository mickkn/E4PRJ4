%RecBuf design
\subsection*{RecBuf}
I dette afsnit beskrives principper og design for RecBuf i Intelligent Lydmonitor.
Samples modtaget fra prefiltret gemmes i den ene af to identiske buffere som RecBuf indeholder. Når den første buffer er fyldt kaldes Analyzers analyze() med værdien 0 eller 1, hvilket repræsenterer hvilken buffer Analyzer skal analysere data fra. Indholdet af den buffer som Analyzer arbejder på skal altså være færdig analyseret inden den andet buffer er fyldt. Der opstår således et hastigheds-krav til analyze og et størrelseskrav til bufferne i Recbuf. 

\begin{verbatim}
//includes
#include "Talkthrough.h"

//defines
#define RB_SIZE 256 /* Must be power of two */
#define RB_MEM_0 0
#define RB_MEM_1 1

//Attributes
extern fract32 rb_mem0[RB_SIZE];
extern fract32 rb_mem1[RB_SIZE];
extern bool rb_currActiveBuffer;
extern unsigned int rb_index;
extern bool rb_ready;

//Prototypes
void rb_init(void);
void rb_storeData(fract32 sample);
\end{verbatim}

Herover ses et udsnit fra headeren for kassen RecBuf. Det er her værd at lægge mærke til at attributter der skal bruges erklæres ved brug af extern, som gør det muligt at tilgå variablerne i forskellige scopes. 

\begin{verbatim}
void rb_init(void)
{
    rb_currActiveBuffer = 0;
    rb_index = 0;
    rb_ready = 0;
    //Initializing memory
    int i;
    for(i=0; i< RB_SIZE; i++)
    {
        rb_mem0[i] = 0;
        rb_mem1[i] = 0;
    }
}
\end{verbatim}

Herover ses funktionen rb\_init, det ses at attributter og buffere initieres til 0.

RecBuf indeholder kun en anden funktion kaldet \textbf{rb\_storeData()} denne funktion er implementeret som en switch case der tjekker på variablen \textbf{rb\_currActiveBuffer}. Case 0 er vist herunder. 
Hvis den private variable rb\_index er lig med RB\_SIZE betyder det at rb\_mem0 er fuld. I dette tilfælde sættes rb\_index lig med 0 og rb\_currActiveBuffer opdateres til 1 så den repræsenterer bufferen rb\_mem1. Herefter sættes den første plads i rb\_mem1 med værdien sample. Slutteligt sættes rb\_ready lig true, hvilket bruges senere til at kalde analyze() fra Analyzer. Hvis ikke rb\_index er lig med RB\_SIZE sættes rb\_mem0[rb\_index] med sample. Slutteligt inkrementeres rb\_index. 

\begin{verbatim}
void rb_storeData(fract32 sample)
{
    switch(rb_currActiveBuffer)
    {
    case 0:
        if(rb_index == RB_SIZE)
        {
            rb_index = 0;
            rb_currActiveBuffer = 1;
            rb_mem1[rb_index] = sample;
            rb_ready = true;
        }
        else
        {
            rb_mem0[rb_index] = sample;
        }
        rb_index++;
        break;
...
\end{verbatim}

Case 1 er vist herunder og er næsten magen til case 0. Forskellen på case 1 og case 0 er følgende: 
rb\_currActiveBuffer sættes til 0 for at repræsentere rb\_mem0 i stedet for rb\_mem1. 

\begin{verbatim}
...
    case 1:
        if(rb_index == RB_SIZE)
        {
            rb_index = 0;
            rb_currActiveBuffer = 0;
            rb_mem0[rb_index] = sample;
            rb_ready = true;
        }
        else
        {
            rb_mem1[rb_index] = sample;
        }
        rb_index++;
        break;

    default:
        break;
    }
}\end{verbatim}





