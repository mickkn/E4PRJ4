%categorizer
\subsection{Categorizer}
Herunder følger beskrivelsen og implementeringen for Categorizer i Intelligent Lydmonitor.
Categorizer har til opgave at kategoriserer BABYCON status på baggrund af analyseresultater. 


Herunder ses et uddrag af headeren for Categorizer. Bemærk den definerede CG\_FREQ\_FAC som er en skaleringsfaktor der ændrer fra Hz til en bin. I Headeren findes ligeledes alle attributter og prototyper.

\begin{verbatim}//defines
#define CG_FS 24000
#define CG_BABYCON1 1
#define CG_BABYCON2 2
#define CG_BABYCON3 3
#define CG_FREQ_FAC ((1.0)/(1/(AN_NFFT/2+1)*(CG_FS/2)))
#define CG_TPR_THRESH 4000.0
\end{verbatim}

Herunder ses funktionen cg\_init som sætter relevante threshholds der benyttes i blandt andet cg\_checkBC1. Herudover beregnes cg\_broadStart og cg\_broadEnd som repræsenterer bin ved frekvensen cq\_broadFreqLow og bin ved frekvensen cg\_broadFreqHigh. 

\begin{verbatim}void cg_init(void)
{
    cg_freq_fac = 0.04275; //((1.0)/(1/(AN_NFFT/2+1)*(CG_FS/2)));
    cg_babyCon = CG_BABYCON3; //Set cg_babyCon default 3
    cg_dBroadThreshHighBC1 = 26;
    cg_dBroadThreshLowBC2 = 14;
    cg_dBroadThreshHighBC2 = 24;
    cg_broadFreqLow = 1000;
    cg_broadFreqHigh = 5500;
    cg_firstPeakFreqBC1 = 1400;
    cg_secondPeakFreqBC1 = 4000;
    cg_freqMargin = 600;

    cg_broadStart = (int)floor(cg_broadFreqLow*cg_freq_fac);
    cg_broadEnd = (int)floor(cg_broadFreqHigh*cg_freq_fac);

    cg_margin = (int)floor(cg_freqMargin*cg_freq_fac);
    cg_BC1First = (int)floor(cg_firstPeakFreqBC1*cg_freq_fac);
    cg_BC1Second = (int)floor(cg_secondPeakFreqBC1*cg_freq_fac);
}
\end{verbatim}

Herunder ses funktionen cg\_categorize som har til opgave at kategoriserer BABYCON status. 
Først opdateres attributten cg\_babyCon med default værdien BABYCON3. 
Herefter beregnes hældningen, cg\_dBroad fra cg\_broadStart til cg\_broadEnd. Denne skal bruges i henholdsvis cg\_checkBC2() og cg\_checkBC1() som der tjekkes for umiddelbart herefter. Er cg\_checkBC2() sand, opdateres cg\_babyCon med BABYCON2, hvis cg\_checkBC1() er sand opdateres cg\_babyCon med BABYCON1.
Slutteligt kaldes Statisticians funktion ss\_calcSignificans med det opdaterede cg\_babyCon.

\begin{verbatim}void cg_categorize(void)
{
    cg_babyCon = CG_BABYCON3; //Set cg_babyCon default 3

    if(true)
    {
        cg_dBroad = an_freqSpecSmooth[cg_broadStart]-an_freqSpecSmooth[cg_broadEnd];

        if(cg_checkBC2())
            cg_babyCon= CG_BABYCON2;

        if(cg_checkBC1())
            cg_babyCon= CG_BABYCON1;

        ss_calcSignificans(cg_babyCon);
    }
}
\end{verbatim}

Herunder ses funktionen cg\_checkBC2() som har til opgave at tjekke for BABYCON2. 
Der tjekkes først at hældningen cg\_dBroad ligger over og under de to threshholds. Er dette tilfældet returneres true ellers returneres false. 

\begin{verbatim}
bool cg_checkBC2(void)
{
    if((cg_dBroad > cg_dBroadThreshLowBC2) && (cg_dBroad < cg_dBroadThreshHighBC2))
    {
        return true;
    }
    else
        return false;
}
\end{verbatim}

Den sidste funktion cg\_checkBC1() som har til opgave at tjekke for BABYCON1 er vist herunder. 
Her skal flere konditioner være opfyldt før der returneres true. Først tjekkes der for at hældningen cg\_dBroad er mindre end threshhold. Desuden skal bins cg\_BC1First og cg\_BC1Second være lokale maksima. 

\begin{verbatim}
bool cg_checkBC1(void)
{
    if((cg_dBroad < cg_dBroadThreshHighBC1) &&
	  (an_freqSpecSmooth[cg_BC1First] > an_freqSpecSmooth[cg_BC1First - cg_margin]) &&
	  (an_freqSpecSmooth[cg_BC1First] > an_freqSpecSmooth[cg_BC1First + cg_margin]) &&
	  (an_freqSpecSmooth[cg_BC1Second] > an_freqSpecSmooth[cg_BC1Second - cg_margin]) &&
	  (an_freqSpecSmooth[cg_BC1Second] > an_freqSpecSmooth[cg_BC1Second + cg_margin]))
    {
        return true;
    }
    else
        return false;
}
\end{verbatim}
