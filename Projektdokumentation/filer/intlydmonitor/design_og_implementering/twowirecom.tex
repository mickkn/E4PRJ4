%TwoWireCom
\subsection*{TwoWireCom}
TwoWireCom står for at sende BABYCON-status til Controller via to GPIO pins. 

Funktionaliteten er implementeret ved hjælp af funktioner fra T\_GPIO\_BANKS. 

Et uddrag af TwoWireCom headeren vises herunder.
Der benyttes to GPIO pins PF4 og PF6, disse skal repræsentere henholdsvis MSB og LSB jf. \ref{Overordnet:Kommunikationsprotokol/BABYCON1} . Yderlige specifikationer kan findes i databladet \citep{AD1836} side 84. 

\begin{lstlisting}[language=C,numbers=none]
#define TW_OUTPUT_PIN_MSB PF4
#define TW_OUTPUT_PIN_LSB PF6
\end{lstlisting}

Herunder ses koden for funktionen tw\_init() som har til opgave at sætte retningen på de to GPIO pins. Desuden kaldes tw\_send med default værdien BABYCON3. 
\begin{lstlisting}[language=C,numbers=none]
void tw_init(void)
{
    /* Set pin direction */
    *pFIO_DIR |= TW_OUTPUT_PIN_LSB;
    *pFIO_DIR |= TW_OUTPUT_PIN_MSB;

    /* Set to babycon 3*/
    tw_send(BABYCON3);
}
\end{lstlisting}

\textit{tw\_send(int bc)} implementeres som en switch case, der switcher på de tre BABYCON-statuser 1, 2 og 3. Som default sendes 0. Dette er koden for error. Det kan også være at funktionen er kaldt med en værdi der ikke er 1, 2 eller 3.  
pFIO\_FLAG's benyttes til at sætte GPIO-pin højt eller lavt, ved at sætte flaget lig med den GPIO pin man ønsker høj eller lav. 

\begin{lstlisting}[language=C,numbers=none]
void tw_send(int bc)
{
    switch(bc)
    {
    case 1:	//babycon_level = 1, must set GPIO OUTPUT_PIN_MSB = 0, OUTPUT_PIN_LSB = 1.
        *pFIO_FLAG_S = TW_OUTPUT_PIN_LSB; // Pointer to BANKS' GPIO write to set register (Set high)
        *pFIO_FLAG_C = TW_OUTPUT_PIN_MSB; //Pointer to BANKS' GPIO Write to Clear register (set low)
        break;

    case 2:
        //babycon_level = 2, must set GPIO OUTPUT_PIN_MSB = 1, OUTPUT_PIN_LSB = 0.
        *pFIO_FLAG_S = TW_OUTPUT_PIN_MSB;
        *pFIO_FLAG_C = TW_OUTPUT_PIN_LSB;
        break;

    case 3:
        //babycon_level = 3, must set GPIO OUTPUT_PIN_MSB = 1, OUTPUT_PIN_LSB = 1.
        *pFIO_FLAG_S = TW_OUTPUT_PIN_LSB;
        *pFIO_FLAG_S = TW_OUTPUT_PIN_MSB;
        break;
    default:
        //babycon_level != 1-3, must set GPIO OUTPUT_PIN_MSB = 0, OUTPUT_PIN_LSB = 0 for error
        *pFIO_FLAG_C = TW_OUTPUT_PIN_MSB;
        *pFIO_FLAG_C = TW_OUTPUT_PIN_LSB;
        break;
    }
}
\end{lstlisting}