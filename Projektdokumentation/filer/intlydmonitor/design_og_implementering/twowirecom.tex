%TwoWireCom
\subsection{TwoWireCom}
TwoWireCom er klassen som står for at sende BABYCON-niveauet til Controller via to GPIO pins. 

Klassen er implementeret ved hjælp af funktioner fra T\_GPIO\_BANKS. Inkluderingen af de nødvendige biblioteker kan ses i kodestykket herunder. 

\begin{verbatim}
#include "adi_initialize.h"
#include <sys\exception.h>
#include <cdefBF533.h>
#include "sysreg.h"
#include "ccblkfn.h"
\end{verbatim}

Som det ses i kodestykket herunder, er klassen TwoWireCom simpel med kun en funktion \textit{send(int bc)} som modtager et BABYCON-niveau i form af en int.
Der benyttes to GPIO pins PF4 og PF6, disse skal repræsentere henholdsvis MSB og LSB jf. \ref{Overordnet:Kommunikationsprotokol/BABYCON1} . Yderlige specifikationer kan findes i databladet \textbf{INDSÆT REFERENCE} side 84. 

\begin{verbatim}
class TwoWireCom {
public:
	TwoWireCom();
	virtual ~TwoWireCom();
	void send(int);
};
\end{verbatim}

\textit{send(int bc)} implementeres som en switch case, der switcher på de tre BABYCON-niveauer 1, 2 og 3. Som default sendes 0. Dette er koden for error. Det kan også være at funktionen er kaldt med en værdi der ikke er 1, 2 eller 3.  
pFIO\_FLAG's benyttes til at sætte GPIO-pin højt eller lavt, ved at sætte flaget lig med den GPIO pin man ønsker høj eller lav. 

\begin{verbatim}
void TwoWireCom::send(int bc)
{
	switch(bc){
	case 1:	//babycon_level = 1, must set GPIO OUTPUT_PIN_MSB = 0, OUTPUT_PIN_LSB = 1.
		*pFIO_FLAG_S = OUTPUT_PIN_LSB; // Pointer to BANKS' GPIO write to set register (Set high)
		*pFIO_FLAG_C = OUTPUT_PIN_MSB; //Pointer to BANKS' GPIO Write to Clear register (set low)
		break;

	case 2:
		//babycon_level = 2, must set GPIO OUTPUT_PIN_MSB = 1, OUTPUT_PIN_LSB = 0.
		*pFIO_FLAG_S = OUTPUT_PIN_MSB;
		*pFIO_FLAG_C = OUTPUT_PIN_LSB;
		break;

	case 3:
		//babycon_level = 3, must set GPIO OUTPUT_PIN_MSB = 1, OUTPUT_PIN_LSB = 1.
		*pFIO_FLAG_S = OUTPUT_PIN_LSB;
		*pFIO_FLAG_S = OUTPUT_PIN_MSB;
		break;
		
	default:
		//babycon_level != 1-3, must set GPIO OUTPUT_PIN_MSB = 0, OUTPUT_PIN_LSB = 0 for error
		*pFIO_FLAG_C = OUTPUT_PIN_MSB;
		*pFIO_FLAG_C = OUTPUT_PIN_LSB;
		break;
	}
\end{verbatim}

