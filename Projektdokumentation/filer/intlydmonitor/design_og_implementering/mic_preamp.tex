%Mikrofon preamp design

\subsection{Mikrofon preamp}

\subsubsection{Mikrofon} 
Som mikrofon er valgt en MCE-100 elektret mikrofon. Et udpluk af specifikationer ses herunder:
\begin{itemize}
	\item Frequency range: 50 Hz to 10 kHz
	\item Sensitivity: 5,6 mV / Pa / 1 kHz
	\item Output impedans: 2 kOhm
	\item Power Supply: 1.5 to 10 V dc, 5 mA
\end{itemize}
 
På Figur \ref{intlyd:mic_preamp:elektret} ses en tegning af en elektret mikrofon som er en videreudvikling af kondensatormikrofonen, hvor bagelektroden har en "indefrosset" ladning i materialet. Kapaciteten i mikrofonen (ca. 10 pF) ændres ved trykvariationer idet afstanden mellem membranen og bagelektroden varieres. 

\figur{0.4}{intlydmonitor/design/elektret_tore}{Tegning af elektrektrostatisk mikrofon. Kilde: Analogteknik}{intlyd:mic_preamp:elektret}

Dog har kablingen mellem mikrofonen og forstærkeren en betydelig kapacitet, så mikrofonpakken er implementeret med en indbygget JFET som buffer, som det ses af Figur \ref{intlyd:mic_preamp:elektret_forsyn}
\figur{0.6}{intlydmonitor/design/elektret_kreds}{Elektret forsyningskredsløb}{intlyd:mic_preamp:elektret_forsyn}

Forsyningen til mikrofonen bliver det halve af Blackfin-kittets 7,5 V, således at det efterfølgende forstærker-kredsløb kan implementeres med +/- 3,75 V forsyning. Således U\begin{tiny}DC\end{tiny} = 3,75 V. 
Vi ved fra databladet for mikrofonen, at den trækker 5mA og at modstanden R\begin{tiny}d\end{tiny} = 2 kOhm (opgivet som "Output Impedance").

Dette giver et spændingsfald på 0.8 V over R\begin{tiny}d\end{tiny} og dermed 2,95 V over den i mikrofonkapslen indbyggede JFET. I dette område af U\begin{tiny}DS\end{tiny} vil transistoren have en strømbegrænsnde virkning og kan derfor bruges som en tilnærmelsesvis lineær strømkilde. Herunder ses for illustration af princippet en JFET med DC arbejdspunktet indtegnet.
\figur{1}{intlydmonitor/design/JFET_Karakteristik}{Karakteristik for JFET'en BF245A. Kilde: Analogteknik}{intlyd:mic_preamp:jfet_karakteristik}

Uden lydtryk vil V\begin{tiny}GS\end{tiny} = 0 V, hvilket resulterer i en I\begin{tiny}D\end{tiny}= 4 mA (markeret på venstre graf i figur \ref{intlyd:mic_preamp:jfet_karakteristik}). Dette DC-arbejdspunkt er for vores mikrofon 5 mA. V\begin{tiny}DS\end{tiny} er ca. 3 V og det ses på højre graf at strømmen i arbejdspunktet, I\begin{tiny}D\end{tiny}, vil ligge lidt lavere. Småsignal strømmen, i\begin{tiny}M\end{tiny}, vil således lade strømmen variere omkring dette punkt.

For figur \ref{intlyd:mic_preamp:elektret_forsyn}, punktet på højre side af kondensatoren C, vil DC-strømmen være sorteret fra og den af mikrofonen modulerede AC-strøm, i\begin{tiny}M\end{tiny}, vil være at finde. Af databladet ved vi, at mikrofonens sensitivitet er S = 5,6 mV/Pa, og i\begin{tiny}M\end{tiny} vi således være givet ved: 
\begin{center}
${ i }_{ M }=\frac { S }{ { R }_{ D } } \cdot p$
\end{center}
Hvor p er lydniveau (Pa)

\subsubsection{PreAmp} 
Det er PreAmpens opgave at omdanne modulationsstrømmen i\begin{tiny}M\end{tiny} til en line level spænding for Blackfin's ADC 

Blackfin ADC'en tager et line-level input på +/- 1,65 V. 
Det ønskes at udnytte det maksimale dynamiske område uden at lade signalet klippe. Det regnes med at mikrofonen ikke udsættes for mere end 2 Pa ved almindelig brug. Den maksimale strøm-amplitude bliver derfor: 
\begin{center}
${ i }_{ M }=\frac { 5,6mV/Pa }{ 2k\Omega  } \cdot 2Pa=5,6 \mu A$
\end{center}
Dette signal skal forstærkes med en TIA op til det ønskede line level på 1,65 V. Den ønskede forstærkning, G, bliver således:
\begin{center}
$G=\frac { 1,65V }{ 5,6\mu A } =2,89\cdot { 10 }^{ 5 }\frac { V }{ A } $
\end{center}

Der benyttes en transimpedansforstærker til at realisere denne forstærkning. 
\figur{0.4}{intlydmonitor/design/TIA.pdf}{Transimpedansforstærker: ${ V }_{ o }={ R }_{ FB }\cdot { I }_{ in }$ }{intlyd:mic_preamp:tia}

Den ønskede transimpedansforstærkning er givet direkte ved værdien af feedbackmodstanden, R\begin{tiny}FB\end{tiny}.
\begin{center}
${ R }_{ FB }=\frac { { V }_{ o } }{ { I }_{ in } } =G=289k\Omega$
\end{center}

\textbf{Valg af OpAmp}

Da kredsen skal bruges til lydbehandling, er forstærkerens \textit{Slew Rate} specifikation vigtig.
Operationsforstærkerens interne kondensator, C\begin{tiny}C\end{tiny}, udgør en begrænsning for hvor hurtigt udgangen kan flytte sig, og for lydbehandling skal denne være så høj som muligt.
\figur{0.5}{intlydmonitor/design/slew_rate}{Illustration af OpAmp parameteren "Slew Rate". Kilde: Analogteknik}{intlyd:mic_preamp:slew_rate}

En forstærkers slew rate er givet ved:
\begin{center}
$SR={ \left[ \frac { du }{ dt }  \right]  }_{ MAX }\Rightarrow SR=2\cdot \pi \cdot f\cdot { U }_{ M }$
\end{center}
Hvor f er højeste arbejdsfrekvens og U\begin{tiny}M\end{tiny} er udgangsspændingens amplitude. Slew Rate er således en værdi for hvor hurtigt operationsforstærkeren kan flytte udgangen et antal volt (målt i MV/s).
For vores applikation har vi en U\begin{tiny}M\end{tiny} = 1.65 V, og som båndbredde, f, der vælges en konservativ værdi på 20 kHz, i det tilfælde, at kredsen senere bruges med en bedre mikrofon. Dette giver følgende SR:
\begin{center}
$SR=2\cdot \pi \cdot 20kHz\cdot 1,65V=0,2MV/s$
\end{center}

Ved denne SR påkræves dog et meget kraftigt indgangssignal, der også vil resultere i høj forvrængning. For at sikre en forvrængning på under 1\% bør u\begin{tiny}M\end{tiny} < 20mV.
\begin{center}
${ u }_{ M }=(2k\Omega ||22k\Omega )\cdot { 5,6µA=10,3mV }$
\end{center}
Dette krav er altså opfyldt.

Ved en grænse på 20mV vil 20 \% af differentialtrinnets udstyringsmulighed på ±I\begin{tiny}E\end{tiny} udnyttes. Den påkrævede SR vil derfor være 5 gange den hidtil beregnede, altså 1MV/s.

Dette betyder altså at vi kan nøjes med at bruge en billig operationsforstærker uden større krav til SR, såsom en OpAmp fra den i lydbehandling almindeligt anvendte TL071-serie ville give (SR = 13MV/s).

\textbf{Stabilitet}

Der er en betydelig kapacitet på forstærkerens indgang idet mikrofonen er koblet med et coax-kabel.

\figur{0.9}{intlydmonitor/design/OpAmp_med_indgangskapacitet}{OpAmp med kapacitiv belastning på indgangen. Kilde: Analogteknik}{intlyd:mic_preamp:tia_ind_kap}

På figur \ref{intlyd:mic_preamp:tia_ind_kap} ses TIA'en som de næste udregninger tager udgangspunkt i. Mikrofonen er kobles med ca. 1m coax kabel med kapaciteten 101pF/m. Desuden foregår der en HF filtrering med et RC filter og en 100nF kondensator i parrallel mellem signal og stel. Se afsnittet med særlige EMC-hensyn INDSÆT REF. Dette giver altså en C\begin{tiny}1\end{tiny} på 100 pF.
Der regnes med en typis GBP på 9MHz. Der testes for nødvendighed af kondensator i tilbagekobling:
\begin{center}
${ C }_{ 1 }<\frac { 1 }{ 8\pi \cdot GBP\cdot { R }_{ 2 } } \Longrightarrow \quad 101nF<\frac { 1 }{ 8\pi \cdot 9MHz\cdot 28,9k\Omega  } \quad \Longrightarrow \quad 101\cdot { 10 }^{ -9 }<153\cdot { 10 }^{ -15 }$
\end{center}
Det er altså nødvendigt at sætte en kondensator i tilbagekoblingen, som illustreret i \ref{intlyd:mic_preamp:tia_fb_kap}.

\figur{0.9}{intlydmonitor/design/OpAmp_med_tilbagekoblingskapacitet}{OpAmp med kapacitiv belastning på indgangen og stabiliserende kondensator i tilbagekoblingen. Kilde: Analogteknik}{intlyd:mic_preamp:tia_fb_kap}

\textit{Feedback kondensatoren} beregnes som følger:
\begin{center}
${ C }_{ 2 }\approx 0.8\sqrt { \frac { { C }_{ 1 } }{ GBP\cdot { R }_{ 2 } }  } \quad \Longrightarrow \quad { C }_{ 2 }=0.8\sqrt { \frac { 101nF }{ 9MHz\cdot 28,9k\Omega  }  } =496pF\quad$
\end{center}

Der benyttes en 500 pF kondensator. Den \textit{resulterende grænsefrekvens} bliver således:
\begin{center}
${ f }_{ H }=\frac { 1 }{ 2\pi \cdot { R }_{ 2 }\cdot { C }_{ 2 } } \quad \Longrightarrow \quad { f }_{ H }=\frac { 1 }{ 2\pi \cdot 28,9k\Omega \cdot 18pF } =11,01kHz$
\end{center}

Dette begrænser altså ikke netop ikke signalets båndbredde.

\textbf{Mic Preamp samlet kredsløb}

På Figur \ref{intlyd:mic_preamp:multisim} ses kredsløbet for Mic Preamp. Der benyttes i virkeligheden en elektret mikrofon. Denne er simuleret ved en spændingskilde, en kondensator og en transistor. Ved implementering skal benyttes et coaxialkabel til at forbinde mikrofon til mikrofonprintet. Coaxialkablet er simuleret ved en modstand, en spole og en kondensator. 

\figur{1}{intlydmonitor/design/mic_preamp_multisim}{Opbygning af Mic preamp i multisim}{intlyd:mic_preamp:multisim}

Af Figur \ref{intlyd:mic_preamp:multisim} er Pos en forsyning på 7.5V, Neg er GND og virGND er en virtual ground på 3.5V. Ved OpAmp'en er der indsat to afkoblingskondensatorer C5 og C7. 
Forsyningen til mikrofonen kommer fra pos gennem R3. 


