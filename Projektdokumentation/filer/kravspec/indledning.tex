\section{Indledning}

Formålet med projektet er at lave en prototype af en intelligent babymonitor til barnevogne med dertilhørende vuggesystem, statushjemmeside og e-mail notifikation.

Systemet har tre tilstande en putte-tilstand, hvor barnet lægges til at sove, en monitorerings-tilstand og undtagelse-tilstand.
I \textbf{putte-tilstanden}, skal systemet vugge babyen i et fastsat tidsinterval, hvorefter monitoreringen overtager.
I \textbf{monitorerings-tilstanden} styres barnevognens vuggefunktion på baggrund af analysen af den aktuelle baby-lydoptagelse.
I \textbf{undtagelses-tilstanden} låses systemet. Barnevognen skal ikke vugges. En e-mail afsendes til den registrerede babypasser hvert 30. sekund, indtil systemet resettes manuelt ude ved barnevognen. 

Statushjemmesiden opdateres løbende, med status på baby, indledt i tre konditioner vist som en BABYCON statusbar.

\begin{itemize}
\item \textbf{BABYCON3} 
\newline På dette niveau kategoriseres lydsignalet fra barnet som roligt. Heraf skal barnevognen ikke vugges. Systemet indsamler lyd og afventer en ændring i lydsignalet, som vil medføre en ændring i niveau. 

\item \textbf{BABYCON2}
\newline På dette niveau kategoriseres lydsignalet fra barnet som uroligt. Heraf skal barnevognen vugge med en forudbestemt frekvens. Systemet indsamler lyd og afventer en ændring i lydsignalet, som vil medføre en ændring i niveau. 

\item \textbf{BABYCON1}
\newline På dette niveau kategoriseres lydsignalet fra barnet som alarmerende. Undtagelses-tilstanden aktiveres. 
\end{itemize}
 
\subsection*{Systemtegning}
\figur{1}{Systemtegning}{Skitse af systemet Baby Watch}{kravspec:systemtegning}

Figur \ref{kravspec:systemtegning} viser illustrativt systemet. Barnevognen er udstyret med en controller, der styrer den intelligente lydmonitor samt vuggesystemet. Controlleren forbindes via Wi-Fi til en server. Babypasseren modtager information på en hjemmeside og modtager desuden mail når systemets tilstand skifter til BABYCON1.