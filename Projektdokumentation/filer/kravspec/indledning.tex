\section{Indledning}

Formålet med projektet er at lave en prototype af en intelligent babymonitor til barnevogne med dertilhørende vuggesystem, statushjemmeside og e-mail notifikation.

Systemet har tre tilstande en putte-tilstand, hvor barnet lægges til at sove, en monitorerings-tilstand og undtagelse-tilstand.
I \textbf{putte-tilstanden}, skal systemet vugge babyen i et fastsat tidsinterval, hvorefter monitoreringen overtager.
I \textbf{monitorerings-tilstanden} styres barnevognens vuggefunktion på baggrund af analysen af den aktuelle baby-lydoptagelse.
I \textbf{undtagelses-tilstanden} låses systemet. Barnevognen skal ikke vugges. En e-mail afsendes til den registrerede babypasser hvert 30. sekund, indtil systemet resettes manuelt ude ved barnevognen. 

Statushjemmesiden opdaterer løbende om babyens tilstand. Babyens tilstand bliver på hjemmesiden kategoriseret i tre konditioner vist via. en BABYCON statusbar:

\begin{itemize}
\item \textbf{BABYCON3} 
\newline På dette niveau kategoriseres lydsignalet fra barnet som roligt. Derfor skal barnevognen ikke vugges. Systemet indsamler lyd og afventer en ændring i lydsignalet, som vil medføre en ændring i niveau. 

\item \textbf{BABYCON2}
\newline På dette niveau kategoriseres lydsignalet fra barnet som uroligt. Heraf skal barnevognen vugge med en forudbestemt frekvens. Systemet indsamler lyd og afventer en ændring i lydsignalet, som vil medføre en ændring i niveau. 

\item \textbf{BABYCON1}
\newline På dette niveau kategoriseres lydsignalet fra barnet som alarmerende. Undtagelses-tilstanden aktiveres. 
\end{itemize}
 
\subsection*{Systemtegning}
\figur{1}{Systemtegning}{Skitse af systemet Baby Watch}{kravspec:systemtegning}

Figur \ref{kravspec:systemtegning} viser illustrativt systemet. Barnevognen er udstyret med en controller som står for at styrer de andre delsystem. På controlleren sidder et brugerpanel som sørger for at brugeren fysisk kan interagerer med systemet via knapper og LED'er.
Den intelligente lydmonitor består af en mikrofon og en digital signal processer (DSP), som sender informationer til controlleren om barnets tilstand ud fra en processering af barnets lyde. Vuggesystemet består af en motor og et selvregulerende system, som sørger for, at barnet altid vugges ud fra et vandret niveau. Vuggesystemet kan opererer i flere vuggetilstande som varieres med forskellige vuggefrekvenser og -amplituder alt efter hvad controlleren giver besked om.  Controlleren indholder også en e-mail server og en http server. Controlleren har forbindelse via Wi-Fi til internettet. Babypasseren kan via disse serverer opsøge information om barnets tilstand på en hjemmeside og modtage e-mails når systemet detekter at barnet skifter til en alarmerende tilstand eller hvis der opstår fejl i systemet.