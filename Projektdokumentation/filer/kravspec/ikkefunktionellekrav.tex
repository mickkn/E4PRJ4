\section{Ikke-funktionelle krav}

\subsection*{Mikrofon}
For at kunne opfange et tilstrækkeligt signal til analyse af babyens gråd, skal systemets mikrofon opfylde følgende krav:
\begin{itemize}
\item Mikrofonen skal have en max SPL rating på min. \SI{120}{\dB}.\footnote{FIXME indsæt reference til studie om gråd volumen}
\item Mikrofonen skal have en jævn frekvens respons på maksimalt +/- \SI{5}{\dB} fra \SI{40}{\hertz} til \SI{10}{\kilo\hertz}.
\end{itemize}

\subsection*{Vuggemekanisme}

\textbf{Nivellering}: \label{kravspec:ikke_funk_nivellering}
\begin{itemize}
	\item Vuggesystemet skal kunne nivellere planet, hvorpå babyen ligger, til vandret position indenfor \SI{2}{\degree}.
	\item Barnevognens understel må stå på et plan med op til \SI{5}{\degree} hældning.
	\item Når systemet er tændt, men ikke skal vugge, nivelleres planet, hvorpå babyen ligger, automatisk til vandret.
\end{itemize}

Ved vugning jf. UC2 gennemgår vugningen af barnet.
Systemets vugge mekanisme skal overholde følgende krav for at sikre en blid vugning:
\begin{itemize}
\item Vuggen skal kunne vippe planet, hvorpå babyen ligger, med op til \SI{10}{\degree} i hver retning fra dets vandrette udgangspunkt, med en fejlmargin på \SI{2}{\degree}.
\item Vuggen skal kunne variere frekvensen, hvormed der vugges fra \SI{0}{\hertz} til \SI{2}{\hertz}, med en fejlmargin på \SI{0.2}{\hertz}.
\item Vuggen skal vende tilbage til vandret indenfor en vinkel på \SI{2}{\degree}, når systemet lukkes ned.
\item Vuggen skal have en begrænsning på vinkelfrekvensen ved \SI{50}{\degree\per\second}. \footnote{De valgte grænseværdier for hastighed og accelleration er bestemt på baggrund af en undersøgelse af almindelig vugning, som dokumenteret på \citep{cd} under textit{/andet/teknologiundersoelse\_af\_alm\_vugning.pdf}}
\item Vuggens vinkel acceleration skal være begrænset ved \SI{250}{\degree\per\square\second}. \footnote{Se fodnote til bestemmelse af grænseværdi for vinkelacceleration.}
\end{itemize}

\textbf{Vuggetilstande}: \label{kravspec:ikke_funk_vuggetilstande}
Ved vugning jf. UC2 gennemgår vugningen af barnet en sekvens af tre vuggetilstande med et interval på 2 min.
\begin{enumerate}

\item Vugning foregår med en frekvens på 0,5 Hz og en vinkel-amplitude på \SI{10}{\degree} +/- \SI{2}{\degree}
\item Vugning foregår med en frekvens på 1 Hz og en vinkel-amplitude på \SI{6}{\degree} +/- \SI{2}{\degree}
\item Vugning foregår med en frekvens på 2 Hz og en vinkel-amplitude på \SI{4}{\degree} +/- \SI{2}{\degree}
\end{enumerate}

Udover de tre sekventielle tilstande har Baby Watch; en manuel vuggetilstand der vugger med en frekvens på 0,75 Hz og en vinkel-amplitude på \SI{8}{\degree} +/- \SI{2}{\degree} og en ''nul''-vuggetilstand, der ikke vugger men bare nivellerer planet, hvorpå barnet ligger til \SI{0}{\degree} +/- \SI{2}{\degree}.

\subsection*{Baby status}
For at sikre at vurderingen af babyens status er pålidelig samt rettidigt tilgængelig for brugeren, skal systemet overholde følgende:
\begin{itemize}
\item Systemets BABYCON-statusbar (se illustration nedenfor) skal opdateres mindst hvert 10. sekund.
\item Statushjemmesiden skal være opdateret, senest 5 sekunder efter controlleren har opdateret babystatus.
\item Når BABYCON-statusbaren opdateres til BABYCON1-niveau skal hjemmesiden afspille en alarmlyd.
\item I undtagelsestilstand afsendes der mails med maksimum 20 sekunders interval. 
\end{itemize}

\subsection*{Controller}
Krav til Controllerens advisering og virkemåder:
\begin{itemize}
\item Wi-Fi-LEDen skal tænde maksimum 15 sekunder efter afbrydelse af netværket.
\item Wi-Fi-LEDen skal slukke maksimum 25 sekunder efter re-etablering af netværket.
\end{itemize}