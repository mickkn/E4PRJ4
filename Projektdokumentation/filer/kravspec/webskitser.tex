\section{Web-skitser}

%\figur{Bredden}{Pdfnavn}{Billedtekst}{Label}

De følgende figurer \ref{kracspec:BABYCON_GUI_3} , \ref{kracspec:BABYCON_GUI_2} samt \ref{kracspec:BABYCON_GUI_1} skitser statushjemmesidens udseende. Hjemmesiden viser aktuel tid, samlet tid som barnet har været i ro, et billede af barnet med tilhørende navn og besked, BABYCON skala fra 1-3 og information omkring hvornår hjemmesiden sidst er opdateret.

\figur{1}{kravspec/GUI_BABYCON_3}{BABYCON3}{kracspec:BABYCON_GUI_3}

Figur \ref{kracspec:BABYCON_GUI_3} illustrerer hjemmesiden når BABYCON niveauet er 3, det niveau hvor Babyen er rolig.

\figur{1}{kravspec/GUI_BABYCON_2}{BABYCON2}{kracspec:BABYCON_GUI_2}

Figur \ref{kracspec:BABYCON_GUI_2} illustrerer hjemmesiden når BABYCON niveauet er 2, det niveau hvor Babyen er urolig, men ikke nok til at udløse alarm til Babypasser.

\figur{1}{kravspec/GUI_BABYCON_1}{BABYCON1}{kracspec:BABYCON_GUI_1}

Figur \ref{kracspec:BABYCON_GUI_1} illustrerer hjemmesiden når BABYCON niveauet er 1. BABYCON1 er niveauet hvor Babyen er alarmerende utilfreds. Babypasseren modtager en e-mail og skal selv trøste Baby. På dette niveau skal hjemmesiden afspille en alarmlyd og det BABYCON røde felt skal blinke.