\subsection*{Hardwaredesign}

Controlleren består af et betjeningspanel samt en raspberry pi der kontrollerer betjeningspanelet og styrer forbindelserne til hhv. den Intelligente Lydmonitor og Vuggesystemet.

Betjeningspanelet består som beskrevet af 3 lysdioder. Én grøn, én gul og én rød der hhv. indikerer at Baby Watch er tændt/slukket, at "Manuel start" er aktiveret/deaktiveret samt Wi-Fi status.Ydermere består betjeninsapanelet af 2 knapper, én til at tænde/slukke for Baby Watch systemet og én til at aktivere Manuel start.
 
\subsubsection*{Lysdioder}
Der benyttes 5mm dioder fra komponentrummet: 

\begin{itemize}
	\item Grøn 5mm LED: KINGBRIGHT L-53 GD
	\item Gul 5mm LED: KINGBRIGHT L-53 YD
	\item Rød 5mm LED: KINGBRIGHT L-53 HD
\end{itemize}


\figur{1}{controller/design/L53_LEDS}{Udsnit af datablad for KINGBRIGHT L53 HD, GD og YD }{controller:ledSpec}

Ud fra figur \ref{controller:ledSpec} ses strømmen som funktion af spændingen over dioderne. Den indtegnede blå linje på hver af 3 kurver angiver spændingsfaldet over hver diode når strømmen er sat til 10 mA. Ud fra aflæsning på kurverne beregnes for modstandene for dioderne

\figur{0.5}{controller/design/L53_resistor}{For modstandsberegninger for de 3 dioder}{controller:resistors}

Ud fra modstandsberegningerne i figur \ref{controller:resistors} er kredsløbsdiagrammet, se figur \ref{betjeningsapanel:schematic}, opbygget. De to knapper for ON/OFF samt Manuel start er medtaget sammen med deres GPIO porte på Rasberry Pi model b.

\figur{0.7}{controller/design/betjeningspanel_multisim}{Kredsløbsdiagram for Controller}{betjeningspanel:schematic}

Tabellen herunder angiver opsætningen af GPIOerne på Raspberry Pi'en for at ovenstående diagram kan fungere.

\begin{center}
\label{ctrl:raspberry_pi_setup}
\begin{longtable}{|p{3cm}|p{4cm}|p{4cm}|}
\hline
\textbf{GPIO}	&\textbf{Opsætning} 		&\textbf{Kommentar} 	\\ \hline
\endfirsthead
\multicolumn{3}{l}{...fortsat fra forrige side} \\ \hline 
\textbf{GPIO}	&\textbf{Opsætning} 		&\textbf{Kommentar}  \\ \hline
\endhead

%% GPIO
%% &Opsætning

%% \\\hline

GPIO14
&Indgang sættes aktiv-lav
&Styrer ON/OFF-LED

\\\hline

GPIO18
&Indgang sættes aktiv-lav
&Styrer Manuelstart-LED

\\\hline


GPIO15
&Indgang sættes aktiv-lav
&Styrer Wi-Fi-LED

\\\hline

GPIO22
&Indgang sættes aktiv-høj, med intern pulldown modstand
&Tænd/sluk signal for Baby Watch

\\\hline

GPIO17
&Indgang sættes aktiv-høj, med intern pulldown modstand
&Aktivering af "Manuel-start"
\\\hline

\end{longtable}
\end{center}

\subsubsection*{Kommunikation til/fra Intelligent Lydmonitor og Vuggesystem}

\textbf{Kommunikation med Intelligent Lydmonitor}
Kommunikationen med den Intelligente Lydmonitor foregår som beskrevet i systemarkitekturen med 2 signalledere (2WireTTL). Herved opnås der fire kombinationsmuligheder, en for hvert BABYCON niveau samt en til indikation  af fejl.

GPIO9 og GPIO10 benyttes til denne to ledet forbindelse til Intelligent Lydmonitor, hvor GPIO9 er LSB delen og GPIO10 er MSB delen. 

Til den fysiske forbindelse designes der efter at benytte et USB kabel. Controllerens del af forbindelsen er et USB hun type B stik. Se figur \ref{controller/design/IL_USB}

\figur{0.3}{controller/design/IL_USB}{USB type B for kommunikation med Intelligent Lydmonitor}{betjeningspanel:IL_USB}
 
Der designes desuden et testprint se figur  \ref{controller/design/IL_testprint}

\figur{0.5}{controller/design/IL_testprint}{Testprint med dipswitch til at angive BABYCON niveau}{betjeningspanel:IL_testprint}

\textbf{Kommunikaiton med Vuggesystem}

I2C forbindelsen består som beskrevet af en clock (SCL) og en datalinje (SDA) samt en fælles ground forbindelse. Ligeledes med med kommunikationen til den Intelligente Lydmonitor benyttes et USB type B stik.

\figur{0.3}{controller/design/VS_USB}{USB type B for kommunikation med Vuggesystem}{betjeningspanel:VS_USB}

Der designes et testprint således at controlleren kan testes med et simpelt PSoC4 program og datakommunikationen kan verificeres vha. en Logic Analyser.

\figur{0.5}{controller/design/VS_testprint}{Testprint til I2C forbindelsen, med pins til PSoC4 samt Logic Analyser}{betjeningspanel:VS_testprint}

\subsubsection*{Samlet HW-design af Controller}

