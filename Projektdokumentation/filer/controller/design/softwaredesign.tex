\subsection{Softwaredesign}

I dette afsnit beskrives hvordan Controllerens software er designet med klassediagrammer

\subsubsection*{Klassebeskrivelser}

\figur{1}{controller/design/klassediagram}{Oversigt over klassediagrammerne for Controller}{controller:klassediagram}

{\centering
\textbf{Betjeningspanel}\par
}
\textbf{Ansvar:} At styre forbindelserne til det hardwarenære på den fysiske controller boks \

\begin{center}
    \begin{tabular}{ | l | p{10cm} |}
    \hline
    \textbf{Funktion}	& \verb+int setLedValue( int pin , int value ) +						\\ \hline
    \textbf{Parametre} 	& Modtager et GPIO pin-nummer og en værdi 0 for ON og 1 for OFF		\\ \hline
    \textbf{Returværdi}	& 0 ved succes. Minus værdi ved fejl 								\\ \hline
    \textbf{Beskrivelse}	& Bruges til at tænde og slukke for LED dioderne på controller		\\ \hline
    \end{tabular}
\end{center}


\verb+int setLedValue( int pin , int value ) +\\
\textbf{Parametre:}   Modtager et GPIO pin-nummer og en værdi 0 for ON og 1 for OFF \\
\textbf{Returværdi:}  0 ved succes. Minus værdi ved fejl \\
\textbf{Beskrivelse:} Bruges til at tænde og slukke for LED dioderne på controller \\

\begin{center}
    \begin{tabular}{ | l | p{10cm} |}
    \hline
    \textbf{Funktion}	& \verb+int getButValue( int pin ) const +						\\ \hline
    \textbf{Parametre} 	& Modtager et GPIO pin-nummer									\\ \hline
    \textbf{Returværdi}	& Status på knaptryk 											\\ \hline
    \textbf{Beskrivelse}	& Bruges til at læse fra trykknapper på controller				\\ \hline
    \end{tabular}
\end{center}

\verb+int getButValue( int pin ) const +\\
\textbf{Parametre:}   Modtager et GPIO pin-nummer \\
\textbf{Returværdi:}  Status på knaptryk \\
\textbf{Beskrivelse:} Bruges til at læse fra trykknapper på controller  \\

\verb+int ledOnOff_ +\\
\textbf{Beskrivelse:} Attribut til at holde GPIO nummer for On/Off LED \\

\verb+int ledMan_ +\\
\textbf{Beskrivelse:} Attribut til at holde GPIO nummer for Manuelstart LED \\

\verb+int ledWifi_ +\\
\textbf{Beskrivelse:} Attribut til at holde GPIO nummer for Netværksstatus LED\\

\verb+int butOnOff_ +\\
\textbf{Beskrivelse:} Attribut til at holde GPIO nummer for On/Off knappen \\

\verb+int butMan_ +\\
\textbf{Beskrivelse:} Attribut til at holde GPIO nummer for ''Manuelstart''-knap \\

{\centering
\textbf{EmailSMTP}\par
}
\textbf{Ansvar:} At sende Alarm og Fejl emails til Babypasser direkte fra main programmet \

\verb+void sendError() +\\
\textbf{Parametre:}   Ingen \\
\textbf{Returværdi:}  Ingen \\
\textbf{Beskrivelse:} Sender en Email med en fejlmeddelelse til Babypasser \\

\verb+void sendAlarm() +\\
\textbf{Parametre:}   Ingen \\
\textbf{Returværdi:}  Ingen \\
\textbf{Beskrivelse:} Sender en Email med en alarm besked til Babypasser \\

{\centering
\textbf{GPIOsocket}\par
}
\textbf{Ansvar:} At oprette et GPIO filarkiv \

\verb+int gpioExport( int pin ) +\\
\textbf{Parametre:}   GPIO pin nummer \\
\textbf{Returværdi:}  0 ved succes. Minus værdi ved fejl \\
\textbf{Beskrivelse:} Opretter et fil arkiv for det modtagne pin-nummer \\

\verb+int gpioUnexport( int pin ) +\\
\textbf{Parametre:}   GPIO pin nummer \\
\textbf{Returværdi:}  0 ved succes. Minus værdi ved fejl \\
\textbf{Beskrivelse:} Fjerner filarkivet for det modtagne pin-nummer \\

\verb+int gpioDirection( int pin , int dir ) +\\
\textbf{Parametre:}   GPIO pin nummer og pin retning \\
\textbf{Returværdi:}  0 ved succes. Minus værdi ved fejl \\
\textbf{Beskrivelse:} Sætter retningen for GPIO pin, INPUT(0) eller OUTPUT(1) \\

{\centering
\textbf{I2Csocket}\par
}
\textbf{Ansvar:} At kommunikere over I2C \

\verb+int writeReg( unsigned char reg_addr , unsigned char data ) +\\
\textbf{Parametre:}   Register adresse og data til skrivning \\
\textbf{Returværdi:}  0 ved succes. Minus værdi ved fejl \\
\textbf{Beskrivelse:} Skriver data til et register på en given enhed \\

\verb+int readReg( unsigned char reg_addr , unsigned char &data ) +\\
\textbf{Parametre:}   Register adresse og en data adresse til at gemme læst data i \\
\textbf{Returværdi:}  0 ved succes. Minus værdi ved fejl \\
\textbf{Beskrivelse:} Læser data fra et register på en given enhed \\

\verb+int openI2C() +\\
\textbf{Parametre:}   Ingen \\
\textbf{Returværdi:}  0 ved succes. Minus værdi ved fejl \\
\textbf{Beskrivelse:} Åbner I2C forbindelsen til en enhed i constructor \\

\verb+int closeI2C() +\\
\textbf{Parametre:}   Ingen \\
\textbf{Returværdi:}  0 ved succes. Minus værdi ved fejl \\
\textbf{Beskrivelse:} Lukker I2C forbindelsen til en enhed i destructor \\

\verb+std::string  i2cFileName +\\
\textbf{Beskrivelse:} Enhedsnavn på Raspberry Pi ''/dev/i2c-0'' eller ''/dev/i2c-1'' \\

\verb+int i2cDescriptor +\\
\textbf{Beskrivelse:} Fil descriptor til åbning af I2C forbindelsen \\

\verb+unsigned char deviceAddress +\\
\textbf{Beskrivelse:} Attribut til at holde Enheds adressen på den enhed der skal kommunikeres med  \\

{\centering
\textbf{Networkstatus}\par
}
\textbf{Ansvar:} At aflæse om der er forbindelse til et netværk \

\verb+bool statusEth0() +\\
\textbf{Parametre:}   Ingen \\
\textbf{Returværdi:}  True ved forbindelse ellers false \\
\textbf{Beskrivelse:} Læser på ''operstate'' for Eth0 forbindelsen på Raspberry Pi og returnere om den er oppe eller nede \\

\verb+bool statusWlan0() +\\
\textbf{Parametre:}   Ingen \\
\textbf{Returværdi:}  True ved forbindelse ellers false \\
\textbf{Beskrivelse:} Læser på ''operstate'' for Wlan0 forbindelsen på Raspberry Pi og returnere om den er oppe eller nede \\

{\centering
\textbf{TWTTLsocket}\par
}
\textbf{Ansvar:} At aflæse status fra Lydmonitoreringen \

\verb+int getPinValue( int pin ) const +\\
\textbf{Parametre:}   GPIO pin nummer \\
\textbf{Returværdi:}  Pin værdi \\
\textbf{Beskrivelse:} Bruges til læsning af MSB og LSB bit fra den Intelligente Lydmonitor \\

\verb+int getBabyconLevel(void) +\\
\textbf{Parametre:}   Ingen \\
\textbf{Returværdi:}  BABYCON niveau 0, 1, 2 eller 3 \\
\textbf{Beskrivelse:} Udlæsning af BABYCON niveau \\

\verb+int msbPin_ +\\
\textbf{Beskrivelse:} Attribut til at holde GPIO nummer for MSB-bit \\

\verb+int lsbPin_ +\\
\textbf{Beskrivelse:} Attribut til at holde GPIO nummer for LSB-bit \\

{\centering
\textbf{Website}\par
}
\textbf{Ansvar:} Til opdatering af hjemmeside \