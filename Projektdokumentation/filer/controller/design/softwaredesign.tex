\subsection{Softwaredesign}

I dette afsnit beskrives hvordan Controllerens software er designet med klassediagrammer

\subsubsection*{Klassebeskrivelser}

\figur{1}{controller/design/klassediagram}{Oversigt over klassediagrammerne for Controller}{controller:klassediagram}

{\centering
\textbf{Betjeningspanel}\par
}
\textbf{Ansvar:} At styre forbindelserne til det hardwarenære på den fysiske controller boks \

\begin{center}
    \begin{tabular}{ | l | p{11,8cm} |}
    \hline
    \textbf{Funktion}	& \verb+int setLedValue( int pin , int value ) +						\\ \hline
    \textbf{Parametre} 	& Modtager et GPIO pin-nummer og en værdi 0 for ON og 1 for OFF		\\ \hline
    \textbf{Returværdi}	& 0 ved succes. Minus værdi ved fejl 								\\ \hline
    \textbf{Beskrivelse}	& Bruges til at tænde og slukke for LED dioderne på controller		\\ \hline
    \end{tabular}
\end{center}


\begin{center}
    \begin{tabular}{ | l | p{11,8cm} |}
    \hline
    \textbf{Funktion}	& \verb+int getButValue( int pin ) const +						\\ \hline
    \textbf{Parametre} 	& Modtager et GPIO pin-nummer									\\ \hline
    \textbf{Returværdi}	& Status på knaptryk 											\\ \hline
    \textbf{Beskrivelse}	& Bruges til at læse fra trykknapper på controller				\\ \hline
    \end{tabular}
\end{center}


\begin{center}
    \begin{tabular}{ | l | p{11,8cm} |}
    \hline
    \textbf{Attribut}		& \verb+int ledOnOff_ +									\\ \hline
    \textbf{Beskrivelse} 	& Attribut til at holde GPIO nummer for On/Off LED		\\ \hline
    \end{tabular}
\end{center}

\begin{center}
    \begin{tabular}{ | l | p{11,8cm} |}
    \hline
    \textbf{Attribut}		& \verb+int ledMan_ +										\\ \hline
    \textbf{Beskrivelse} 	& Attribut til at holde GPIO nummer for Manuelstart LED		\\ \hline
    \end{tabular}
\end{center}

\begin{center}
    \begin{tabular}{ | l | p{11,8cm} |}
    \hline
    \textbf{Attribut}		& \verb+int ledWifi_ +										\\ \hline
    \textbf{Beskrivelse} 	& Attribut til at holde GPIO nummer for Netværksstatus LED	\\ \hline
    \end{tabular}
\end{center}

\begin{center}
    \begin{tabular}{ | l | p{11,8cm} |}
    \hline
    \textbf{Attribut}		& \verb+int butOnOff_ +										\\ \hline
    \textbf{Beskrivelse} 	& Attribut til at holde GPIO nummer for On/Off knappen		\\ \hline
    \end{tabular}
\end{center}

\begin{center}
    \begin{tabular}{ | l | p{11,8cm} |}
    \hline
    \textbf{Attribut}		& \verb+int butMan_ +											\\ \hline
    \textbf{Beskrivelse} 	& Attribut til at holde GPIO nummer for ''Manuelstart''-knap		\\ \hline
    \end{tabular}
\end{center}



{\centering
\textbf{EmailSMTP}\par
}
\textbf{Ansvar:} At sende Alarm og Fejl emails til Babypasser direkte fra main programmet \



\begin{center}
    \begin{tabular}{ | l | p{11,8cm} |}
    \hline
    \textbf{Funktion}	& \verb+void sendError() +									\\ \hline
    \textbf{Parametre} 	& Ingen														\\ \hline
    \textbf{Returværdi}	& Ingen 														\\ \hline
    \textbf{Beskrivelse}	& Sender en Email med en fejlmeddelelse til Babypasser		\\ \hline
    \end{tabular}
\end{center}

\begin{center}
    \begin{tabular}{ | l | p{11,8cm} |}
    \hline
    \textbf{Funktion}	& \verb+void sendAlarm() +									\\ \hline
    \textbf{Parametre} 	& Ingen														\\ \hline
    \textbf{Returværdi}	& Ingen 														\\ \hline
    \textbf{Beskrivelse}	& Sender en Email med en alarm besked til Babypasser		\\ \hline
    \end{tabular}
\end{center}


{\centering
\textbf{GPIOsocket}\par
}
\textbf{Ansvar:} At oprette et GPIO filarkiv \

\begin{center}
    \begin{tabular}{ | l | p{11,8cm} |}
    \hline
    \textbf{Funktion}	& \verb+int gpioExport( int pin ) +						\\ \hline
    \textbf{Parametre} 	& GPIO pin nummer										\\ \hline
    \textbf{Returværdi}	& 0 ved succes. Minus værdi ved fejl 					\\ \hline
    \textbf{Beskrivelse}	& Opretter et fil arkiv for det modtagne pin-nummer		\\ \hline
    \end{tabular}
\end{center}

\begin{center}
    \begin{tabular}{ | l | p{11,8cm} |}
    \hline
    \textbf{Funktion}	& \verb+int gpioUnexport( int pin ) +					\\ \hline
    \textbf{Parametre} 	& GPIO pin nummer										\\ \hline
    \textbf{Returværdi}	& 0 ved succes. Minus værdi ved fejl 					\\ \hline
    \textbf{Beskrivelse}	& Fjerner filarkivet for det modtagne pin-nummer			\\ \hline
    \end{tabular}
\end{center}

\begin{center}
    \begin{tabular}{ | l | p{11,8cm} |}
    \hline
    \textbf{Funktion}	& \verb+int gpioDirection( int pin , int dir ) +					\\ \hline
    \textbf{Parametre} 	& GPIO pin nummer og pin retning									\\ \hline
    \textbf{Returværdi}	& 0 ved succes. Minus værdi ved fejl								\\ \hline
    \textbf{Beskrivelse}	& Sætter retningen for GPIO pin, INPUT(0) eller OUTPUT(1)			\\ \hline
    \end{tabular}
\end{center}



{\centering
\textbf{I2Csocket}\par
}
\textbf{Ansvar:} At kommunikere over I2C \


\begin{center}
    \begin{tabular}{ | l | p{11,8cm} |}
    \hline
    \textbf{Funktion}	& \verb+int writeReg( unsigned char reg_addr , unsigned char data ) +		\\ \hline
    \textbf{Parametre} 	& Register adresse og data til skrivning									\\ \hline
    \textbf{Returværdi}	& 0 ved succes. Minus værdi ved fejl										\\ \hline
    \textbf{Beskrivelse}	& Skriver data til et register på en given enhed							\\ \hline
    \end{tabular}
\end{center}

\begin{center}
    \begin{tabular}{ | l | p{11,8cm} |}
    \hline
    \textbf{Funktion}	& \verb+int readReg( unsigned char reg_addr , unsigned char &data ) +		\\ \hline
    \textbf{Parametre} 	& Register adresse og en data adresse til at gemme læst data i			\\ \hline
    \textbf{Returværdi}	& 0 ved succes. Minus værdi ved fejl										\\ \hline
    \textbf{Beskrivelse}	& Læser data fra et register på en given enhed							\\ \hline
    \end{tabular}
\end{center}

\begin{center}
    \begin{tabular}{ | l | p{11,8cm} |}
    \hline
    \textbf{Funktion}	& \verb+int openI2C() +													\\ \hline
    \textbf{Parametre} 	& Ingen																	\\ \hline
    \textbf{Returværdi}	& 0 ved succes. Minus værdi ved fejl										\\ \hline
    \textbf{Beskrivelse}	& Åbner I2C forbindelsen til en enhed i constructor						\\ \hline
    \end{tabular}
\end{center}

\begin{center}
    \begin{tabular}{ | l | p{11,8cm} |}
    \hline
    \textbf{Funktion}	& \verb+int closeI2C() +													\\ \hline
    \textbf{Parametre} 	& Ingen																	\\ \hline
    \textbf{Returværdi}	& 0 ved succes. Minus værdi ved fejl										\\ \hline
    \textbf{Beskrivelse}	& Lukker I2C forbindelsen til en enhed i destructor						\\ \hline
    \end{tabular}
\end{center}


\begin{center}
    \begin{tabular}{ | l | p{11,8cm} |}
    \hline
    \textbf{Attribut}		& \verb+std::string  i2cFileName +								\\ \hline
    \textbf{Beskrivelse} 	& Enhedsnavn på Raspberry Pi ''/dev/i2c-0'' eller ''/dev/i2c-1''	\\ \hline
    \end{tabular}
\end{center}

\begin{center}
    \begin{tabular}{ | l | p{11,8cm} |}
    \hline
    \textbf{Attribut}		& \verb+int i2cDescriptor +										\\ \hline
    \textbf{Beskrivelse} 	& Fil descriptor til åbning af I2C forbindelsen					\\ \hline
    \end{tabular}
\end{center}

\begin{center}
    \begin{tabular}{ | l | p{11,8cm} |}
    \hline
    \textbf{Attribut}		& \verb+unsigned char deviceAddress + 								\\ \hline
    \textbf{Beskrivelse} 	& Attribut til at holde Enheds adressen på den enhed der skal kommunikeres med					\\ \hline
    \end{tabular}
\end{center}



{\centering
\textbf{Networkstatus}\par
}
\textbf{Ansvar:} At aflæse om der er forbindelse til et netværk \

\begin{center}
    \begin{tabular}{ | l | p{11,8cm} |}
    \hline
    \textbf{Funktion}	& \verb+bool statusEth0() +												\\ \hline
    \textbf{Parametre} 	& Ingen																	\\ \hline
    \textbf{Returværdi}	& True ved forbindelse ellers false										\\ \hline
    \textbf{Beskrivelse}	& Læser på ''operstate'' for Eth0 forbindelsen på Raspberry Pi og returnere om den er oppe eller nede						\\ \hline
    \end{tabular}
\end{center}

\begin{center}
    \begin{tabular}{ | l | p{11,8cm} |}
    \hline
    \textbf{Funktion}	& \verb+bool statusWlan0() +												\\ \hline
    \textbf{Parametre} 	& Ingen																	\\ \hline
    \textbf{Returværdi}	& True ved forbindelse ellers false										\\ \hline
    \textbf{Beskrivelse}	& Læser på ''operstate'' for Wlan0 forbindelsen på Raspberry Pi og returnere om den er oppe eller nede					\\ \hline
    \end{tabular}
\end{center}


{\centering
\textbf{TWTTLsocket}\par
}
\textbf{Ansvar:} At aflæse status fra Intelligent Lydmonitor \

\begin{center}
    \begin{tabular}{ | l | p{11,8cm} |}
    \hline
    \textbf{Funktion}	& \verb+int getPinValue( int pin ) const + 								\\ \hline
    \textbf{Parametre} 	& GPIO pin nummer														\\ \hline
    \textbf{Returværdi}	& Pin værdi																\\ \hline
    \textbf{Beskrivelse}	& Bruges til læsning af MSB og LSB bit fra den Intelligente Lydmonitor	\\ \hline
    \end{tabular}
\end{center}

\begin{center}
    \begin{tabular}{ | l | p{11,8cm} |}
    \hline
    \textbf{Funktion}	& \verb+int getBabyconLevel(void) +		 								\\ \hline
    \textbf{Parametre} 	& Ingen																	\\ \hline
    \textbf{Returværdi}	& BABYCON niveau 0, 1, 2 eller 3											\\ \hline
    \textbf{Beskrivelse}	& Udlæsning af BABYCON niveau											\\ \hline
    \end{tabular}
\end{center}


\begin{center}
    \begin{tabular}{ | l | p{11,8cm} |}
    \hline
    \textbf{Attribut}		& \verb+int msbPin_ +										\\ \hline
    \textbf{Beskrivelse} 	& Attribut til at holde GPIO nummer for MSB-bit				\\ \hline
    \end{tabular}
\end{center}

\begin{center}
    \begin{tabular}{ | l | p{11,8cm} |}
    \hline
    \textbf{Attribut}		& \verb+int lsbPin_ +		 								\\ \hline
    \textbf{Beskrivelse} 	& Attribut til at holde GPIO nummer for LSB-bit				\\ \hline
    \end{tabular}
\end{center}




{\centering
\textbf{Website}\par
}
\textbf{Ansvar:} Til opdatering af hjemmeside \