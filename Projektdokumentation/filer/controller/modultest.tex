\section{Modultest}

Modultesten af Controller er opstillet som en modultest hvor Controlleren er forbundet til et testprint der simulerer BABYCON-niveauer og en PSoC4 med et testprogram til at kommunikere vha. I2C, hvor statusværdien kan ændres manuelt.

\subsection*{''ON/OFF''-knap}

\subsubsection*{Formål}

At teste om ''ON/OFF''-LEDen lyser og programmet starter sit ON-loop når knappen er ON. Samt kontrollere om ''ON/OFF''-LEDen slukker når knappen er OFF og programmet forlader sit ON-loop.

\subsubsection*{Fremgangsmåde}
\begin{enumerate}
\item Tilkoble strøm til Controlleren med ''ON/OFF'' knappen i OFF position
\item Start programmet på Raspberry Pi'en
\item Sæt ''ON/OFF'' knappen i ON position
\item Iagttag ''ON/OFF''-LEDen samt terminal udskrift
\item Sæt ''ON/OFF'' knappen i OFF position
\item Iagttag ''ON/OFF''-LEDen samt terminal udskrift
\end{enumerate}

\subsubsection*{Forventet resultat} 
Det forventes at LEDen tænder når ''ON/OFF'' knappen bringes i ON position. Det forventes at terminal udskriften viser at programmet reagerer med udskrifterne [INSIDE ON LOOP] når knappen er i ON position og [OUTSIDE ON LOOP] når knappen er i OFF position.

\subsubsection*{Resultat} 
LEDen tænder og slukker som forventet og terminaludskriften passer med forventningerne. Se terminaludskriften på figuren herunder \ref{controller:Modul_ONOFF}
\figur{0.8}{controller/modultest/ON_OFF.pdf}{Terminaludskrift - Modultest:''ON/OFF''-knap}{controller:Modul_ONOFF}

\textit{Testen er godkendt}


\subsection*{''Manuel start''-knap}

\subsubsection*{Formål}
At sikre ''Manuel start'' knappen aktiverer ISR rutinen, her sættes en variabel som får en funktion til at køre. 

\subsubsection*{Fremgangsmåde}
\begin{enumerate}
\item Start programmet og sæt ''ON/OFF'' knappen i ON position
\item Aktiver ''Manuel start''-knappen ved at trykke på den 
\item Iagtag ''Manuel start''-LED'en og terminal udskriften
\item Afvent at LED'en slukker og iagtag terminal udskiften 
\end{enumerate}

\subsubsection*{Forventet resultat} 
Det forventes at ''Manuel start''-LED'en tænder umiddelbart når ''Manuel start''-knappen aktiveres. Terminal udskriften reagerer ved [MANUELSTART ISR] når knappen aktiveres, [MANUELSTART RUNNING] funktionen kører og [MANUELSTART DONE] når funktionen er kørt. Herefter slukkes ''Manuel start''-LED'en 

\subsubsection*{Resultat} 
''Manuel start''-LED'en tænder med det samme når ''Manuel start''-knappen aktiveres, som forventet. Og LED'en slukker når funktionen er kørt. Figur \ref{controller:Modul_manStart} viser terminal udskriften. [MANUELSTART ISR] skrives flere gange, det skyldes kontaktprel og er håndteret ved at der sættes en variabel første gang knappen aktiveres, denne variabel nulstilles af funktionen selv.

\figur{0.8}{controller/modultest/manStart.pdf}{Terminaludskrift - Modultest:''Manuel start''-knap}{controller:Modul_manStart}


\textit{Testen er godkendt}

\subsection*{Internet status - Wi-Fi LED}

\subsubsection*{Formål}
At teste systemet med og uden internet tilkoblet

\subsubsection*{Fremgangsmåde}
\begin{enumerate}
\item Start programmet og sæt ''ON/OFF'' knappen i ON position
\item Fjern internet forbindelsen ved at afbryde routeren fra netværket
\item Iagtag Wi-Fi LED'en og terminal udskrift
\item Tilføj internet forbindelsen ved at tilkoble routeren fra netværket
\item Iagtag Wi-Fi LED'en og terminal udskrift
\end{enumerate}

\subsubsection*{Forventet resultat} 
Det forventes at Wi-Fi LED'en lyser efter maks 10 sekunder og slukker igen maks 20 sekunder efter routeren er tilkoblet netværket. Terminalen viser tilsvarende ping udskrifter som viser status for internetforbindelsen.

\subsubsection*{Resultat} 


\textit{Testen er godkendt}

\subsection*{BABYCON niveau fra testprint}

\subsubsection*{Formål}
At teste systemts reaktion på de 3 BABYCON niveauter samt ved error/ikke tilkoblet Intelligent Lydmonitor.

\subsubsection*{Fremgangsmåde}
\begin{enumerate}
\item 

\end{enumerate}

\subsubsection*{Forventet resultat} 


\subsubsection*{Resultat} 


\textit{Testen er godkendt}

\subsection*{I2C kommunikation med PSoC testprogram}

\subsubsection*{Formål}

\subsubsection*{Fremgangsmåde}
\begin{enumerate}
\item 

\end{enumerate}

\subsubsection*{Forventet resultat} 


\subsubsection*{Resultat} 


\textit{Testen er godkendt}