\newpage
\section{Modultest}

Modultesten af Controller er opstillet som en modultest, hvor Controlleren er forbundet til et testprint, der simulerer BABYCON-niveauer og en PSoC4 med et testprogram til at kommunikere vha. I2C, hvor statusværdien kan ændres manuelt. I kildekoden er der implementeret en MODULTEST variabel, der kan sættes til true eller false, denne bliver refereret til løbende i modultestbeskrivelsen.

\subsection*{''ON/OFF''-knap}

\subsubsection*{Formål}

At teste om ''ON/OFF''-LEDen lyser, og programmet starter sit ON-loop, når knappen er ON. Samt kontrollere om ''ON/OFF''-LEDen slukker, når knappen er OFF, og programmet forlader sit ON-loop.

\subsubsection*{Fremgangsmåde}
\begin{enumerate}
\item Tilkoble strøm til Controlleren med ''ON/OFF'' knappen i OFF position
\item Start programmet i MODULTEST-tilstand på Raspberry Pien
\item Sæt ''ON/OFF'' knappen i ON position
\item Iagttag ''ON/OFF''-LEDen samt terminal udskrift
\item Sæt ''ON/OFF'' knappen i OFF position
\item Iagttag ''ON/OFF''-LEDen samt terminal udskrift
\end{enumerate}

\subsubsection*{Forventet resultat} 
Det forventes at LEDen tænder, når ''ON/OFF'' knappen bringes i ON position. Det forventes at terminal udskriften viser, at programmet reagerer med udskrifterne [INSIDE ON LOOP], når knappen er i ON position og [OUTSIDE ON LOOP], når knappen er i OFF position.

\subsubsection*{Resultat} 
LEDen tænder og slukker som forventet, og terminaludskriften passer med forventningerne. Se terminaludskriften på figuren herunder \ref{controller:Modul_ONOFF}.
\figur{0.8}{controller/modultest/ON_OFF.pdf}{Terminaludskrift - Modultest:''ON/OFF''-knap}{controller:Modul_ONOFF}

\textit{Testen er godkendt.}


\subsection*{''Manuel start''-knap}

\subsubsection*{Formål}
At sikre ''Manuel start''-knappen aktiverer ISR-rutinen, heri sættes en variabel, som får en funktion til at køre. 

\subsubsection*{Fremgangsmåde}
\begin{enumerate}
\item Start programmet i MODULTEST-tilstand og sæt ''ON/OFF'' knappen i ON position
\item Aktiver ''Manuel start''-knappen ved at trykke på den 
\item Iagttag ''Manuel start''-LED'en og terminaludskriften
\item Afvent at LEDen slukker og iagttag terminaludskriften 
\end{enumerate}

\subsubsection*{Forventet resultat} 
Det forventes at ''Manuel start''-LED'en tænder umiddelbart, når ''Manuel start''-knappen aktiveres. Terminaludskriften reagerer ved: [MANUELSTART ISR] når knappen aktiveres, [MANUELSTART RUNNING] når funktionen kører og [MANUELSTART DONE] når funktionen er kørt. Herefter slukkes ''Manuel start''-LEDen. 

\subsubsection*{Resultat} 
''Manuel start''-LEDen tænder med det samme, når ''Manuel start''-knappen aktiveres som forventet. Og LEDen slukker, når funktionen er kørt. Figur \ref{controller:Modul_manStart} viser terminaludskriften. [MANUELSTART ISR] skrives flere gange, det skyldes kontaktprel og er håndteret ved, at der sættes en global variabel første gang knappen aktiveres, denne variabel nulstilles af funktionen selv.

\figur{0.8}{controller/modultest/manStart}{Terminaludskrift - Modultest:''Manuel start''-knap}{controller:Modul_manStart}

\textit{Testen er godkendt.}

\subsection*{Internet status - Wi-Fi LED}

\subsubsection*{Formål}
At teste systemet med og uden internet tilkoblet.

\subsubsection*{Fremgangsmåde}
\begin{enumerate}
\item Start programmet i MODULTEST-tilstand og sæt ''ON/OFF'' knappen i ON position
\item Fjern internetforbindelsen ved at afbryde routeren fra netværket
\item Iagttag Wi-Fi LEDen og terminaludskrift
\item Tilføj internetforbindelsen ved at tilkoble routeren fra netværket
\item Iagttag Wi-Fi LEDen og terminaludskrift
\end{enumerate}

\subsubsection*{Forventet resultat} 
Det forventes at Wi-Fi LEDen lyser efter maksimum 10 sekunder, og slukker igen maksimum 20 sekunder efter routeren er tilkoblet netværket. Terminalen viser tilsvarende ping udskrifter, som viser status for internetforbindelsen.

\subsubsection*{Resultat} 
Wi-Fi-LEDen lyser som forventet ved afkobling af internettet, og slukker igen ved re-etablering. Tiderne ligger inde for rammerne i de ikke-funktionelle krav. 

\textit{Testen er godkendt.}

\subsection*{BABYCON niveau fra testprint}

\subsubsection*{Formål}
At teste systemts reaktion på de tre BABYCON-niveauer samt ved error/ikke tilkoblet Intelligent lydmonitor.

\subsubsection*{Fremgangsmåde}
\begin{enumerate}
\item Sæt testprintet i BABYCON3
\item Start programmet i MODULTEST-tilstand og sæt ''ON/OFF'' knappen i ON position
\item Iagttag terminaludskriften
\item Sæt testprintet i BABYCON2
\item Iagttag terminaludskriften
\item Sæt testprintet i BABYCON1
\item Iagttag terminaludskriften
\item Sæt testprintet i BABYCON0
\item Sæt ON/OFF-knap i OFF position
\item Sæt ON/OFF-knap i ON position
\item Iagttag terminaludskriften
\end{enumerate}

\subsubsection*{Forventet resultat} 
Terminaludskriften skal vise de respektive BABYCON-niveauer, samt andre MODULTEST udskrifter, der fortæller om, hvilket LOOP man befinder sig i, og hvilke e-mails der sendes.

\subsubsection*{Resultat} 
Resultatet kan ses på følgende terminalskærmbillede figur \ref{controller:BABYCON_test_CTRL}. Det ses at BABYCON-niveauerne skifter som forventet.

\figur{0.8}{controller/modultest/BABYCON_test_CTRL}{Terminaludskrift - Modultest: BABYCON niveauer fra testprint}{controller:BABYCON_test_CTRL} 

\textit{Testen er godkendt.}



\subsection*{Webserver}

\subsubsection*{Formål}
Formålet med denne modultest er, at teste om Baby Watch webserveren kan vise og skifte mellem de tre BABYCON-niveauer, samt om hjemmesiden opdaterer hvert 10. sek.

\subsubsection*{Fremgangsmåde}
\begin{enumerate}
\item Start en Windows 8.1 computer \citep{website:Windows_8_1}, der er på netværk med Baby Watch systemet
\item Lav forbindelse til Baby Watch Controller via ssh
\item Start test programmet på Baby Watch Controller \textit{BW\_website\_test}, som kan findes i \citep{cd} via \textit{/SW/Controller/BW\_website\_test}
\item På computeren åbnes en internetbrowser, der forbindes til hjemmesiden "10.0.2.5"
\item På hjemmesiden vises BABYCON3-niveau jf. afsnit \vref{Ikke-funk:webskitser} i Kravspecifikationen Ikke-funktionelle krav om webskitser
\item I testprogrammet via ssh forbindelsen skrives 2
\item Hjemmesiden opdateres indenfor max. 10 sekunder til BABYCON2-niveau
\item I testprogrammet via ssh forbindelsen skrives 1
\item Hjemmesiden opdateres indenfor max. 10 sekunder til BABYCON1-niveau
\item Der ventes i max. 10 sekunder på, at hjemmesiden opdaterer igen
\end{enumerate}

\subsubsection*{Forventet resultat} 
Det forventes at hjemmesiden kan skifte mellem alle BABYCON-niveauer, og at hjemmesiden opdateres hvert 10. sekund.

\subsubsection*{Resultat} 
Hjemmesiden og derved webserveren gør som forventet:
\figur{0.55}{controller/modultest/server_BC3_test.pdf}{Baby Watch hjemmesiden viser BABYCON3-niveau}{controller:BW_Web3} 


\figur{0.55}{controller/modultest/server_BC2_test.pdf}{Baby Watch hjemmesiden viser BABYCON2-niveau}{controller:BW_Web2}

\figur{0.55}{controller/modultest/server_BC1_test.pdf}{Baby Watch hjemmesiden viser BABYCON1-niveau}{controller:BW_Web1}

\figur{0.55}{controller/modultest/server_BC1_test_10sec.pdf}{Hjemmesiden er opdateret 10 sekunder efter i forhold til foregående billede (markeret med rød cirkel)}{controller:BW_Web1_10sek}

\textit{Testen er godkendt.}












\subsection*{I2C kommunikation med PSoC testprogram}

\subsubsection*{Formål}
Formålet med denne modultest er, at teste I2C kommunikationen med Vuggesystemet vha. et testprogram på PSoCen, der udskriver i et terminalvindue via UART.

\subsubsection*{Fremgangsmåde}
\begin{enumerate}
\item Forbind PSoCen (med testprogrammet på) til computeren og start TeraTerm
\item Test UART ved tryk på reset på PSoC - se terminal udskrift ''Baby Watch - USB UART!''
\item Sæt testprintet i BABYCON3
\item Sæt ON/OFF-knap i ON
\item Iagttag terminaludskriften
\item Sæt testprintet i BABYCON2 
\item Iagttag terminaludskriften
\item Sæt testprintet i BABYCON1
\item Sæt ON/OFF-knap i OFF
\end{enumerate}

\subsubsection*{Forventet resultat} 
TeraTerm-udskrifterne skal udskrive værdier, der passer med testværdierne.
\begin{enumerate}
\item ON:  0xF0
\item OFF: 0x0F
\item BABYCON1 -  Frekvens: 0x00 Vinkel: 0x00
\item BABYCON2A - Frekvens: 0x32 Vinkel: 0xC8
\item BABYCON2B - Frekvens: 0x64 Vinkel: 0x78
\item BABYCON2C - Frekvens: 0xC8 Vinkel: 0x50
\item BABYCON3 -  Frekvens: 0x00 Vinkel: 0x00
\end{enumerate}

\subsubsection*{Resultat} 
TeraTerm udskriver som forventet de afsendte værdier i figur \ref{controller:BABYCON_test_VS}. På figuren er indsat BABYCON-niveauer for forståelsens skyld.

\figur{0.9}{controller/modultest/BABYCON_test_VS}{TeraTer-udskrift - Modultest: I2C kommunikation med PSoC testprogram}{controller:BABYCON_test_VS} 

\textit{Testen er godkendt.}

\subsection*{Fejlhåndtering}

\subsubsection*{Formål}
Formålet med denne modultest er, at teste om Controller reagerer korrekt på en error fra Vuggesystemet. Testprogrammet er programmeret med en statusværdi på \verb+0b10000000(0x80)+. 

\subsubsection*{Fremgangsmåde}
\begin{enumerate}
\item Forbind PSoCen (med testprogrammet på) til computeren og start TeraTerm
\item Test UART ved tryk på reset på PSoC - se terminaludskrift ''Baby Watch - USB UART!''
\item Sæt testprintet i BABYCON3
\item Sæt ON/OFF-knap i ON
\item Iagttag terminaludskriften
\end{enumerate}

\subsubsection*{Forventet resultat} 
TeraTerm-udskrifterne skal udskrive den første BABYCON3 skrivning fra TeraTerm, som slutter med at aflæse status. 

\subsubsection*{Resultat} 
På figur \ref{controller:error_vs} ses, at den første BABYCON3 sekvens modtages, og i statusregisteret står 0x10 som forventet. Dette handles der på i Controller og den går i fejl-tilstand og stopper vugningen.  

\figur{0.4}{controller/modultest/error_vs}{TeraTerm-udskrift - Modultest: Fejlhåndtering}{controller:error_vs} 

\textit{Testen er godkendt.}

\subsection*{Powercontrol}

\subsubsection*{Formål}
At teste funktionaliteten af powerOn og powerOff signal fra Controller til PSU.

\subsubsection*{Fremgangsmåde}
\begin{enumerate}
\item Start programmet
\item Sæt ON/OFF-knap til ON
\item Mål udgangsspænding (powOn\_out) i forhold til ground
\item Sæt ON/OFF-knap til OFF
\item Mål udgangsspænding (powOn\_out) i forhold til ground

\end{enumerate}

\subsubsection*{Forventet resultat}
Når ON/OFF-knappen sættes i ON position, forventes det at spændingen på (powOn\_out) i forhold til ground, er 3,3 V. Omvendt når ON/OFF-knappen sættes i OFF position forventes det, at spændingen afbrydes, og derfor er 0 V.

\subsubsection*{Resultat} 
Når ON/OFF-knappen sættes i ON position er spændingen 3,3 V, og når ON/OFF-knappen sættes i OFF position er spændingen 0 V som forventet.

\textit{Testen er godkendt.}