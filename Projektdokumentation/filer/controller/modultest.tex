\section{Modultest}

Modultesten af Controller er opstillet som en modultest hvor Controlleren er forbundet til et testprint der simulerer BABYCON-niveauer og en PSoC4 med et testprogram til at kommunikere vha. I2C, hvor statusværdien kan ændres manuelt. I kildekoden er der implementeret en MODULTEST variabel, der kan sættes til true eller false, denne bliver refereret til løbende i modultest beskrivelsen.

\subsection*{''ON/OFF''-knap}

\subsubsection*{Formål}

At teste om ''ON/OFF''-LEDen lyser og programmet starter sit ON-loop når knappen er ON. Samt kontrollere om ''ON/OFF''-LEDen slukker når knappen er OFF og programmet forlader sit ON-loop.

\subsubsection*{Fremgangsmåde}
\begin{enumerate}
\item Tilkoble strøm til Controlleren med ''ON/OFF'' knappen i OFF position
\item Start programmet i MODULTEST-tilstand på Raspberry Pi'en
\item Sæt ''ON/OFF'' knappen i ON position
\item Iagttag ''ON/OFF''-LEDen samt terminal udskrift
\item Sæt ''ON/OFF'' knappen i OFF position
\item Iagttag ''ON/OFF''-LEDen samt terminal udskrift
\end{enumerate}

\subsubsection*{Forventet resultat} 
Det forventes at LEDen tænder når ''ON/OFF'' knappen bringes i ON position. Det forventes at terminal udskriften viser at programmet reagerer med udskrifterne [INSIDE ON LOOP] når knappen er i ON position og [OUTSIDE ON LOOP] når knappen er i OFF position.

\subsubsection*{Resultat} 
LEDen tænder og slukker som forventet og terminaludskriften passer med forventningerne. Se terminaludskriften på figuren herunder \ref{controller:Modul_ONOFF}
\figur{0.8}{controller/modultest/ON_OFF.pdf}{Terminaludskrift - Modultest:''ON/OFF''-knap}{controller:Modul_ONOFF}

\textit{Testen er godkendt}


\subsection*{''Manuel start''-knap}

\subsubsection*{Formål}
At sikre ''Manuel start'' knappen aktiverer ISR-rutinen, her sættes en variabel som får en funktion til at køre. 

\subsubsection*{Fremgangsmåde}
\begin{enumerate}
\item Start programmet i MODULTEST-tilstand og sæt ''ON/OFF'' knappen i ON position
\item Aktiver ''Manuel start''-knappen ved at trykke på den 
\item Iagttag ''Manuel start''-LED'en og terminaludskriften
\item Afvent at LED'en slukker og iagttag terminaludskriften 
\end{enumerate}

\subsubsection*{Forventet resultat} 
Det forventes at ''Manuel start''-LED'en tænder umiddelbart når ''Manuel start''-knappen aktiveres. Terminal udskriften reagerer ved [MANUELSTART ISR] når knappen aktiveres, [MANUELSTART RUNNING] funktionen kører og [MANUELSTART DONE] når funktionen er kørt. Herefter slukkes ''Manuel start''-LED'en 

\subsubsection*{Resultat} 
''Manuel start''-LED'en tænder med det samme når ''Manuel start''-knappen aktiveres, som forventet. Og LED'en slukker når funktionen er kørt. Figur \ref{controller:Modul_manStart} viser terminal udskriften. [MANUELSTART ISR] skrives flere gange, det skyldes kontaktprel og er håndteret ved at der sættes en global variabel første gang knappen aktiveres, denne variabel nulstilles af funktionen selv.

\figur{0.8}{controller/modultest/manStart}{Terminaludskrift - Modultest:''Manuel start''-knap}{controller:Modul_manStart}

\textit{Testen er godkendt}

\subsection*{Internet status - Wi-Fi LED}

\subsubsection*{Formål}
At teste systemet med og uden internet tilkoblet

\subsubsection*{Fremgangsmåde}
\begin{enumerate}
\item Start programmet i MODULTEST-tilstand og sæt ''ON/OFF'' knappen i ON position
\item Fjern internet forbindelsen ved at afbryde routeren fra netværket
\item Iagttag Wi-Fi LED'en og terminaludskrift
\item Tilføj internet forbindelsen ved at tilkoble routeren fra netværket
\item Iagttag Wi-Fi LED'en og terminaludskrift
\end{enumerate}

\subsubsection*{Forventet resultat} 
Det forventes at Wi-Fi LEDen lyser efter maks. 10 sekunder og slukker igen maks. 20 sekunder efter routeren er tilkoblet netværket. Terminalen viser tilsvarende ping udskrifter som viser status for internetforbindelsen.

\subsubsection*{Resultat} 
Wi-Fi-LEDen lyser som forventet ved afkobling af internettet. Og slukker igen ved re-tilkobling.

\textit{Testen er godkendt}

\subsection*{BABYCON niveau fra testprint}

\subsubsection*{Formål}
At teste systemts reaktion på de 3 BABYCON niveauer samt ved error/ikke tilkoblet Intelligent Lydmonitor.

\subsubsection*{Fremgangsmåde}
\begin{enumerate}
\item Sæt testprintet i BABYCON3
\item Start programmet i MODULTEST-tilstand og sæt ''ON/OFF'' knappen i ON position
\item Iagttag terminaludskriften
\item Sæt testprintet i BABYCON2
\item Iagttag terminaludskriften
\item Sæt testprintet i BABYCON1
\item Iagttag terminaludskriften
\item Sæt testprintet i BABYCON0
\item Sæt ON/OFF-knap i OFF position
\item Sæt ON/OFF-knap i ON position
\item Iagttag terminaludskriften
\end{enumerate}

\subsubsection*{Forventet resultat} 
Terminal udskriften skal vise de respektive BABYCON niveauer, samt andre MODULTEST udskrifter, der fortæller om hvilke LOOP man befinder sig i, og hvilke e-mails der sendes

\subsubsection*{Resultat} 
Resultatet kan ses på følgende terminal skærmbillede figur \ref{ctrl}. Det ses at BABYCON niveauerne skifter som forventet.

\figur{0.8}{controller/modultest/BABYCON_test_CTRL}{Terminaludskrift - Modultest: BABYCON niveauer fra testprint}{controller:BABYCON_test_CTRL} 

\textit{Testen er godkendt}

\subsection*{I2C kommunikation med PSoC testprogram}

\subsubsection*{Formål}
Formålet med denne modultest er at teste I2C kommunikationen med vuggesystemet vha. et testprogram på PSoCen, der udskriver i et terminalvindue via UART.

\subsubsection*{Fremgangsmåde}
\begin{enumerate}
\item Start TeraTerminal, med PSoCen forbundet med testprogrammet på
\item Test UART ved tryk på reset på PSoC - se terminal udskrift ''Baby Watch - USB UART!''
\item Sæt testprintet i BABYCON3
\item Sæt ON/OFF-knap i ON
\item Iagttag terminaludskriften
\item Sæt testprintet i BABYCON2 
\item Iagttag terminaludskriften
\item Sæt testprintet i BABYCON1
\item Sæt ON/OFF-knap i OFF
\end{enumerate}

\subsubsection*{Forventet resultat} 
TeraTerminalen udskrifterne skal udskrive værdier der passer med testværdierne.
\begin{enumerate}
\item ON:  0xF0
\item OFF: 0x0F
\item BABYCON1 -  Frekvens: 0x00 Vinkel: 0x00
\item BABYCON2A - Frekvens: 0x32 Vinkel: 0xC8
\item BABYCON2B - Frekvens: 0x64 Vinkel: 0x78
\item BABYCON2C - Frekvens: 0xC8 Vinkel: 0x50
\item BABYCON3 -  Frekvens: 0x00 Vinkel: 0x00
\end{enumerate}

\subsubsection*{Resultat} 
TeraTerminalen udskriver som forventet de afsendte værdier i figur \ref{controller:BABYCON_test_VS}. På figuren er indsat BABYCON niveauer for forståelsens skyld.

\figur{0.9}{controller/modultest/BABYCON_test_VS}{TeraTerminaludskrift - Modultest: I2C kommunikation med PSoC testprogram}{controller:BABYCON_test_VS} 

\textit{Testen er godkendt}

\subsection*{Fejlhåndtering}

\subsubsection*{Formål}
Formålet med denne modultest er at teste om Controller reagerer korrekt på en error fra Vuggesystemet. Testprogrammet er programmeret med en statusværdi på \verb+0b10000000(0x80)+. 

\subsubsection*{Fremgangsmåde}
\begin{enumerate}
\item Start TeraTerminal, med PSoCen forbundet med testprogrammet på
\item Test UART ved tryk på reset på PSoC - se terminal udskrift ''Baby Watch - USB UART!''
\item Sæt testprintet i BABYCON3
\item Sæt ON/OFF-knap i ON
\item Iagttag terminaludskriften
\end{enumerate}

\subsubsection*{Forventet resultat} 
TeraTerminalen udskrifterne skal udskrive den første BABYCON3 skrivning fra TeraTerminalen, som slutter med at aflæse status. 

\subsubsection*{Resultat} 
På figur \ref{controller:error_vs} ses at den første BABYCON3 sekvens modtages, og i statusregisteret står 0x10 som forventet, dette handles der på i Controller og den går i fejl-tilstand og stopper vugningen.  

\figur{0.4}{controller/modultest/error_vs}{TeraTerminaludskrift - Modultest: Fejlhåndtering}{controller:error_vs} 

\textit{Testen er godkendt}

\subsection*{I2C kommunikation med PSoC testprogram}

\subsubsection*{Formål}
Formålet med denne modultest er at teste I2C kommunikationen med vuggesystemet vha. et testprogram på PSoCen, der udskriver i et terminalvindue via UART.

\subsubsection*{Fremgangsmåde}
\begin{enumerate}
\item Start TeraTerminal, med PSoCen forbundet med testprogrammet på
\item Test UART ved tryk på reset på PSoC - se terminal udskrift ''Baby Watch - USB UART!''
\item Sæt testprintet i BABYCON3
\item Sæt ON/OFF-knap i ON
\item Iagttag terminaludskriften
\item Sæt testprintet i BABYCON2 
\item Iagttag terminaludskriften
\item Sæt testprintet i BABYCON1
\item Sæt ON/OFF-knap i OFF
\end{enumerate}

\subsubsection*{Forventet resultat} 
TeraTerminalen udskrifterne skal udskrive værdier der passer med testværdierne.
\begin{enumerate}
\item ON:  0xF0
\item OFF: 0x0F
\item BABYCON1 -  Frekvens: 0x00 Vinkel: 0x00
\item BABYCON2A - Frekvens: 0x32 Vinkel: 0xC8
\item BABYCON2B - Frekvens: 0x64 Vinkel: 0x78
\item BABYCON2C - Frekvens: 0xC8 Vinkel: 0x50
\item BABYCON3 -  Frekvens: 0x00 Vinkel: 0x00
\end{enumerate}

\subsubsection*{Resultat} 
TeraTerminalen udskriver som forventet de afsendte værdier i figur \ref{controller:BABYCON_test_VS}. På figuren er indsat BABYCON niveauer for forståelsens skyld.

\figur{0.9}{controller/modultest/BABYCON_test_VS}{TeraTerminaludskrift - Modultest: I2C kommunikation med PSoC testprogram}{controller:BABYCON_test_VS} 

\textit{Testen er godkendt}

\subsection*{Powercontrol}

\subsubsection*{Formål}
At teste funktionaliteten af power on og off signal fra Controller til PSU.

\subsubsection*{Fremgangsmåde}
\begin{enumerate}
\item 
\item Sæt ON/OFF-knap til ON
\

\end{enumerate}

\subsubsection*{Forventet resultat} 


\subsubsection*{Resultat} 

\textit{Testen er godkendt}