\section{Systemarkitektur}

I dette afsnit beskrives systemarkitekturen for Controller.

Softwarearkitekturen er beskrevet i sekvensdiagrammer, der giver et indblik i, hvordan mainprogrammet interagerer med den Intelligente lydmonitor og Vuggesystemet.

Hardwarearkitekturen beskriver den fysiske grænseflade til den Intelligente lydmonitor og Vuggesystemet.

\subsection{Softwarearkitektur}

\subsubsection{Sekvensdiagram (BABYCON-niveauer)}

\figur{0.8}{controller/sysark/Controller_Sekvensdiagram_Babycon}{Sekvensdiagram for Babycon niveauer}{controller:sd_babycon}

Sekvensdiagrammet for BABYCON-niveauerne beskriver hvordan Controlleren reagere på BABYCON-niveauer fra Intelligent lydmonitor. Samt hvad der kommunikeres ud til Vuggesystemet.
Det er tænkt som et stort loop der hele tiden kigger på variabler og handler på disse. BABYCON har så 3 niveauer og en fejltilstand 0, som den reagerer på. BABYCON2 har sine egne loops at switche imellem, disse kan dog afbrydes af fejl og slukning af systemet.

\subsubsection{Sekvensdiagram (''Manuel start'')}

\figur{0.9}{controller/sysark/Controller_Sekvensdiagram_ManStart}{Sekvensdiagram for ''Manuel Start''}{controller:sd_manstart}

''Manuel start'' skal køre et loop i 5 min. fordelt på hhv. 2 min. ''forced'' loop hvor der vugges uden hensyntagen til BABYCON-niveau og i 3 min. hvor BABYCON1 kan afbryde vugningen. ''Manuel start'' kan selvfølgelig afbrydes af et Power OFF eller fejl i vuggesystemet.

\subsubsection{Sekvensdiagram (Power Off)}

\figur{0.9}{controller/sysark/Controller_Sekvensdiagram_PowerOff}{Sekvensdiagram for nedlukning}{controller:sd_poweroff}

Power off skal starte et loop der kigger på om vuggesystemet er klar til at lukke ned. Dvs. der sendes et power off signal til Vuggesystemet og ventes på et ''shutdown ready'' fra samme. Når signalet er modtaget, kan der afbrydes for strømmen til Vuggesystemet og Int. lydmonitor.  


\subsection{Hardware arkitektur}

\figur{1}{controller/sysark/Controller_IBD}{IBD diagram for Controller}{controller:ibd}

Figur \ref{controller:ibd} viser IBDet for Controlleren. Signaltabellen, tabel \ref{controller:signaltabel},  specificerer de interne forbindelser. De to lidt bredere sorte forbindelser i IBDet angiver en forbindelse, der består af flere ledere frem for de alm. forbindelser, der udgør en enkelt ledet forbindelse. 

Controlleren består af to dele:
\begin{description}
\item[Raspberry Pi] Linux baseret computer der styrer hele systemet. 
\item[Betjeningspanel] Panel med tre lysdioder og to trykknapper til brugerinteraktion.
\end{description}

\newpage

\subsection{Signaltabel}

I signaltabellen beskrives hvilke signaler, der sendes imellem blokkene i Controller.  

\begin{center}
\label{controller:signaltabel}
\begin{longtable}{|p{2,5cm}|p{1,8cm}|p{2,6cm}|p{2,8cm}|p{3cm}|}
\caption{Signaltabel for Controller}\\
\hline
\textbf{Signal-navn}	&\textbf{Type} 		&\textbf{Port 1} 	&\textbf{Port 2} 			&\textbf{Kommentar} \\ \hline
\endfirsthead
\multicolumn{5}{l}{...fortsat fra forrige side} \\ \hline 
\textbf{Signal-navn}	&\textbf{Type} 		&\textbf{Port 1} 	&\textbf{Port 2} 			&\textbf{Kommentar} \\ \hline
\endhead

%% Signal-navn
%% &Type
%% &Område
%% &Port1
%% &Port2
%% &Kommentar
%% \\\hline

manStartOn
&touch
&Betjeningspanel \newline (manStart\_in)
&Babypasser \newline (touch)
&Babypasser der laver et tryk på knappen
\\\hline

babyWatchOn
&touch
&Betjeningspanel \newline (onOff\_in)
&Babypasser \newline (touch)
&Babypasser der laver et tryk på knappen
\\\hline

power
&5 VDC
&RasberryPi \newline(micro\_USB) 
&Controller \newline(powC\_in)
&5 V forsyning til controller
\\\hline
power
&3,3 VDC
&RasberryPi \newline(3V3) 
&Betjeningspanel \newline (pow\_in)
&3,3 V forsyning til Betjeningspanel
\\\hline


powerOn			
&TTL 0-3,3 VDC	
&RasberryPi \newline (GPIO23)				
&Controller \newline (powOn\_out) 			
&Benyttes til at tænde og slukket for strømforsyningen   				\\\hline

2WireTTL		
&TTL 0-3,3 VDC			
&RasberryPi \newline (2WTTLRPI) \newline	
	GPIO9: LSB \newline	
	GPIO10: MSB \newline	
&Controller \newline (I2CController) 			
&Seriel kommunikation
\\\hline

I2C			
&Seriel		
&RasberryPi \newline (I2CRPI) \newline	
	GPIO0: SDA \newline	
	GPIO1: SCL \newline	
&Controller \newline (Controller) 			
&BABYCON niveau
\\\hline

USB		
&Seriel				
&RasberryPi \newline (USB\_in) 			
&Wi-Fi dongle \newline (USB\_out) \newline	
&Seriel kommunikation for Wi-Fi forbindelse
\\\hline

ledControl
&TTL 0-3,3 VDC		
&RaspberryPi \newline (GPIO14) \newline
 RaspberryPi \newline (GPIO15) \newline
 RaspberryPi \newline (GPIO18)
&Betjeningspanel \newline (onOffLed\_in) \newline
 Betjeningspanel \newline (manLed\_in) \newline
 Betjeningspanel \newline (wifiLed\_in)
&Logisk signal til at styre LED på Betjenningspanel
\\\hline

butPress
&TTL 0-3,3 VDC		
&RaspberryPi \newline (GPIO17) \newline
 RaspberryPi \newline (GPIO22)
&Betjeningspanel \newline (onOff\_out) \newline
 Betjeningspanel \newline (manStart\_out)
&Logisk signal til kontrol af knapper på betjeningspanel
\\\hline

\end{longtable}
\end{center}

Figur \ref{controller:RPI_GPIO} viser den 26 pins GPIO port på Raspberry Pi'en. Der er desuden tilføjet enkelte portes specielle muligheder (SPI, I2C mm.), hvis de er tilstede. 

\figur{0.7}{controller/RaspberryGPIO.pdf}{Raspberry Pi Model Bs GPIO port}{controller:RPI_GPIO}


