%psu
\chapter{Power Supply Unit}

Dette kapitel omhandler strømdistributionen for Baby Watch. Som beskrevet kapitlet Overordnet så styrer Controlleren hvorvidt Vuggesystemet og den Intelligente Lydmonitor skal forsynes med strøm. Der benyttes et relæ til at styre dette. Forsyningsbatteriet skal reguleres ned til 3 forskellige spændinger:

\begin{itemize}
\item \textbf{Controller:} Kræver 5 V / 1200 mA (Raspberry Pi + Betjeningspanel)

\item \textbf{Vuggesystem:} Kræver 3,3 V / 500 mA (PSoC + Motorstyringskreds)

\item \textbf{Intelligent Lydmonitor:} Kræver 7.5 V / 1500 mA (BlackFin 533 kit + Mikrofonkreds)
\end{itemize}

Ved nedkonverteringen til 7,5 V benyttes en DC-DC konverter (LM2596S). Fra denne spænding reguleres der til hhv. 5 V og 3,3 V vha. spændingsregulatorer (LM317). 


\subsection{DC-DC konverter (LM2596S)}

DC-DC konverteren (LM2596S), se figur \ref{psu:DCkonv}, kan tage spændingsinput på 5 V til 35 V og outputte 1,25 V til 30 V. Maks strømoutput er 4 A og overstiger effekten 15 W, så skal den køles. I dette tilfælde skal DC-DC konverteren konvertere 12 V DC til 7,5 V DC. Ved 7,5 V og 15 W, så er der 2 A, da der skal benyttes maksimum 2,5 A, ca. 19 W, derfor er der i implementeringen påsat køleplade på begge sider af den regulator der skal køles. 

\figur{0.4}{psu/LM2596S_fabriksbillede}{Fabriksbillede af LM2596S - DC-DC konverter}{psu:DCkonv}

Som figur \ref{psu:DCkonv} er der 3 indstilige potentiometre samt 4 lodde øer. De 4 lodde øer er VIN-, VIN+, VOUT- og VOUT+. VIN- og VOUT- er ikke isoleret fra hinanden og kobles til samme stelplan. De 3 potentiometre bruges til at indstille udgangsspænding, maksimum udgangsstrøm og det sidste bruges til at styre en LED som lyser når strømmen når over det indstillede niveau. 

\subsection{Spændingsregulatorer (LM317)}

Spændingsregulatoren LM317 er variabel justerbar mellem 1,2 V til 33 V og kan levere op til 3 A. Lm317 benyttes derfor til at regulere fra 7,5 V til hhv. 5 V og 3,3 V. 

\figur{0.4}{psu/LM317_schematic.pdf}{Schematic for LM317}{psu:LM317_schematic}

Den variable udgangsspænding beregnes via formlen \ref{eq:Vout}. $I_{Q}$ er strømmen der løber fra ben 1, den er opgivet til at være <100uA og typisk 50uA, som er den værdi der vælges at benytte i formlen. Spændingsforskellen mellem udgangen (ben 2) og referencen (ben 1) er de 1,25 V der også optræder i formlen for $V_{out}$   

\begin{equation} 
{ V }_{ out }=1,25V\cdot \left( 1+\frac { R2 }{ R1 }  \right) +{ I }_{ Q }\cdot R2
\label{eq:Vout}
\end{equation}

Foruden den afsætte effekt til det objekt regulatoren skal forsyne, så løber der en strøm gennem R1 og R2, dette er et effekttab som skal regnes med i den samlede effekt afsættelse der afgør hvorvidt regulatoren skal køles. Strømen gennem R1 og R2 kaldes $I_{G}$ og beregnes vha. ohms lov:

\begin{equation} 
{ I }_{ G }=\frac { 1,25V }{ 240\Omega  } +50uA=5,26mA
\label{eq:IG}
\end{equation}

Den samlede effekt beregnes ved formel \ref{eq:P_lm317} 

\begin{equation} 
{ P }=({ V }_{ in }-{ V }_{ out })\cdot { I }_{ L }+{ V }_{ in }\cdot { I }_{ G }
\label{eq:P_lm317}
\end{equation}


\textbf{5 V til Controller} \newline
5 V skal benyttes til at forsynes Raspberry Pi'en. Ligning \ref{eq:Vout5V} viser beregningen af R2 når ønsket udgangsspænding er 5 V.  

\begin{equation} 
{ V }_{ out }=1,25V\cdot \left( 1+\frac { 720\Omega  }{ 240\Omega  }  \right) +50uA\cdot 720\Omega =5V
\label{eq:Vout5V}
\end{equation}

R2 er beregnet til 720$\Omega$ og tilhørende simulering ses på figur \ref{psu:LM3175Vsim}. Simuleringen viser som forventet at udgangsspændingen er 5V. 

\figur{0.6}{psu/LM3175V_simulering.pdf}{Simulering af LM317 justeret til 5 V}{psu:LM3175Vsim}

Raspberry Pi'en forbruger maksimalt 1200mA og typisk 700mA i den ønskede opstilling. Der regnes med den maksimale værdi i forhold til at analyse på worst case tilfældet. Strømmen til Raspberry Pi'en skrives som $I_{L}$

\begin{equation} 
{ P }_{ 5V }=(7,5V-5V)\cdot 1200mA+7,5V\cdot 5,26mA=3,39W
\label{eq:P5V}
\end{equation}

Ud fra formel \ref{eq:P5V} så er den maksimalt afsatte effekt i regulatoren 3,39W hvilket kræver køleplade og aktiv køling f.eks i form af en blæser, da effektafsættelsen er større en 1W som er grænsen for hvornår regulatoren kræver køling. Hvis der regnes med en $I_{L}=700mA$ så bliver effektafsættelsen reduceret til 2,14W.



\textbf{3,3 V til Vuggesystem}\newline
3,3 V til Vuggesystemet skal bruges til at forsyne PSoC'en, motorstyringskreds samt sensorer. Strømforbruget er sat til et maksimum på 500mA.

Ligning \ref{eq:Vout3.3V} viser at R2 er fundet til 390$\Omega$ når udgangsspændingen skal være 3,3 V. Figur \ref{psu:LM31733Csim} viser simuleringen og verificerer beregningen. 

\begin{equation} 
{ V }_{ out }=1,25V\cdot \left( 1+\frac { 390\Omega  }{ 240\Omega  }  \right) +50uA\cdot 390\Omega =3,3V
\label{eq:Vout3.3V}
\end{equation}

\figur{0.6}{psu/LM31733V_simulering.pdf}{Simulering af LM317 justeret til 3,3 V}{psu:LM31733Vsim}

Der foretages også effektberegning på denne regulator. Effektafsættelsen er mindre end ved 5 V reguleringen, men stadig over 1 W, så der skal også monteres køleplade på denne regulator.

\begin{equation} 
{ P }_{ 5V }=(7,5V-3,3V)\cdot 500mA+7,5V\cdot 5,26mA=2,5W
\label{eq:P3.3V}
\end{equation}


\subsection{Relæ styring}

Forsyningen til den Intelligente Lydmonitor og til Vuggesystemet skal kunne afbrydes af Controlleren, når ON/OFF-knappen på betjeningspanelet sættes i OFF position. Vuggesystemet giver signal via I2C at forsyningen kan afbrydes og herefter sender Controlleren via signalet PowerOn et signal til PSU'en at 3,3 V og 7,5 V forsyningen skal afbrydes. Afbrydelsen foregår vha. et relæ. Relæet der benyttes er et Finder 40.52S som er godkendt til en middel belastning på 8 A hvilket dækker behovet. 

\figur{0.6}{psu/relay4052.pdf}{Udsnit af datablad for Finder 40.52S}{psu:relayDatasheet}

Figur \ref{psu:relayDatasheet} viser et udsnit af databladet for Finder 40.52S. Figuren viser at 5 V relæet fungerer med en spænding over spolen på 3,7 V til 7,5 V samt at relæet kræver 100 mA ved 5 V for at trække.

Raspberry Pi'ens GPIO23 skal sende powerOn signalet jf. Systemarkitekturen. Raspberryen kan hverken forsyne relæet med 5 V eller 100 mA, derfor benyttes en transistor. BC517 kan løse opgaven da denne kan tåle 500 mA i kollektor strøm. Strømforstækningen max $h_{fe}$ er 30000, så i princippet burde en basestrøm på 3 uA være nok til de 100 mA. Der er dog valgt at basestrømmen skal være 0,1 mA for ikke at gå helt til grænsen for hvad der er minimum for at opnå de 100 mA. 

Raspberry Pi'en afgiver 3,3 V og BC517 har et base emitter spændingsfald på 1,4 V da det er en Darlington. 

\begin{equation}
{ R }_{ base }=\frac { 3,3V-1,4V }{ 0,1mA } =\frac { 1,9V }{ 0,1mA } =19k\Omega
\label{eq:R_base}
\end{equation}

Ligning \ref{eq:R_base} viser udregningen af base modstanden. Modstandens værdi vælges til 18$\Omega$ da dette er en E24 standart.  

Da relæet kan operere ved 7,5 V, benyttes denne spænding til at trække relæet og spændingen leveres af DC-DC konverteren. 

Samlet giver det anledning til nedenstående kredsløbsdiagram for relæstyringen se figur \ref{psu:relaySim}

\figur{0.7}{psu/relaycontrol.pdf}{Kredsløbsdiagram for relæstyring}{psu:relaySim}

Dioden (1N4007) i figur \ref{psu:relaySim} er indsat for at beskytte transistoren mod tilbageløbende strøm når relæet klikker. 


\subsection{Samlet PSU print}

\figur{1}{psu/PSUmultisim.pdf}{Samlet kredsløbsdiagram for PSU'en}{psu:PSUsamlet}

Figur \ref{psu:PSUsamlet} viser det samlede kredsløbsdiagram for PSU'en med tilhørende powOn\_out pin som Raspberry Pi'ens GPIO23 skal forbindes til. Der er desuden indtegnet en hovedafbryder til PSU'en samt 12 V forbindelse til den blæser der skal køle spændingsregulatorerne.

EAGLE billeder af printudlæg 